\chapter{Wärmelehre}\label{chap: Waermelehre}
Die Erkenntnis, dass Wärme eine Form von Energie ist -- genauer gesagt, die mikroskopische Bewegungsenergie der Atome und Moleküle eines Körpers -- und dass mechanische Arbeit direkt in Wärme umgewandelt werden kann, ist eine der fundamentalen Säulen der modernen Physik. Diese Einsicht, die uns heute fast selbstverständlich erscheint, wurde maßgeblich durch die Arbeiten von Pionieren wie Julius Robert Mayer geprägt, der 1842 das Prinzip der Energieerhaltung bei der Umwandlung von mechanischer in thermische Energie formulierte.

Die Entwicklung der \textbf{kinetischen Gastheorie} im 19. Jahrhundert lieferte schließlich die mikroskopische Deutung der Wärmeenergie als die Summe der kinetischen und potenziellen Energien der Moleküle eines Körpers.

\section{Temperatur und kinetische Gastheorie}\label{sec: Temperatur_kinetische_Gastheorie}
In der kinetischen Gastheorie wird die Temperatur eines Systems direkt mit der durchschnittlichen kinetischen Energie seiner Teilchen verknüpft. Die \textbf{absolute Temperatur} $T$, gemessen in Kelvin (\si{\kelvin}), ist proportional zur mittleren kinetischen Energie der Translation $\overline{E_\mathrm{kin}}$ der Moleküle:
\begin{equation}\label{eq:def_temperatur_kinetisch}
    \overline{E_\mathrm{kin}} = \frac{1}{2}m\overline{v^2} = \frac{3}{2} k_B T \mDot
\end{equation}
Hierbei ist $m$ die Masse eines Moleküls, $\overline{v^2}$ das mittlere Geschwindigkeitsquadrat und $k_B = \SI{1.380649e-23}{\joule\per\kelvin}$ die \textbf{Boltzmann-Konstante}. Diese Gleichung ist eine der zentralen Aussagen der kinetischen Gastheorie und verbindet die makroskopische Größe der Temperatur mit der mikroskopischen Welt der Teilchenbewegung.

\begin{rememberbox}{Das thermodynamische System}
    Als thermodynamisches System bezeichnen wir eine Ansammlung von Atomen oder Molekülen, die mit ihrer Umgebung Energie in Form von Wärme oder mechanischer Arbeit austauschen kann. Ein solches System wird durch makroskopische Zustandsgrößen wie Temperatur, Druck, Volumen und Teilchenzahl beschrieben.
\end{rememberbox}

\section{Temperaturmessung und Skalen}\label{sec: Temperaturmessung_Skalen}
Zur Messung der Temperatur können prinzipiell alle physikalischen Eigenschaften genutzt werden, die sich vorhersagbar mit der Temperatur ändern. Beispiele hierfür sind:
\begin{itemize}[itemsep=1.5pt]
    \item Die \textbf{thermische Ausdehnung} von Festkörpern, Flüssigkeiten oder Gasen.
    \item Der \textbf{elektrische Widerstand} von Materialien, der bei Metallen typischerweise mit der Temperatur ansteigt.
    \item Die \textbf{elektrische Kontaktspannung} zwischen zwei unterschiedlichen Metallen (Thermoelement).
    \item Die \textbf{Strahlungsleistung}, die von einem heißen Körper emittiert wird.
\end{itemize}
Geräte zur Temperaturmessung werden als \textbf{Thermometer} bezeichnet. Die Funktionsweise und Genauigkeit eines Thermometers hängen von der gewählten physikalischen Eigenschaft und der definierten Temperaturskala ab. Im alltäglichen Gebrauch findet man vor allem Thermometer auf Basis der Volumenänderung von Flüssigkeiten (\textbf{Flüssigkeitsthermometer}) oder Thermometer auf Basis der Änderung der Kontaktspannung zwischen zwei Metallen (\textbf{Thermoelemente}). 

\subsection{Die Celsius- und Fahrenheit-Skala}\label{subsec: Celsius_Fahrenheit}
Historisch wurden verschiedene Temperaturskalen entwickelt, die auf unterschiedlichen Fixpunkten basieren.

\paragraph{Celsius-Skala}
Der schwedische Astronom Anders Celsius schlug 1742 eine Skala vor, die auf zwei leicht reproduzierbaren Fixpunkten basiert:
\begin{itemize}
    \item \textbf{\SI{0}{\celsius}} entspricht dem Schmelzpunkt von Eis.
    \item \textbf{\SI{100}{\celsius}} entspricht dem Siedepunkt von Wasser bei Normaldruck (\SI{1013.25}{\hecto\pascal}).
\end{itemize}
Der Bereich zwischen diesen beiden Punkten wird in 100 gleiche Teile unterteilt, die Grade Celsius ($\si{\celsius}$) genannt werden. Jedes Skalenteil entspricht daher $\SI{1}{\celsius}$.

\paragraph{Fahrenheit-Skala}
Diese Skala, die heute vor allem in den USA gebräuchlich ist, wurde von Daniel Gabriel Fahrenheit definiert. Ihre Fixpunkte sind der Schmelzpunkt einer speziellen Eis-Salz-Mischung (\SI{0}{\fahrenheit} $\approx \SI{-17.8}{\celsius}$) und die angenommene menschliche Körpertemperatur (\SI{100}{\fahrenheit} $\approx \SI{37.8}{\celsius}$). Auf dieser Skala gefriert Wasser bei \SI{32}{\fahrenheit} und siedet bei \SI{212}{\fahrenheit}. Auch auf dieser Skala wird der Bereich zwischen \SI{0}{\fahrenheit} und \SI{100}{\fahrenheit} in 100 Skalenteile eingeteilt. 

\begin{figure}[htb]
    \centering
    \includegraphics[width=0.55\linewidth]{Bilder/Kapitel_Waermelehre/fixpunkte_celsius_fahrenheit.png}
    \caption{Vergleich der Celsius- und Fahrenheit-Skala mit ihren jeweiligen Fixpunkten (in rot).}\label{fig: temp_skalen_vergleich}
\end{figure}

Die Umrechnung von Fahrenheit ($T_F$) nach Celsius ($T_C$) erfolgt über folgende Formel:
\begin{equation}\label{eq: fahrenheit_celsius_umrechnung}
    T_C\, [\si{\celsius}] = \frac{5}{9} (T_F\, [\si{\fahrenheit}]- 32) \mDot
\end{equation}

\subsection{Die Kelvin-Skala}\label{subsec: Kelvin_Skala}
Die physikalisch fundamentale Skala ist die \textbf{absolute Temperaturskala} oder \textbf{Kelvin-Skala}, benannt nach Lord Kelvin. Sie ist die SI-Basiseinheit der Temperatur.
Im Gegensatz zur Celsius-Skala benötigt die Kelvin-Skala eigentlich nur einen einzigen Fixpunkt:
\begin{itemize}
    \item Der \textbf{absolute Nullpunkt} bei \SI{0}{\kelvin} ist so gesehen kein Fixpunkt. Dies ist die theoretisch tiefstmögliche Temperatur, bei der die Teilchen eines Systems ihre geringstmögliche Energie besitzen.
    \item Der zweite Referenzpunkt ist der \textbf{Tripelpunkt von Wasser}. Dies ist der Punkt, an dem die feste, flüssige und gasförmige Phase von Wasser im thermodynamischen Gleichgewicht existieren. Seine Temperatur ist per Definition exakt auf \SI{273.16}{\kelvin} festgelegt, was \SI{0.01}{\celsius} entspricht.
\end{itemize}

\begin{figure}[htb]
    \centering
    \includegraphics[width=0.5\linewidth]{Bilder/Kapitel_Waermelehre/Phasendiagramm_H2O.png}
    \caption{Schematisches Phasendiagramm $(T, p)$ von Wasser, das den Tripelpunkt zeigt, an dem alle drei Phasen (fest, flüssig, gasförmig) koexistieren.}\label{fig: phasendiagramm_wasser}
\end{figure}

Die Celsius-Skala ist heute über die Kelvin-Skala definiert. Die Umrechnung lautet:
\begin{equation}\label{eq: kelvin_celsius_umrechnung}
    T_C\, [\si{\celsius}] = T\,[\si{\kelvin}] - \num{273.15} \mDot
\end{equation}

\begin{importantbox}[]{Temperaturdifferenzen}
    Die Einteilung der Kelvin- und der Celsius-Skala ist identisch. Das bedeutet, eine Temperaturdifferenz von \SI{1}{\kelvin} entspricht exakt einer Temperaturdifferenz von \SI{1}{\celsius}:
    \begin{equation}
        \Delta T = \SI{1}{\kelvin} \,\Longleftrightarrow \,\Delta T = \SI{1}{\celsius}
    \end{equation}
\end{importantbox}


\subsection{Flüssigkeitsthermometer}\label{subsec: fluessigkeitsthermometer}
Das wohl bekannteste Messinstrument ist das \textbf{Flüssigkeitsthermometer}. Es nutzt die thermische Ausdehnung von Flüssigkeiten, um die Temperatur anzuzeigen. Typischerweise wird dafür Quecksilber (Hg) oder gefärbter Alkohol in einem dünnen Glasröhrchen (Kapillare) verwendet.
\begin{figure}[htb]
    \centering
    \includegraphics[width=0.22\linewidth]{Bilder/Kapitel_Waermelehre/hg_alkohol_fluessigthermometer_vergleich.png}
    \caption{Vergleich eines Quecksilber- (Hg) und eines Alkoholthermometers. Das unterschiedliche Ausdehnungsverhalten der Flüssigkeiten führt zu einer nicht-linearen Skala, wenn eine hohe Genauigkeit erforderlich ist.}\label{fig: hg_alkohol_thermometer}
\end{figure}
Die angezeigte Temperaturskala hängt dabei nicht nur von der Art der Flüssigkeit ab, sondern auch von der Glasart des Röhrchens, da sich auch das Glas selbst ausdehnt. Ein wichtiger Punkt ist, dass sich die Flüssigkeiten im Allgemeinen nicht perfekt linear über den gesamten Temperaturbereich ausdehnen. Vergleicht man beispielsweise ein Alkohol- mit einem Quecksilberthermometer, die beide bei \SI{0}{\celsius} und \SI{100}{\celsius} kalibriert wurden, können bei Zwischentemperaturen leicht unterschiedliche Werte angezeigt werden. Für alltägliche Zwecke ist diese Abweichung oft vernachlässigbar, für präzise wissenschaftliche Messungen sind jedoch gleichmäßigere Skalen erforderlich, wie sie beispielsweise Gasthermometer bieten.


\section{Thermische Ausdehnung}\label{sec: thermische_Ausdehnung}
Die meisten Materialien dehnen sich bei Erwärmung aus und ziehen sich bei Abkühlung zusammen. Dieses Phänomen ist die Grundlage für viele Thermometer und muss in technischen Anwendungen, wie dem Brückenbau oder bei Eisenbahnschienen, berücksichtigt werden.

\subsection{Lineare Ausdehnung fester Körper}\label{subsec: lineare_ausdehnung}
Betrachten wir einen Stab der Länge $L_0$ bei einer Anfangstemperatur $T_0$, dargestellt in \cref{fig: lineare_ausdehnung_L0_L}. Erwärmt man den Stab auf eine Temperatur $T = T_0 + \Delta T$, so ändert sich seine Länge. Für einen begrenzten Temperaturbereich ist die Längenänderung $\Delta L = L-L_0$ in guter Näherung proportional zur Temperaturänderung $\Delta T = T - T_0$ und zur ursprünglichen Länge $L_0$:
\begin{equation}\label{eq: lineare_ausdehnung_delta_L}
    \Delta L = \alpha \cdot L_0 \cdot \Delta T \mDot
\end{equation}
Die Längenänderung $\Delta L$ fällt demnach größer aus, je länger der Stab $(L_0)$ bei der Ausgangstemperatur ist und je mehr man ihn erhitzt ($\Delta T$). Die neue Länge $L(T)$ ist somit:
\begin{equation}\label{eq: lineare_ausdehnung_L_T}
    L(\Delta T) =L_0 + \Delta L =  L_0 (1 + \alpha \cdot \Delta T) \mDot
\end{equation}
Der Proportionalitätsfaktor $\alpha$ wird als \textbf{linearer Längenausdehnungskoeffizient} bezeichnet. Er ist eine materialspezifische Eigenschaft und gibt die relative Längenänderung pro Kelvin an. Aus \cref{eq: lineare_ausdehnung_delta_L} erhalten wir
\begin{equation}\label{eq: def_L_von_DeltaT}
    \alpha = \frac{1}{L_0} \frac{\Delta L}{\Delta T} \mComma 
\end{equation}
wobei $[\alpha] = \si{\kelvin^{-1}}$.
\begin{figure}[htb]
    \centering
    \includegraphics[width=0.5\linewidth]{Bilder/Kapitel_Waermelehre/lineare_ausdehnung_L0_L.png}
    \caption{Ein Stab der Länge $L_0$ bei Temperatur $T_0$ dehnt sich bei einer Erwärmung auf $T = T_0 + \Delta T$ auf die Länge $L = L_0 + \Delta L$ aus.}\label{fig: lineare_ausdehnung_L0_L}
\end{figure}
Genauere Messungen zeigen, dass der Ausdehnungskoeffizient $\alpha$ schwach von der Temperatur abhängt. Für die meisten technischen Anwendungen ist die Annahme eines konstanten Wertes jedoch eine ausreichend gute Näherung. Die in \Cref{tab: lineare_ausdehnungskoeffizienten} zusammengefassten linearen Ausdehnungskoeffizienten verdeutlichen, wie stark sich die Längenausdehnung verschiedener Materialien unterscheiden kann – teilweise um zwei Größenordnungen oder mehr.
\begin{table}[htb]
    \centering
    \caption{Linearer Längenausdehnungskoeffizient $\alpha$ verschiedener fester Stoffe bei Raumtemperatur.}\label{tab: lineare_ausdehnungskoeffizienten}
    \vspace{4pt}
    \begin{tabular}{lc}
        \toprule
        \textbf{Fester Stoff} & \textbf{Linearer Ausdehnungskoeffizient $\alpha$ (\SI{e-6}{\kelvin^{-1}})} \\
        \midrule
        Aluminium     & 23,8 \\
        Eisen         & 12   \\
        V2A-Stahl     & 16   \\
        Kupfer        & 16,8 \\
        Natrium       & 71   \\
        Wolfram       & 4,3  \\
        Invar         & 1,5  \\
        Hartgummi     & 75--100 \\
        \bottomrule
    \end{tabular}
\end{table}


\subsection{Nichtlineare Ausdehnung}\label{subsec: nichtlineare_ausdehnung}
Für größere Temperaturänderungen oder bei Materialien, die ein ausgeprägt nichtlineares Verhalten zeigen, reicht die lineare Näherung aus \cref{eq: lineare_ausdehnung_L_T} nicht mehr aus. In solchen Fällen muss die Temperaturabhängigkeit des Ausdehnungskoeffizienten $\alpha$ selbst berücksichtigt werden. Eine genauere Beschreibung wird oft durch einen quadratischen Ansatz erreicht:
\begin{equation}\label{eq: alpha_von_DeltaT}
    \alpha(\Delta T) = \alpha_0 + \beta \cdot \Delta T \mDot
\end{equation}
Hierbei ist $\alpha_0$ der lineare Ausdehnungskoeffizient bei der Referenztemperatur $T_0$ und $\beta$ ist ein weiterer materialabhängiger Koeffizient, der den quadratischen Anteil der Ausdehnung beschreibt. Setzt man diesen temperaturabhängigen Koeffizienten in die Längengleichung ein, erhält man:
\begin{equation}\label{eq:nichtlineare_ausdehnung}
    L(\Delta T) = L_0 (1+ \alpha(\Delta T)\cdot \Delta T) = L_0 (1 + \alpha_0 \Delta T + \beta \Delta T^2) \mDot
\end{equation}
Der Term $\beta \Delta T^2$ repräsentiert die Abweichung von der linearen Ausdehnung.

\begin{examplebox}[]{Beispiel: Nichtlineare Ausdehnung von Aluminium}
    Für Aluminium bei \SI{0}{\celsius} betragen die Ausdehnungskoeffizienten etwa $\alpha_0 = \SI{23.8e-6}{\kelvin^{-1}}$ und $\beta = \SI{1.8e-8}{\kelvin^{-2}}$. Das Verhältnis der quadratischen Ausdehnung zur linearen Ausdehnung beträgt in diesem Fall
    \begin{equation}
        \frac{\beta \Delta T^2}{\alpha_0 \Delta T} = \frac{\beta \Delta T}{\alpha_0} = \frac{\num{1.8e-8}}{\num{23.8e-6}} \cdot \Delta T \approx \num{7.56e-4} \Delta T \mDot
    \end{equation}
    Bei einer Erhöhung der Temperatur um $\Delta T = \SI{100}{\celsius}$ ändert sich der lineare Ausdehnungskoeffizient von $\alpha_0$ auf $\alpha_0(1+\num{0.0756})$. Der nichtlineare Anteil macht in diesem Fall also bereits etwa \SI{7.56}{\percent} der gesamten Ausdehnung aus.
\end{examplebox}

\subsection{Bimetallthermometer}\label{subsec: Bimetallthermometer}
Das Prinzip der unterschiedlichen thermischen Ausdehnung wird im \textbf{Bimetallthermometer} genutzt. Hierbei werden zwei Streifen aus unterschiedlichen Metallen (z.B. Stahl und Kupfer) fest miteinander verbunden. Da die Metalle unterschiedliche Ausdehnungskoeffizienten haben, krümmt sich der Verbundstreifen bei einer Temperaturänderung. Im Ausgangszustand $T_0$ haben beide Materialien dieselbe Länge, siehe \cref{fig: bimetallstreifen_bimetallthermometer}. Das Material des schwarzen Streifens hat einen höheren Ausdehnungskoeffizienten $\alpha_{\text{schwarz}} > \alpha_{\text{rot}}$. Bei Erwärmung biegt sich der Bimetallstreifen so, dass das Material mit dem größeren Ausdehnungskoeffizienten länger wird. Je größer der Unterschied in den Ausdehnungskoeffizienten, desto größer ist die Biegung. \\

Diese Krümmung kann auf eine geeichte Skala übertragen werden, um die Temperatur anzuzeigen. Dazu rollt man den Bimetallstreifen auf und befestigt ihn derart, dass eine Erwärmung $\Delta T$ über einen Zeiger auf einer geeigneten Skala ablesbar wird. 
\begin{figure}[htb]
    \centering
    \includegraphics[width=0.5\linewidth]{Bilder/Kapitel_Waermelehre/bimetallthermometer_beschreibung.png}
    \caption{(Links) Funktionsweise eines Bimetallstreifens. Das Material mit dem größeren $\alpha$ (schwarz) dehnt sich stärker aus und zwingt den Streifen zur Krümmung. (Rechts) In einem Zeigerthermometer wird ein aufgerollter Bimetallstreifen verwendet, um eine Drehbewegung zu erzeugen.}\label{fig: bimetallstreifen_bimetallthermometer}
\end{figure}

\subsection{Volumenausdehnung}\label{subsec: volumenausdehnung}
Analog zur linearen Ausdehnung erfahren Körper auch eine Volumenausdehnung. Der \textbf{Volumenausdehnungskoeffizient} $\gamma$ (auch kubischer Ausdehnungskoeffizient) ist analog zu \cref{eq: def_L_von_DeltaT} definiert als:
\begin{equation}
    \gamma = \frac{1}{V_0} \frac{\Delta V}{\Delta T} \mDot
\end{equation}
Die Volumenänderung $\Delta V$ bei einer Temperaturänderung $\Delta T$ ist gegeben durch:
\begin{equation}\label{eq: volumenausdehnung}
    \Delta V = \gamma \cdot V_0 \cdot \Delta T \quad \implies \quad V(\Delta T) = V_0 (1 + \gamma \cdot \Delta T) \mDot
\end{equation}
Für homogene, isotrope Festkörper, die sich in alle Richtungen gleich ausdehnen, gilt der Zusammenhang:
\begin{equation}
    \gamma = 3\alpha \mComma
\end{equation}
wenn $\alpha$ der lineare Längenausdehnungskoeffizient ist. Diese Näherung ergibt sich aus 
\begin{equation}
    V(\Delta T) = L{(\Delta T)}^3 = {[L_0 \cdot (1+\alpha\Delta T)]}^3 = V_0\cdot{(1+\alpha\Delta T)}^3 \mDot
\end{equation} 
Der kubische Term ergibt 
\begin{equation}
    {(1+\alpha \Delta T)}^3 = 1 + 3\alpha \Delta T + 3\alpha^2 \Delta T^2 + \alpha^3 \Delta T^3 
\end{equation}
und für Festkörper gilt meist $\alpha\Delta T \ll 1$. Daher können die quadratischen und kubischen Terme in $\Delta T$ vernachlässigt werden. Somit bleibt 
\begin{equation}
    {(1+\alpha \cdot \Delta T)}^3 \approx 1 + 3\alpha \Delta T = 1 + \gamma \Delta T\mComma
\end{equation}
sofern $\alpha \Delta T \ll 1$. 
In \cref{tab: volumen_ausdehnungskoeffizienten} sind einige Volumenausdehnungskoeffizienten für verschiedene feste, flüssige und gasförmige Stoffe aufgezählt. 
\begin{table}[htb]
    \centering
    \caption{Volumenausdehnungskoeffizienten $\gamma$ für ausgewählte feste, flüssige und gasförmige Stoffe.}\label{tab: volumen_ausdehnungskoeffizienten}
    \vspace{4pt}
    \begin{tabular}{llc}
        \toprule
        \textbf{Stoff} & \textbf{Phase} & \textbf{Ausdehnungskoeffizient $\gamma$ (\SI{e-6}{\kelvin^{-1}})} \\
        \midrule
        Aluminium      & fest           & 75   \\
        Eisen          & fest           & 35   \\
        Kupfer         & fest           & 50   \\
        \midrule
        Benzin         & flüssig        & 950  \\
        Quecksilber    & flüssig        & 180  \\
        Ethanol        & flüssig        & 1100 \\
        Wasser         & flüssig        & 210  \\ % Korrigierter Wert, 2 aus PDF ist zu niedrig für den Bereich bis 100°C
        \midrule
        Ideales Gas    & gasförmig      & 3661 \\
        Helium (He)    & gasförmig      & 3660 \\
        Argon (Ar)     & gasförmig      & 3671 \\
        Sauerstoff ($\text{O}_2$) & gasförmig  & 3674 \\
        Kohlendioxid ($\text{CO}_2$) & gasförmig & 3726 \\
        \bottomrule
    \end{tabular}
\end{table}


\subsection{Ausdehnung von Gasen (Gesetze von Gay-Lussac)}\label{subsec: ausdehnung_gase}
Auch Gase dehnen sich bei Erwärmung aus, jedoch deutlich stärker als Festkörper oder Flüssigkeiten. Die Ausdehnungskoeffizienten von Gasen sind ungefähr \qtyrange{2}{3}{} Größenordnungen größer als die von festen oder flüssigen Körpern. Die Gesetze von Gay-Lussac beschreiben dieses Verhalten für ideale Gase:
\begin{enumerate}
    \item \textbf{Bei konstantem Druck (isobar):} Hält man den Druck eines Gases konstant, so ist das Volumen direkt proportional zur \textbf{absoluten Temperatur} $T$ (in Kelvin). Wir starten mit der Volumenausdehnung als Funktion der Celsius-Temperatur $\vartheta$:
    \begin{equation}\label{eq: gay_lussac_isobar}
        V(\vartheta) = V_0(1 + \gamma \vartheta), % \quad \text{oder} \quad \frac{V}{T} = \const
    \end{equation}
    wobei für ideale Gase $\gamma = \dfrac{1}{\num{273.15}}\,\si{\celsius^{-1}}$ eine Konstante ist und $\vartheta$ die Temperatur in $\si{\celsius}$ ist. Bezieht man die Volumenänderung auf eine Referenztemperatur von $\vartheta_0 = \SI{0}{\celsius}$ ($T_0 = \SI{273.15}{\kelvin}$), so lässt sich \cref{eq: gay_lussac_isobar} auch auf die absolute Temperatur $T = \qty{273.15} + \vartheta$ umformen  
    \begin{equation} \begin{gathered}
        V(\vartheta) = V_0(1+\gamma \vartheta) = V_0 \left(1 + \frac{\vartheta}{\num{273.15}} \right) = V_0\left( \frac{\num{273.15} + \vartheta}{\num{273.15}} \right) = \\
        \implies V(T) = V_0 \left( \frac{T}{T_0} \right) \mDot
    \end{gathered}\end{equation}
    Somit findet man, dass 
    \begin{equation}\label{eq: gay_lussac_V_ueber_V0}
        \frac{V}{V_0} = \frac{T}{T_0} \quad \text{oder} \quad \frac{V}{T} = \frac{V_0}{T_0} = \const \mDot
    \end{equation}
    \item \textbf{Bei konstantem Volumen (isochor):} Hält man das Volumen eines Gases konstant, so ist der Druck eines Gases direkt proportional zur absoluten Temperatur. Auch hier beginnt man mit der Druckänderung als Funktion der Celsius-Temperatur $\vartheta$
    \begin{equation}
        p(\vartheta) = p_0(1 + \gamma_p \vartheta)
    \end{equation}
    und formt dies auf die absolute Temperatur $T$ (in Kelvin) um. Somit erhält man 
    \begin{equation}\label{eq: gay_lussac_p_ueber_p0}
        \frac{p}{p_0} = \frac{T}{T_0} \quad \text{oder} \quad \frac{p}{T} = \frac{p_0}{T_0} = \const \mDot
    \end{equation}
\end{enumerate}
Diese beiden Gesetze werden in dem Gesetz von Gay-Lussac zusammengefasst, indem man die \cref{eq: gay_lussac_V_ueber_V0,eq: gay_lussac_p_ueber_p0} etwas umformt, wie im Folgenden für die isochore Druckänderung dargestellt: 
\begin{align}
    \frac{p}{T} &= \frac{p_0}{T_0} \nonumber \\
    p &= p_0 \cdot \frac{T}{T_0} \nonumber \\
    p - p_0 &=  \left( p_0 \cdot \frac{T}{T_0} \right) - p_0 = p_0 \left( \frac{T}{T_0} - 1 \right) = p_0 \left( \frac{T - T_0}{T_0} \right) = \nonumber \\
    &= \frac{p_0}{T_0} \left( T - T_0 \right) \mDot
\end{align}
Somit folgt für die Druckänderung 
\begin{equation}
    (p-p_0) \propto (T-T_0) \mDot
\end{equation}
Die Volumenänderung bei konstantem Druck lässt sich analog herleiten. Zusammengefasst ergibt sich das sogenannte \textit{Gesetz von Gay-Lussac}.
\begin{importantbox}[]{Gesetz von Gay-Lussac}
    Das Gay-Lussac Gesetz beschreibt die Druckänderung bei konstantem Volumen (isochor) und die Volumenänderung bei konstantem Druck (isobar)
    \begin{equation}\begin{aligned}
        (p - p_0) &\propto (T - T_0) \mComma \quad \text{für konstantes Volumen}\\
        (V - V_0) &\propto (T - T_0) \mComma \quad \text{für konstanten Druck.}
    \end{aligned}\end{equation}
\end{importantbox}

\subsection{Gasthermometer}\label{subsec: Gasthermometer}
Bei einem \textbf{Gasthermometer} mit konstantem Volumen dient die Druckänderung als Maß für die Temperatur. Da sich Gase sehr gleichmäßig ausdehnen und Flüssigkeiten näherungsweise inkompressibel sind, sind Gasthermometer besonders präzise und werden zur Kalibrierung anderer Thermometer verwendet. 

Die Funktionsweise eines Gasthermometers erklärt sich anhand der \cref{fig: gasthermometer}. Das System besteht aus einem mit Gas gefüllten Kolben ($B_1$), der über einen flexiblen Schlauch mit einem Quecksilber-Manometer verbunden ist. Das entscheidende Anwendungsmerkmal eines Gasthermometers ist das \textbf{konstante Volumen}:
Um eine Messung durchzuführen, wird der bewegliche Behälter $B_3$ vertikal so verschoben, dass der Quecksilberspiegel im Schenkel $B_2$ exakt auf der Referenzmarke (bei $h=0$) steht. Damit nimmt das Gas in $B_1$ immer dasselbe Volumen ein.\\

Der absolute Druck $p$ im Inneren des Gasgefäßes $B_1$ wird durch die Höhendifferenz $h$ zwischen den beiden Quecksilbersäulen bestimmt. Der Druck $p$ des Gases in $B_1$ muss genau den äußeren Luftdruck $p_{\atm}$ und den hydrostatischen Druck der Quecksilbersäule kompensieren:
\begin{equation}\label{eq: druck_gasthermometer}
    p = p_{\atm} + \rho_{\text{Hg}} \cdot g \cdot h \mDot
\end{equation}
Hierbei ist $\rho_{\text{Hg}}$ die Dichte des Quecksilbers und $g$ die Erdbeschleunigung.

\begin{figure}[htb]
    \centering
    \includegraphics[width=0.5\linewidth]{Bilder/Kapitel_Waermelehre/gasthermometer.png}
    \caption{Aufbau eines Gasthermometers mit konstantem Volumen. Der Behälter $B_3$ wird gehoben oder gesenkt, um den Meniskus in $B_2$ stets bei der Nullmarke zu halten.}\label{fig: gasthermometer}
\end{figure}

\subsubsection*{Kalibrierung}
Um dem gemessenen Druck eine Temperatur in Grad Celsius zuzuordnen, muss das Thermometer kalibriert werden. Dazu werden zwei Fixpunkte genutzt, üblicherweise der Gefrierpunkt und der Siedepunkt von Wasser.

\begin{enumerate}
    \item \textbf{Messung am Eispunkt ($\SI{0}{\celsius}$):}
    Das Gefäß $B_1$ wird in ein Eiswasserbad getaucht. Nachdem sich das thermische Gleichgewicht eingestellt hat, wird $B_3$ justiert, bis der Pegel in $B_2$ wieder bei Null steht. Man misst die Höhendifferenz $h_0$ und berechnet den Druck $p_0$:
    \begin{equation}
        p_0 = p_{\atm} + \rho_{\text{Hg}} \cdot g \cdot h_0 \mDot
    \end{equation}
    
    \item \textbf{Messung am Siedepunkt ($\SI{100}{\celsius}$):}
    Anschließend wird das Gefäß in kochendes Wasser (oder Dampf) gebracht. Durch die Erwärmung steigt der Druck, wodurch das Quecksilber in $B_2$ nach unten gedrückt würde. Um das Volumen konstant zu halten, muss $B_3$ stark angehoben werden, bis der Pegel in $B_2$ wieder die Nullmarke erreicht. Die neue Höhendifferenz $h_{100}$ liefert den Druck $p_{100}$:
    \begin{equation}
        p_{100} = p_{\atm} + \rho_{\text{Hg}} \cdot g \cdot h_{100} \mDot
    \end{equation}
\end{enumerate}

Da der Druck bei konstantem Volumen linear mit der Temperatur steigt (Gesetz von Amontons), kann nun für jeden beliebigen gemessenen Druck $p_T$ die zugehörige Temperatur $T$ interpoliert werden.

\begin{rememberbox}{Temperaturskala am Gasthermometer}
    Die Temperatur $T$ in Grad Celsius ergibt sich aus dem Verhältnis der Druckänderung zur Druckdifferenz der beiden Fixpunkte:
    \begin{equation}\label{eq: formel_gasthermometer_celsius}
        T[\si{\celsius}] = \frac{p_T - p_0}{p_{100} - p_0} \cdot \SI{100}{\celsius} \mDot
    \end{equation}
    Hierbei ist $p_T$ der Druck bei der zu messenden Temperatur und $p_0$ bzw. $p_{100}$ sind die Drücke am Gefrier- bzw. Siedepunkt von Wasser.
\end{rememberbox}

\subsubsection{Herleitung der Formel}
Die Herleitung von \cref{eq: formel_gasthermometer_celsius} basiert auf der Annahme, dass sich der Druck eines idealen Gases bei konstantem Volumen linear mit der Temperatur ändert (Gesetz von Amontons). Geometrisch entspricht dies einer Geraden im $(p, T)$-Diagramm, die durch zwei experimentell bestimmte Punkte verläuft: Den Eispunkt $(p_0, T_0)$ und den Siedepunkt $(p_{100}, T_{100})$ von Wasser. Wir verwenden die Zweipunktform der Geradengleichung mit dem unbekannten Punkt $(p_T, T)$:
\begin{equation}\begin{aligned}
    \frac{T - T_{0}}{p_T - p_{0}} &= \frac{T_{100} - T_{0}}{p_{100} - p_{0}} \mComma \\
    \frac{T - 0}{p_T - p_0} &= \frac{100 - 0}{p_{100} - p_0} \mDot
\end{aligned}\end{equation}
Durch Umformen nach $T$ folgt direkt die gesuchte Beziehung:
\begin{align}
    \frac{T}{p_T - p_0} &= \frac{100}{p_{100} - p_0} & | \cdot (p_T - p_0) \notag \\
    T &= \frac{100}{p_{100} - p_0} \cdot (p_T - p_0) \notag \\
    T &= \frac{p_T - p_0}{p_{100} - p_0} \cdot \SI{100}{\celsius} \mDot \label{eq: herleitung_ende}
\end{align}
Diese lineare Skala ist die \textbf{empirische Celsiusskala} des Gasthermometers.
\subsubsection*{Vereinfachung in der Praxis}
In der Praxis ist es oft nicht nötig, die absoluten Drücke zu berechnen. Setzen wir die Definition des hydrostatischen Drucks $p = p_{\atm} + \rho_{\text{Hg}} g h$ in die Temperaturformel ein, so sehen wir, dass sich der Atmosphärendruck $p_{\atm}$ durch die Differenzbildung eliminiert:
\begin{align}
    p_T - p_0 &= (p_{\atm} + \rho_{\text{Hg}} g h_T) - (p_{\atm} + \rho_{\text{Hg}} g h_0) \notag \\
    &= \rho_{\text{Hg}} g (h_T - h_0) \mDot
\end{align}
Dasselbe gilt für den Nenner ($p_{100} - p_0$). Setzt man dies in \cref{eq: formel_gasthermometer_celsius} ein, kürzen sich auch die Dichte $\rho_{\text{Hg}}$ und die Erdbeschleunigung $g$ heraus:

\begin{equation}
    T = \frac{\cancel{\rho_{\text{Hg}} g} (h_T - h_0)}{\cancel{\rho_{\text{Hg}} g} (h_{100} - h_0)} \cdot 100 
    = \frac{h_T - h_0}{h_{100} - h_0} \cdot \SI{100}{\celsius} \mDot
\end{equation}

\begin{importantbox}{Praktische Messung}
    Solange der Atmosphärendruck während der Kalibrierung und Messung konstant bleibt, genügt es, die \textbf{Höhendifferenzen} der Quecksilbersäulen zu messen. Die Temperatur ist proportional zum Verhältnis der Höhenänderungen:
    \begin{equation}
        T[\si{\celsius}] = \frac{h_T - h_0}{h_{100} - h_0} \cdot \SI{100}{\celsius} \mDot
    \end{equation}
\end{importantbox}


\section{Das ideale Gas und die Zustandsgleichung}\label{sec: ideales_gas}
Die Funktionsweise des Gasthermometers in \cref{subsec: Gasthermometer} basierte auf der Beobachtung, dass der Druck eines Gases linear mit der Temperatur steigt, sofern das Volumen konstant gehalten wird (Gesetz von Amontons). Dies ist jedoch nur einer von mehreren Aspekten, wie sich Gase verhalten.

Im Gegensatz zu Festkörpern und Flüssigkeiten, deren Teilchen dicht gepackt sind und starke Bindungskräfte aufweisen, füllen Gase jeden verfügbaren Raum aus und lassen sich leicht komprimieren. Um das Verhalten von Gasen vollständig zu beschreiben, benötigen wir einen Zusammenhang zwischen allen drei Zustandsgrößen: Druck $p$, Volumen $V$ und Temperatur $T$.

Wir betrachten hierzu das Modell des \textbf{idealen Gases}. In diesem idealisierten Modell vernachlässigen wir das Eigenvolumen der Gasteilchen sowie die Kräfte zwischen ihnen.

\subsection{Das Gesetz von Boyle-Mariotte (Isotherme Zustandsänderung)}\label{subsec: boyle_mariotte}
Während das Gasthermometer bei konstantem Volumen arbeitet, betrachten wir nun, was passiert, wenn wir das Volumen ändern, aber die Temperatur konstant halten ($T = \const$).
Ein solches Experiment lässt sich mit einem Kolben in einem Zylinder durchführen (siehe \cref{fig: kolben_zylinder_gas_pV}). Drückt man den Kolben langsam herunter, verringert sich das Volumen. Gleichzeitig beobachtet man, dass der Druck im Inneren steigt.

\begin{figure}[tb]
    \centering
    \resizebox{0.35\linewidth}{!}{
    \begin{tikzpicture}[>=Latex] 
        % Parameter für einfache Größenanpassung
        \def\cylW{3.2}   % Breite des Zylinders
        \def\cylH{3.8}   % Höhe der Wände
        \def\pistY{2.0}  % y-Position des Kolbenbodens (etwas tiefer gesetzt für Kompression)
        \def\pistH{0.5}  % Dicke des Kolbens

        % 1. Zylinder
        \draw[thick] (0, \cylH) -- (0, 0) -- (\cylW, 0) -- (\cylW, \cylH);
        % 2. Kolben
        \filldraw[fill=red!20, draw=black, thick] (0, \pistY) rectangle (\cylW, \pistY+\pistH);
        
        % 3. Gas (Punkte)
        \foreach \i in {1,...,50} {
            \fill[blue!70] (0.03 + rnd*\cylW*0.96, 0.03 + rnd*\pistY*0.96) circle (1.5pt);
        }
        
        % 4. Beschriftung 
        \node[draw , fill=white] at (0.5*\cylW, 0.5*\pistY) {\large $V, p$};
        \node[right=0.1] at (\cylW, 0.5*\pistY) {\large $T=\const$};
        
        % 5. Kraft Pfeil
        \draw[-{Stealth}, ultra thick] (0.5*\cylW, \cylH + 0.5) -- (0.5*\cylW, \pistY+\pistH);
        \node[right] at (0.5*\cylW, \cylH + 0.2) {Kompression};
    \end{tikzpicture}
    }
    \caption{Gesetz von Boyle-Mariotte: Wird das Volumen $V$ bei konstanter Temperatur verringert, steigt der Druck $p$, da die Teilchen pro Zeiteinheit öfter gegen die Wand stoßen.}\label{fig: kolben_zylinder_gas_pV}
\end{figure}

Robert Boyle und Edme Mariotte fanden unabhängig voneinander heraus, dass für eine abgeschlossene Gasmenge bei \textbf{konstanter Temperatur} der Druck umgekehrt proportional zum Volumen ist ($p \propto 1/V$). Das bedeutet, das Produkt aus Druck und Volumen ist konstant:
\begin{equation}\label{eq: boyle_mariotte_gesetz}
    p \cdot V = \const \qquad (\text{für } T = \const) \mDot
\end{equation}
Halbiert man beispielsweise das Volumen eines Gases bei konstanter Temperatur, verdoppelt sich sein Druck.

\subsection{Die ideale Gasgleichung}\label{subsec: ideale_gasgleichung}
Fassen wir nun die experimentellen Befunde zusammen:
\begin{enumerate}
    \item \textbf{Gesetz von Amontons (Isochor):} $p \propto T$ (bei $V=\const$)
    \item \textbf{Gesetz von Gay-Lussac (Isobar):} $V \propto T$ (bei $p=\const$)
    \item \textbf{Gesetz von Boyle-Mariotte (Isotherm):} $p \propto 1/V$ (bei $T=\const$)
\end{enumerate}
Alle drei Gesetze sind Spezialfälle einer einzigen, fundamentalen Beziehung. Kombiniert man die Proportionalitäten ($p \propto T$ und $p \propto 1/V$), erhält man:
\begin{equation}
    p \propto \frac{T}{V} \quad \Longleftrightarrow \quad p \cdot V \propto T \mDot
\end{equation}
Um aus der Proportionalität eine Gleichung zu machen, benötigen wir eine Proportionalitätskonstante. Diese hängt von der Menge des Gases ab.

\begin{importantbox}{Ideale Gasgleichung}
    Betrachten wir die Anzahl der Teilchen $N$ im Gas, so erhalten wir die \textbf{thermische Zustandsgleichung idealer Gase}:
    \begin{equation}\label{eq: ideale_gasgleichung}
        p \cdot V = N \cdot k_B \cdot T
    \end{equation}
    Hierbei ist $p$ der absolute Druck in \si{\pascal}, $V$ das Volumen in \si{\cubic\meter}, $N$ die absolute Anzahl der Gasteilchen, $T$ die absolute Temperatur in \si{\kelvin} und $k_B \approx \SI{1.38e-23}{\joule\per\kelvin}$ die Boltzmann-Konstante.
\end{importantbox}
Diese Gleichung ist das Herzstück der Wärmelehre für Gase. Sie erlaubt es, eine unbekannte Größe zu berechnen, wenn die anderen bekannt sind. Zudem impliziert sie alle vorherigen Gesetze: Ist zum Beispiel $T$ konstant, wird die rechte Seite zu einer Konstanten und wir erhalten wieder $p \cdot V = \text{const}$ (Boyle-Mariotte).


\section{Die barometrische Höhenformel}\label{sec: barometrische_hoehenformel}

\begin{figure}[tb]
    \centering
    \resizebox{0.45\linewidth}{!}{
    \begin{tikzpicture}[>=Latex]
        % Parameter
        \def\cylW{3.5}       % Breite des Zylinders
        \def\cylH{5.0}       % Höhe
        \def\hBase{1.2}      % Höhe h
        \def\dh{1.8}         % Abstand dh (relative Höhe zu h)
        \def\hTop{\hBase+\dh}% Höhe h + dh
        \def\hNull{0.7}

        % 1. Füllung (Linearer Farbverlauf orange -> weiß)
        % 'middle color' kann optional für feinere Steuerung genutzt werden, 
        % hier reicht bottom zu top.
        \shade[bottom color=orange!60, top color=white] 
            (0,\hBase) rectangle (\cylW, \cylH);

        % 2. Zylinderwände (nur links und rechts)
        \draw[thick] (0,0) -- (0,\cylH);
        \draw[thick] (\cylW,0) -- (\cylW,\cylH);

        % 3. Markierungslinien
        % Untere Linie bei h (dick orange + gestrichelt schwarz für den Effekt)
        \draw[line width=4pt, orange] (0, \hBase) -- (\cylW, \hBase);
        \draw[thick, dashed] (0, \hBase) -- (\cylW, \hBase);
        % Obere Linie bei h + dh (nur gestrichelt)
        \draw[thick, dashed] (0, \hTop) -- (\cylW, \hTop);

        % 4. Koordinatenachse z (links außen)
        \draw[->, ultra thick] (-1.1, 3*\cylH/4) -- (-1.1, \cylH) node[right, midway] {\Large $z$};

        % 5. Druck-Pfeil p(z) (von oben kommend)
        \draw[->, ultra thick] (\cylW/2, \cylH + 0.5) -- (\cylW/2, \hBase + 0.2) 
            node[right=0.1, pos=0.05] {\Large $p(z)$};

        % 6. Beschriftungen
        % Links (Höhen)
        \node[left=0.2] at (0, \hBase) {\large $h$};
        \node[left=0.2] at (0, \hTop) {\large $h + dh$};

        % Rechts (Dichte/Druck Werte)
        \node[right=0.2] at (\cylW, \hBase) {\large $\rho(h), p(h)$};
        % Multiline Node für den oberen Text
        \node[right=0.2, align=left] at (\cylW, \hTop) {\large $\rho(h + dh),$ \\ \large $p(h + dh)$};

        % Mitte (Fläche A) - in orange
        \node[orange!90!black] at (\cylW/2, \hBase - 0.6) {\Large $A$};
    \end{tikzpicture}
    }
    \caption{Mit zunehmender Höhe $h$ nimmt das Gewicht der darüberliegenden Luftsäule ab, wodurch Druck und Dichte sinken.}\label{fig: luftsaeule_hoehenformel_anschauung}
\end{figure}

Der Luftdruck, den wir auf der Erdoberfläche messen, entsteht durch die Gewichtskraft der darüber liegenden Luftsäule (siehe \cref{fig: luftsaeule_hoehenformel_anschauung}). Da die Dichte der Luft mit der Höhe abnimmt, ist auch der Druck nicht konstant. Wir wollen eine Formel für den Druck als Funktion der Höhe $h$ herleiten.

Bei einem infinitesimalen Höhenanstieg von $h$ auf $h + \dd h$ nimmt das Gewicht der Luftsäule um $\rho\cdot g \cdot \dd V = \rho\cdot g \cdot A \cdot \dd h$ ab und daher sinkt der Druck $p$ um
\begin{equation}
    dp = -\rho(h) \cdot g \cdot dh \mComma
\end{equation}
wobei $\rho(h)$ die Dichte in der Höhe $h$ ist. Das negative Vorzeichen zeigt, dass der Druck mit zunehmender Höhe abnimmt.

Im Gegensatz zu einer inkompressiblen Flüssigkeit ist die Dichte der Luft vom Druck abhängig. Aus dem Gesetz von Boyle-Mariotte folgt für eine isotherme Atmosphäre ($T = \const$), dass $\rho/p = \const = \rho_0/p_0$. Wir können also $\rho$ ersetzen:
\begin{equation}
    dp = - \frac{\rho_0}{p_0} p \cdot g \cdot dh \implies \frac{dp}{p} = -\frac{\rho_0 g}{p_0} dh \mDot
\end{equation}
Integration dieser Differentialgleichung von Höhe $0$ (mit Druck $p_0$ und Dichte $\rho_0$) bis zur Höhe $h$ (mit Druck $p(h)$) liefert:
\begin{equation}
    \int_{p_0}^{p(h)} \frac{1}{p} dp = -\int_0^h \frac{\rho_0 g}{p_0} dh \implies \ln\left(\frac{p(h)}{p_0}\right) = -\frac{\rho_0 g}{p_0} h \mDot
\end{equation}
\begin{figure}[tb]
    \centering
    \includegraphics[width=0.48\textwidth]{Bilder/Kapitel_Waermelehre/hoehenformel.pdf}
    \caption{Der exponentiell abnehmende Luftdruck $p(z)$ über der Höhe $h$ in km.}\label{fig: barom_hoehenformel_grafik}\label{fig: barometrische_hoehenformel_exp_abnahme}
\end{figure}
\begin{importantbox}{Barometrische Höhenformel}
    Durch Auflösen nach $p(h)$ erhält man die \textbf{barometrische Höhenformel} für eine isotherme Atmosphäre:
    \begin{equation}\label{eq: barometrische_hoehenformel}
        p(h) = p_0 \cdot e^{-\frac{\rho_0 g}{p_0}h} = p_0 \cdot e^{-h/H} \mDot
    \end{equation}
    Der Druck in der Atmosphäre nimmt exponentiell mit der Höhe ab (siehe \cref{fig: barometrische_hoehenformel_exp_abnahme}). Die Größe $H = p_0/(\rho_0 g)$ wird als Skalenhöhe bezeichnet und beträgt für Luft bei \SI{0}{\celsius} etwa \SI{8}{\kilo\meter}. In einer Höhe von ca. \SI{5.8}{\kilo\meter} beträgt der Luftdruck nur noch die Hälfte des Bodenwertes.
\end{importantbox}
Laut dieser isothermen Näherung erhält man für den Luftdruck auf der Spitze des Mount Everest (\SI{8849}{\meter}) nur noch $\approx \SI{34.5}{\percent}$ des Luftdrucks auf Meereshöhe, was nur ungefähr $\SI{2}{\percent}$ vom realen Wert abweicht. 

Die \textbf{Druckabnahme} einer isothermen Lufthülle folgt also einem \textbf{Exponentialgesetz} im Gegensatz zur \textbf{linearen Druckabnahme in einer Flüssigkeit}. Dieser Unterschied ist in \cref{fig: gassaeule_fluessigkeit_gegenueberstellung} dargestellt. Die Kompressibilität der Luft führt zu einer exponentiellen (schneller als linear) Abnahme des Drucks. Die Lufthülle der Erde hat jedoch keine scharfe Grenze. In der realen Erdatmosphäre ändert sich die Temperatur allerdings mit der Höhe, weshalb die barometrische Höhenformel nur für die untere Atmosphäre eine gute Näherung darstellt. 

\begin{figure}[tb]
    \centering
    \resizebox{0.90\linewidth}{!}{
        \begin{tikzpicture}[>=Latex]
            % =========================
            % Gemeinsame Parameter
            % =========================
            \def\cylW{4.4}       % Breite des Zylinders
            \def\cylH{5.0}       % Gesamthöhe
            \def\hBase{1.2}      % Referenzhöhe h (etwas angehoben, damit man sie sieht)
            \def\dh{1.0}         % Abstand dh
            \def\offset{8.0}     % Abstand zwischen den Zylindern
            \def\AOffset{0.4}    % Vertikaler Versatz für die A-Beschriftung
            % ============================================================
            % LINKER ZYLINDER: GAS (Kompressibel)
            % ============================================================
            \begin{scope}
                \node[above, font=\bfseries] at (\cylW/2, \cylH+0.5) {Gas (kompressibel)};

                % 1. Füllung: Linearer Farbverlauf (orange -> weiß)
                % Visualisiert abnehmende Dichte mit der Höhe
                \shade[bottom color=orange!60, top color=white] (0,0) rectangle (\cylW, \cylH*0.99);

                % 2. Zylinderwände
                \draw[thick] (0,0) -- (0,\cylH);
                \draw[thick] (\cylW,0) -- (\cylW,\cylH);
                \draw[thick] (0,0) -- (\cylW,0); % Boden

                % 3. Markierungslinien für das Volumenelement
                % Untere Linie bei h
                \draw[line width=4pt, orange!80!black, opacity=0.7] (0, \hBase) -- (\cylW, \hBase);

                % 4. Koordinatenachse z
                \draw[->, ultra thick] (-1.4, \cylH*0.7) -- (-1.4, \cylH) node[right, above] {\Large $z$};
                
                % 5. Druck-Pfeil und Formel
                % Exponentielle Abnahme
                \draw[->, ultra thick, orange!90!black] (\cylW/2, \cylH-1.2) -- (\cylW/2, \hBase+0.1) 
                    node[pos=-0.2, align=center, fill=white, fill opacity=0.0, text opacity=1, rounded corners] 
                    {$p(z) = p_0 \cdot e^{-\frac{z}{H}}$\\[0.3em] \small (Exponentielle Abnahme)};

                % 6. Beschriftungen
                % Links (Höhen)
                \node[left=0.2] at (0, \hBase) {\large $p(h), \rho(h)$};
                \draw[line width=2pt] (-0.2, \hBase) -- (0.0, \hBase); % Tick mark
                % p_0
                \node[left=0.2] at (0, 0) {\large $p_0, \rho_0$};
                \draw[line width=2pt] (-0.2, 0.03) -- (0.0, 0.03); % Tick mark

                % Rechts (Dichte/Druck Werte)
                % Dichte variiert mit der Höhe

                % Mitte (Fläche A)
                \node[orange!90!black, font=\bfseries] at (\cylW/2, \hBase - \AOffset) {\Large $A$};
            \end{scope}

            % ============================================================
            % RECHTER ZYLINDER: FLÜSSIGKEIT (Inkompressibel)
            % ============================================================
            \begin{scope}[xshift=\offset cm]
                \node[above, font=\bfseries] at (\cylW/2, \cylH+0.5) {Flüssigkeit (inkompressibel)};

                % 1. Füllung: Solide Farbe (blau)
                % Visualisiert konstante Dichte
                \fill[blue!40] (0,0) rectangle (\cylW, \cylH);
                % Optional: Eine leichte Schattierung für 3D-Effekt, aber die Dichte wirkt konstant
                \shade[left color=blue!50, right color=blue!30, opacity=0.3] (0,0) rectangle (\cylW, \cylH);


                % 2. Zylinderwände
                \draw[thick] (0,0) -- (0,\cylH);
                \draw[thick] (\cylW,0) -- (\cylW,\cylH);
                \draw[thick] (0,0) -- (\cylW,0); % Boden
                % Wasseroberfläche andeuten
                \draw[thick, blue!80!black, decorate, decoration={snake, amplitude=1pt, segment length=5pt}] (0,\cylH) -- (\cylW,\cylH);

                % 3. Markierungslinien
                % Untere Linie bei h
                \draw[line width=4pt, blue!80!black, opacity=0.7] (0, \hBase) -- (\cylW, \hBase);

                % 5. Druck-Pfeil und Formel
                % Lineare Abnahme mit der Höhe (oder Zunahme mit der Tiefe)
                \draw[->, ultra thick, blue!90!black] (\cylW/2, \cylH-1.2) -- (\cylW/2, \hBase+0.1) 
                    node[pos=-0.2, align=center, fill=white, fill opacity=0.0, text opacity=1, rounded corners] 
                    {$p(z) = p_{0} - \rho g z$\\[0.3em] \small (Lineare Abnahme)};

                % 6. Beschriftungen
                % Links (Höhen)
                \node[left=0.2] at (0, \hBase) {\large $p(h), \rho_0$};
                \draw[line width=2pt] (-0.2, \hBase) -- (0.0, \hBase);

                \node[left=0.2] at (0, 0) {\large $p_0, \rho_0$};
                \draw[line width=2pt] (-0.2, 0.03) -- (0.0, 0.03);

                % Mitte (Fläche A)
                \node[blue!90!black, font=\bfseries] at (\cylW/2, \hBase - \AOffset) {\Large $A$};
            \end{scope}
        \end{tikzpicture}
    }
    \caption{Gegenüberstellung von Gas (kompressibel) und Flüssigkeit (inkompressibel) bezüglich Druck- und Dichteverteilung. Gase zeigen einen exponentiellen Druckabfall mit steigender Höhe, während Flüssigkeiten eine lineare Abnahme aufweisen.}\label{fig: gassaeule_fluessigkeit_gegenueberstellung}
\end{figure}

\subsubsection{Archimedisches Prinzip}
Gleich wie bei Flüssigkeiten gilt auch in Gasen das Archimedische Prinzip:
\begin{rememberbox}{Archimedisches Prinzip für Gase}
    Ein in ein Gas eingetauchter Körper erfährt eine Auftriebskraft $F_A$, die dem Gewicht des von ihm verdrängten Gases entspricht:
    \begin{equation}\label{eq: auftrieb_gase}
        F_{\text{Auf}} = \rho_{\text{Gas}} \cdot V_{\text{Körper}} \cdot g = m_{\text{verdr. Gas}} \cdot g = F_{\text{Gew}} \mDot
    \end{equation}
    Die Auftriebskraft wirkt jedoch der Gewichtskraft des Körpers entgegen, $\ivecS{F}{\text{Gew}} = -\ivecS{F}{\text{Auf}}$.
\end{rememberbox}
Für Gase ist die Dichte $\rho_{\text{Gas}}$ eigentlich nicht konstant über die Höhe des Körpers. In der Praxis ist die Variation jedoch meist vernachlässigbar klein, weshalb \cref{eq: auftrieb_gase} praktisch exakt ist.



\section{Kinetische Gastheorie}\label{sec: kinetische_gastheorie}
Die kinetische Gastheorie, entwickelt in der zweiten Hälfte des 19. Jahrhunderts unter anderem von James Clerk Maxwell, Ludwig Boltzmann und Rudolf Clausius, erklärt die makroskopischen Eigenschaften von Gasen (wie Druck und Temperatur) durch die Bewegung und die Wechselwirkung der Gasatome auf mikroskopischer Ebene.

Das einfachste Gasmodell ist das des idealen Gases, bei dem die Gasteilchen als punktförmige, nicht wechselwirkende Massen betrachtet werden, die sich mit statistisch verteilten Geschwindigkeiten bewegen und elastische Stöße ausführen.
\subsection{Das Modell des idealen Gases}\label{subsec: modell_ideales_gas}
Das Modell des idealen Gases basiert auf folgenden Annahmen:
\begin{itemize}
    \item Das Gas besteht aus einer großen Anzahl von Teilchen (Atomen/Molekülen), die als punktförmige Massen betrachtet werden können. Ihr Eigenvolumen ist vernachlässigbar gegenüber dem Gesamtvolumen.
    \item Die Teilchen bewegen sich zufällig (statistisch verteilt) in alle Richtungen.
    \item Zwischen den Teilchen gibt es keine anziehenden oder abstoßenden Kräfte, außer bei direkten Kollisionen.
    \item Alle Stöße der Teilchen untereinander und mit den Wänden des Behälters sind vollkommen elastisch, \gDh, die kinetische Gesamtenergie bleibt erhalten.
\end{itemize}
\textit{Erklärendes Beispiel:} Um zu beurteilen, ob ein reales Gas als ideales Gas betrachtet werden kann, vergleichen wir den mittleren Abstand $\lrangle{r}$ der Gasteilchen mit ihrem effektiven Durchmesser $r_0$. Ist der mittlere Abstand deutlich größer als der Durchmesser ($r_0/\lrangle{r} \ll 1$), sind die Wechselwirkungen zwischen den Teilchen vernachlässigbar, und das Gas verhält sich ideal.
Bei einem Druck von \SI{1}{\barPr} und Raumtemperatur enthält \SI{1}{\centi\meter\cubed} eines Gases etwa $3\cdot 10^{19}$ Moleküle. Ihr mittlerer Abstand ist dann $\lrangle{r} \approx \SI{3}{\nano\meter}$. Für Heliumatome ist $r_0 \approx \SI{0.05}{\nano\meter}$, weshalb $r_0/\lrangle{r} \approx \qty{0.017} \ll 1$. Helium ist daher bei einem Druck von \SI{1}{\barPr} als ideales Gas zu betrachten.

\subsection{Herleitung des Gasdrucks}\label{subsec: herleitung_gasdruck}
\begin{figure}[htb]
    \centering
    \resizebox{0.5\linewidth}{!}{
    \begin{tikzpicture}[>=Latex, font=\large]
        % --- Definitionen für Konsistenz ---
        \definecolor{myblue}{RGB}{0, 102, 204} % Dunkelblau
        \def\wallHeight{3.0}
        \def\wallWidth{1.0}
        
        \def\xBall{2.2} % x-Position des Balls
        \def\yBall{2.0} % y-Position des Balls
        \def\factPos{0.75} % Faktor für die Position des ausgehenden Balls
        
        \def\vecLenX{1.3} % Länge der x-Vektorkomponente
        \def\vecLenY{\yBall*\vecLenX/\xBall} % Länge der y-Vektorkomponente

        % Koordinaten oben und unten mit gleichem Winkel
        \coordinate (O) at (0,0); % Aufprallpunkt
        \coordinate (BallIn) at (-\xBall, -\yBall); % Position Ball unten
        \coordinate (BallOut) at (-\factPos*\xBall, \factPos*\yBall); % Position Ball oben

        % 1. Wand (rechts)
        \fill[red!15] (0, -\wallHeight) rectangle (\wallWidth, \wallHeight);
        \draw[thick] (0, -\wallHeight) -- (0, \wallHeight);

        % 2. Hilfslinien für die Bahn (dünn)
        \draw[thin, gray] (BallIn) -- (O);
        \draw[thin, gray] (O) -- (BallOut);

        % 3. Impulsänderung Delta p (Pfeil nach rechts aus der Wand)
        \draw[->, thick] (O) -- (2.5, 0) node[right] {$\Delta p = 2mv_x$};

        % --- UNTEN: Eingehender Ball (Dunkelblau) ---
        \filldraw[fill=myblue, draw=black] (BallIn) circle (0.25);
        
        % Vektoren unten
        % vx (schwarz, horizontal)
        \draw[->, thick] (BallIn) -- ++(\vecLenX, 0) node[midway, below] {$v_x$};
        % vy (gestrichelt, vertikal)
        \draw[->, thick] (BallIn) ++(\vecLenX, 0) -- ++(0, \vecLenY) node[midway, right] {$v_y$};
        % v (rot, diagonal)
        \draw[->, red, thick] (BallIn) -- ++(\vecLenX, \vecLenY) node[midway, above left] {$\ivec{v}$};

        % % --- OBEN: Ausgehender Ball (Hellblau, gestrichelt) ---
        \filldraw[fill=myblue!30, draw=black, dashed] (BallOut) circle (0.25);
        
        % Vektoren oben
        % vy (schwarz, vertikal nach oben)
        \draw[->, thick] (BallOut) -- ++(0, \vecLenY) node[midway, right] {$v_y$};
        % -vx (schwarz, horizontal nach links, startet an der Spitze von vy)
        \draw[->, thick] (BallOut) ++(0, \vecLenY) -- ++(-\vecLenX, 0) node[midway, above] {$-v_x$};
        % v (rot, diagonal nach links oben)
        \draw[->, red, thick] (BallOut) -- ++(-\vecLenX, \vecLenY) node[midway, below left] {$\ivec{v}$};

        % 4. Textbeschriftung Mitte
        \node[below right, align=left] at (-3.25, 0.45) {$\Delta v_x = 2v_x$\\ $\Delta v_y = 0$};
        
        % 5. Koordinatenachsen x,y
                \draw[->, ultra thick] (2.5,1.5) -- (3.5, 1.5) node[right] {\Large $x$};
                \draw[->, ultra thick] (2.5,1.5) -- (2.5, 2.5) node[above] {\Large $y$};
    \end{tikzpicture}
    }
    \caption{Impulsübertragung beim elastischen Stoß auf eine Wand beträgt $\Delta p = 2mv_x$, wenn $v_x$ die Geschwindigkeitskomponente senkrecht zur Wand ist.}\label{fig: impulsuebertrag_wand}
\end{figure}

Der Druck, den ein Gas auf eine Wand ausübt, entsteht durch den Impulsübertrag der unzähligen Teilchen, die pro Zeiteinheit auf die Wand prallen.
In \cref{fig: impulsuebertrag_wand} betrachten wir ein Teilchen der Masse $m$, das sich mit der Geschwindigkeitskomponente $v_x$ auf eine Wand in der $(y,z)$-Ebene zubewegt. Beim elastischen Stoß kehrt sich die $x$-Komponente der Geschwindigkeit um ($v_x \rightarrow -v_x$), während die anderen Komponenten unverändert bleiben. Der Impulsübertrag auf die Wand entspricht der negativen Änderung des Impulses des Teilchens:
\begin{equation}
    \Delta p_{x, \text{Wand}} = -(p_{x,\text{nachher}} - p_{x,\text{vorher}}) = -(-mv_x - mv_x) = 2mv_x
\end{equation}

Der Impulsübertrag findet demnach immer durch die Normalkomponente der Geschwindigkeit zur Wand statt. Wir erinnern uns, dass der Impuls $\ivec{p} = m\cdot \ivec{v}$ ist und eine Kraft $\ivec{F}$ gleich der zeitlichen Änderung des Impulses ist:
\begin{equation}
    \ivec{F} = \frac{\dd \ivec{p}}{\dd t} \mDot
\end{equation}
 Um den Druck $p$ zu berechnen, betrachten wir ein kleines Flächenelement $A$ der Wand. Der Druck ist definiert als die Kraft pro Fläche:
\begin{equation}
    p = \frac{F}{A} = \frac{\dd \ivec{p}/\dd t}{A} = \frac{\dd}{\dd t} \left( \frac{\text{auf Fläche $A$ übertragener Impuls}}{\text{Fläche $A$}}\right)  \mDot 
\end{equation}
Die Schwierigkeit besteht darin, den gesamten Impuls zu bestimmen, der in der Zeit $\Delta t$ auf die Fläche $A$ übertragen wird, da viele Teilchen mit unterschiedlichen Geschwindigkeiten auf die Wand prallen. 
Wir betrachten daher zunächst das Volumen $dV$ in \cref{fig: teilchen_in_box}. Damit ein Teilchen innerhalb des Zeitintervalls $\dd t$ auf die rechte (graue) Wand treffen kann, muss es eine genügend große Geschwindigkeitskomponente $v_x$ haben -- die Komponenten $v_y$ und $v_z$ sind bezüglich des Abstands zur rechten Wand unerheblich. Fixieren wir die betrachtete Geschwindigkeit $v_x$, so erreichen nur jene Teilchen die Wand, die maximal einen Abstand $\dd s$ von der Wand haben:
\begin{equation}
    \dd s = v_x \cdot \dd t \mDot
\end{equation}
Dies definiert ein gedankliches Volumen $V_S$ (siehe \cref{fig: teilchen_in_box}), einen Quader mit der Grundfläche $A$ und der Länge $\dd s$:
\begin{equation}
    V_S = A \cdot \dd s = A \cdot v_x \cdot \dd t \mDot
\end{equation}
Alle Teilchen in diesem Volumen, die sich auf die Wand zubewegen, werden im Zeitintervall $\dd t$ mit ihr kollidieren. Sei $N$ die Gesamtzahl der Teilchen im Gesamtvolumen $V$ und $n = N/V$ die Teilchendichte. Die Anzahl der Teilchen im Volumen $V_S$ beträgt somit $n \cdot V_S$.
\begin{figure}[htb]
    \centering
    \includegraphics[width=0.6\textwidth]{Bilder/Kapitel_Waermelehre/particles_in_box.pdf}
    \caption{Im kleinen Volumenelement $V_S$ erreichen alle Teilchen mit Geschwindigkeit $v_x$ die Wand $A$ im Zeitintervall $\dd t$, weil die Seitenlänge des Quaders $\dd s=v_x \dd t$.}\label{fig: teilchen_in_box}
\end{figure}
Da sich die Teilchen jedoch statistisch ungeordnet bewegen (siehe \cref{sec: maxwell_boltzmann_verteilung}), bewegt sich im Durchschnitt nur die Hälfte der Teilchen in positive $x$-Richtung (auf die Wand zu), die andere Hälfte bewegt sich davon weg. Die Anzahl der stoßenden Teilchen $\dd N_{\text{Stoß}}$ ist daher:
\begin{equation}
    \dd N_{\text{Stoß}} = \frac{1}{2} \cdot n \cdot V_S = \frac{1}{2} n A v_x \dd t \mDot
\end{equation}
Der gesamte Impulsübertrag $\dd P$ auf die Wand in der Zeit $\dd t$ ergibt sich aus der Anzahl der Stöße multipliziert mit dem Impulsübertrag pro Stoß ($\Delta p = 2mv_x$):
\begin{equation}
    \dd P = \dd N_{\text{Stoß}} \cdot (2 m v_x) = \left( \frac{1}{2} n A v_x \dd t \right) \cdot 2 m v_x = n m v_x^2 A \dd t \mDot
\end{equation}
Die Kraft $F$, die auf die Wand wirkt, ist die zeitliche Änderung des Impulses $F = \dd P / \dd t$:
\begin{equation}
    F = n m v_x^2 A \mDot
\end{equation}
Daraus folgt, dass die Teilchen mit Geschwindigkeit $v_x$ den Druckanteil ($p = F/A$)
\begin{equation}
    p = n m v_x^2
\end{equation}
ausüben. In einem realen Gas haben nicht alle Teilchen dieselbe Geschwindigkeit $v_x$. Wir müssen daher das mittlere Geschwindigkeitsquadrat $\overline{v_x^2}$ betrachten. Zudem bewegen sich die Teilchen im dreidimensionalen Raum. Aufgrund der \textit{Isotropie des Raumes} (keine Raumrichtung ist ausgezeichnet) muss das mittlere Geschwindigkeitsquadrat in alle drei Richtungen gleich groß sein:
\begin{equation}
    \overline{v_x^2} = \overline{v_y^2} = \overline{v_z^2} \mDot
\end{equation}
Das Quadrat der Gesamtgeschwindigkeit $\ivec{v}$ setzt sich nach dem Satz des Pythagoras aus den Komponenten zusammen: $\overline{v^2} = \overline{v_x^2} + \overline{v_y^2} + \overline{v_z^2} = 3 \overline{v_x^2}$.
Daraus folgt für die $x$-Komponente:
\begin{equation}
    \overline{v_x^2} = \frac{1}{3} \overline{v^2} \mDot
\end{equation}
Setzen wir dies in den Ausdruck für den Druck ein, erhalten wir die Grundgleichung der kinetischen Gastheorie.

\begin{importantbox}{Grundgleichung der kinetischen Gastheorie}
    Der Druck eines idealen Gases ergibt sich aus der Teilchendichte $n=N/V$, der Masse $m$ eines Teilchens und dem mittleren Geschwindigkeitsquadrat $\overline{v^2}$:
    \begin{equation}\label{eq: grundglg_kinetische_gastheorie}
        p = \frac{1}{3} n m \overline{v^2} \mDot
    \end{equation}
\end{importantbox}

\subsubsection{Interpretation als Energiedichte}
Die Gleichung lässt sich physikalisch sehr anschaulich interpretieren, wenn man sie leicht umformt und die \textbf{mittlere kinetische Energie eines Teilchens} $\overline{\Ekin} = \frac{1}{2}m\overline{v^2}$ einführt:
\begin{equation}
    p = \frac{1}{3} n m \overline{v^2} = \frac{2}{3} n \cdot \left( \frac{1}{2} m \overline{v^2} \right) = \frac{2}{3} \cdot \frac{N}{V} \cdot \overline{\Ekin} \mDot
\end{equation}
Bringt man das Volumen auf die andere Seite, erhält man
\begin{equation}\label{eq: druck_energie_beziehung}
    p \cdot V = \frac{2}{3} N \cdot \overline{\Ekin} \mDot
\end{equation}
Vergleicht man dieses Ergebnis mit der idealen Gasgleichung aus der Thermodynamik, $p \cdot V = N \cdot \kB \cdot T$, so findet man den direkten mikroskopischen Zugang zum Begriff der Temperatur.

\begin{rememberbox}{Mikroskopische Deutung der Temperatur}
    Durch Gleichsetzen von \cref{eq: druck_energie_beziehung} und der idealen Gasgleichung folgt:
    \begin{equation}\label{eq: Verbindung_EKin_mit_abs_T}
        \frac{2}{3} N \overline{\Ekin} = N \kB T \quad \implies \quad \overline{\Ekin} = \frac{3}{2} \kB T \mDot
    \end{equation}
    Die absolute Temperatur $T$ ist also ein Maß für die mittlere kinetische Energie der Gasteilchen.
\end{rememberbox}
\Cref{eq: Verbindung_EKin_mit_abs_T} ergibt sich für ein ideales Gas mit drei Freiheitsgraden (Bewegungsmöglichkeiten in $x$-, $y$- und $z$-Richtung). Jedes Freiheitsgrad trägt dabei den Anteil $\frac{1}{2} \kB T$ zur mittleren kinetischen Energie bei. 

\section{Maxwell-Boltzmann-Verteilung}\label{sec: maxwell_boltzmann_verteilung}
Im vorangegangenen Abschnitt haben wir den Gasdruck mithilfe des mittleren Geschwindigkeitsquadrats $\overline{v^2}$ hergeleitet. Dies könnte den Eindruck erwecken, dass alle Gasteilchen mit derselben Geschwindigkeit unterwegs sind.
In der Realität herrscht im Mikrokosmos jedoch ein ständiges Chaos: Die Gasteilchen stoßen permanent miteinander und tauschen dabei kinetische Energie aus. Ein Teilchen kann durch einen zufälligen Stoß stark beschleunigt werden, während ein anderes fast bis zum Stillstand abgebremst wird.

Selbst im thermischen Gleichgewicht, wo makroskopische Größen wie $p$ und $T$ konstant sind, ändern sich die Geschwindigkeiten der einzelnen Teilchen fortwährend.
Betrachtet man jedoch eine sehr große Anzahl von Teilchen ($N_A \approx \SI{6.022e23}{\mole^{-1}}$), so stellt sich eine zeitlich konstante \textbf{Geschwindigkeitsverteilung} ein.

\subsection{Die Verteilungsfunktion}\label{subsec: verteilungsfunktion}
In der folgenden Erklärung muss dezidiert zwischen den Komponenten der Geschwindigkeit $v_x,v_y,v_z$ und dem Betrag der Geschwindigkeit $v = \sqrt{v_x^2 + v_y^2 + v_z^2}$ unterschieden werden.
Die Wahrscheinlichkeit, ein Teilchen mit einer bestimmten Geschwindigkeit $v$ anzutreffen, wird durch die \textbf{Maxwell-Boltzmann-Verteilung} $f(v)$ beschrieben.
Genauer gesagt gibt die Größe $f(v)\,\dd v$ den Bruchteil der Teilchen an, deren Geschwindigkeitsbetrag im Intervall $[v, v+\dd v]$ liegt (siehe \cref{fig: f-v-dv_verteilung_anschauung}).

\begin{figure}[htb]
    \centering
    \resizebox{0.55\linewidth}{!}{
        \begin{tikzpicture}[
            >=Latex, font=\large,
            % Definition der Funktion
            % Maxwell-Boltzmann Form: f(v) ~ v^2 * exp(-a * v^2)
            % Die Konstanten 4.0 und 0.4 sind gewählt, damit die Kurve
            % optisch gut in das Koordinatensystem (6x4) passt.
            declare function={
                mb(\x) = 4.0 * \x^2 * exp(-0.4 * \x^2);
            }
            ]
            % Parameter für den Streifen
            % vpos etwas verschoben, damit der Streifen schön auf der Flanke liegt
            \def\vpos{1.8}       
            \def\dv{0.6}         
            \def\vend{\vpos+\dv} 

            % 1. Fläche füllen (mit mb statt gauss)
            \fill[red!15] (\vpos, 0) -- (\vpos, {mb(\vpos)})
                -- plot[domain=\vpos:\vend, samples=50] (\x, {mb(\x)})
                -- (\vend, 0) -- cycle;

            % 2. Achsen
            \draw[->, thick] (0,0) -- (6,0) node[below] {\Large $v$};
            \draw[->, thick] (0,0) -- (0,4) node[left] {\large $f(v)$};
            \node[below left] at (0,-0.1) {$0$};

            % 3. Graph (mit mb statt gauss)
            % Startet bei 0, nicht bei einem y-Offset!
            \draw[red!85!black, ultra thick, samples=100, domain=0:5.5] 
                plot (\x, {mb(\x)});

            % 4. Begrenzungslinien
            \draw[thick] (\vpos, 0) -- (\vpos, {mb(\vpos)});
            \draw[thick] (\vend, 0) -- (\vend, {mb(\vend)});

            % 5. Beschriftung x-Achse
            \node[below=6pt] at (\vpos, 0) {$v$};
            \node[below right=4pt] at (\vend-0.25, 0) {$v + \dd v$};

            % 6. Pfeile für dv
            % Position der Pfeile leicht angepasst auf y=0.6, das passt gut unter die Kurve
            \draw[->] (\vpos - 0.6, 0.6) -- (\vpos, 0.6) node[midway, above] {$\dd v$};
            \draw[->] (\vend + 0.6, 0.6) -- (\vend, 0.6); 

            % 7. Beschriftung Fläche
            % Position relativ zum Streifen angepasst
            \draw[-, thick] (\vpos + 0.45*\dv, {mb(\vpos)*0.6}) -- ++(0.5, 1.7) node[above right] {$f(v)\,\dd v$};
        \end{tikzpicture}
    }
    \caption{Die schraffierte Fläche $f(v)\dd v$ unter der Kurve entspricht dem Anteil der Moleküle mit Geschwindigkeitsbeträgen im Intervall zwischen $v$ und $v+\dd v$.}\label{fig: f-v-dv_verteilung_anschauung}
\end{figure}

Die analytische Form dieser Verteilung für den Betrag der Geschwindigkeit $v$ lautet:
\begin{equation}\label{eq:maxwell_boltzmann}
    f(v) = \underbrace{4\pi \left(\frac{m}{2\pi k_B T}\right)^{3/2}}_{\text{Normierung}} \cdot \underbrace{v^2}_{\text{Geometrie}} \cdot \underbrace{e^{-\frac{mv^2}{2k_B T}}}_{\text{Boltzmann-Faktor}} \mDot
\end{equation}
Der Verlauf der Kurve in \cref{fig: f-v-dv_verteilung_anschauung} ergibt sich aus dem Produkt zweier konkurrierender Terme:
\begin{itemize}
    \item \textbf{Der geometrische Anteil ($v^2$):} Dieser Term nimmt quadratisch zu. Er berücksichtigt, dass es mit zunehmendem Geschwindigkeitsbetrag rein geometrisch mehr Realisierungsmöglichkeiten für die Geschwindigkeit gibt (das Volumen einer Kugelschale im Geschwindigkeitsraum wächst mit $4\pi v^2$).
    \item \textbf{Der Boltzmann-Faktor ($e^{-E_{kin}/k_B T}$):} Dieser Term fällt exponentiell ab. Er drückt aus, dass Zustände mit sehr hoher Energie (hoher Geschwindigkeit) extrem unwahrscheinlich sind.
\end{itemize}
Das Produkt dieser Terme führt dazu, dass die Kurve bei $f(v=0) = 0$ beginnt, ein Maximum durchläuft (die wahrscheinlichste Geschwindigkeit) und für sehr große Geschwindigkeiten gegen Null geht.
\subsection{Warum ist die Verteilung nicht symmetrisch?}
Betrachtet man nur eine einzige Richtung (z.\,B. die $x$-Richtung), so verhalten sich die Gasteilchen vollkommen symmetrisch. Bewegungen nach links ($-v_x$) sind genauso wahrscheinlich wie nach rechts ($+v_x$). Die Wahrscheinlichkeit, ein Teilchen mit der Geschwindigkeitskomponente $v_x$ zu finden, folgt einer \textit{Gaußschen Normalverteilung} (Glockenkurve):
\begin{equation}
    f(v_x) \propto e^{-\frac{m v_x^2}{2 k_B T}} \mDot
\end{equation}
Das Maximum dieser Kurve liegt bei $v_x = 0$. Das bedeutet: Betrachtet man nur eine Komponente, ist \gDQ{Stillstand} der häufigste Zustand.

Warum liegt dann das Maximum der Maxwell-Boltzmann-Verteilung für den \textbf{Betrag der Geschwindigkeit} $v$ \textit{nicht} bei Null?

\subsubsection*{Der Wettstreit zwischen Energie und Entropie}
Um die Wahrscheinlichkeit für einen bestimmten Betrag $v$ zu erhalten, müssen wir alle Geschwindigkeitsvektoren $\ivec{v}=(v_x, v_y, v_z)$ zusammenzählen, die dieselbe Länge $v$ haben ($v^2 = v_x^2 + v_y^2 + v_z^2$).
Geometrisch betrachtet liegen alle Vektoren mit Länge $v$ auf der Oberfläche einer Kugel mit Radius $v$.

Die Wahrscheinlichkeit $P(v)$, ein Teilchen mit Geschwindigkeitsbetrag $v$ zu finden, setzt sich aus zwei Faktoren zusammen:
\begin{enumerate}
    \item \textbf{Wie schwierig ist es energetisch, diese Geschwindigkeit zu erreichen?} \\
    Dies wird durch den \textbf{Boltzmann-Faktor} beschrieben. Hohe Energien sind exponentiell unwahrscheinlicher. Die \gDQ{Punktdichte} im Geschwindigkeitsraum nimmt nach außen hin rapide ab.
    \begin{equation*}
        \text{Wahrscheinlichkeit pro Zustand} \propto e^{-\frac{m v^2}{2 k_B T}}
    \end{equation*}
    
    \item \textbf{Wie viele Kombinationen $(v_x, v_y, v_z)$ gibt es, um einen Geschwindigkeitsbetrag $v$ zu realisieren?} \\
    Hier kommt die Geometrie ins Spiel. Alle Vektoren mit Betrag $v$ bilden eine Kugelschale. Die Fläche dieser Kugelschale wächst quadratisch mit dem Radius $v$:
    \begin{equation*}
        \text{Anzahl der Zustände (Kugelfläche)} \propto 4\pi v^2
    \end{equation*}
\end{enumerate}



\begin{rememberbox}{Zusammenhang: Wahrscheinlichkeit $\times$ Anzahl der Möglichkeiten}
    Bei sehr kleinen Geschwindigkeiten ($v \approx 0$) ist der Boltzmann-Faktor groß (ca.\ 1), aber die Kugelschale ist winzig ($v^2 \approx 0$). Es gibt kaum Möglichkeiten, sich \gDQ{langsam} zu bewegen, denn es müssen alle 3 Geschwindigkeitskomponenten gleichzeitig klein sein.
    
    Bei sehr großen Geschwindigkeiten ist die Kugelschale riesig ($v^2$ sehr groß), denn es gibt viele Möglichkeiten ein großes $v$ zu realisieren, aber der Boltzmann-Faktor geht gegen Null, weil es sehr unwahrscheinlich ist, dass ein Teilchen durch elastische Stöße so viel Energie erhält.
    
    Die Maxwell-Boltzmann-Verteilung ist das Produkt dieser beiden Terme:
    \begin{equation}
        f(v) \propto \underbrace{v^2}_{\text{Kugelschale}} \cdot \underbrace{e^{-v^2}}_{\text{Boltzmann-Faktor}} \mComma
    \end{equation}
    dargestellt in \cref{fig: entstehung_maxwell_boltzmann}.
    Das Produkt aus einer steigenden Parabel und einer fallenden Exponentialfunktion erzwingt ein Maximum bei einer Geschwindigkeit $v > 0$ und eine verschwindende Wahrscheinlichkeit bei $v \to \infty$.
\end{rememberbox}

\begin{figure}[htb]
    \centering
    \begin{tikzpicture}[xscale=1.2, yscale=4.5, >=Latex, font=\small]
        % Achsen
        \draw[->] (0,0) -- (4,0) node[right] {$v$};
        \draw[->] (0,0) -- (0,1.1) node[above] {Wahrscheinlichkeit};
        
        % 1. Parabel (v^2) - Skaliert für die Optik
        \draw[blue, thick, dashed, domain=0:2.1] plot (\x, {0.2*\x*\x}) node[right] {$v^2$ (Volumen)};
        
        % 2. Exponential (e^-v^2)
        \draw[green!60!black, thick, dashed, domain=0:3.7] plot (\x, {exp(-0.8*\x*\x)}) node[below=1pt, xshift=-1cm] {$e^{-v^2}$ (Boltzmann)};
        
        % 3. Produkt (Maxwell-Boltzmann)
        \draw[red, ultra thick, domain=0:3.7, samples=100] plot (\x, {2.0 * \x^2 * exp(-0.8*\x*\x)}) node[above=4pt] {Resultat $f(v)$};
        
        % Beschriftung
        \node[below left] at (0,0) {0};
    \end{tikzpicture}
    \caption{Die Maxwell-Boltzmann-Verteilung (rot) entsteht aus dem Produkt der quadratisch anwachsenden Zustandsdichte (blau) und der exponentiell fallenden Besetzungswahrscheinlichkeit (grün).}
    \label{fig: entstehung_maxwell_boltzmann}
\end{figure}


\subsection{Einfluss der Temperatur}
Die Form der Verteilung hängt stark von der Temperatur $T$ und der Teilchenmasse $m$ ab.
Erhöht man die Temperatur, steht mehr thermische Energie zur Verfügung. Das Maximum der Kurve verschiebt sich zu höheren Geschwindigkeiten und die Verteilung wird breiter und flacher. Die Fläche unter der Kurve bleibt dabei stets 1 (da die Summe aller Wahrscheinlichkeiten 100\,\% ergeben muss).
\begin{figure}[htb]
    \centering
    \includegraphics[width=0.63\textwidth]{Bilder/Kapitel_Waermelehre/maxwell-boltzmann_He_T100-1000K.pdf}
    \caption{Maxwell-Boltzmann-Verteilung für verschiedene Temperaturen $T$ für Helium. Mit steigender Temperatur verschiebt sich das Maximum zu höheren Geschwindigkeiten, und die Verteilung wird breiter.}
    \label{fig: maxwell_boltzmann_temp_variation_Helium}
\end{figure}

\subsection{Charakteristische Geschwindigkeiten}
Da die Verteilung nicht symmetrisch ist, fallen der häufigste Wert und der Mittelwert nicht zusammen. Wir unterscheiden drei wichtige Kennzahlen:

\begin{enumerate}
    \item \textbf{Die wahrscheinlichste Geschwindigkeit $v_W$}:
    Sie entspricht dem Maximum der Kurve (dort, wo die Ableitung $f'(v)=0$ ist). Die meisten Teilchen besitzen eine Geschwindigkeit nahe diesem Wert.
    \begin{equation}
        v_W = \sqrt{\frac{2k_B T}{m}}
    \end{equation}
    
    \item \textbf{Die mittlere Geschwindigkeit $\overline{v}$}:
    Dies ist das arithmetische Mittel der Geschwindigkeitsbeträge.
    \begin{equation}
        \overline{v} = \sqrt{\frac{8k_B T}{\pi m}} \approx 1{,}13 \, v_W
    \end{equation}
    
    \item \textbf{Die mittlere quadratische Geschwindigkeit $\sqrt{\overline{v^2}}$}:
    Die rms(\textit{root-mean-square})-Geschwindigkeit ist entscheidend für die kinetische Energie ($\overline{E_{\kin}} \propto \overline{v^2}$). Sie ist etwas größer als die mittlere Geschwindigkeit, da schnelle Teilchen quadratisch stärker ins Gewicht fallen.
    \begin{equation}
        v_{\text{rms}} = \sqrt{\overline{v^2}} = \sqrt{\frac{3k_B T}{m}} \approx 1{,}22 \, v_W
    \end{equation}
\end{enumerate}

\begin{examplebox}{Beispiel: Stickstoff in der Luft}
    Für Stickstoffmoleküle ($N_2$) mit $m \approx \SI{4.67e-26}{\kilo\gram}$ bei Raumtemperatur ($T = \SI{300}{\kelvin}$) ergeben sich folgende Werte:
    \begin{itemize}
        \item $v_W \approx \SI{422}{\meter\per\second}$
        \item $\overline{v} \approx \SI{476}{\meter\per\second}$
        \item $\sqrt{\overline{v^2}} \approx \SI{517}{\meter\per\second}$
    \end{itemize}
    Diese Geschwindigkeiten sind bemerkenswert hoch und liegen im Bereich der Schallgeschwindigkeit ($v_{\text{Schall}} \approx \SI{343}{\meter\per\second}$ bei Raumtemperatur).
\end{examplebox}

% NEW SECTION
\section{Wärmemenge und Wärmekapazität}\label{sec: waermemenge_waermekapazitaet}
Bisher haben wir gesehen, dass die Temperatur ein Maß für die mikroskopische Bewegungsenergie der Teilchen ist ($T \propto \overline{E_{\kin}}$). 
Daraus ergibt sich eine direkte Konsequenz: Möchten wir die Temperatur eines Körpers erhöhen, müssen wir die Geschwindigkeit seiner Teilchen erhöhen. Wir müssen dem System also Energie zuführen.
Diese Energie, die aufgrund einer Temperaturdifferenz zwischen einem System und seiner Umgebung übertragen wird, nennen wir \textbf{Wärme} $Q$ (oder Wärmemenge).
\begin{importantbox}{Wärmekapazität}
    Führt man einem Körper Wärme $\Delta Q$ zu, so steigt in der Regel seine Temperatur um einen Betrag $\Delta T$. Für diesen Prozess gilt:
    \begin{equation}\label{eq: waermemenge_def}
        \Delta Q = C \cdot \Delta T = c \cdot m \cdot \Delta T \mDot
    \end{equation}
    Die Proportionalitätskonstante $C$ ist die \textbf{Wärmekapazität} des gesamten Körpers. 
    Beziehen wir diese auf die Masse $m$, erhalten wir die Materialkonstante $c$, die \textbf{spezifische Wärmekapazität}.
    Sie gibt an, wie viel Energie notwendig ist, um \SI{1}{\kilo\gram} eines Stoffes um \SI{1}{\kelvin} zu erwärmen. Ihre Einheit ist $[\si{\joule\per\kilogram\per\kelvin}]$.
\end{importantbox}

\begin{rememberbox}{Die Einheit Kalorie}
    Historisch wurde für die Wärmemenge oft die Einheit \textbf{Kalorie} (\si{\calorie}) verwendet.
    Eine Kalorie ist definiert als die Wärmemenge, die benötigt wird, um \SI{1}{\gram} Wasser von \SI{14.5}{\celsius} auf \SI{15.5}{\celsius} zu erwärmen.
    Heute ist die SI-Einheit für Energie, das Joule, gebräuchlich. Der Umrechnungsfaktor beträgt:
    \begin{equation}
        \SI{1}{\calorie} = \SI{4.1868}{\joule} \mDot
    \end{equation}
\end{rememberbox}
Im alltäglichen Sprachgebrauch wird hingegen oft von \textbf{Kalorien} (mit großem K) gesprochen, wenn es um den Energiegehalt von Lebensmitteln geht. Dabei ist allerdings eine \textbf{Kilokalorie} (\si{\kilo\calorie}, kcal) gemeint, also \num{1000} Kalorien: 
\begin{equation}
    \SI{1}{\kilo\calorie} = \SI{1000}{\calorie} = \SI{4186.8}{\joule} \mDot
\end{equation}

\subsection{Molare Größen und die innere Energie}\label{subsec: molare_groessen_innere_energie}
In der Physik der Gase und in der Chemie ist es oft sinnvoller, statt der Masse $m$ die Stoffmenge $\nu$ in \textbf{Mol} zu betrachten, da die physikalischen Eigenschaften (wie der Druck) direkt von der Anzahl der Teilchen abhängen.
\begin{itemize}
    \item \textbf{Ein Mol} ist die Stoffmenge, die genauso viele Teilchen enthält, wie Atome in \SI{12}{\gram} des Kohlenstoff-Isotops ${}^{12}C$ enthalten sind.
    \item Diese Anzahl an Teilchen pro Mol ist eine universelle Konstante, die \textbf{Avogadro-Konstante} $N_A$:
    \begin{equation}
        N_A \approx \SI{6.022e23}{\mol^{-1}}
    \end{equation}
    \item Das \textbf{Molvolumen} $V_M$ ist das Volumen, das ein Mol eines Stoffes einnimmt. Für ideale Gase unter Normbedingungen ($p=\SI{1}{bar}$, $T=\SI{0}{\celsius}$) beträgt es $V_M = \SI{22.7}{\deci\meter\cubed}$.
\end{itemize}

\noindent Die ideale Gasgleichung lässt sich damit in ihrer molaren Form schreiben:
\begin{equation}\label{eq:ideale_gasgleichung_molar}
    p \cdot V = \nu \cdot R \cdot T \mComma
\end{equation}
wobei $R = N_A \cdot k_B \approx \SI{8.314}{\joule\per\mol\per\kelvin}$ die \textbf{allgemeine Gaskonstante} ist.

\subsubsection{Die Innere Energie idealer Gase}
Betrachten wir nun die gesamte Energie, die in einem idealen Gas gespeichert ist. Da wir potenzielle Energien zwischen den Teilchen vernachlässigen, ist die \textbf{Innere Energie} $U$ einfach die Summe der kinetischen Energien aller $N$ Teilchen.
Aus der kinetischen Gastheorie wissen wir, dass pro (Translations-)Freiheitsgrad $f$ eine mittlere Energie von $\frac{1}{2}k_B T$ gespeichert wird (Gleichverteilungssatz).
Für ein Gas mit $f$ Freiheitsgraden gilt daher:
\begin{equation}
    U = N \cdot \overline{E_\mathrm{kin}} = N \cdot \frac{f}{2} \kB T = \nu \cdot N_A \cdot \frac{f}{2} \kB T \mDot
\end{equation}
Da $N_A \cdot \kB = R$, folgt der fundamentale Zusammenhang:
\begin{equation}\label{eq: innere_energie_f}
    U(T) = \frac{f}{2} \cdot \nu \cdot R \cdot T \mDot
\end{equation}
Die Innere Energie eines idealen Gases hängt also \textit{nur} von der Temperatur und der Anzahl der Freiheitsgrade ab.

\subsection{Wärmekapazität von Gasen}\label{subsec: waermekapazitaet_gase}
Da die Innere Energie direkt von $T$ abhängt, können wir nun leicht berechnen, wie viel Wärme wir zuführen müssen, um $T$ zu erhöhen. Dabei müssen wir jedoch unterscheiden, \textit{wie} wir das Gas erwärmen: bei konstantem Volumen oder bei konstantem Druck.

\subsubsection{Bei konstantem Volumen ($C_V$)}
\begin{figure}[htb]
    \centering
    \includegraphics[width=0.6\linewidth]{Bilder/Kapitel_Waermelehre/cv_over_T_gases.png}
    \caption{Die molare Wärmekapazität $C_V$ (normiert auf $R$) für verschiedene Gase. 
    Einatomige Gase (wie He) haben nur Translationsfreiheitsgrade ($f=3 \implies C_V \approx 3R/2$). Zweiatomige Gase (wie $N_2$) haben zusätzliche Rotationsfreiheitsgrade ($f=5 \implies C_V \approx 5R/2$). Bei sehr hohen Temperaturen kommen noch Vibrationen hinzu.}\label{fig: cv_gase}
\end{figure}
Halten wir das Volumen fest (isochor), kann das Gas keine mechanische Arbeit verrichten (keine Expansion). Die gesamte zugeführte Wärme $\Delta Q$ fließt direkt in die Erhöhung der Inneren Energie ($\Delta Q = \Delta U$).
Für ein Mol Gas ($\nu = 1$) folgt aus \cref{eq: innere_energie_f,eq: waermemenge_def}:
\begin{equation}\label{eq: cv_definition}
    C_V = \frac{\Delta Q}{\Delta T} = \frac{\Delta U}{\Delta T} = \frac{d}{dT} \left(\frac{f}{2} R T\right) = \frac{f}{2} R \mDot
\end{equation}
\begin{importantbox}{Molare Wärmekapazität bei konstantem Volumen}
    Die molare Wärmekapazität eines idealen Gases bei konstantem Volumen beträgt:
    \begin{equation}\label{eq: cv_definition_ideal_gas}
        C_V = \frac{f}{2} R \mComma
    \end{equation}
    wobei $f$ die Anzahl der Freiheitsgrade des Gasmoleküls ist.
\end{importantbox}


\subsubsection{Bei konstantem Druck ($C_p$)}
Halten wir den Druck konstant (isobar), dehnt sich das Gas beim Erwärmen aus. Dabei leistet es Arbeit gegen die Umgebung. Wir müssen also \textit{mehr} Wärme zuführen: einen Teil für die Erhöhung der Inneren Energie (Temperaturanstieg) und einen Teil für die Verrichtung der Arbeit.
\begin{importantbox}{Molare Wärmekapazität bei konstantem Druck}
    Die molare Wärmekapazität eines idealen Gases bei konstantem Druck ergibt sich aus der \textit{Mayer-Relation}\footnote{Die Herleitung dieser Relation ist mithilfe des Ersten Hauptsatzes der Thermodynamik besonders einfach, der erst in \cref{chap: Hauptsaetze_Thermodynamik} behandelt wird. Daher finden Sie die Herleitung in \cref{subsec: herleitung_cp_mayer-relation}.}:
    \begin{equation}\label{eq: cp_definition_ideal_gas}
        C_p = C_V + R = \frac{f+2}{2} R \mComma
    \end{equation}
    die also um $R$ größer ist als $C_V$.
\end{importantbox}
Die universelle Gaskonstante $R$ entspricht demnach genau der Arbeit, die ein Mol Gas bei isobarer Erwärmung um \SI{1}{\kelvin} gegen den Umgebungsdruck verrichtet. \\
\\
Gase haben also zwei Wärmekapazitäten, wobei stets $C_p > C_V$ ist. Das Verhältnis $\kappa = C_p/C_V$ nennt man Adiabatenexponent. 
In \cref{fig: cv_gase} erkennt man, dass die molare Wärmekapazität von Gases von der Temperatur abhängt und keine Materialkonstante ist. Aus \cref{eq: cv_definition} ist ersichtlich, dass $C_V$ nur durch die die Anzahl der Freiheitsgrade $f$ des Gasmoleküls beeinflusst wird. Bisher haben wir nur Translationsfreiheitsgrade betrachtet (3 Freiheitsgrade in $x$-, $y$- und $z$-Richtung). Zwei- oder mehratomige Moleküle besitzen jedoch zusätzlich Rotations- und Vibrationsfreiheitsgrade, die bei höheren Temperaturen angeregt werden können.
Daher ist $f$ im Allgemeinen die Gesamtanzahl der Freiheitsgrade des Moleküls 
\begin{equation}\label{eq: f_gesamt_freiheitsgrade}
    f = f_\text{trans} + f_\text{rot} + f_\text{vib}
\end{equation}

\subsection{Wärmekapazität von Festkörpern}\label{subsec: waermekapazitaet_festkoerper}
In Festkörpern sind die Atome an feste Gitterplätze gebunden. Sie können nicht frei umherfliegen (keine Translation) und nicht rotieren. Sie können jedoch in drei Raumrichtungen um ihre Ruhelage schwingen.
Jede Schwingung besitzt im Mittel kinetische Energie \textit{und} potenzielle Energie (wie eine Feder). Pro Raumrichtung zählt man daher 2 energetische Freiheitsgrade.
Bei 3 Raumrichtungen ergeben sich somit $f = 3 \times 2 = 6$ Freiheitsgrade pro Atom.

Setzen wir dies in unsere Formel für die Wärmekapazität ein, erhalten wir das \textbf{Gesetz von Dulong-Petit}:
\begin{equation}\label{eq: dulong_petit}
    C_V = \frac{6}{2} R = 3R \approx \SI{24.9}{\joule\per\mol\per\kelvin} \mDot
\end{equation}
Erstaunlicherweise haben fast alle einfachen Festkörper (wie Metalle) bei Raumtemperatur näherungsweise dieselbe molare Wärmekapazität.
(Hinweis: Bei tiefen Temperaturen weichen Festkörper davon ab, was erst durch die Quantenmechanik erklärt werden konnte).

\begin{figure}[htb]
    \centering
    % Unterbild (a)
    \begin{subfigure}[b]{0.48\textwidth}
        \centering
        \includegraphics[width=\linewidth]{Bilder/Kapitel_Waermelehre/transversale_und_longitudinale_Schwingungen_festkoerper.png}
        \caption{Transversale (links) und longitudinale Schwingungen (rechts).}
        \label{fig: schwingungen_festkoerper}
    \end{subfigure}
    \hfill % Erzeugt horizontalen Abstand zwischen den Bildern
    % Unterbild (b)
    \begin{subfigure}[b]{0.48\textwidth}
        \centering
        \includegraphics[width=\linewidth]{Bilder/Kapitel_Waermelehre/cv_festkoerper.png}
        \caption{Normierte molare Wärmekapazität verschiedener Festkörper.}
        \label{fig: cv_festkoerper}
    \end{subfigure}
    \caption{Wärmekapazität von Festkörpern: (a) Veranschaulichung verschiedener transversaler und longitudinaler Schwingungsmodi im Gitter. (b) Bei hoher Temperatur nähern sich viele Stoffe dem klassischen Grenzwert von $3R$ an.}
    \label{fig: festkoerper_kombiniert}
\end{figure}


\section{Phasenübergänge}\label{sec: phasenuebergaenge}
Die Zustände fest, flüssig und gasförmig werden als \textbf{Aggregatzustände} oder \textbf{Phasen} eines Stoffes bezeichnet. Die Übergänge zwischen diesen Phasen haben spezifische Namen, wie Schmelzen, Gefrieren, Verdampfen, Kondensieren, Sublimieren und Resublimieren.
\begin{figure}[htb]
    \centering
    \includegraphics[width=0.5\linewidth]{Bilder/Kapitel_Waermelehre/Aggregatszustände.png}
    \caption{Die drei Aggregatzustände und die Übergänge zwischen ihnen.}
    \label{fig: phasenuebergaenge}
\end{figure}
Wir wollen in diesem Abschnitt untersuchen, unter welchen Bedingungen diese Phasenübergänge stattfinden, welche Energien dafür benötigt werden und unter welchen Bedingungen verschiedene Phasen koexistieren können.

\subsection{Beispiel: Die Erwärmungskurve von Wasser}\label{subsec: erwaermungskurve_beispiel}
Betrachten wir ein Experiment, bei dem Eis mit einer Anfangstemperatur von $\SI{-20}{\celsius}$ in einem geschlossenen Gefäß mit einer konstanten Heizleistung erwärmt wird. Man würde erwarten, dass die Temperatur kontinuierlich steigt. Die Messung in \cref{fig: erwaermungskurve_wasser} zeigt jedoch einen anderen Verlauf.
% \begin{figure}[htb]
%     \centering
%     \includegraphics[width=0.7\linewidth]{Bilder/Kapitel_Waermelehre/temperaturverlauf_erwärmen_von_eis.png}
%     \caption{Temperaturverlauf beim Erwärmen von Eis mit konstanter Wärmezufuhr. Während der Phasenübergänge (Schmelzen bei $t_1$ bis $t_2$, Verdampfen bei $t_3$ bis $t_4$) bleibt die Temperatur konstant, obwohl weiterhin Wärme zugeführt wird.}\label{fig: erwaermungskurve_wasser}
% \end{figure}
\begin{figure}[htb]
    \centering
    \begin{tikzpicture}[
            scale=1.0, font=\small,
            axis/.style={-{Stealth}, line width=1.0pt},
            box/.style={draw, fill=blue!10, rounded corners=2pt, minimum height=0.6cm, align=center, font=\footnotesize},
            curve/.style={line width=1.5pt, red!80!black},
            guide/.style={dashed, thin, gray}
            ]

            % --- Definition der Koordinaten (Zeit t und Temperatur T) ---
            % Y-Achse
            \def\yStart{0.5}   % Start bei z.B. -20 Grad
            \def\yMelt{2.0}    % 0 Grad
            \def\yBoil{5.5}    % 100 Grad
            \def\yEnd{7.0}     % > 100 Grad

            % X-Achse (Zeitpunkte)
            \def\xZero{0}
            \def\xOne{1.0}     % Ende Eis-Erwärmung
            \def\xTwo{2.0}     % Ende Schmelzen
            \def\xThree{5.0}   % Ende Wasser-Erwärmung
            \def\xFour{10.0}   % Ende Verdampfen (Langes Plateau!)
            \def\xEnd{10.6}    % Ende Dampf-Erwärmung

            % Y-Position der Beschriftungsboxen oben
            \def\yBox{6.3}

            % --- 1. Achsen zeichnen ---
            \draw[axis] (0, \yMelt) -- (\xEnd+0.5, \yMelt) node[right] {$t \,[\si{\second}]$}; 
            \draw[axis] (0, 0) -- (0, \yBox+1) node[left] {\large $T\, [\si{\celsius}]$};
            
            % --- 2. Ticks und Labels Y-Achse ---
            \draw[thick] (-0.15, \yStart) -- (0.15, \yStart) node[left=0.3cm] {$T_a$}; % Anfang
            \draw[thick] (-0.15, \yMelt) -- (0.15, \yMelt) node[left=0.3cm] {$T_s = \SI{0}{\celsius}$};
            \draw[thick] (-0.15, \yBoil) -- (0.15, \yBoil) node[left=0.3cm] {$T_v = \SI{100}{\celsius}$};
            \draw[thick, dashed, gray] (0, \yBoil) -- (\xEnd+0.5, \yBoil); % Hilfslinie bei Siedetemp.
            % --- 3. Kurve zeichnen ---
            \draw[curve] 
                (\xZero, \yStart) coordinate (Start)
                -- (\xOne, \yMelt) coordinate (MeltStart)
                -- (\xTwo, \yMelt) coordinate (MeltEnd)
                -- (\xThree, \yBoil) coordinate (BoilStart)
                -- (\xFour, \yBoil) coordinate (BoilEnd)
                -- (\xEnd, \yEnd) coordinate (End);

            % --- 4. Hilfslinien & Ticks X-Achse ---
            % t1
            \draw[guide] (MeltStart) -- (\xOne, 0) node[below, black] {$t_1$};
            % t2
            \draw[guide] (MeltEnd) -- (\xTwo, 0) node[below, black] {$t_2$};
            % t3
            \draw[guide] (BoilStart) -- (\xThree, 0) node[below, black] {$t_3$};
            % t4
            \draw[guide] (BoilEnd) -- (\xFour, 0) node[below, black] {$t_4$};

            % --- 5. Beschriftungsboxen (Phasen) ---
            % Schmelzen
            \node[align=center, text width=1.8cm] at ({(\xOne+\xTwo)/2}, \yMelt+0.5) {Schmelzen};
            % Verdampfen
            \node[align=center, text width=3.8cm] at ({(\xThree+\xFour)/2}, \yBoil+0.5) {Verdampfen};
            
            % --- 6. Zusätzliche Infos im Plot ---
            % Aggregatzustände unter die Kurve schreiben
            \node[red!60!black, font=\footnotesize] at ({(\xZero+\xOne)/2}, \yStart+0.1) {fest};
            \node[red!60!black, font=\footnotesize] at ({(\xTwo+\xThree)/2}, \yMelt+0.9) {flüssig};
            \node[red!60!black, font=\footnotesize] at (\xFour+0.86, \yEnd-1.2) {gasförmig};

            % Pfeil für Latente Wärme
            \draw[<->, >=Latex, thin] (\xThree, \yBoil-0.3) -- (\xFour, \yBoil-0.3) node[midway, below] {$\Delta Q = m \cdot \lambda_v$};

            % dT/dt Pfeile während der Erwärmungsphasen
            \draw[->, >=Latex, thin] ({(\xZero+\xOne)/2}, \yStart+0.65) -- (\xTwo+0.35, \yStart+0.1) node[right] {$\frac{\dd T}{\dd t} \propto \frac{1}{C_V\text{(fest)}}$};
            \draw[->, >=Latex, thin] ({(\xTwo+\xThree)/2 + 0.1}, \yMelt+1.77) -- (\xThree+0.35, \yMelt+1) node[right] {$\frac{\dd T}{\dd t} \propto \frac{1}{C_V\text{(flüssig)}}$};
            \draw[->, >=Latex, thin] (\xFour+0.3, \yBoil+1) -- (\xFour-0.35, \yBoil+1.5) node[left] {$\frac{\dd T}{\dd t} \propto \frac{1}{C_V\text{(gas)}}$};
    \end{tikzpicture}
    \caption{Temperaturverlauf beim Erwärmen von Wasser mit konstanter Wärmezufuhr pro Zeiteinheit. Während der Phasenübergänge (Schmelzen bei $t_1$ bis $t_2$, Verdampfen bei $t_3$ bis $t_4$) bleibt die Temperatur konstant, obwohl weiterhin Wärme zugeführt wird.}\label{fig: erwaermungskurve_wasser}
\end{figure}

Man beobachtet zwei markante Bereiche (Plateaus), in denen die Temperatur trotz Wärmezufuhr konstant bleibt:
\begin{enumerate}
    \item Beim Schmelzpunkt ($T_S = \SI{0}{\celsius}$) schmilzt das Eis zu Wasser.
    \item Beim Siedepunkt ($T_V = \SI{100}{\celsius}$) verdampft das Wasser zu Wasserdampf.
\end{enumerate}
Erst wenn die gesamte Substanz den Phasenübergang vollzogen hat, steigt die Temperatur wieder an. Die Steigung der Erwärmungskurve in den Bereichen zwischen den Plateaus hängt von der spezifischen Wärmekapazität des jeweiligen Aggregatzustands ab. Je größer die Wärmekapazität, desto langsamer steigt die Temperatur bei gegebener Wärmezufuhr. Für Wasser gilt:
\begin{equation}
    c_\text{Wasser} \approx \SI{4.18}{\kilo\joule\per\kilogram\per\kelvin} > c_\text{Eis} \approx \SI{2.1}{\kilo\joule\per\kilogram\per\kelvin} > c_\text{Dampf} \approx \SI{2.0}{\kilo\joule\per\kilogram\per\kelvin} \mDot
\end{equation}

\subsection{Schmelz- und Verdampfungswärme}\label{subsec: schmelz_verdampfungswaerme}
Die im obigen Beispiel beobachteten Temperaturplateaus führen uns zum Begriff der \textbf{latenten Wärme}. Die in diesen Phasen zugeführte Energie wird nicht zur Erhöhung der kinetischen Energie der Teilchen (was wir als Temperaturanstieg messen würden), sondern zur Erhöhung der potenziellen Energie verwendet. Konkret werden Bindungen zwischen den Molekülen gelockert (Schmelzen) oder fast vollständig getrennt (Verdampfen).

Wir definieren daher zwei spezifische Materialkonstanten:
\begin{itemize}
    \item Die \textbf{spezifische Schmelzwärme} $\lambda_s$ ist die Energiemenge, die benötigt wird, um \SI{1}{\kilo\gram} eines festen Stoffes bei konstanter Schmelztemperatur zu verflüssigen.
    \item Die \textbf{spezifische Verdampfungswärme} $\lambda_v$ ist die Energiemenge, die benötigt wird, um \SI{1}{\kilo\gram} einer Flüssigkeit bei konstanter Siedetemperatur zu verdampfen.
\end{itemize}

Beide Größen haben die Einheit $[\lambda] = \si{\joule\per\kilogram}$. In \cref{tab: waermekapazitaet_schmelz_verdampfungswaermen} sind typische Werte für verschiedene Stoffe aufgelistet. Man beachte, dass die Verdampfungswärme meist deutlich größer ist als die Schmelzwärme, da beim Verdampfen die Teilchenbindungen vollständig überwunden werden müssen.

\begin{table}[htb]
    \centering
    \caption{Spezifische Wärmekapazität $c$, Schmelzwärme $\lambda_s$ und Verdampfungswärme $\lambda_v$ ausgewählter Stoffe.}\label{tab: waermekapazitaet_schmelz_verdampfungswaermen}
    \vspace{4pt}
    \begin{tabular}{l S[table-format=1.3] S[table-format=3.1] S[table-format=5.0]}
        \toprule
        \textbf{Stoff} & {\textbf{$c$ (\si{\kilo\joule\per\kilogram\per\kelvin})}} & {\textbf{$\lambda_s$ (\si{\kilo\joule\per\kilogram})}} & {\textbf{$\lambda_v$ (\si{\kilo\joule\per\kilogram})}} \\
        \midrule
        Wasser (flüssig)  & 4,182  & {--}   & 2256  \\
        Quecksilber       & 0,14   & 12,4   & 285   \\
        Aluminium         & 0,896  & 397    & 10900 \\
        Eisen             & 0,45   & 277    & 6340  \\
        Kupfer            & 0,383  & 205    & 4790  \\
        Eis (\SI{0}{\celsius}) & 2,1    & 332,8  & {--}   \\
        \bottomrule
    \end{tabular}
\end{table}



\section{Reale Gase und Phasendiagramme}\label{sec: reale_gase_phasendiagramme}
Das ideale Gasmodell vernachlässigt sowohl das Eigenvolumen der Teilchen als auch die Anziehungskräfte zwischen ihnen. Bei hohem Druck und niedriger Temperatur, wenn die Teilchen nahe beieinander sind, werden diese Effekte jedoch relevant und das Verhalten realer Gase weicht von der idealen Gasgleichung ab.

\subsection{Die Van-der-Waals-Gleichung}\label{subsec: van_der_waals}
Um reale Gase zu beschreiben, entwickelten Johannes Diderik van der Waals im Jahr 1873 eine angepasste Zustandsgleichung. Wir leiten diese heuristisch her, indem wir zwei Korrekturen an der idealen Gasgleichung ($p V = \nu R T$) vornehmen.

\subsubsection*{1. Volumenkorrektur (Kovolumen)}
Reale Gasteilchen sind keine punktförmigen Objekte, sondern besitzen ein endliches Eigenvolumen. Das Volumen $V$, in dem sich die Teilchen frei bewegen können, ist daher kleiner als das Volumen des Behälters.
Wir ersetzen das Volumen $V$ durch das \textbf{effektive Volumen}:
\begin{equation}
    V \rightarrow V - \nu \cdot b
\end{equation}
Hierbei ist $b$ das sogenannte \textbf{Kovolumen}, das proportional zum Eigenvolumen der Teilchen eines Mols ist.

\subsubsection*{2. Druckkorrektur (Binnendruck)}
Zwischen den Gasteilchen wirken Anziehungskräfte (Van-der-Waals-Kräfte). Befindet sich ein Teilchen im Inneren des Gases, heben sich die Kräfte der Nachbarteilchen im Mittel auf. Ein Teilchen nahe der Wand erfährt jedoch eine resultierende Kraft in das Innere des Gases (siehe Kohäsion). Dadurch wird der Impulsübertrag auf die Wand -- und somit der messbare Druck $p$ -- verringert.
Um den \gDQ{wahren} thermodynamischen Druck zu erhalten, den das Gas ohne diese Anziehung hätte, müssen wir zum gemessenen Druck $p$ einen Korrekturterm $p_{i}$ (Binnendruck) addieren.

Dieser Binnendruck ist proportional zur Teilchendichte zum Quadrat ($\rho^2 \propto (\nu/V)^2$), da die Anziehung sowohl von der Anzahl der ziehenden Teilchen als auch von der Anzahl der gezogenen Teilchen abhängt:
\begin{equation}
    p \rightarrow p + a \cdot \frac{\nu^2}{V^2}
\end{equation}
Der Parameter $a$ ist ein Maß für die Stärke der Anziehungskräfte zwischen den Teilchen.

\begin{importantbox}{Van-der-Waals-Gleichung}
    Führt man beide Korrekturen in die ideale Gasgleichung ein, erhält man die \textbf{Van-der-Waals-Gleichung}:
    \begin{equation}\label{eq: van_der_waals}
        \left( p + a \frac{\nu^2}{V^2} \right) \cdot (V - \nu b) = \nu R T \mDot
    \end{equation}
    Hierbei sind $a$ und $b$ materialspezifische Konstanten, die für jedes Gas experimentell bestimmt werden müssen.
\end{importantbox}

\subsubsection{Isothermen realer Gase}
Trägt man den Druck $p$ als Funktion des Volumens $V$ für verschiedene Temperaturen auf (Isothermen), so zeigt sich das charakteristische Verhalten realer Gase, wie in \cref{fig: vanderwaals_isothermen} für Kohlendioxid ($\text{CO}_2$) dargestellt.

\begin{itemize}
    \item \textbf{Hohe Temperaturen ($T > T_K$):} Die Isothermen ähneln Hyperbeln. Das Gas verhält sich näherungsweise wie ein ideales Gas.
    \item \textbf{Kritische Temperatur ($T_K$):} Bei einer bestimmten Temperatur (für $\text{CO}_2$ ca. \SI{304}{\kelvin}) besitzt die Isotherme einen Sattelpunkt (kritischer Punkt). Oberhalb dieser Temperatur kann das Gas durch Druck allein \textit{nicht} mehr verflüssigt werden.
    \item \textbf{Niedrige Temperaturen ($T < T_K$):} Die theoretische Van-der-Waals-Kurve zeigt einen schlangenförmigen Verlauf (\gDQ{Van-der-Waals-Schleife}). In der Realität findet hier ein Phasenübergang statt: Das Gas verflüssigt sich. Der Druck bleibt dabei konstant (horizontales Plateau), bis das gesamte Gas kondensiert ist (Maxwell-Konstruktion).
\end{itemize}
Das heißt die van-der-Waals-Gleichung beschreibt das Verhalten realer Gase nur näherungsweise und weicht bei Phasenübergängen vom gemessenen Verhalten ab. Die van-der-Waals-Isotherme in \cref{fig: vanderwaals_isothermen} zeigt dies exemplarisch für $\text{CO}_2$ bei einer Temperatur unterhalb der kritischen Temperatur. Komprimiert man $\text{CO}_2$-Gas bei $T = \SI{0}{\celsius}$, so folgt die gemessene Kurve zunächst der van-der-Waals-Kurve bis zum Punkt $A$. Danach bleibt der Druck konstant (horizontales Plateau) bis zum Punkt $C$, während das Gas verflüssigt wird. Erst danach steigt der Druck wieder an. Zwischen
$A$ und $C$ können also Gas und Flüssigkeit gleichzeitig existieren
(Koexistenzbereich). Der steile Anstieg von $p$ nach Erreichen des Punktes $C$ liegt an der im Vergleich mit Gasen sehr kleinen Kompressibilität von Flüssigkeiten.

\begin{figure}[htb]
    \centering
    \includegraphics[width=0.6\linewidth]{Bilder/Kapitel_Waermelehre/van-der-Waals_Isothermen_CO2.png}
    \caption{Isothermen eines realen Gases ($\text{CO}_2$) nach dem Van-der-Waals-Modell. Unterhalb der kritischen Temperatur $T_K$ treten Bereiche auf (rosa Fläche), in denen Gas und Flüssigkeit koexistieren. Die gestrichelte Linie markiert den Bereich des Phasenübergangs (konstanter Dampfdruck).}\label{fig: vanderwaals_isothermen}
\end{figure}

\section{Phasendiagramme}\label{sec: phasendiagramme}
Ein Phasendiagramm (oder Zustandsdiagramm) visualisiert die Existenzbereiche der Aggregatzustände (fest, flüssig, gasförmig) in Abhängigkeit von den Zustandsgrößen Druck $p$ und Temperatur $T$. Die Linien im Diagramm trennen die verschiedenen Phasen und markieren Zustände, in denen zwei Phasen im Gleichgewicht koexistieren.

\begin{itemize}
    \item \textbf{Sublimationskurve:} Trennt fest und gasförmig.
    \item \textbf{Siedekurve (Dampfdruckkurve):} Trennt flüssig und gasförmig. Sie endet im kritischen Punkt.
    \item \textbf{Schmelzkurve:} Trennt fest und flüssig.
\end{itemize}

Es gibt zwei ausgezeichnete Punkte in einem solchen Diagramm:
\begin{enumerate}
    \item \textbf{Der Tripelpunkt:} Hier treffen alle drei Phasengrenzlinien aufeinander. Alle drei Phasen (fest, flüssig, gasförmig) existieren gleichzeitig im thermodynamischen Gleichgewicht. Der Tripelpunkt von Wasser liegt bei $T_{\text{Tr}} = \SI{273.16}{\kelvin}$ und $p_{\text{Tr}} = \SI{6.1}{\hecto\pascal}$ und dient zur Definition der Kelvin-Skala.
    \item \textbf{Der kritische Punkt:} Er bildet das Ende der Siedekurve. Oberhalb von kritischem Druck $p_K$ und kritischer Temperatur $T_K$ sind Flüssigkeit und Gas nicht mehr unterscheidbar; man spricht von einem überkritischen Fluid.
\end{enumerate}

\subsection{Vergleich: Wasser und Kohlendioxid}
Die Phasendiagramme verschiedener Stoffe sehen qualitativ ähnlich aus, weisen aber wichtige Unterschiede auf, insbesondere in der Steigung der Schmelzkurve.
\begin{figure}[htb]
    \centering
    % Wir nutzen subfigure oder minipages für den Vergleich, falls das Paket subcaption geladen ist (im Preamble oben ja).
    \begin{subfigure}[b]{0.42\textwidth}
        \centering
        \includegraphics[width=\linewidth]{Bilder/Kapitel_Waermelehre/Phasendiagramm_H2O.png}
        \caption{Phasendiagramm von Wasser: Die Schmelzkurve (fest/flüssig) verläuft nach links oben.}\label{fig: phasendiagramm_wasser_links}
    \end{subfigure}
    \hfill
    \begin{subfigure}[b]{0.42\textwidth}
        \centering
        \includegraphics[width=\linewidth]{Bilder/Kapitel_Waermelehre/Phasendiagramm_CO2.png}
        \caption{Phasendiagramm von $\text{CO}_2$: Die Schmelzkurve verläuft klassisch nach rechts oben.}
        \label{fig: phasendiagramm_co2_rechts}
    \end{subfigure}
    \caption{Gegenüberstellung der Phasendiagramme von Wasser und Kohlendioxid. Die unterschiedlichen Steigungen der Schmelzkurven resultieren aus den Dichteunterschieden zwischen fester und flüssiger Phase.}
    \label{fig: phasendiagramme_vergleich}
\end{figure}

\subsubsection{Wasser ($\text{H}_2\text{O}$) -- Die Anomalie}
Wie in \cref{fig: phasendiagramme_vergleich} (links) zu sehen ist, besitzt die Schmelzkurve von Wasser eine \textbf{negative Steigung}. Das bedeutet: Erhöht man den Druck auf Eis, sinkt dessen Schmelzpunkt.
\begin{rememberbox}{Anomalie des Wassers}
    Da flüssiges Wasser eine höhere Dichte hat als Eis (Eis schwimmt oben), begünstigt eine Druckerhöhung den Zustand mit dem kleineren Volumen -- also die flüssige Phase. Daher schmilzt Eis unter hohem Druck.
\end{rememberbox}

\subsubsection{Kohlendioxid ($\text{CO}_2$) -- Normales Verhalten}
Bei den meisten Stoffen, so auch bei $\text{CO}_2$ (siehe \cref{fig: phasendiagramme_vergleich} rechts), ist die Schmelzkurve nach rechts geneigt (\textbf{positive Steigung}). Der Feststoff hat eine höhere Dichte als die Flüssigkeit. Durch Druckerhöhung wird der Festkörper stabilisiert, die Schmelztemperatur steigt.

Eine Besonderheit von $\text{CO}_2$ ist die Lage des Tripelpunkts bei $p_{\text{Tr}} \approx \SI{5.1}{\barPr}$. Da unser Atmosphärendruck ($\approx \SI{1}{\barPr}$) unterhalb dieses Drucks liegt, gibt es bei Normalbedingungen kein flüssiges $\text{CO}_2$. Festes $\text{CO}_2$ (Trockeneis) geht beim Erwärmen direkt in den gasförmigen Zustand über (Sublimation).



\subsection{Dampfdruckkurve und Clausius-Clapeyron-Gleichung}\label{subsec: dampfdruckkurve_clausius_clapeyron}
Bringt man eine Flüssigkeit in ein abgeschlossenes Gefäß, das sie nur teilweise ausfüllt, so stellt man fest, dass ein Teil der Flüssigkeit verdampft, auch wenn man keine zusätzliche Wärme zuführt. Im Volumen oberhalb der Flüssigkeit bildet sich eine Dampfphase, die einen Druck $p_S(T)$ auf die Wände und die Flüssigkeitsoberfläche ausübt.

Bei einer festen Temperatur $T$ stellt sich ein dynamisches Gleichgewicht ein: Pro Zeiteinheit verlassen genauso viele Moleküle die Flüssigkeit (Verdampfung), wie aus dem Dampf in die Flüssigkeit zurückkehren (Kondensation). Der sich einstellende Druck wird als \textbf{Sättigungsdampfdruck} $p_S(T)$ bezeichnet.

\begin{figure}[htb]
    \centering
    \includegraphics[width=0.48\linewidth]{Bilder/Kapitel_Waermelehre/Messung_Dampfdruckkurve.png}
    \caption{Links: Prinzipielle Messanordnung zur Bestimmung des Dampfdrucks. Rechts: Typischer Verlauf der Dampfdruckkurve $p_S(T)$ im $(p,T)$-Diagramm. Sie trennt den flüssigen vom gasförmigen Bereich und endet im kritischen Punkt.}\label{fig: messung_dampfdruckkurve}
\end{figure}

Dass dieser Druck stark mit der Temperatur ansteigt, lässt sich mit der kinetischen Gastheorie erklären: Auch in einer Flüssigkeit sind die kinetischen Energien der Moleküle nach einer Maxwell-Boltzmann-Verteilung verteilt. Um die Flüssigkeit zu verlassen, muss ein Molekül genügend kinetische Energie besitzen, um die anziehenden Bindungskräfte an der Oberfläche (Oberflächenspannung) zu überwinden. 

Je höher die Temperatur, desto breiter wird die Verteilung und desto größer ist der Anteil der Moleküle, deren Energie diesen Schwellenwert überschreiten kann. Folglich steigt der Dampfdruck exponentiell an.

\subsubsection{Die Clausius-Clapeyron-Gleichung}
Der quantitative Zusammenhang zwischen der Steigung der Dampfdruckkurve im $(p,T)$-Diagramm und der umgesetzten Wärme wird durch die \textbf{Clausius-Clapeyron-Gleichung} beschrieben.
\begin{importantbox}{Clausius-Clapeyron-Gleichung}
    Die Steigung der Dampfdruckkurve $\frac{\dd p_S}{\dd T}$ ist verknüpft mit der spezifischen Verdampfungswärme $\lambda_v$ (bzw. der molaren Verdampfungswärme $\Lambda_v$), der absoluten Temperatur $T$ und der Volumenänderung beim Phasenübergang:
    \begin{equation}\label{eq: clausius_clapeyron}
        \frac{\dd p_S}{\dd T} = \frac{\lambda_v}{T \cdot (v_D - v_{Fl})} \mDot
    \end{equation}
    Hierbei sind $v_D$ und $v_{Fl}$ die spezifischen Volumina (Volumen pro Masse, $[\si{\cubic\meter\per\kilogram}]$) des Dampfes bzw. der Flüssigkeit.
\end{importantbox}

Da das Volumen des Gases $v_D$ in der Regel sehr viel größer ist als das der Flüssigkeit ($v_D \gg v_{Fl}$), kann man näherungsweise $v_D - v_{Fl} \approx v_D$ setzen. Betrachtet man den Dampf zudem als ideales Gas ($v_D = RT/p$), lässt sich die Gleichung integrieren, was zu dem exponentiellen Verlauf der Dampfdruckkurve führt.

\subsection{Verdampfungswärme}\label{subsec: ursachen_verdampfungswaerme}
Die spezifische Verdampfungswärme $\lambda_v$ ist die Energie, die aufgebracht werden muss, um die Phase von flüssig zu gasförmig zu ändern. Physikalisch betrachtet setzt sich diese Energie aus zwei Beiträgen zusammen:
\begin{enumerate}
    \item \textbf{Verschiebungsarbeit:} Das Volumen des Stoffes vergrößert sich beim Verdampfen drastisch (von $V_{Fl}$ auf $V_D$). Diese Expansion muss gegen den äußeren Druck $p$ verrichtet werden. Der Arbeitsanteil ist $W_{\text{ext}} = p \cdot (V_D - V_{Fl})$.
    \item \textbf{Innere Arbeit (Dissoziationsarbeit):} Arbeit muss verrichtet werden, um die Anziehungskräfte der Moleküle vollständig zu überwinden und den mittleren Molekülabstand auf den des Gases zu vergrößern. Dies entspricht einer Erhöhung der inneren Energie.
\end{enumerate}

Der zweite Anteil ist in der Regel wesentlich größer als der erste, wie das folgende Beispiel zeigt.

\begin{examplebox}{Beispiel: Verdampfung von Wasser}
    Wir betrachten $m = \SI{1}{\kilogram}$ Wasser, das bei $T = \SI{100}{\celsius}$ und einem äußeren Druck von $p = \SI{1.013}{\barPr}$ ($\approx \SI{1e5}{\pascal}$) verdampft wird.
    \begin{itemize}
        \item Volumen der Flüssigkeit: $V_{Fl} \approx \SI{1}{\deci\meter\cubed} = \SI{0.001}{\cubic\meter}$.
        \item Volumen des Dampfes: $V_D \approx \SI{1700}{\deci\meter\cubed} = \SI{1.7}{\cubic\meter}$.
    \end{itemize}
    Die Volumenänderung beträgt $\Delta V \approx \SI{1.7}{\cubic\meter}$. Die gegen den äußeren Luftdruck verrichtete Arbeit ist:
    \begin{equation}
        W_{\text{vol}} = p \cdot \Delta V \approx \SI{1e5}{\pascal} \cdot \SI{1.7}{\cubic\meter} = \SI{170}{\kilo\joule} \mDot
    \end{equation}
    Die gemessene spezifische Verdampfungswärme von Wasser beträgt jedoch $\lambda_v = \SI{2257}{\kilo\joule\per\kilogram}$.
    
    \textbf{Vergleich:} Der Anteil der Volumenarbeit beträgt
    \begin{equation}
        \frac{W_{\text{vol}}}{\lambda_v} = \frac{\SI{170}{\kilo\joule}}{\SI{2257}{\kilo\joule}} \approx \SI{7.5}{\percent} \mDot
    \end{equation}
    Über \SI{92}{\percent} der zugeführten Energie werden also benötigt, um die molekularen Bindungen (bei Wasser insbesondere die Wasserstoffbrückenbindungen) zu trennen.
\end{examplebox}

\subsection{Schmelzkurve}\label{subsec: schmelzkurve}
Erhöht man die Temperatur eines festen Stoffes bei konstantem Druck, so beginnt der Stoff bei einer charakteristischen Temperatur $T_S$ (Schmelztemperatur) zu schmelzen. Die Schmelzkurve $p_{fs}(T)$ im Phasendiagramm trennt die feste von der flüssigen Phase. Nur auf dieser Linie können beide Phasen im thermodynamischen Gleichgewicht koexistieren.

Die Steigung der Schmelzkurve wird analog zur Dampfdruckkurve durch die Clausius-Clapeyron-Gleichung bestimmt:
\begin{equation}\label{eq: clausius_clapeyron_schmelzen}
    \frac{\dd p_{fs}}{\dd T} = \frac{\lambda_s}{T \cdot (v_{Fl} - v_{Fest})} \mDot
\end{equation}
Da die Schmelzwärme $\lambda_s$ stets positiv ist ($\lambda_s > 0$) und die absolute Temperatur $T > 0$ ist, wird das Vorzeichen der Steigung ausschließlich durch die Volumendifferenz $\Delta v = v_{Fl} - v_{Fest}$ bestimmt.

Hierbei unterscheiden wir zwei Fälle, die zu qualitativ unterschiedlichen Phasendiagrammen führen (siehe \cref{fig: phasendiagramme_vergleich_schmelz}).

\begin{figure}[htb]
    \centering
    \begin{subfigure}[b]{0.43\textwidth}
        \centering
        \includegraphics[width=\linewidth]{Bilder/Kapitel_Waermelehre/Phasendiagramm_H2O_Schmelzkurve.png}
        \caption{Wasser ($\text{H}_2\text{O}$): Anomalie. Die Schmelzkurve (grün) hat eine \textit{negative} Steigung.}\label{fig: phasendiagramm_schmelzkurve_wasser}
    \end{subfigure}
    \hfill
    \begin{subfigure}[b]{0.43\textwidth}
        \centering
        \includegraphics[width=\linewidth]{Bilder/Kapitel_Waermelehre/Phasendiagramm_CO2_Schmelzkurve.png}
        \caption{Kohlendioxid ($\text{CO}_2$): Normalverhalten. Die Schmelzkurve (grün) hat eine \textit{positive} Steigung.}\label{fig: phasendiagramm_schmelzkurve_co2}
    \end{subfigure}
    \caption{Gegenüberstellung der Phasendiagramme von Wasser und Kohlendioxid. Die unterschiedlichen Steigungen resultieren aus den Dichteunterschieden der Phasen.}
    \label{fig: phasendiagramme_vergleich_schmelz}
\end{figure}

\subsubsection{Normales Verhalten (z.B. CO\textsubscript{2})}
Bei den meisten Stoffen nimmt das Volumen beim Schmelzen zu ($v_{Fl} > v_{Fest}$), da die Teilchen in der Flüssigkeit weniger dicht gepackt sind als im Kristallgitter.
\begin{itemize}
    \item Daraus folgt: $\Delta v > 0 \implies \frac{\dd p}{\dd T} > 0$.
    \item Die Schmelzkurve ist nach \textbf{rechts} geneigt (siehe \cref{fig: phasendiagramm_schmelzkurve_co2}).
    \item Eine Druckerhöhung führt zu einer Erhöhung der Schmelztemperatur (der feste Zustand wird stabilisiert, da er das kleinere Volumen hat).
\end{itemize}

\subsubsection{Anomalie des Wassers}
Wasser zeigt ein besonderes Verhalten. Aufgrund der offenen Struktur des Eisgitters (Wasserstoffbrückenbindungen bilden Hohlräume) ist die Dichte von Eis geringer als die von flüssigem Wasser ($\rho_{Eis} < \rho_{Wasser}$). Das Volumen nimmt beim Schmelzen also ab ($v_{Fl} < v_{Fest}$).
\begin{itemize}
    \item Daraus folgt: $\Delta v < 0 \implies \frac{\dd p}{\dd T} < 0$.
    \item Die Schmelzkurve ist nach \textbf{links} geneigt (siehe \cref{fig: phasendiagramm_schmelzkurve_wasser}).
\end{itemize}

\begin{rememberbox}{Druckaufschmelzen}
    Aufgrund der negativen Steigung der Schmelzkurve sinkt der Schmelzpunkt von Eis bei Druckerhöhung. Dies ist die \textbf{Anomalie des Wassers}.
    
    Ein praktisches Beispiel ist das Verhalten von Gletschern: An der Unterseite von Gletschern kann der immense Druck des Eises dazu führen, dass das Eis schmilzt, obwohl die Temperatur leicht unter \SI{0}{\celsius} liegt. Dieser Wasserfilm erleichtert das Gleiten des Gletschers.
\end{rememberbox}

Da die Volumenänderung beim Schmelzen ($v_{Fl} - v_{Fest}$) wesentlich kleiner ist als beim Verdampfen ($v_{D} - v_{Fl}$), ist der Nenner in \cref{eq: clausius_clapeyron_schmelzen} sehr klein. Dies führt dazu, dass die Schmelzkurve im $(p,T)$-Diagramm extrem steil verläuft (viel steiler als die Dampfdruckkurve).

Die Schmelzkurve $p_{fs}(T)$ und die Dampfdruckkurve $p_S(T)$ schneiden sich im \textbf{Tripelpunkt} $(p_{Tr}, T_{Tr})$. Dies ist der einzige Punkt, an dem alle drei Phasen -- fest, flüssig und gasförmig -- gleichzeitig im thermodynamischen Gleichgewicht existieren können.

% Chapter end - always start new page after chapter
\newpage