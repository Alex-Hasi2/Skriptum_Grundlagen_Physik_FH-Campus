\chapter{Messgenauigkeit und Messfehler}\label{chap: MessgenauigkeitMessfehler}
\section{Signifikante Stellen}\label{sec: signifikante_Stellen}

\begin{rememberbox}[]{}
    Signifikante Stellen geben die Genauigkeit einer Zahl an, die aus einer Messung oder Berechnung hervorgeht. Sie zeigen an, welche Ziffern tatsächlich zur Genauigkeit eines Wertes beitragen und zudem nicht nur Platzhalter sind. Die Berücksichtigung signifikanter Stellen ist wichtig, um die Unsicherheit in Messungen und Berechnungen korrekt darzustellen.
\end{rememberbox}
Führende Nullen sind generell nicht aussagekräftig. Nullen am Ende einer Zahl können signifikant sein. 
Alle Zahlen hinter der letzten signifikanten Stelle sind bedeutungslos. Die letzte signifikante Stelle ist mit einer Unsicherheit behaftet und nur eine Schätzung innerhalb der gegebenen Präzision. 
Es kann sowohl zu einer falschen Präzisionsangabe führen, wenn man zu viele (nicht signifikante) Stellen mitnimmt, als auch zu überhaupt falschen Zahlenwerten. 
Für exakte Werte wendet man das Prinzip der signifikanten Stellen nicht an.
Signifikante Stellen sind nicht dasselbe wie Nachkommastellen.

\begin{center}
\begin{NiceTabular}{l c c}
\CodeBefore
  \rowcolor{boxcol_title_navyblue}{1}
\Body
\toprule
\color{white}\textbf{Zahl} & \color{white}\textbf{Signifikante Stellen} & \color{white}\textbf{Nachkommastellen} \\
\midrule
\num{15.23} & 4 & 2 \\
\num{0.1523} & 4 & 4 \\
\num{1.523e3} & 4 & 3 \\
\num{4500} & 2 & 0 \\
\num{4500.0} & 5 & 1 \\
\num{4.5e3} & 2 & 1 \\
\num{4.50e2} & 3 & 2 \\
\num{4.500e3} & 4 & 3 \\
\bottomrule
\end{NiceTabular}
\end{center}
\vspace{0.3cm}
Die wissenschaftliche Schreibweise als 
\begin{equation}
    \mathrm{Zahl} = \mathrm{Mantisse} \cdot 10^{\mathrm{Exponent}}
\end{equation}
erlaubt eine einfache Angabe der signifikanten Stellen, da auch endende Nullen sofort als signifikant gekennzeichnet sind. 

Vorsicht ist geboten beim Wechsel von Einheiten oder Zehnerpotenzen (die Genauigkeit darf dadurch nicht verändert werden):
\begin{equation}
    \SI{3.2}{\meter} = \SI{320}{\centi\meter}\mDot
\end{equation}
Die Umrechnung stimmt und die Meterangabe hat definitiv 2 signifikante Stellen. Die Zentimeterangabe könnte allerdings 2 oder 3 signifikante Stellen haben. Besser wäre es also $\SI{3.2e2}{\centi\meter}$ zu schreiben.

Eine Zahl in nicht wissenschaftlichem Format wird meist so interpretiert, dass die letzte Stelle gerundet ist. Die Zahl $50$ kann also zwischen $\num{49.5}$ und $\num{50.4}$ liegen. Die signifikanten Stellen enden an der letzten Stelle, die nach dem Runden noch angegeben werden kann.

\section{Messgenauigkeit und Messfehler}\label{sec: Messgenauigkeit_Messfehler}
\textbf{Jede Messung ist fehlerbehaftet}. Der Messwert muss daher mit einer Fehlerangabe versehen werden, um seine Genauigkeit feststellen zu können. 
Man unterscheidet grundlegend zwischen systematischen Fehlern und statistischen (zufälligen) Fehlern. 
\subsection{Systematische Fehler}\label{subsec: systematische_Fehler}
\begin{rememberbox}[]{}
    Ein \textbf{systematischer Fehler} ergibt eine fehlerhafte Abweichung des Messwertes vom wahren Wert, die immer gleich ist, wenn die Messung unter den gleichen Bedingungen wiederholt wird. 
\end{rememberbox}
Systematische Fehler ergeben sich durch ein falsches Messverfahren, durch die Messapparatur oder durch die Nichtberücksichtigung von äußeren Einflüssen. 
Häufig werden systematische Fehler unterschätzt!

\textit{Beispiel:} Messungen der Elektronenmasse \newline
Man sieht in \cref{fig: elektronenmasse_ungenauigkeit}, dass die Fehlerbalken allesamt den heutigen Bestwert nicht inkludieren und damit die systematischen Fehler unterschätzt werden.
 \begin{figure}
     \centering
     \includegraphics[width=0.7\linewidth]{Bilder/Kapitel_Grundlagen/Elektronenmasse_Messungenauigkeit.png}
     \caption{Historische Messwerte für die Elektronenmasse in Einheiten von $\SI{e-31}{\kilo\gram}$. Dargestellt sind die relativen Abweichungen $\Delta m/m$ vom heutigen Bestwert.}
     \label{fig: elektronenmasse_ungenauigkeit}
 \end{figure}

 \begin{examplebox}[sidebyside, sidebyside align=center, lower separated=false, righthand width=5.5cm]{Beispiel}
     \textbf{Raummaße messen mit Maßband:}
     \begin{itemize}[itemsep=1.5pt]
        \item Maßband hat einen systematischen Fehler (Kalibrierung)
        \item Schiefe Messung
        \item Parallaxenfehler (Ablesefehler)
        \item Nullpunktsverschiebung (Metallende könnte verbogen sein)
        \item Dehnung des Maßbands unter Belastung 
        \item Temperaturbedingte Ausdehnung (sehr heißer oder kalter Raum)
     \end{itemize} 
     \tcblower
     \begin{center}
     \includegraphics[width=0.9\linewidth]{Bilder/Kapitel_Grundlagen/massband_raum.png}
     \end{center}
 \end{examplebox}

\subsection{Statistische Fehler}\label{subsec: Statistische_Fehler}
Wiederholte Messungen ergeben im Allgemeinen (abseits von systematischen Fehlern) trotzdem nicht den gleichen Wert. Das kann an einer ungenauen Ablesung liegen, an Vibrationen des Messinstruments oder auch an intrinsischen Schwankungen der Messgröße selbst. 

\begin{rememberbox}[]{}
    Ein rein \textbf{statistischer Fehler} ergibt bei wiederholten Messungen eine Verteilung der Messwerte $(x_i),$ um einen Mittelwert $(\bar{x}$).
    Bei Abwesenheit von systematischen Fehlern nähert sich der Mittelwert dem wahren Wert $(x_\mathrm{W})$ für unendlich viele Messungen an. 

\end{rememberbox}
Da man nicht unendlich viele Messungen durchführen kann, bleibt der wahre Wert im Allgemeinen unbekannt. Die Breite der Messwerte-Verteilung ist ein Maß für die Güte der Messung. Das arithmetische Mittel aller Messungen, 
\begin{equation}
    \bar{x} = \frac{1}{n} \sum_{i=1}^{n} x_i \mComma
\end{equation}
nähert sich dem wahren Wert bei unendlich vielen Messungen 
\begin{equation}
    x_\mathrm{W} = \lim_{n\rightarrow \infty} \frac{1}{n} \sum_{i=1}^n x_i \mDot
\end{equation}
\begin{figure}
    \centering
    \includegraphics[width=0.5\linewidth]{Bilder/Kapitel_Grundlagen/stat_Verteilung_messwerte.png}
    \caption{Typische Verteilung von Messwerten $x_i$ um den Mittelwert $\bar{x}$ bei statistischer Fehlerverteilung.}
    \label{fig: stat_Verteilung_Messwerte}
\end{figure}
Liegen nur statistische Fehler vor, erhält man für die Verteilung der Messwerte eine Normalverteilung bei einer hohen Anzahl an Messwerten.\footnote{Vorausgesetzt die Messgröße ist kontinuierlich und die Schwankung auch.}

\begin{importantbox}[]{}
    Das \textbf{arithmetische Mittel der Einzelmessungen}
    \begin{equation}
        \bar{x} = \frac{1}{n} \sum_{i=1}^n x_i \mDot
    \end{equation}
    Die \textbf{Standardabweichung der Einzelmessungen}
    \begin{equation}
        \sigma = \sqrt{\frac{\sum \left( \bar{x} - x_i \right)^2}{n-1}} \mDot
    \end{equation}
    Mittlere Fehler des arithmetischen Mittels
    \begin{equation}
        \sigma_m = \frac{\sigma}{\sqrt{n}} = \sqrt{\frac{\sum \left(\bar{x} - x_i \right)^2}{n(n-1}} \mDot
    \end{equation}
\end{importantbox}
Bei sehr vielen Messungen nähert sich der Mittelwert dem wahren Wert
\begin{equation}
    x_\mathrm{W} = \lim_{n \rightarrow \infty} \frac{1}{n} \sum_{i=1}^n x_i
\end{equation}
Die Definition der Standardabweichung bedingt, dass bei sogenannten Normalverteilungen 
\begin{itemize}[itemsep=1.5pt]
    \item $68,3 \%$ der Werte innerhalb der \\einfachen Standardabweichung $[\bar{x} - \sigma,\, \bar{x}+\sigma]$,
    \item $95,4\%$ der Werte innerhalb der \\zweifachen Standardabweichung $[\bar{x} - 2\sigma,\, \bar{x} + 2\sigma]$, und
    \item $99,7\%$ der Werte liegen innerhalb der \\dreifachen Standardabweichung $[\bar{x} - 3\sigma,\, \bar{x} + 3\sigma]$
\end{itemize}
liegen.

\begin{examplebox}[sidebyside, sidebyside align=top, lower separated=false, righthand width=5.0cm]{Beispiel}
    Schwingungsdauer\footnotemark{} eines Fadenpendels: \newline
    Man misst 
    Daraus ergibt sich ein Mittelwert von $\SI{2.00}{\second}$ und eine Standardabweichung von $\sigma = \SI{0.05}{\second}$: 
    \begin{equation*}
        T_W = \SI{2.00 +- 0.05}{\second}
    \end{equation*}
    \tcblower
    \begin{center}
    \begin{tabular}{C{1.6cm} | C{1.6cm}}
    \hline
    % \rowcolor{boxcol_title_navyblue}
    % \multicolumn{2}{c}{\textbf{\color{white}Messwerte [s]}} \\
    \multicolumn{2}{c}{\textbf{Messwerte [s]}} \\
    \hline \num{2.08} & \num{2.02} \\
    \hline \num{2.06} & \num{1.98} \\
    \hline \num{1.96} & \num{2.00} \\
    \hline \num{2.02} & \num{1.94} \\
    \hline \num{1.98} & \num{1.96} \\
    \hline
    \end{tabular}
    \end{center}
\end{examplebox}
\footnotetext{Die Schwingungsdauer ist die Länge des Zeitabschnitts, nach der sich die Bewegung wiederholt (nicht nur derselbe Punkt erreicht wird).}

% Chapter end - always start new page after chapter
\newpage