\documentclass[ngerman]{scrreprt}
\usepackage[utf8]{inputenc}
\usepackage[ngerman]{babel}
\usepackage[colorlinks=true,citecolor=blue,linkcolor=blue,urlcolor=blue,linktoc=page]{hyperref}
\usepackage{lmodern} % Sorgt für eine schönere Schrift
\usepackage[dvipsnames, table]{xcolor}
\usepackage[most]{tcolorbox}
\tcbuselibrary{fitting}
\tcbuselibrary{raster}

\usepackage{comment} % to comment out large sections
\usepackage{needspace} % to calculate the needed space for the box before moving it to next page

\usepackage{colortbl}
\usepackage{enumitem} % for lists
\usepackage[framemethod=tikz]{mdframed}
\usepackage{cancel}

\usepackage{caption}
\captionsetup{
  format=plain, % Kein Einzug
  labelfont=it, % Kursiv für "Abbildung X"
  textfont=normal, % Normaler Text für die restliche Caption
}
\usepackage{multirow}

\usepackage{tikz}
\usetikzlibrary{positioning,arrows.meta, angles, calc, quotes}
\usepackage{pgfplots}
\pgfplotsset{compat=1.18}


\usepackage[table]{xcolor}
\usepackage{array}
\usepackage{booktabs}
\usepackage{tabularx}
\usepackage{nicematrix} % nicematrix statt tabularx laden

\usepackage{multicol}
\usepackage{makecell} 
\renewcommand\theadfont{\bfseries} 

\usepackage{graphicx}
\usepackage{epstopdf}

\usepackage{eurosym}  % euro symbol € 
\usepackage{amsmath,amssymb,latexsym,amsfonts,subfigure,bbm, bm, bbold}
\usepackage{mathtools}
\usepackage[b]{esvect} % For \vv (option [d] is a nice default)

% \usepackage{unicode-math} % clashes with si-unit x
\usepackage{dsfont,relsize}

\usepackage{siunitx}
\DeclareSIUnit{\litre}{l}
\DeclareSIUnit{\calorie}{cal}
\DeclareSIUnit{\bar}{bar}
\DeclareSIUnit{\atm}{atm}
\sisetup{
    locale = DE, 
    angle-symbol-degree = ^\circ,
    separate-uncertainty, 
    inter-unit-product = \,,
    range-units = single, 
    list-units = single, 
    per-mode=symbol-or-fraction,
    range-phrase = { bis }
}

% colors for the cref package
\definecolor{myblue}{RGB}{51,51,255}
\definecolor{myRefcolor}{RGB}{51,51,200}
\definecolor{myCiteColor}{RGB}{33,140,33}

\usepackage[nameinlink, noabbrev]{cleveref} % noabbrev für ausgeschriebene Referenzen

% --- NEUE BOXEN FÜR AUFGABEN UND LÖSUNGEN ---
\definecolor{boxcol_back_lgray}{RGB}{245, 245, 245}
\definecolor{boxcol_frame_dgray}{RGB}{180, 180, 180}
\definecolor{boxcol_title_gray}{RGB}{200, 210, 210}

\newtcbtheorem[auto counter, number within=chapter,crefname={Aufgabe}{Aufgaben}]{aufgabebox}{Aufgabe}{
    enhanced,
    colback=white,
    colframe=boxcol_frame_dgray,
    coltitle=black,
    colbacktitle=boxcol_title_gray,
    fonttitle=\bfseries,
    attach boxed title to top left={xshift=8mm, yshift=-3mm},
    boxed title style={
        colframe=boxcol_frame_dgray,
        arc=2mm,
    },
    breakable,
    pad at break=6mm,
    enlarge left by=-10pt,
    enlarge right by=-10pt,
    top=6mm,  
    bottom=5mm,
    before skip balanced=0.7cm,
    after skip balanced=0.7cm,
}{aufg}

% Neue Box für Lösungen
\definecolor{boxcolbacklgreen}{RGB}{232, 245, 233}
\definecolor{boxcolframedgreen}{RGB}{129, 199, 132}
\newmdenv[
    linecolor=boxcolframedgreen,
    backgroundcolor=white,
    linewidth=1.5pt,
    frametitlerule=true,
    frametitlebackgroundcolor=boxcolframedgreen,
    frametitlefont=\bfseries,
    skipabove=0.7cm,
    skipbelow=0.7cm,
    leftmargin=-10pt,
    rightmargin=-10pt,
    innerleftmargin=20pt,
    innerrightmargin=20pt,
    innertopmargin=16pt,
    innerbottommargin=16pt,
    roundcorner=5pt,
]{loesungboxstyle}

\newenvironment{loesungbox}[1]
  {\needspace{5\baselineskip}\begin{loesungboxstyle}[frametitle={#1}]}
  {\end{loesungboxstyle}}


% Stil für die Boxen der Begriffe
\newtcolorbox{termbox}{
    colback=RoyalBlue!80!white,   % A pleasant medium blue background
    coltext=white,                % White text color
    colframe=RoyalBlue!60!black,  % A slightly darker border
    fontupper=\bfseries,
    arc=5pt,
    halign=center,
    valign=center,
    fontupper=\bfseries,
    height=1.5cm, 
    left=4pt,
    right=4pt,
    top=4pt,
    bottom=4pt
    }

% Stil für die Boxen der Beispiele
\newtcolorbox{examplebox}[1]{
    colback=boxcol_back_lgray,
    colframe=boxcol_frame_dgray, % Verwendet deine vordefinierte Farbe
    boxrule=1pt,
    title=\textbf{Beispiel #1},
    colbacktitle=white, % Macht den Titel-Hintergrund unsichtbar
    coltitle=black,
    fonttitle=\small\bfseries,
}

\newtcolorbox{solutiontermbox}[1]{
    colback=white,
    colframe=boxcol_frame_dgray,
    boxrule=1pt,
    title=\textbf{#1}, % The title is now the term itself (e.g., "Hypothese")
    colbacklower=black!5!white, % A light gray background for the lower part
    fontlower=\small\itshape, % A smaller, italic font for the definition
}

% ----- own definitions

% Needed for the bibliography style (.bst files)
\providecommand{\Verfuegbar}{Verf{\"u}gbar}


%defs for the paper
\newcommand{\mDot}{\,.}
\newcommand{\mComma}{\,{,}\,}

% text subscripts
\newcommand{\kin}{\mathrm{kin}}
\newcommand{\pot}{\mathrm{pot}}
\newcommand{\rot}{\mathrm{rot}}
\newcommand{\trans}{\mathrm{trans}}
\newcommand{\atm}{\mathrm{atm}}
\newcommand{\minText}{\mathrm{min}}
\newcommand{\maxText}{\mathrm{max}}
% quantities with text subscripts
\newcommand{\Ekin}{E_{\kin}}
\newcommand{\Epot}{E_{\pot}}
\newcommand{\Erot}{E_{\rot}}
\newcommand{\Etrans}{E_{\trans}}
\newcommand{\kB}{k_{\mathrm{B}}}

% Text 
\newcommand{\Schro}{Schr\"o\-din\-ger }

% vectors 
\newcommand{\ivec}[1]{\vv{#1}} % using esvect package
\newcommand{\ivecS}[2]{\vv*{#1}{\!#2}} % using esvect package
% column vector
\newcommand{\icolTwo}[2]{\begin{pmatrix} #1 \\ #2 \end{pmatrix}}
\newcommand{\icolThree}[3]{\begin{pmatrix} #1 \\ #2 \\ #3 \end{pmatrix}}
% row vector (INLINE)
\newcommand{\inlrowTwo}[2]{(#1, #2)}
\newcommand{\inlrowThree}[3]{(#1, #2, #3)}

% point 
\newcommand{\ipTwo}[2]{(#1\!\mid\! #2)}
\newcommand{\ipThree}[3]{(#1\!\mid\! #2\!\mid\! #3)}

% rangle, langle 
\newcommand{\lrangle}[1]{{\langle{#1}\rangle}}
% measurement units
\newcommand{\Unit}[1]{\,\mathrm{#1}}

\newcommand{\msSp}{\;}
\newcommand{\mdSp}{\;\;}
\newcommand{\mtSp}{\;\;\;}
\newcommand{\mqSp}{\;\;\;\;}

  
%mathematical symbols
\newcommand{\defeq}{\vcentcolon=}
\newcommand{\eqdef}{=\vcentcolon}
\newcommand*\conj[1]{\bar{#1}}
\newcommand{\eqexcl}{\stackrel{!}{=}}
\newcommand{\eqquestion}{\stackrel{?}{=}}
\newcommand\equalhatInl{\mathrel{\stackon[1.0pt]{=}{\stretchto{%
    \scalerel*[\widthof{=}]{\wedge}{\rule{1ex}{3ex}}}{0.45ex}}}}
\newcommand\equalhat{\mathrel{\stackon[4.8pt]{=}{\stretchto{%
    \scalerel*[\widthof{=}]{\wedge}{\rule{1ex}{3ex}}}{0.45ex}}}}
\newcommand{\mAND}{\land}
\newcommand{\mOR}{\lor}
\newcommand{\mNOT}{\lnot}

% text editing
\newcommand{\textunderscript}[1]{$_{\text{#1}}$}
\newcommand{\textupperscript}[1]{$^{\text{#1}}$}
\newcommand{\eqqref}[1]{eq.\!~(\ref{#1})}
\newcommand{\Eqqref}[1]{Eq.\!~(\ref{#1})}
\newcommand{\figref}[1]{fig.\!~(\ref{#1})}
\newcommand{\Figref}[1]{Fig.\!~(\ref{#1})}
\newcommand{\secref}[1]{sec.\!~(\ref{#1})}
\newcommand{\Secref}[1]{Sec.\!~(\ref{#1})}

%general abbreviations (in German)
\newcommand{\wA}{\mbox{w.\,A.\ }}
\newcommand{\fA}{\mbox{f.\,A.\ }}
\newcommand{\zB}{\mbox{z.\,B.\ }}
\newcommand{\bzw}{\mbox{bzw.\ }}
\newcommand{\gDh}{\mbox{d.\,h.\ }}
\newcommand{\gDQ}[1]{\glqq #1\grqq}
\newcommand{\oBdA}{\mbox{o.\,B.\,d.\,A.\ }}
\newcommand{\sEUR}{\text{\euro}}

%latin abbreviations
\newcommand{\etal}{\mbox{\emph{et al.\ }}}
\newcommand{\exgrat}{\mbox{e.g.\ }}
\newcommand{\idest}{\mbox{i.e.\ }}

%general math terms
\newcommand{\const}{\mathrm{const}}
\newcommand{\bigO}{\mathcal{O}}

% Lorem ipsum
\newcommand*{\QEDA}{\hfill\ensuremath{\blacksquare}}%
\newcommand*{\QEDB}{\hfill\ensuremath{\square}}%

%  ------------------ abbreviations

%matrix operations
\newcommand{\T}{T}
\DeclareMathOperator{\arcsinh}{arcsinh}
\DeclareMathOperator{\Tr}{Tr}
\DeclareMathOperator{\argg}{arg}
\DeclareMathOperator{\Arg}{arg}
\DeclareMathOperator{\codim}{codim}
\DeclareMathOperator{\atanTwo}{atan2}
\DeclareMathOperator{\diag}{diag}

%real and complex numbers latin Letters
\newcommand{\Real}{\mathbb{R}}
\newcommand{\Complex}{\mathbb{C}}
\newcommand{\Integer}{\mathbb{N}}

% differentials 
\newcommand{\dd}{\mathrm{d}}




\title{Übungsaufgaben: Grundlagen der Physik}
\author{Alexander Schumer}
\date{\today}



\begin{document}

    % Title
    {\Huge \bfseries Übungsbeispiele\par}
    \vspace{0.2cm}

    {\Large \bfseries Physikalische Grundlagen\par} 
    \vspace{1cm}
    % University Name
    {\large FH Campus Wien\par}

    % Subtitle
    {\Large HKLS ILV | SS2025\par}

    \vfill % Fills the space in the middle of the page

    % Author
    {\large Dr. Dipl.-Ing. Alexander Schumer\par}
    \vspace{1cm}

    % Date
    {\large August 2025\par}

\tableofcontents

%\begin{comment}

\chapter{Mathematische Grundlagen}\label{chap:mathe_grundlagen}

\section{Aufgaben}\label{sec:mathe_grundlagen_aufgaben}


\begin{aufgabebox}{Ableitungen}{ableitungen}
Bilden Sie die erste Ableitung folgender Funktionen:
\begin{enumerate}
    \item $f_{1}(x)=3x^{4}-8x^{2}+x+9$
    \item $f_{2}(x)=\sin(7x^{3})$
    \item $f_{3}(x)=A\cdot e^{\lambda\cdot x}$
    \item $f_{4}(x)=\cos(e^{4x^{2}})$, \\
    Tipp: $f_{4}(x)=g(h(k(x)))$ mit $g(x)=\cos(x)$, $h(x)=e^{x}$, $k(x)=4x^{2}$
    \item $f_{5}(x)=\frac{1}{4x^{5}}$
    \item $f_{6}(x)=\sqrt{x}$
\end{enumerate}
\end{aufgabebox}


\begin{aufgabebox}{Integralrechnung}{integral_fx_OS_US}
Betrachten Sie die Funktion
\begin{equation}\label{eq:integral_func}
    f(x)=\frac{1}{2}x^{3}-3x^{2}+\frac{3}{2}x+5 \mComma
\end{equation}
\begin{center}
    \begin{tikzpicture}
        \begin{axis}[
            % Achsen und Beschriftungen
            axis lines=middle, % Achsen kreuzen sich im Ursprung (0,0)
            axis line style = thick,
            xlabel={\Large $x$},      % Beschriftung der x-Achse
            ylabel={\Large $f(x)$},    % Beschriftung der y-Achse
            % Achsen-Grenzen
            xmin=-1.5, xmax=5.5,
            ymin=-6.8, ymax=6.8,
            % Ticks und Gitter
            xtick={-1,0,1,2,3,4,5},
            ytick={-6,-4,-2,0,2,4,6},
            grid=both, % Haupt- und Untergitter anzeigen
            minor tick num=1, % Eine Gitterlinie zwischen Hauptlinien
            % Stil des Plots
            width=12cm, % Breite der Grafik
            height=8cm, % Höhe der Grafik
            ticklabel style={font=\normalsize}, % Schriftgröße der Tick-Zahlen
            label style={font=\normalsize}, % Schriftgröße der Achsenbeschreibungen
            xlabel style={at={(ticklabel* cs:1.01)}, anchor=west},
            ylabel style={at={(ticklabel* cs:0.94)}, anchor=south west},
            grid style={line width=.1pt, draw=gray!20} % Stil für das Gitter
        ]
            % Der eigentliche Plot-Befehl
            \addplot[
                domain=-1.2:5.2, % Definitionsbereich, für den gezeichnet wird
                samples=200,     % Anzahl der Punkte (für eine glatte Kurve)
                color=blue,      % Farbe der Kurve
                line width=1.6pt % Linienstärke
            ]
            % --- KORRIGIERTE FUNKTION ---
            {0.5*x*x*x - 3*x*x + 1.5*x + 5}; 
        \end{axis}
    \end{tikzpicture}
    %\captionof{figure}{Graph der Funktion $f(x)$ aus \Cref{eq:integral_func}.}
    \label{fig:integral_func_aufg}
\end{center}
die in der Abbildung dargestellt ist.
\begin{enumerate}
    \item Schätzen Sie die Flächen unter der Kurve in den Intervallen $x\in[-1,2]$ und $x\in[2,5]$ ab, und zwar mittels
    \begin{enumerate}
        \item der Obersumme,
        \item der Untersumme,
    \end{enumerate}
    jeweils mit der Schrittweite $\Delta x = \num{0.5}$. Skizzieren Sie die entsprechenden Rechtecke.
    \item Berechnen Sie das unbestimmte Integral der Funktion $f(x)$, die sogenannte Stammfunktion $F(x)$. Beachten Sie die Integrationskonstante C.
    \item Verwenden Sie den ersten Hauptsatz der Integralrechnung, um nun die exakte Fläche in den Intervallen $[-1,2]$ und $[2, 5]$ separat zu berechnen. Setzen Sie das Resultat in Relation zu den Schätzungen aus 1).
    \item Was ergibt das Integral über $f(x)$ von $x = -1$ bis $x = 5$?
\end{enumerate}
\end{aufgabebox}


\begin{aufgabebox}{Kreisgleichung}{kreisgleichung}
\begin{enumerate}
    \item Zeigen Sie, dass die folgende Gleichung einen Kreis beschreibt. Ermitteln Sie dazu Mittelpunkt und Radius des Kreises:
    \begin{equation}
        x^{2}+y^{2}+4x-6y+9=0 \mDot
    \end{equation}
    \item Liegt der Punkt $Q(-5|\,-5)$ auf dem Kreis? Wie sieht es mit $S(-4|\,3)$ aus?
    \item Berechnen Sie je einen
    \begin{itemize}
        \item Punkt $I$, der innerhalb des Kreises liegt,
        \item Punkt $O$, der außerhalb des Kreises liegt,
        \item Punkt $P$, der auf dem Kreis liegt.
    \end{itemize}
\end{enumerate}
\end{aufgabebox}


\begin{aufgabebox}{Vektorrechnung}{vektorrechnung}
Betrachten Sie die Punkte $A\ipTwo{0}{0}$, $B\ipTwo{1}{2}$, $C\ipTwo{4}{2}$ und $D\ipTwo{3}{0}$ in der $(x,y)$-Ebene.
\begin{center}
    \begin{tikzpicture}
        \begin{axis}[
            xlabel={$x$},
            ylabel={$y$},
            xmin=-2, xmax=5,
            ymin=-1, ymax=3,
            xtick={-2,-1,0,1,2,3,4,5},
            ytick={-1,0,1,2,3},
            axis lines=box,
            axis on top,
            grid=both,
            grid style={line width=.1pt, draw=gray!20},
            tick label style={font=\normalsize},
            every axis x label/.append style={font=\Large, at={(ticklabel* cs:0.5)},anchor=north,yshift=-0.6cm},
            every axis y label/.append style={font=\Large, at={(ticklabel* cs:0.5)},anchor=south,yshift=0.6cm},
        ]
        % 1. Nur die Punkte zeichnen
        \addplot[only marks, mark=*, mark options={fill=red}, mark size=2pt] coordinates {
            (0,0)
            (1,2)
            (4,2)
            (3,0)
        };
        % 2. Die Beschriftungen als separate TikZ-Nodes hinzufügen
        %    'axis cs:' stellt sicher, dass die Koordinaten des Achsensystems verwendet werden.
        \node[above left, xshift=-2pt] at (axis cs:0,0) {$A$};
        \node[above left, xshift=-2pt] at (axis cs:1,2) {$B$};
        \node[above right, xshift=2pt] at (axis cs:4,2) {$C$};
        \node[below right, xshift=2pt] at (axis cs:3,0) {$D$};
        \end{axis}
    \end{tikzpicture}
\end{center}

\begin{enumerate}
    \item Was sind die beiden definierenden Eigenschaften eines Vektors?
    \item Wie lautet der Vektor $\ivec{AB}$ und wie lautet der Vektor $\ivec{DC}$? \newline Begründen Sie das Ergebnis!
    \item Welche Länge hat der Vektor $\ivec{p}=\ivec{AC}$? \newline Normieren Sie diesen Vektor!
    \item Welche Länge hat der Vektor $\ivec{q}=\ivec{BD}$? \newline Normieren Sie diesen Vektor!
    \item Welchen Winkel schließen $\ivec{p}$ und $\ivec{q}$ miteinander ein? \newline
    \textit{Tipp: Inneres Produkt}
    \item Zeigen Sie, dass $\ivec{AB}+\ivec{BC}=\ivec{AC}$. \newline 
    Begründen Sie das Ergebnis!
    \item Was ist der Flächeninhalt des Parallelogramms $(ABCD)$? \newline
    \textit{Tipp: Erweiterung in 3D und Äußeres Produkt.}
\end{enumerate}
\end{aufgabebox}




\begin{aufgabebox}{Häufigkeitsverteilung und Wahrscheinlichkeitsdichte}{haeufigkeitsverteilung}
Bei einem Test mit $N=2000$ Studierenden, bei dem es 10 Punkte zu erreichen gab, hat ein Kollege bereits die Tabelle mit der Verteilung der erreichten Punktezahlen erstellt.

\begin{center}
\begin{tabular}{ccc}
\toprule
\textbf{Punkte} & \textbf{Anzahl Studierende} & \textbf{Relative Häufigkeit} \\
\textbf{(x)} & \textbf{($N_x$)} & \textbf{($n_x$)} \\
\midrule
0 & 14 & \\
1 & 102 & \\
2 & 159 & \\
3 & 186 & \\
4 & 216 & \\
5 & 415 & \\
6 & 374 & \\
7 & 271 & \\
8 & 160 & \\
9 & 75 & \\
10 & 28 & \\
\bottomrule
\end{tabular}
\end{center}

\begin{enumerate}
    \item Berechnen Sie die durchschnittliche Punktezahl mittels
    \begin{equation}\label{eq:mittelwert_diskret}
        \overline{x}=\frac{1}{N}\sum_{i=0}^{10}x_{i}\cdot N_{i} \mComma
    \end{equation}
    wobei $N_{i}$ die absolute Häufigkeit der Punktezahl $x_{i}$ ist.
    
    \item Berechnen Sie die relativen Häufigkeiten $n_{i}=N_{i}/N$ für alle Punktezahlen!
    
    \item Verändern Sie die Formel für $\overline{x}$ in \cref{eq:mittelwert_diskret}, sodass Sie die relativen Häufigkeiten $n_{i}$ verwenden können.
    
    \item Nehmen Sie nun an, es wären $N=\infty$ Studierende und die Punkte zwischen 0 und 10 seien kontinuierlich. Sie kennen die Wahrscheinlichkeitsdichte $f(x)$ der Punktezahl $x$
    \begin{equation}\label{eq:wahrscheinlichkeitsdichte_punkte}
        f(x)=\frac{1}{6}\sin^2\left(\frac{x}{4}\right) \mComma
    \end{equation}
    die in der Abbildung dargestellt ist. Für Wahrscheinlichkeitsdichten gilt 
    $$\int_{0}^{10}f(x)\dd x=1 \mDot$$
    % \begin{center}
    %     \includegraphics[width=0.6\textwidth]{Bilder/Uebungsaufgaben/wahrscheinlichkeitsverteilung_dx.png}
    % \end{center}

    \begin{center}
        \begin{tikzpicture}
            \begin{axis}[
                axis lines=left,
                axis line style = thick,
                xlabel={\Large $x$},
                ylabel={\Large $f(x)$},
                xmin=0, xmax=10.8,
                ymin=0, ymax=0.19,
                xtick={0,2,4,6,8,10},
                ytick={0,0.05,0.10,0.15},
                yticklabel style={/pgf/number format/fixed},
                enlargelimits=false,
                clip=false,
                samples=200, % Erhöht die Glätte der Kurve
                domain=0:10,
                axis equal image=false,
            ]
            
            \def\myfunction{(1/6)*(sin(deg(x/4)))^2}
            \addplot[blue, line width=1.5pt] {\myfunction};
            
            % Wert der Funktion bei x=4
            \pgfmathsetmacro{\funcvalue}{(1/6)*(sin(deg(4/4)))^2}
            \addplot[gray, dashed, thin] coordinates {(0, \funcvalue) (10, \funcvalue)};
        
            \addplot[draw=black, fill=violet!70!black, ybar, bar width=0.5] coordinates {(4, \funcvalue)};
        
            % Beschriftung für dx
            \draw[-{Stealth}, line width=1.0pt] (3.0, 0.13) -- (3.75, 0.13);
            \draw[{Stealth}-, , line width=1.0pt] (4.25, 0.13) -- (5.00, 0.13);
            \draw[line width=1.0pt] (3.75, 0.12) -- (3.75, 0.14);
            \draw[line width=1.0pt] (4.25, 0.12) -- (4.25, 0.14);
            \node[] (center) at (4,0.155) {\Large $dx$};
            % Beschriftung für n(x)
            \draw[{Stealth}-{Stealth}, line width=1.0pt] (4.65, 0) -- (4.65, \funcvalue) node[midway, left, rotate=90, anchor=center, yshift=-3.5mm] {\Large $n(x)$};
        
            \end{axis}
        \end{tikzpicture}
    \end{center}
    Erklären Sie, warum die mittlere Punktezahl im kontinuierlichen Fall durch das Integral
    \begin{equation}\label{eq:mittelwert_kontinuierlich}
        \overline{x}=\int_{0}^{10}x\cdot f(x)\cdot \dd x
    \end{equation}
    berechnet wird. Berechnen Sie die mittlere Punktezahl der Studierenden! \\
    \textbf{Hinweis:} $\int x \cdot A \cdot \sin^2\left(\frac{x}{\sigma}\right)\dd x = A\left(\frac{x^2}{4}-\frac{\sigma^2}{8}\cos(\frac{2x}{\sigma})-\frac{x\sigma}{4}\sin(\frac{2x}{\sigma})\right) + C$.
\end{enumerate}
\textbf{Einschub:} In einer kontinuierlichen Wahrscheinlichkeitsverteilung wie $f(x)$ ist $f(x)$ keine Wahrscheinlichkeit für die Punktezahl $x$, sondern eine Wahrscheinlichkeitsdichte. Das bedeutet: 
$f(x)$ gibt an, wie dicht die Wahrscheinlichkeiten um $x$ herum liegen. \\
Um tatsächliche Wahrscheinlichkeiten zu bekommen, dass die Punktezahl in einem kleinen Intervall [$x$, $x + \dd x$] liegt, muss man die Dichte mit der Breite des Intervalls $\dd x$ multiplizieren. \\
\textit{Beispiel:} Die Wahrscheinlichkeit, dass Studierende zwischen $7,15$ und $7,15+\dd x$ Punkte haben, lautet $f(7,15)\cdot \dd x$. 
\end{aufgabebox}





\newpage
\section{Lösungen}\label{sec:mathe_grundlagen_loesungen}

\begin{loesungbox}{Lösung zu \Cref{aufg:ableitungen}}
\begin{enumerate}
    \item \textbf{Polynom (Summenregel):}
    \begin{equation}
        f_{1}'(x)=12x^{3}-16x+1
    \end{equation}
    
    \item \textbf{Verkettete Funktion (Kettenregel):}
    Mit $g(h) = \sin(h)$ und $h(x)=7x^{3}$ folgt $f_2(x) = g(h(x))$. Die Ableitungen sind $g'(h) = \cos(h)$ und $h'(x) = 21x^2$.
    \begin{equation}
        f_{2}'(x) = g'(h(x)) \cdot h'(x) = \cos(7x^{3}) \cdot 21x^{2}
    \end{equation}

    \item \textbf{Exponentialfunktion (Kettenregel):}
    Die Ableitung der äußeren Funktion $e^{(\cdot)}$ ist wieder die e-Funktion. Die innere Ableitung von $\lambda x$ ist $\lambda$.
    \begin{equation}
        f_{3}'(x)=A\cdot e^{\lambda\cdot x}\cdot \lambda
    \end{equation}
    
    \item \textbf{Mehrfach verkettete Funktion (Kettenregel):}
    Mit $g(x)=\cos(x)$, $h(x)=e^{x}$ und $k(x)=4x^{2}$ ist $f_4(x)=g(h(k(x)))$. Die Ableitungen sind $g'(x)=-\sin(x)$, $h'(x)=e^{x}$ und $k'(x)=8x$.
    \begin{align}
        f_{4}'(x) &= g'(h(k(x)))\cdot h'(k(x))\cdot k'(x) \nonumber \\
        &= -\sin(e^{4x^{2}})\cdot e^{4x^{2}}\cdot 8x \nonumber \\
        &= -8x\cdot e^{4x^{2}} \sin(e^{4x^{2}})
    \end{align}
    
    \item \textbf{Gebrochenrationale Funktion:}
    \begin{itemize}
        \item \textit{Weg (a) - Quotientenregel:} $f_5(x) = \frac{u(x)}{v(x)}$ mit $u(x)=1, v(x)=4x^5$.
        \begin{equation}
             f_{5}'(x)=\frac{u'v - uv'}{v^2} = \frac{0\cdot4x^{5}-1\cdot20x^{4}}{(4x^{5})^{2}}=-\frac{20x^{4}}{16x^{10}}=-\frac{5}{4x^{6}}
        \end{equation}
        \item \textit{Weg (b) - Potenzregel:} $f_5(x) = \frac{1}{4}x^{-5}$.
        \begin{equation}
            f_{5}'(x)=\frac{1}{4}(-5x^{-6})=-\frac{5}{4x^{6}}
        \end{equation}
    \end{itemize}
    
    \item \textbf{Wurzelfunktion (Potenzregel):}
    Wir schreiben die Wurzel als Potenz: $f_6(x) = \sqrt{x} = x^{1/2}$.
    \begin{equation}
        f_{6}'(x)=\frac{1}{2}x^{-1/2}=\frac{1}{2\sqrt{x}}
    \end{equation}
\end{enumerate}
\end{loesungbox}

\begin{loesungbox}{Lösung zu \Cref{aufg:integral_fx_OS_US}}
\begin{enumerate}
    \item Die Flächen werden durch die Summe der Rechtecksflächen angenähert.
    \begin{itemize}
        \item[(a)] \textbf{Obersumme:} $A_1^{\text{OS}} \approx 12,5$, $A_2^{\text{OS}} \approx -7,35$.
        \item[(b)] \textbf{Untersumme:} $A_1^{\text{US}} \approx 7,4$, $A_2^{\text{US}} \approx -12,4$.
    \end{itemize}
    \begin{center}
        \begin{tikzpicture}
            \begin{axis}[
                % Achsen und Beschriftungen
                axis lines=middle, % Achsen kreuzen sich im Ursprung (0,0)
                axis line style = thick,
                xlabel={\Large $x$},      % Beschriftung der x-Achse
                ylabel={\Large $f(x)$},    % Beschriftung der y-Achse
                % Achsen-Grenzen
                xmin=-1.5, xmax=5.5,
                ymin=-6.8, ymax=6.8,
                % Ticks und Gitter
                xtick={-1,0,1,2,3,4,5},
                ytick={-6,-4,-2,0,2,4,6},
                grid=both, % Haupt- und Untergitter anzeigen
                minor tick num=1, % Eine Gitterlinie zwischen Hauptlinien
                % Stil des Plots
                width=12cm, % Breite der Grafik
                height=8cm, % Höhe der Grafik
                ticklabel style={font=\normalsize}, % Schriftgröße der Tick-Zahlen
                label style={font=\normalsize}, % Schriftgröße der Achsenbeschreibungen
                xlabel style={at={(ticklabel* cs:1.01)}, anchor=west},
                ylabel style={at={(ticklabel* cs:0.94)}, anchor=south west},
                grid style={line width=.1pt, draw=gray!20} % Stil für das Gitter
            ]
                % --- NEU: Rechtecke für die Obersumme ---
                \addplot[
                    ybar interval,
                    fill=green!10,
                    draw=green!40!black,
                    line width=0.5pt,
                ] coordinates {
                    (-1.0, 3.4375)
                    (-0.5, 5.0)
                    (0.0,  5.196)
                    (0.5,  5.063)
                    (1.0,  4.0)
                    (1.5,  2.1875)
                    (2.0,  0.0)
                    (2.5,  -2.1875)
                    (3.0,  -4.0)
                    (3.5,  -5) % Technisch korrekter Wert für die Obersumme
                    (4.0,  -3.4375)
                    (4.5,  0.0)
                    (5.0,  0) % Letzter x-Wert definiert nur das Ende des letzten Balkens
                };
                % Der eigentliche Plot-Befehl
                \addplot[
                    domain=-1.2:5.2, % Definitionsbereich, für den gezeichnet wird
                    samples=200,     % Anzahl der Punkte (für eine glatte Kurve)
                    color=blue,      % Farbe der Kurve
                    line width=1.6pt % Linienstärke
                ]
                % --- KORRIGIERTE FUNKTION ---
                {0.5*x*x*x - 3*x*x + 1.5*x + 5}; 
            
                % --- NEU: Beschriftungen aus dem Bild ---
                % Titel-Box
                \node[draw, fill=white, font=\large, anchor=north west] at (axis cs:3.2, 5.9) {Obersumme};
            
                % Flächenberechnungen
                \node[anchor=west,fill=white] at (axis cs:-0.7, -1.5) {\small $A_1^{OS} \approx 0,5 \cdot 25 = 12,5$};
                \node[anchor=west,fill=white] at (axis cs:2.3, 1.5) {\small $A_2^{OS} \approx 0,5 \cdot (-14,7) = -7,35$};
                
                % Funktionswerte in den Rechtecken (positive Fläche)
                \node[rotate=90] at (axis cs:-0.75, 2.5) {\small $f \approx 3,5$};
                \node[rotate=90] at (axis cs:-0.25, 2.5) {\small $f \approx 5$};
                \node[rotate=90] at (axis cs:0.25, 2.5) {\small $f \approx 5,2$};
                \node[rotate=90] at (axis cs:0.75, 2.5) {\small $f \approx 5,1$};
                \node[rotate=90] at (axis cs:1.25, 2.5) {\small $f \approx 4$};
                \node[rotate=90] at (axis cs:1.75, 1.3) {\small $f \approx 2,2$};
                \node[rotate=90] at (axis cs:2.25, -0.7) {\small $f \approx 0,0$};
                
                % Funktionswerte in den Rechtecken (negative Fläche)
                \node[rotate=90] at (axis cs:2.75, -1.5) {\small $f \approx -2,2$};
                \node[rotate=90] at (axis cs:3.25, -2.5) {\small $f \approx -4,0$};
                \node[rotate=90] at (axis cs:3.75, -2.5) {\small $f \approx -5,0$};
                \node[rotate=90] at (axis cs:4.25, -2.5) {\small $f \approx -3,5$};
                \node[rotate=90] at (axis cs:4.75, -0.7) {\small $f \approx 0,0$};
            \end{axis}
        \end{tikzpicture}
        \begin{tikzpicture}
            \begin{axis}[
                % Achsen und Beschriftungen
                axis lines=middle, % Achsen kreuzen sich im Ursprung (0,0)
                axis line style = thick,
                xlabel={\Large $x$},      % Beschriftung der x-Achse
                ylabel={\Large $f(x)$},    % Beschriftung der y-Achse
                % Achsen-Grenzen
                xmin=-1.5, xmax=5.5,
                ymin=-6.8, ymax=6.8,
                % Ticks und Gitter
                xtick={-1,0,1,2,3,4,5},
                ytick={-6,-4,-2,0,2,4,6},
                grid=both, % Haupt- und Untergitter anzeigen
                minor tick num=1, % Eine Gitterlinie zwischen Hauptlinien
                % Stil des Plots
                width=12cm, % Breite der Grafik
                height=8cm, % Höhe der Grafik
                ticklabel style={font=\normalsize}, % Schriftgröße der Tick-Zahlen
                label style={font=\normalsize}, % Schriftgröße der Achsenbeschreibungen
                xlabel style={at={(ticklabel* cs:1.01)}, anchor=west},
                ylabel style={at={(ticklabel* cs:0.94)}, anchor=south west},
                grid style={line width=.1pt, draw=gray!20} % Stil für das Gitter
            ]
                % --- NEU: Rechtecke für die Obersumme ---
                \addplot[
                    ybar interval,
                    fill=red!10,
                    draw=red!40!black,
                    line width=0.5pt,
                ] coordinates {
                    (-1.0, 0.0)
                    (-0.5, 3.4375)
                    (0.0,  5.0625)
                    (0.5,  4.0)
                    (1.0,  2.1875)
                    (1.5,  0.0)
                    (2.0,  -2.1875)
                    (2.5,  -4.0)
                    (3.0,  -5.0625)
                    (3.5,  -5.196) % Technisch korrekter Wert für die Obersumme
                    (4.0,  -5.0)
                    (4.5,  -3.4375)
                    (5.0,  0) % Letzter x-Wert definiert nur das Ende des letzten Balkens
                };
                % Der eigentliche Plot-Befehl
                \addplot[
                    domain=-1.2:5.2, % Definitionsbereich, für den gezeichnet wird
                    samples=200,     % Anzahl der Punkte (für eine glatte Kurve)
                    color=blue,      % Farbe der Kurve
                    line width=1.6pt % Linienstärke
                ]
                % --- KORRIGIERTE FUNKTION ---
                {0.5*x*x*x - 3*x*x + 1.5*x + 5}; 
            
                % --- NEU: Beschriftungen aus dem Bild ---
                % Titel-Box
                \node[draw, fill=white, font=\large, anchor=north west] at (axis cs:3.2, 5.9) {Untersumme};
            
                % Flächenberechnungen
                \node[anchor=west,fill=white] at (axis cs:-0.7, -1.5) {\small $A_1^{US} \approx 0,5 \cdot 14,7 = 7,4$};
                \node[anchor=west,fill=white] at (axis cs:2.0, 1.2) {\small $A_2^{US} \approx 0,5 \cdot (-24,8) = -12,4$};
                
                % Funktionswerte in den Rechtecken (positive Fläche)
                \node[rotate=90] at (axis cs:-0.75, 1.3) {\small $f \approx 0,0$};
                \node[rotate=90] at (axis cs:-0.25, 1.9) {\small $f \approx 3,5$};
                \node[rotate=90] at (axis cs:0.25, 2.9) {\small $f \approx 5,0$};
                \node[rotate=90] at (axis cs:0.75, 2.5) {\small $f \approx 4,0$};
                \node[rotate=90] at (axis cs:1.25, 1.5) {\small $f \approx 2,2$};
                \node[rotate=90] at (axis cs:1.75, 1.3) {\small $f \approx 0,0$};
                
                % Funktionswerte in den Rechtecken (negative Fläche)
                \node[rotate=90] at (axis cs:2.25, -1.7) {\small $f \approx -2,2$};
                \node[rotate=90] at (axis cs:2.75, -1.7) {\small $f \approx -4,0$};
                \node[rotate=90] at (axis cs:3.25, -2.5) {\small $f \approx -5,0$};
                \node[rotate=90] at (axis cs:3.75, -2.5) {\small $f \approx -5,2$};
                \node[rotate=90] at (axis cs:4.25, -2.5) {\small $f \approx -5,0$};
                \node[rotate=90] at (axis cs:4.75, -1.7) {\small $f \approx -3,4$};
            \end{axis}
            \end{tikzpicture}
        \label{fig:ober_unter_summe}
    \end{center}

    \item Die \textbf{Stammfunktion} $F(x)$ ergibt sich durch unbestimmte Integration:
    \begin{equation}\label{eq:stammfunktion_f}
        F(x) = \int \left(\frac{1}{2}x^{3}-3x^{2}+\frac{3}{2}x+5\right) \dd x = \frac{x^{4}}{8}-x^{3}+\frac{3x^{2}}{4}+5x+C \mDot
    \end{equation}
    
    \item Nach dem \textbf{Hauptsatz der Differential- und Integralrechnung} gilt \newline $\int_{a}^{b} f(x)\,\dd x = F(b)-F(a)$.
    \begin{align}
        A_1 &= \int_{-1}^{2} f(x)\,\dd x = F(2) - F(-1) = 7 - \left(-\frac{25}{8}\right) = \frac{81}{8} \approx 10,125 \\
        A_2 &= \int_{2}^{5} f(x)\,\dd x = F(5) - F(2) = -\frac{25}{8} - 7 = -\frac{81}{8} \approx -10,125
    \end{align}
    Wie erwartet, liegt der exakte Wert für beide Flächen zwischen der Ober- und der Untersumme:
    \begin{gather*}
        7,4 = A_{1}^{\text{US}} \le 10,125 \le A_{1}^{\text{OS}} = 12,5 \\
        -12,4 = A_{2}^{\text{US}} \le -10,125 \le A_{2}^{\text{OS}} = -7,35
    \end{gather*}
    
    \item Das Integral über das Gesamtintervall ist die Summe der Teilflächen:
    \begin{equation}
         \int_{-1}^{5}f(x)\,\dd x = A_1 + A_2 = \frac{81}{8} - \frac{81}{8} = 0 \mDot
    \end{equation}
    Die positive und die negative Fläche heben sich exakt auf.
\end{enumerate}
\end{loesungbox}



\begin{loesungbox}{Lösung zu \Cref{aufg:kreisgleichung}}
\begin{enumerate}
    \item \textbf{Umformung zur Kreisgleichung:}
    Wir formen den gegebenen Ausdruck durch quadratische Ergänzung auf die allgemeine Kreisgleichung $$(x-x_M)^2 + (y-y_M)^2 = r^2$$ um.
    \begin{align*}
        x^2+y^2+4x-6y+9 &= 0 \\
        (x^2+4x) + (y^2-6y) + 9 &= 0 \\
        (x^2+4x\underbrace{+4-4}_{\text{quadr. Erg.}}) + (y^2-6y\underbrace{+9-9}_{\text{quadr. Erg.}}) + 9 &= 0 \\
        (x+2)^2 - 4 + (y-3)^2 - 9 + 9 &= 0 \\
        (x+2)^2 + (y-3)^2 - 4 &= 0
    \end{align*}
    Daraus folgt die finale Kreisgleichung:
    \begin{equation}\label{eq:kreisgleichung_final}
        (x+2)^2 + (y-3)^2 = 4 = 2^2
    \end{equation}
    Dies ist die Gleichung eines Kreises mit dem Mittelpunkt $M\ipTwo{-2}{3}$ und dem Radius $r = 2$.

    \item \textbf{Punktprobe:} Wir setzen die Koordinaten der Punkte in die linke Seite der \Cref{eq:kreisgleichung_final} ein.
    \begin{itemize}
        \item \textbf{Punkt} $Q\ipTwo{-5}{-5}$\textbf{:}
        \begin{equation*}
            (-5+2)^2 + (-5-3)^2 = (-3)^2 + (-8)^2 = 9 + 64 = 73
        \end{equation*}
        Da $73 \neq 4$, liegt $Q$ nicht auf dem Kreis. Da $73 > 4$, liegt der Punkt außerhalb des Kreises.
        \item \textbf{Punkt} $S\ipTwo{-4}{3}$\textbf{:}
        \begin{equation*}
            (-4+2)^2 + (3-3)^2 = (-2)^2 + 0^2 = 4
        \end{equation*}
        Da $4 = 4$, liegt $S$ exakt auf dem Kreis.
    \end{itemize}

    \item \textbf{Beispielpunkte finden:}
    \begin{itemize}
        \item \textbf{Innerhalb ($I$):} Der Abstand zum Mittelpunkt $M\ipTwo{-2}{3}$ muss kleiner als der Radius $r=2$ sein. Jeder Punkt im $x$-Intervall $[-4, 0]$ und $y$-Intervall $[1, 5]$ ist ein Kandidat. Wir wählen $I = \ipTwo{-3}{4}$ und überprüfen 
        $$I = (-3+2)^2 + (4-3)^2 = (-1)^2 + (1)^2 = 1+1 = 2 < 4 \mDot$$
        Der Punkt $I\ipTwo{-3}{4}$ liegt daher innerhalb des Kreises.
        \begin{center}\begin{tikzpicture}
            \begin{axis}[
                xlabel={$x$},
                ylabel={$y$},
                xmin=-5, xmax=1,
                ymin=0, ymax=6,
                xtick={-5,-4,-3,-2,-1,0,1},
                ytick={0,1,2,3,4,5,6},
                axis lines=box,
                axis equal,
                grid=both,
                grid style={line width=.1pt, draw=gray!20},
                tick label style={font=\normalsize},
                every axis x label/.append style={font=\Large, at={(ticklabel* cs:0.5)},anchor=north,yshift=-0.6cm},
                every axis y label/.append style={font=\Large, at={(ticklabel* cs:0.5)},anchor=south,yshift=0.6cm},
            ]
            \fill[red!10] (axis cs:-4,1) rectangle (axis cs:0,5);
            % 1. Nur die Punkte zeichnen
            \addplot[only marks, mark=*, mark options={fill=black}, mark size=1pt] coordinates {
                (-2,3)
            };
            \node[above left] at (axis cs:-2,3) {$M$};
        
            % 3. Kreis um den Mittelpunkt M mit Radius 2 zeichnen
            \draw[blue, thick] (axis cs:-2,3) circle (2);
            % 4. Links-Rechts-Pfeil von x=0 bis x=4 auf der Höhe y=0.5
            \draw[{Stealth}-{Stealth}, red, thick] (axis cs:-4,1.) -- (axis cs:0,1.);
            \draw[{Stealth}-{Stealth}, red, thick] (axis cs:-4.,1) -- (axis cs:-4.,5);
            \end{axis}
        \end{tikzpicture}\end{center}
        \item \textbf{Außerhalb ($O$):} Der Abstand zum Mittelpunkt muss größer als $r=2$ sein.
        Beispiel: $O = (10|0)$, da $$(10+2)^2 + (0-3)^2 = 144+9=153 > 4 \mDot$$
        
        \item \textbf{Auf dem Kreis ($P$):} Wir wählen eine $x$-Koordinate aus dem Intervall der möglichen $x$-Werte $[-4, 0]$, \zB $x=-1$, und berechnen das zugehörige $y$:
        \begin{align*}
            (-1+2)^2 + (y-3)^2 &= 4 \\
            1 + (y-3)^2 &= 4 \\
            (y-3)^2 &= 3 \implies y-3 = \pm\sqrt{3} \implies y = 3 \pm \sqrt{3}
        \end{align*}
        Beispiel: $P = \ipTwo{-1}{3+\sqrt{3}}$.
    \end{itemize}
\end{enumerate}
\end{loesungbox}



\begin{loesungbox}{Lösung zu \Cref{aufg:vektorrechnung}}
Gegebene Punkte: $A\ipTwo{0}{0}$, $B\ipTwo{1}{2}$, $C\ipTwo{4}{2}$, $D\ipTwo{3}{0}$.
\begin{enumerate}
    \item Ein Vektor ist durch seine \textbf{Länge (Betrag)} und seine \textbf{Richtung} definiert. Ein Vektor wird durch Parallelverschiebung nicht verändert.

    \item Vektoren werden durch \gDQ{Spitze minus Schaft} berechnet.
    \begin{align*}
        \ivec{AB} &= B - A = \icolTwo{1-0}{2-0} = \icolTwo{1}{2} \\
        \ivec{DC} &= C - D = \icolTwo{4-3}{2-0} = \icolTwo{1}{2}
    \end{align*}
    Die Vektoren sind identisch, da sie dieselbe Länge und Richtung haben. Im gegebenen Parallelogramm sind sie die gegenüberliegenden Seiten.
    
    \item \textbf{Länge und Normierung von $\ivec{p} = \ivec{AC}$:}
    \begin{equation*}
        \ivec{p} = \ivec{AC} = C - A = \icolTwo{4-0}{2-0} = \icolTwo{4}{2}
    \end{equation*}
    Die Länge (der Betrag) ist:
    \begin{equation*}
        |\ivec{p}| = \sqrt{4^2 + 2^2} = \sqrt{16+4} = \sqrt{20} = 2\sqrt{5}
    \end{equation*}
    Der normierte Vektor (Einheitsvektor) ist $\ivecS{e}{p} = \frac{1}{|\ivec{p}|}\ivec{p}$:
    \begin{equation*}
        \ivecS{e}{p} = \frac{1}{2\sqrt{5}}\icolTwo{4}{2} = \frac{1}{\sqrt{5}}\icolTwo{2}{1}
    \end{equation*}

    \item \textbf{Länge und Normierung von $\ivec{q} = \ivec{BD}$:}
    \begin{equation*}
        \ivec{q} = \ivec{BD} = D - B = \icolTwo{3-1}{0-2} = \icolTwo{2}{-2}
    \end{equation*}
    Die Länge ist:
    \begin{equation*}
        |\ivec{q}| = \sqrt{2^2 + (-2)^2} = \sqrt{4+4} = \sqrt{8} = 2\sqrt{2}
    \end{equation*}
    Der normierte Vektor ist $\ivecS{e}{q} = \frac{1}{|\ivec{q}|}\ivec{q}$:
    \begin{equation*}
        \ivecS{e}{q} = \frac{1}{2\sqrt{2}}\icolTwo{2}{-2} = \frac{1}{\sqrt{2}}\icolTwo{1}{-1}
    \end{equation*}
    
    \item \textbf{Winkel zwischen $\ivec{p}$ and $\ivec{q}$:}
    Wir verwenden das innere Produkt (Skalarprodukt): $$\ivec{p} \cdot \ivec{q} = |\ivec{p}| |\ivec{q}| \cos(\phi)\mDot$$ Der Winkel zwischen den Vektoren $\ivec{p}$ und $\ivec{q}$ muss derselbe sein, wie der Winkel zwischen ihren jeweiligen Einheitsvektoren $\ivecS{e}{p}$ und $\ivecS{e}{q}$: 
    \begin{equation*}
        \ivecS{e}{p} \cdot \ivecS{e}{q} = \underbrace{|\ivecS{e}{p}|}_{=1} \underbrace{|\ivecS{e}{q}|}_{=1} \cos(\phi) = \cos(\phi) \mDot 
    \end{equation*}
    \begin{equation*}
        \cos(\phi) = \frac{\ivec{p} \cdot \ivec{q}}{|\ivec{p}| |\ivec{q}|} \equiv \ivecS{e}{p} \cdot \ivecS{e}{q} = \frac{1}{\sqrt{5}} \icolTwo{2}{1} \cdot \frac{1}{\sqrt{2}}\icolTwo{1}{-1} = \frac{1}{\sqrt{10}} (2\cdot 1 + 1\cdot (-1)) = \frac{1}{\sqrt{10}}
    \end{equation*}
    \begin{equation*}
        \implies \phi = \arccos\left(\frac{1}{\sqrt{10}}\right) \approx \SI{71.57}{\degree} \approx \SI{1.25}{\radian}
    \end{equation*}
    \begin{center}
        \begin{tikzpicture}
            \begin{axis}[
                xlabel={$x$},
                ylabel={$y$},
                xmin=-2, xmax=5,
                ymin=-1, ymax=3,
                xtick={-2,-1,0,1,2,3,4,5},
                ytick={-1,0,1,2,3},
                axis lines=box,
                grid=both,
                grid style={line width=.1pt, draw=gray!20},
                tick label style={font=\normalsize},
                every axis x label/.append style={font=\Large, at={(ticklabel* cs:0.5)},anchor=north,yshift=-0.6cm},
                every axis y label/.append style={font=\Large, at={(ticklabel* cs:0.5)},anchor=south,yshift=0.6cm},
            ]
            % 1. Nur die Punkte zeichnen
            \addplot[only marks, mark=*, mark options={fill=red}, mark size=2pt] coordinates {
                (0,0)
                (1,2)
                (4,2)
                (3,0)
            };
            %    'axis cs:' stellt sicher, dass die Koordinaten des Achsensystems verwendet werden.
            \node[above left, xshift=-2pt] at (axis cs:0,0) {$A$};
            \node[above left, xshift=-2pt] at (axis cs:1,2) {$B$};
            \node[above right, xshift=2pt] at (axis cs:4,2) {$C$};
            \node[below right, xshift=2pt] at (axis cs:3,0) {$D$};
            \draw[-{Stealth}, blue, line width=1.2pt] (axis cs:0,0) -- (axis cs:4,2);
            \draw[-{Stealth}, blue, line width=1.2pt] (axis cs:1,2) -- (axis cs:3,0);
            \node[right] at (axis cs:2.3,1) {\Large $\phi$};
            \end{axis}
        \end{tikzpicture}
    \end{center}

    \item \textbf{Vektoraddition:}
    \begin{equation*}
         \ivec{BC} = C - B = \icolTwo{4-1}{2-2} = \icolTwo{3}{0}
    \end{equation*}
    \begin{equation*}
         \ivec{AB} + \ivec{BC} = \icolTwo{1}{2} + \icolTwo{3}{0} = \icolTwo{1+3}{2+0} = \icolTwo{4}{2} = \ivec{AC}
    \end{equation*}
    Das Ergebnis ist korrekt (\wA). Geometrisch bedeutet die Addition von Vektoren, sie aneinander zu hängen. Der Weg von $A$ nach $B$ und dann von $B$ nach $C$ ist dasselbe wie der direkte Weg von $A$ nach $C$.

    \item \textbf{Flächeninhalt des Parallelogramms:}
    Der Betrag des äußeren Produkts (Kreuzprodukts) von zwei Vektoren, entspricht dem Flächeninhalt des Parallelogramms, das sie aufspannen. Wir erweitern die Vektoren $\ivec{AB}$ und $\ivec{AD}$ in 3D, indem wir die $z$-Koordinate auf $0$ setzen.
    \begin{equation*}
        \ivec{AD} = D - A = \icolTwo{3}{0} \implies \ivecS{AD}{\,3D} = \icolThree{3}{0}{0}
    \end{equation*}
    \begin{equation*}
        \ivecS{AB}{\,3D} = \icolThree{1}{2}{0}
    \end{equation*}
    Das Kreuzprodukt ergibt den Normalvektor $\ivec{n}$:
    \begin{equation*}
        \ivec{n} = \ivecS{AB}{\,3D} \times \ivecS{AD}{\,3D} = \icolThree{1}{2}{0} \times \icolThree{3}{0}{0} = \icolThree{2\cdot0 - 0\cdot0}{-(1\cdot0 - 0\cdot3)}{1\cdot0 - 2\cdot3} = \icolThree{0}{0}{-6}
    \end{equation*}
    Der Flächeninhalt $F$ ist der Betrag des Normalvektors:
    \begin{equation}
        F = |\ivec{n}| = \sqrt{0^2+0^2+(-6)^2} = 6
    \end{equation}
\end{enumerate}
\end{loesungbox}



\begin{loesungbox}{Lösung zu \Cref{aufg:haeufigkeitsverteilung}}
\begin{enumerate}
    \item Wir setzen die Werte aus der Tabelle in die Formel ein:
    \begin{align}
        \overline{x} &= \frac{1}{2000} (0\cdot14 + 1\cdot102 + 2\cdot159 + 3\cdot186 + 4\cdot216 + 5\cdot415 \nonumber \\ 
        &\quad + 6\cdot374 + 7\cdot271 + 8\cdot160 + 9\cdot75 + 10\cdot28) \nonumber \\
        &= \frac{1}{2000} (0+102+318+558+864+2075+2244+1897+1280+675+280) \nonumber \\
        &= \frac{10293}{2000} = \num{5.1465} \text{ Punkte} \mDot
    \end{align}
    
    \item Die relativen Häufigkeiten $n_i = N_i/N$ mit $N=2000$:
    \begin{center}
    \begin{tabular}{ccc}
    \toprule
    \textbf{Punkte ($x_i$)} & \textbf{Anzahl ($N_i$)} & \textbf{Relative Häufigkeit ($n_i$)} \\
    \midrule
    0 & 14 & 0,0070 \\
    1 & 102 & 0,0510 \\
    2 & 159 & 0,0795 \\
    3 & 186 & 0,0930 \\
    4 & 216 & 0,1080 \\
    5 & 415 & 0,2075 \\
    6 & 374 & 0,1870 \\
    7 & 271 & 0,1355 \\
    8 & 160 & 0,0800 \\
    9 & 75 & 0,0375 \\
    10 & 28 & 0,0140 \\
    \bottomrule
    \end{tabular}
    \end{center}

    \item Wir formen die Summe um, indem wir $N$ in die Summe ziehen:
    \begin{equation}
        \overline{x} = \frac{1}{N}\sum_{i=0}^{10}x_{i}\cdot N_{i} = \sum_{i=0}^{10}x_{i}\cdot \frac{N_{i}}{N} = \sum_{i=0}^{10}x_{i}\cdot n_{i} \mDot
    \end{equation}
    
    \item Im kontinuierlichen Fall wird die Summe zu einem Integral. Der Term $f(x)dx$ entspricht der relativen Häufigkeit $n_i$, also der Wahrscheinlichkeit, dass ein Ergebnis im infinitesimalen Intervall $[x, x+dx]$ liegt. Das Produkt $x \cdot (f(x)dx)$ ist dann der Beitrag dieses Intervalls zum Mittelwert. Das Integral summiert all diese Beiträge über den gesamten möglichen Punktebereich auf.
    
    Wir berechnen das Integral mit $A=1/6$ und $\sigma=4$:
    \begin{align}
        \overline{x} &= \int_{0}^{10}x\cdot\frac{1}{6}\sin^2\left(\frac{x}{4}\right)\dd x = \frac{1}{6} \left[ \frac{x^2}{4}-\frac{16}{8}\cos\left(\frac{2x}{4}\right)-\frac{4x}{4}\sin\left(\frac{2x}{4}\right) \right]_{0}^{10} \nonumber \\
        &= \frac{1}{6} \left[ \frac{x^2}{4}-2\cos\left(\frac{x}{2}\right)-x\sin\left(\frac{x}{2}\right) \right]_{0}^{10} \nonumber \\
        &= \frac{1}{6} \left( \left( \frac{100}{4}-2\cos(5)-10\sin(5) \right) - \left( 0-2\cos(0)-0 \right) \right) \nonumber \\
        &= \frac{1}{6} \left( 25 - 2\cos(5) - 10\sin(5) + 2 \right) \nonumber \\
        &= \frac{1}{6} (27 - \num{0.567} + \num{9.589}) \approx \frac{\num{36.022}}{6} \approx \num{6.004} \text{ Punkte} \mDot
    \end{align}
\end{enumerate}
\end{loesungbox}



\chapter{Physikalische Grundbegriffe und Einheiten}\label{chap:grundbegriffe}
\section{Aufgaben}\label{sec:grundbegriffe_aufgaben}

\begin{aufgabebox}{Die Physik}{die_physik}
Verbinden Sie die nachstehenden Beispiele mit dem passenden Begriff aus der Wissenschaftstheorie. Begründen Sie Ihre Entscheidung! Sie werden feststellen, dass die Zuordnung nicht immer eindeutig ist.

\medskip\medskip\noindent\textbf{Begriffe}\par\medskip
\begin{tcbraster}[raster rows=1, raster columns=5]
    \begin{termbox}Hypothese\end{termbox}
    \begin{termbox}Axiom\end{termbox}
    \begin{termbox}Satz\end{termbox}
    \begin{termbox}Gesetz\end{termbox}
    \begin{termbox}Theorie\end{termbox}
\end{tcbraster}

\bigskip\noindent\textbf{Beispiele}\par\medskip
\begin{tcbraster}[raster columns=2, raster equal height=rows]
    \begin{examplebox}{1}
        Durch zwei Punkte kann genau eine Gerade gelegt werden.
    \end{examplebox}
    \begin{examplebox}{2}
        Der elektrische Widerstand ist gleich der Spannung geteilt durch den Strom.
    \end{examplebox}
    \begin{examplebox}{3}
        Alle von einem Halbkreis umschriebenen Dreiecke sind rechtwinklig.
    \end{examplebox}
    \begin{examplebox}{4}
        Alle Arten von Lebewesen haben sich über lange Zeiträume entwickelt.
    \end{examplebox}
    \begin{examplebox}{5}
        Wenn man Salz in Wasser auflöst, erhöht sich der Siedepunkt des Wassers.
    \end{examplebox}
    \begin{examplebox}{6}
        Für $n>2$ gibt es keine positiven ganzen Zahlen $a, b, c$, die $a^n + b^n = c^n$ erfüllen.
    \end{examplebox}
    \begin{examplebox}{7}
        Zwei parallele Linien schneiden sich nie.
    \end{examplebox}
    \begin{examplebox}{8}
        Pflanzen wachsen bei rotem Licht schneller als bei blauem Licht.
    \end{examplebox}
    \begin{examplebox}{9}
        Die Summe der Kathetenquadrate ist gleich dem Hypotenusenquadrat.
    \end{examplebox}
    \begin{examplebox}{10}
        Die physikalischen Gesetze sind in allen Inertialsystemen gleich.
    \end{examplebox}
    \begin{examplebox}{11}
        Jede Masse zieht jede andere Masse an.
    \end{examplebox}
    \begin{examplebox}{12}
        Wenn man den Druck auf ein Gas erhöht, sinkt sein Volumen.
    \end{examplebox}
\end{tcbraster}
\end{aufgabebox}


% \begin{aufgabebox}{Die Physik}{die_physik}
% Verbinden Sie die nachstehenden Beispiele mit dem passenden Begriff aus der Wissenschaftstheorie. Begründen Sie Ihre Entscheidung! Sie werden feststellen, dass die Zuordnung nicht immer eindeutig ist.
% \begin{center}
%     \begin{tabular}{ll}
%         \toprule
%         \textbf{Begriffe} & \textbf{Beispiele} \\
%         \midrule
%         \begin{tabular}[t]{@{}l@{}}
%             Hypothese \\
%             Axiom \\
%             Satz \\
%             Gesetz \\
%             Theorie
%         \end{tabular} &
%         \begin{tabular}[t]{@{}l@{}}
%             1. Durch zwei Punkte kann genau eine Gerade gelegt werden. \\
%             2. Der elektrische Widerstand ist gleich der Spannung geteilt durch den Strom. \\
%             3. Alle von einem Halbkreis umschriebenen Dreiecke sind rechtwinklig. \\
%             4. Alle Arten von Lebewesen haben sich über lange Zeiträume entwickelt. \\
%             5. Wenn man Salz in Wasser auflöst, erhöht sich der Siedepunkt des Wassers. \\
%             6. Für $n>2$ gibt es keine positiven ganzen Zahlen $a, b, c$, die $a^n + b^n = c^n$ erfüllen. \\
%             7. Zwei parallele Linien schneiden sich nie. \\
%             8. Pflanzen wachsen bei rotem Licht schneller als bei blauem Licht. \\
%             9. Die Summe der Kathetenquadrate ist gleich dem Hypotenusenquadrat. \\
%             10. Die physikalischen Gesetze sind in allen Inertialsystemen gleich. \\
%             11. Jede Masse zieht jede andere Masse an. \\
%             12. Wenn man den Druck auf ein Gas erhöht, sinkt sein Volumen. \\
%         \end{tabular} \\
%         \bottomrule
%     \end{tabular}
% \end{center}
% \end{aufgabebox}

% \begin{aufgabebox}{Umrechnung von Einheiten}{umrechnung_einheiten}
% Rechnen Sie die Einheiten auf der linken Seite in die Einheiten auf der rechten Seite um!
% \begin{enumerate}
%     \item \SI{14,3}{\nano\second} = \underline{\hspace{2.5cm}} \si{\hour}
%     \item \SI{81,23}{\meter\per\second} = \underline{\hspace{2.5cm}} \si{\kilo\meter\per\hour}
%     \item \SI{0,78}{\litre} = \underline{\hspace{2.5cm}} \si{\milli\meter\cubed}
%     \item \SI{1}{\kilo\watt\hour} = \underline{\hspace{2.5cm}} \si{\joule}
%     \item \SI{15000}{\ampere\hour} (bei einer Spannung von $U = \SI{3,7}{\volt}$) = \underline{\hspace{2.5cm}} \si{\kilo\watt\hour}
% \end{enumerate}
% \end{aufgabebox}

\begin{aufgabebox}{Umrechnung von Einheiten}{umrechnung_einheiten}
Rechnen Sie die Einheiten auf der linken Seite in die Einheiten auf der rechten Seite um!
\begin{multicols}{2}
\setlength{\jot}{8pt}
\allowdisplaybreaks
\begin{alignat*}{2}
    \text{1.} \quad & \SI{14,3}{\nano\second} &&=\quad \underline{\hspace{1.5cm}}\; \si{\hour} \\
    \text{2.} \quad & \SI{81,23}{\meter\per\second} &&=\quad \underline{\hspace{1.5cm}} \;\si{\kilo\meter\per\hour} \\
    \text{3.} \quad & \SI{0,78}{\litre} &&=\quad \underline{\hspace{1.5cm}} \;\si{\milli\meter\cubed} \\
    \text{4.} \quad & \SI{1}{\kWh} &&=\quad \underline{\hspace{1.5cm}} \;\si{\joule} \\
    \text{5.} \quad & \SI[product-units=repeat]{15000}{\ampere\hour} &&=\quad \underline{\hspace{1.5cm}} \;\si{\kWh} \\
    & \multicolumn{2}{l}{\footnotesize{(bei $U = \SI{3,7}{\volt}$)}} \\
    \text{6.} \quad & \SI{101325}{\pascal} &&=\quad \underline{\hspace{1.5cm}} \;\si{\bar} \\
    \text{7.} \quad & \SI{19,3}{\gram\per\centi\meter\cubed} &&=\quad \underline{\hspace{1.5cm}} \;\si{\kilo\gram\per\meter\cubed} \\
    \text{8.} \quad & 120\,\text{PS} &&=\quad \underline{\hspace{1.5cm}} \;\si{\kilo\watt} \\
    & \multicolumn{2}{l}{\footnotesize{($1\,\text{PS} = \SI{735.5}{\watt}$)}} \\
    \text{9.} \quad & \SI{3000}{\per\minute} &&=\quad \underline{\hspace{1.5cm}} \;\si{\radian\per\second} \\
    \text{10.} \quad & \SI{500}{\kilo\joule} &&=\quad \underline{\hspace{1.5cm}} \;\si{\electronvolt} 
\end{alignat*}
\end{multicols}
\end{aufgabebox}


\begin{aufgabebox}{Physikalische Dimensionsanalyse}{dimensionsanalyse}
Lösen Sie folgende Beispiele, indem Sie die Einheiten der physikalischen Größen in Beziehung setzen!
\begin{enumerate}[itemsep=8pt]
    \item Der Staudruck $p$ ($[p] = \si{\pascal} = \si{\newton\per\meter\squared}$) in einer bewegten Flüssigkeit hängt von ihrer Dichte $\rho$ ($[\rho] = \si{\kilo\gram\per\meter\cubed}$) und von ihrer Geschwindigkeit $v$ ($[v] = \si{\meter\per\second}$) ab. Finden Sie eine einfache Kombination von Dichte und Geschwindigkeit, die die korrekte Dimension für den Druck ergibt!\\
    \textit{Tipp: Drücken Sie die Einheit Newton durch die Basiseinheiten (\si{\kilo\gram}, \si{\meter}, \si{\second}) aus.}
    
    \item Die Fallzeit $t$ ($[t] = \si{\second}$) eines Objektes hängt von der Fallhöhe $H$ ($[H] = \si{\meter}$) und von der Erdbeschleunigung $g$ ($[g] = \si{\meter\per\second\squared}$) ab. Finden Sie den Zusammenhang zwischen $t$, $H$ und $g$!
    
    \item Die Auftriebskraft $\ivec{F}$ ($[F] = \si{\newton}$), die auf einen Körper in einer Flüssigkeit wirkt, hängt von seinem Volumen $V$ ($[V] = \si{\meter\cubed}$), der Dichte der Flüssigkeit $\rho$ ($[\rho] = \si{\kilo\gram\per\meter\cubed}$) und der Erdbeschleunigung $g$ ($[g] = \si{\meter\per\second\squared}$) ab. Finden Sie den physikalischen Zusammenhang zwischen der Auftriebskraft und den anderen drei Größen!
\end{enumerate}
\end{aufgabebox}



\newpage
\section{Lösungen}\label{sec:grundbegriffe_loesungen}

\begin{loesungbox}{Lösung zu \Cref{aufg:die_physik}}
\begin{tcbraster}[raster columns=2, raster equal height=rows]
    \begin{solutiontermbox}{Axiom}
        Durch zwei Punkte kann genau eine Gerade gelegt werden.
        \tcblower
        \textbf{Begründung:} Dies ist eine fundamentale Annahme der Euklidischen Geometrie. Sie wird nicht bewiesen, sondern als wahr vorausgesetzt, um daraus weitere Sätze (Theoreme) logisch abzuleiten.
    \end{solutiontermbox}
    \begin{solutiontermbox}{Gesetz}
        Der elektrische Widerstand ist gleich der Spannung geteilt durch den Strom.
        \tcblower
        \textbf{Begründung:} Dies beschreibt eine universell gültige, empirisch bestätigte Beziehung ($R=U/I$) in der Natur. Es ist eine aus Beobachtungen abgeleitete Regel, die ein fundamentales Verhalten beschreibt.
    \end{solutiontermbox}
    \begin{solutiontermbox}{Satz}
        Alle von einem Halbkreis umschriebenen Dreiecke sind rechtwinklig.
        \tcblower
        \textbf{Begründung:} Dies ist der Satz des Thales. Er ist keine grundlegende Annahme, sondern eine Aussage, die mithilfe der Axiome der Geometrie rigoros bewiesen werden kann.
    \end{solutiontermbox}
    \begin{solutiontermbox}{Theorie}
        Alle Arten von Lebewesen haben sich über lange Zeiträume entwickelt.
        \tcblower
        \textbf{Begründung:} Die Evolutionstheorie ist ein umfassendes Erklärungsmodell. Sie verbindet Gesetze, Fakten und unzählige geprüfte Hypothesen zu einem in sich schlüssigen System, das die Entwicklung des Lebens erklärt.
    \end{solutiontermbox}
    \begin{solutiontermbox}{Hypothese}
        Wenn man Salz in Wasser auflöst, erhöht sich der Siedepunkt des Wassers.
        \tcblower
        \textbf{Begründung:} Dies ist eine konkrete, überprüfbare Behauptung. Man kann ein Experiment durchführen, um sie zu bestätigen (verifizieren) oder zu widerlegen (falsifizieren), was sie zu einer klassischen Hypothese macht.
    \end{solutiontermbox}
    \begin{solutiontermbox}{Satz}
        Für $n>2$ gibt es keine pos. ganzen Zahlen $a, b, c$, die $a^n + b^n = c^n$ erfüllen.
        \tcblower
        \textbf{Begründung:} Dies war jahrhundertelang eine berühmte Vermutung (Hypothese). Seit dem Beweis durch Andrew Wiles im Jahr 1994 ist es ein bewiesener mathematischer Satz (der Große Fermatsche Satz).
    \end{solutiontermbox}
    \begin{solutiontermbox}{Axiom}
        Zwei parallele Linien schneiden sich nie.
        \tcblower
        \textbf{Begründung:} Dies ist das Parallelenaxiom, eine weitere Grundannahme der Euklidischen Geometrie. Die Änderung dieses einen Axioms führt zu völlig neuen, nicht-euklidischen Geometrien.
    \end{solutiontermbox}
    \begin{solutiontermbox}{Hypothese}
        Pflanzen wachsen bei rotem Licht schneller als bei blauem Licht.
        \tcblower
        \textbf{Begründung:} Dies ist eine spezifische, testbare Vorhersage. Ein kontrolliertes Experiment mit zwei Pflanzengruppen unter verschiedenfarbigem Licht kann diese Behauptung direkt überprüfen.
    \end{solutiontermbox}
    \begin{solutiontermbox}{Satz}
        Die Summe der Kathetenquadrate ist gleich dem Hypotenusenquadrat.
        \tcblower
        \textbf{Begründung:} Dies ist der Satz des Pythagoras. Es handelt sich nicht um eine bloße Beobachtung, sondern um eine mathematische Wahrheit, die aus den Axiomen der Geometrie logisch und eindeutig bewiesen werden kann.
    \end{solutiontermbox}
    \begin{solutiontermbox}{Theorie}
        Die physikalischen Gesetze sind in allen Inertialsystemen gleich.
        \tcblower
        \textbf{Begründung:} Dies ist eines der beiden Postulate, auf denen Albert Einstein seine Spezielle Relativitätstheorie aufgebaut hat – ein komplexes Gedankengebäude, das die klassische Mechanik erweitert und Phänomene bei hohen Geschwindigkeiten erklärt.
    \end{solutiontermbox}
    \begin{solutiontermbox}{Gesetz}
        Jede Masse zieht jede andere Masse an.
        \tcblower
        \textbf{Begründung:} Dies ist eine qualitative Formulierung des Newtonschen Gravitationsgesetzes. Es beschreibt ein fundamentales und universell beobachtbares Muster in der Natur, ohne es im Detail zu erklären.
    \end{solutiontermbox}
    \begin{solutiontermbox}{Hypothese}
        Wenn man den Druck auf ein Gas erhöht, sinkt sein Volumen.
        \tcblower
        \textbf{Begründung:} Dies ist eine einfache, überprüfbare Annahme über das Verhalten von Gasen. Experimente können diesen Zusammenhang bestätigen und quantifizieren, was zur Formulierung von Gasgesetzen (z.B. Gesetz von Boyle-Mariotte) führt.
    \end{solutiontermbox}
\end{tcbraster}
\end{loesungbox}



\begin{loesungbox}{Lösung zu \Cref{aufg:umrechnung_einheiten}}
Umrechnen von Einheiten:
\begin{enumerate}
    \item Wir verwenden die Umrechnungen $\SI{1}{\nano\second} = 10^{-9}\,\si{\second}$ und $\SI{1}{\hour} = \SI{3600}{\second}$.
    \begin{equation}
        \SI{14.3}{\nano\second} = \SI{14.3e-9}{\second} = \num{14.3} \cdot 10^{-9} \cdot \frac{1}{3600}\si{\hour} \approx \SI{3.972e-12}{\hour}
    \end{equation}

    \item Wir verwenden $\SI{1}{\meter} = \SI{e-3}{\kilo\meter}$ und $\SI{1}{\second} = \frac{1}{3600}\,\si{\hour}$.
    \begin{equation}
        \SI{81.23}{\meter\per\second} = \num{81.23} \cdot \frac{10^{-3}\,\si{\kilo\meter}}{1/3600\,\si{\hour}} = \num{81.23} \cdot \SI{3.6}{\kilo\meter\per\hour} \approx \SI{292.43}{\kilo\meter\per\hour}
    \end{equation}

    \item Es gilt $\SI{1}{\liter} = \SI{1}{\deci\meter\cubed}$. Wir rechnen Dezimeter in Millimeter um: $\SI{1}{\deci\meter} = \SI{100}{\milli\meter}$.
    \begin{equation}
        \SI{1}{\litre} = \SI{1}{\deci\meter\cubed} = (\SI{100}{\milli\meter})^3 = \SI{e6}{\milli\meter\cubed}
    \end{equation}
    Damit folgt:
    \begin{equation}
        \SI{0.78}{\litre} = \num{0.78} \cdot \SI{e6}{\milli\meter\cubed} = \SI{780000}{\milli\meter\cubed} = \SI{7.8e5}{\milli\meter\cubed}
    \end{equation}

    \item Eine Wattsekunde ist ein Joule ($\SI{1}{\watt\second} = \SI{1}{\joule}$).
    \begin{equation}
        \SI{1}{\kWh} = \SI{1000}{\watt} \cdot \SI{3600}{\second} = \SI{3.6e6}{\watt\second} = \SI{3.6e6}{\joule}
    \end{equation}

    \item Die elektrische Energie $E$ ist das Produkt aus Leistung $P$ und Zeit $t$, wobei $P=U \cdot I$.
    \begin{equation}
        E = U \cdot I \cdot t = \SI{3,7}{\volt} \cdot (\SI{15000}{\ampere\hour}) = \SI{55500}{\watt\hour}
    \end{equation}
    Umrechnung in Kilowattstunden:
    \begin{equation}
        \SI{55500}{\watt\hour} = \frac{\num{55500}}{\num{1000}} \si{\kilo\watt\hour} = \SI{55.5}{\kilo\watt\hour}
    \end{equation}

    \item Die Definition von Bar ist $\SI{1}{\bar} = \SI{e5}{\pascal}$.
    \begin{equation}
        \SI{101325}{\pascal} = \frac{\num{101325}}{\num{e5}}\,\si{\bar} = \SI{1.01325}{\bar}
    \end{equation}

    \item Der Umrechnungsfaktor beträgt 1000, da $\SI{1}{\gram} = \SI{e-3}{\kilo\gram}$ und \\$\SI{1}{\centi\meter\cubed} = (\SI{e-2}{\meter})^3 = \SI{e-6}{\meter\cubed}$.
    \begin{equation}
        \SI{19.3}{\gram\per\centi\meter\cubed} = 19.3 \cdot \frac{\SI{e-3}{\kilo\gram}}{\SI{e-6}{\meter\cubed}} = 19.3 \cdot \SI{e3}{\kilo\gram\per\meter\cubed} = \SI{19300}{\kilo\gram\per\meter\cubed}
    \end{equation}

    \item Eine metrische Pferdestärke entspricht ca. $\SI{735.5}{\watt}$.
    \begin{equation}
        120\,\text{PS} \approx 120 \cdot \SI{735.5}{\watt} = \SI{88260}{\watt} = \SI{88.26}{\kilo\watt}
    \end{equation}
    
    \item Eine Umdrehung entspricht $2\pi\,\si{\radian}$ und eine Minute hat $\SI{60}{\second}$.
    \begin{equation}
        \SI{3000}{\per\minute} = 3000 \cdot \frac{2\pi\,\si{\radian}}{\SI{60}{\second}} = 100\pi\,\si{\radian\per\second} \approx \SI{314.16}{\radian\per\second}
    \end{equation}

    \item Die Umrechnung basiert auf der Elementarladung: $\SI{1}{\electronvolt} \approx \SI{1.602e-19}{\joule}$.
    \begin{equation}
        \SI{500}{\kilo\joule} = \SI{5e5}{\joule} \cdot \frac{\SI{1}{\electronvolt}}{\SI{1.602e-19}{\joule}} \approx \SI{3.12e24}{\electronvolt}
    \end{equation}
\end{enumerate}
\end{loesungbox}



\begin{loesungbox}{Lösung zu \Cref{aufg:dimensionsanalyse}}
\begin{enumerate}
    \item \textbf{Druck aus Dichte und Geschwindigkeit:}
    Wir stellen den Ansatz 
    \begin{equation}
        p \propto \rho^x \cdot v^y
    \end{equation} 
    auf und betrachten die Einheiten. Zuerst drücken wir Newton in SI-Basiseinheiten aus: $\si{\newton} = \si{\kilo\gram} \cdot \si{\meter\per\second\squared}$.
    \begin{equation}
        [p] = \frac{\si{\newton}}{\si{\meter\squared}} = \frac{\si{\kilo\gram\meter}}{\si{\second\squared}\cdot\si{\meter\squared}} = \frac{\si{\kilo\gram}}{\si{\meter\cdot\second\squared}}
    \end{equation}
    Nun setzen wir dies in den Einheitenansatz ein:
    \begin{equation}\begin{aligned}
        [p] &= [\rho]^x [v]^y \\
        \si{\kilo\gram} \cdot \si{\meter}^{-1} \cdot \si{\second}^{-2} = \frac{\si{\kilo\gram}}{\si{\meter\cdot\second\squared}} &= \left(\frac{\si{\kilo\gram}}{\si{\meter\cubed}}\right)^x \cdot \left(\frac{\si{\meter}}{\si{\second}}\right)^y \\
        &= \frac{\si{\kilo\gram}^x \cdot \si{\meter}^y}{\si{\meter}^{3x} \cdot \si{\second}^y} = \si{\kilo\gram}^x \cdot \si{\meter}^{y-3x} \cdot \si{\second}^{-y}
    \end{aligned}\end{equation}
    Durch Vergleich der Exponenten für jede Basiseinheit erhalten wir ein Gleichungssystem:
    \begin{itemize}
        \item \si{\kilo\gram}: $1 = x$
        \item \si{\second}: $-2 = -y \implies y = 2$
        \item \si{\meter}: $-1 = y - 3x \implies -1 = 2 - 3(1) \implies -1 = -1$ \;(\wA)
    \end{itemize}
    Die Exponenten sind also $x=1$ und $y=2$. Der gesuchte Zusammenhang ist:
    \begin{equation}\label{eq:druck_rho_v}
        p \propto \rho \cdot v^2
    \end{equation}
    
    \item \textbf{Fallzeit aus Höhe und Erdbeschleunigung:}
    Der Ansatz lautet 
    \begin{equation}
        t \propto H^x \cdot g^y \mDot
    \end{equation}
    Wir setzen die Einheiten ein:
    \begin{equation}\begin{aligned}
        \si{\second} &= (\si{\meter})^x \cdot \left(\frac{\si{\meter}}{\si{\second\squared}}\right)^y \\
        &= \frac{\si{\meter}^x \cdot \si{\meter}^y}{\si{\second}^{2y}} = \si{\meter}^{x+y} \cdot \si{\second}^{-2y}
    \end{aligned}\end{equation}
    Der Exponentenvergleich ergibt:
    \begin{itemize}
        \item \si{\second}: $1 = -2y \implies y = -1/2$
        \item \si{\meter}: $0 = x+y \implies x = -y = 1/2$
    \end{itemize}
    Der Zusammenhang ist also $t \propto H^{1/2} \cdot g^{-1/2}$.
    \begin{equation}\label{eq:fallzeit_h_g}
        t \propto \sqrt{\frac{H}{g}}
    \end{equation}
    
    \item \textbf{Auftriebskraft aus Volumen, Dichte und Erdbeschleunigung:}
    Der Ansatz lautet 
    \begin{equation}
        F \propto V^x \cdot \rho^y \cdot g^z \mDot
    \end{equation}
    Einsetzen der Einheiten:
    \begin{equation}\begin{aligned}
        \frac{\si{\kilo\gram\cdot\meter}}{\si{\second\squared}} &= (\si{\meter\cubed})^x \cdot \left(\frac{\si{\kilo\gram}}{\si{\meter\cubed}}\right)^y \cdot \left(\frac{\si{\meter}}{\si{\second\squared}}\right)^z \\
        &= \si{\meter}^{3x-3y+z} \cdot \si{\kilo\gram}^y \cdot \si{\second}^{-2z}
    \end{aligned}\end{equation}
    Der Exponentenvergleich liefert:
    \begin{itemize}
        \item \si{\kilo\gram}: $1 = y$
        \item \si{\second}: $-2 = -2z \implies z=1$
        \item \si{\meter}: $1 = 3x - 3y + z \implies 1 = 3x - 3(1) + 1 \implies 3 = 3x \implies x=1$
    \end{itemize}
    Alle Exponenten sind 1. Der gesuchte physikalische Zusammenhang ist das Archimedische Prinzip:
    \begin{equation}\label{eq:auftrieb}
        F \propto V \cdot \rho \cdot g
    \end{equation}
\end{enumerate}
\end{loesungbox}




\chapter{Fehlerrechnung und Messunsicherheit}\label{chap:fehlerrechnung}
\section{Aufgaben}\label{sec:fehlerrechnung_aufgaben}

\begin{aufgabebox}{Statistische Fehler}{stat_fehler}
\textbf{Schwingungsdauer eines Fadenpendels}

Messen Sie die Schwingungsdauer des Fadenpendels (z.B. aus einer Animation) mit Hilfe einer Stoppuhr insgesamt 10 Mal. Die Schwingungsdauer ist die Zeit für eine volle Hin- und Herbewegung.
\begin{center}
    \includegraphics[width=0.6\textwidth]{Bilder/Uebungsaufgaben/pendel_schwingungsdauer.png}
\end{center}
Tragen Sie Ihre Ergebnisse in eine Tabelle ein und berechnen Sie daraus:
\begin{enumerate}[label=(\alph*)]
    \item den Mittelwert,
    \item die Standardabweichung und
    \item den mittleren Fehler des arithmetischen Mittels.
\end{enumerate}
Erklären Sie, welche statistischen und welche systematischen Fehler Sie begehen, vor allem im Hinblick auf Ihr Ergebnis. Wie genau ist Ihr gemessener Wert?
\vspace{1em}
\begin{small}
\textit{Hinweis: Die Schwingungsdauer sollte im Intervall $T \in [\num{2}, \num{3}] \si{\second}$ liegen.}
\end{small}
\end{aufgabebox}



\newpage
\section{Lösungen}\label{sec:loesung_fehlerrechnung}

\begin{loesungbox}{Lösung zu \Cref{aufg:stat_fehler}}
Anhand von 10 Beispielmessungen werden die geforderten Größen berechnet.
\begin{center}
\begin{tabular}{c|c|c|c}
\toprule
 & \textbf{Messwerte} [\si{\second}] && \\
\hline
1. & \num{2.08} & & \\
2. & \num{2.02} & & \textbf{Mittelwert [\si{\second}]}\\
3. & \num{2.06} & &  \SI{2.00}{\second} \\
4. & \num{1.98} & &  \\
5. & \num{1.96} & & \textbf{Standardabweichung [\si{\second}]} \\
6. & \num{2.00} & & $\SI{0.045216}{\second} \approx \SI{0.05}{\second}$ \\
7. & \num{2.02} & & \\
8. & \num{1.94} & & \textbf{Mittlerer Fehler des}\\
9. & \num{1.98} & & \textbf{arithmetischen Mittels} [\si{\second}]\\
10. & \num{1.96} & & $\SI{0.014298}{\second} \approx \SI{0.01}{\second}$ \\
\bottomrule
\end{tabular}
\end{center}
\textbf{Statistische Fehler} (zufällige Fehler) entstehen durch nicht kontrollierbare Einflüsse während der Messung. Hier sind das \zB die menschliche Reaktionszeit beim Starten und Stoppen der Uhr oder das ungenaue Abschätzen des exakten Umkehrpunktes des Pendels. Diese Fehler streuen um den wahren Wert und können durch Mittelwertbildung reduziert werden. Der \textit{mittlere Fehler des arithmetischen Mittels} (\num{0.01}) gibt an, wie genau der von uns berechnete Mittelwert ist.

\textbf{Systematische Fehler} verschieben das gesamte Ergebnis in eine Richtung. Beispiele hierfür wären eine Stoppuhr, die konsequent zu schnell oder zu langsam läuft, oder wenn man die Schwingungsdauer systematisch zu kurz misst (\zB nur von der Mitte bis zum Umkehrpunkt). Solche Fehler werden durch Wiederholung der Messung nicht eliminiert.

Das Ergebnis der Messung wird oft als $T = (\text{Mittelwert} \pm \text{Unsicherheit})$ angegeben. Je nach Konvention wird als Unsicherheit die Standardabweichung oder der mittlere Fehler des Mittels verwendet. Nimmt man die Standardabweichung, lautet das Ergebnis:
\begin{equation}
    T = (\num{2.00} \pm \num{0.05})\,\si{\second}
\end{equation}
Dies bedeutet, dass ca. 68\% aller Einzelmessungen im Intervall $[\SI{1.95}{\second}, \SI{2.05}{\second}]$ liegen. Die Genauigkeit des Mittelwerts selbst ist jedoch höher und wird durch den mittleren Fehler von $\pm\SI{0.01}{\second}$ beschrieben.
\end{loesungbox}







\chapter{Kinematik}\label{chap:kinematik}
\section{Aufgaben}\label{sec:kinematik_aufgaben}

\begin{aufgabebox}{Bewegung eines Düsentriebwerks}{rakete}
Ein Düsentriebwerk bewegt sich auf einer Versuchsstrecke entlang der $x$-Achse. Sein Ort in Abhängigkeit der Zeit ist gegeben durch die Gleichung
\begin{equation}\label{eq:rakete_ort}
    x(t)=A t^{2}+B
\end{equation}
mit $A=\SI{2,10}{\meter\per\second\squared}$ und $B=\SI{3,0}{\meter}$. Das Triebwerk soll als punktförmige Masse angenommen werden.
\begin{center}
    \includegraphics[width=0.7\textwidth]{Bilder/Uebungsaufgaben/rakete_triebwerk.png}
    \refstepcounter{figure} % Erhöht den Abbildungszähler, ohne etwas auszugeben
    \label{fig: rakete_strecke}
\end{center}

\begin{enumerate}
    \item Ermitteln Sie die Durchschnittsgeschwindigkeit im Zeitintervall $t_{1}=\SI{3,0}{\second}$ bis $t_{2}=\SI{5,0}{\second}$.
    \item Ermitteln Sie die Momentangeschwindigkeit zum Zeitpunkt $t = \SI{4,2}{\second}$.
    \item Welchen Weg hat das Triebwerk zwischen $t_{1}=\SI{2,5}{\second}$ und $t_{2}=\SI{4,0}{\second}$ zurückgelegt?
    \item Wie groß ist die Durchschnittsbeschleunigung des Triebwerks zwischen $t_{1}=\SI{0}{\second}$ und $t_{2}=\SI{5}{\second}$? Wie groß ist die Momentanbeschleunigung?
\end{enumerate}
\end{aufgabebox}

\begin{aufgabebox}{Bahnkurve eines Fadenpendels}{fadenpendel}
Die Bahnkurve der (roten) Kugel am Ende eines Fadenpendels der Länge $l$, wie in der Skizze dargestellt, lautet
\begin{equation}\label{eq:pendel_r_t}
    \ivec{r}(t)=\icolTwo{l\cdot \sin(\phi(t))}{-l\cdot \cos(\phi(t))} \mComma
\end{equation}
wobei der Winkel gegeben ist durch
\begin{equation}\label{eq:pendel_phi_t}
    \phi(t)=\phi_{\max}\cdot \cos\left(\sqrt{\frac{g}{l}}t\right) \mDot
\end{equation}
Hier sind $l=\SI{4}{\meter}$ die Länge des Fadens, $g\approx\SI{9,81}{\meter\per\second\squared}$ die Erdbeschleunigung und $\phi_{\max}=\SI{0,5}{\radian}$ die maximale Auslenkung.\newline
\textit{Tipp: Rechnen Sie das Beispiel zunächst ohne Zahlen und setzen Sie die Werte erst am Ende zum Berechnen von Zahlenwerten ein!}
\begin{center}
    \centering
    \includegraphics[width=0.4\textwidth]{Bilder/Uebungsaufgaben/bahnkurve_fadenpendel.png}
    \refstepcounter{figure} % Erhöht den Abbildungszähler, ohne etwas auszugeben
    \label{fig: fadenpendel_skizze}
\end{center}
\begin{enumerate}
    \item Zu welchem Zeitpunkt $t>0$ ist die Kugel das erste Mal genau in der Mitte (bei $\phi=0$)?
    \item Berechnen Sie die Momentangeschwindigkeit $\ivec{v}(t)$ als Funktion der Zeit. Wenden Sie dazu die Kettenregel auf $\ivec{r}(\phi(t))$ an.
    \item Was ist die Maximalgeschwindigkeit $v_{\max}$ der Kugel? \newline
    \textit{Tipp: } $v(t) = |\ivec{v}(t)|$
    \item Die Graphen für $r_x(t)$ und $v_x(t)$ sind gegeben. Bestimmen Sie die Steigung der Tangente von $r_x(t)$ zu den Zeitpunkten $t \in \{1,0;\, 2,0;\, 3,5\}\,\si{\second}$ und überprüfen Sie Ihre Werte anhand der Kurve $v_x(t)$.\newline
    \textit{Hinweis:} 
    \begin{equation}
    \ivec{r}(t) = \icolTwo{r_x(t)}{r_y(t)} \mComma \quad \ivec{v}(t) = \icolTwo{v_x(t)}{v_y(t)} \mDot
    \end{equation}
    % \begin{center}
    %     \centering
    %     \includegraphics[width=0.6\textwidth]{Bilder/Uebungsaufgaben/fadenpendel_r_x_von_t.png}
    %     \includegraphics[width=0.6\textwidth]{Bilder/Uebungsaufgaben/fadenpendel_v_x_von_t.png}
    %     \refstepcounter{figure} % Erhöht den Abbildungszähler, ohne etwas auszugeben
    %     \label{fig: fadenpendel_rx_vx_von_t}
    % \end{center}

    \begin{center}
            \begin{tikzpicture}
                \begin{axis}[
                    grid=major,
                    axis lines=middle,
                    axis line style = thick,
                    xlabel={\Large $t$},
                    ylabel={\Large $r_x(t)$},
                    ylabel style={at={(ticklabel cs:1.07)}, anchor=west, rotate=0},
                    xmin=0, xmax=4,
                    ymin=-2.2, ymax=2.2,
                    xtick={0,1,2,3,4},
                    ytick={-2,-1,0,1,2},
                    minor tick num=1,
                    domain=0:4,
                    samples=200, % More samples for smoother curve
                    legend pos=south east,
                    enlargelimits=false,
                    axis equal image=false,
                ]
                \pgfmathsetmacro{\lval}{4}
                \pgfmathsetmacro{\gval}{9.81}
                \pgfmathsetmacro{\phimaxval}{0.5} % radians
                \pgfmathsetmacro{\omegavall}{sqrt(\gval/\lval)} % omega = sqrt(g/l) approx 1.566
            
            
                \addplot[blue, line width=1.5pt] { \lval * sin(deg(\phimaxval * cos(deg(\omegavall * x)))) };
                \addlegendentry{\large $r_x(t)$}
                \end{axis}
            \end{tikzpicture}
            
            \begin{tikzpicture}
                \begin{axis}[
                    grid=major,
                    axis lines=middle,
                    axis line style = thick,
                    xlabel={\Large $t$},
                    ylabel={\Large $v_x(t)$},
                    ylabel style={at={(ticklabel cs:1.07)}, anchor=west, rotate=0},
                    xmin=0, xmax=4,
                    ymin=-3.5, ymax=3.5,
                    xtick={0,1,2,3,4},
                    ytick={-3,-2,-1,0,1,2,3},
                    minor tick num=1,
                    domain=0:4,
                    samples=200, % More samples for smoother curve
                    legend pos=south east,
                    enlargelimits=false,
                    axis equal image=false,
                ]
                \pgfmathsetmacro{\lval}{4}
                \pgfmathsetmacro{\gval}{9.81}
                \pgfmathsetmacro{\phimaxval}{0.5} % radians
                \pgfmathsetmacro{\omegavall}{sqrt(\gval/\lval)} % omega = sqrt(g/l) approx 1.566
                \addplot[orange,line width=1.5pt] { -\lval * \phimaxval * \omegavall * cos(deg(\phimaxval * cos(\omegavall * x))) * sin(deg(\omegavall * x)) };
                \addlegendentry{\large $v_x(t)$}
                \end{axis}
            \end{tikzpicture}
            \label{fig: fadenpendel_rx_vx_von_t}
    \end{center}

\end{enumerate}
\end{aufgabebox}


\begin{aufgabebox}{Geschwindigkeit eines Läufers}{laeufer}
Usain Bolt absolvierte die \SI{100}{\meter} bei seinem Weltrekord im Jahr 2009 in \SI{9.58}{\second}. Die folgende Tabelle zeigt seine Zwischenzeiten.
\begin{center}
\begin{tabular}{l*{5}{c}}
\toprule
Reaktionszeit & \SI{20}{\meter} & \SI{40}{\meter} & \SI{60}{\meter} & \SI{80}{\meter} & \SI{100}{\meter} \\
\midrule
\SI{0.146}{\second} & \SI{2.89}{\second} & \SI{4.64}{\second} & \SI{6.31}{\second} & \SI{7.92}{\second} & \SI{9.58}{\second} \\
\bottomrule
\end{tabular}
\includegraphics[width=0.35\textwidth]{Bilder/Uebungsaufgaben/laeufer_bolt_weltrekord.png}
\end{center}
\begin{enumerate}
    \item Welche Durchschnittsgeschwindigkeit hatte er bei diesem Lauf?
    \item Die \SI{40}{\meter}-Marke erreichte er nach \SI{4.64}{\second}. Berechnen Sie die Durchschnittsgeschwindigkeit bis dahin und die für die restlichen \SI{60}{\meter}!
    \item Warum ist der gewichtete Mittelwert dieser beiden Durchschnittsgeschwindigkeiten gleich der Durchschnittsgeschwindigkeit aus Punkt 1?
    \item Tragen Sie seine Zeiten in ein Weg-Zeit-Diagramm ein. Wie können Sie die Durchschnittsgeschwindigkeiten der einzelnen Teilintervalle aus dem Graphen ablesen? Berücksichtigen Sie die Reaktionszeit.
\end{enumerate}
\end{aufgabebox}



\begin{aufgabebox}{Freier Fall}{freier_fall}
Ein Apfel fällt unter dem Einfluss der konstanten Erdbeschleunigung $g=\SI{9.81}{\meter\per\second\squared}$ aus einem Baum zu Boden. Seine Beschleunigung beträgt somit $a_x = \SI{-9.81}{\meter\per\second\squared}$. Zu Beginn befindet sich der Apfel auf einer Höhe von $x_0 = \SI{4.00}{\meter}$ mit einer Anfangsgeschwindigkeit von $v_{0,x} = \SI{0.00}{\meter\per\second}$. 
\begin{center}
    \includegraphics[width=0.45\textwidth]{Bilder/Uebungsaufgaben/freier_fall_apfel.png}
    \label{fig:freier_fall}
\end{center}
\begin{enumerate}
    \item Leiten Sie aus der konstanten Beschleunigung durch Integration die Momentangeschwindigkeit $v_x(t)$ her. Verwenden Sie dabei das angegebene Koordinatensystem und die gegebenen Anfangsbedingungen. 
    \item Auf einem Ast in $\SI{2.50}{\meter}$ Höhe sitzt ein Vogel. Mit welcher Geschwindigkeit fliegt der Apfel am Vogel vorbei? 
    \item Mit welcher Geschwindigkeit schlägt der Apfel am Boden auf? 
    \item Leiten Sie aus der Momentangeschwindigkeit $v_x(t)$ die Ortskurve $x(t)$ durch bestimmte Integration her. 
    \item Wie lange dauert es, bis der Apfel am Boden aufschlägt? 
\end{enumerate}
\end{aufgabebox}


\begin{aufgabebox}{Der schräge Wurf}{schraeger_wurf}
Ein Teilchen wird unter dem Einfluss der Erdbeschleunigung, die in negative $z$-Richtung wirkt ($a_z = -g = \SI{-9.81}{\meter\per\second\squared}$), mit einer Anfangsgeschwindigkeit $\ivecS{v}{0} = \inlrowThree{v_{0,x}}{0}{v_{0,z}}$ abgeworfen (siehe Skizze). Der Startpunkt liegt bei $\ivecS{r}{0} = \inlrowThree{0}{0}{h}$. Rechnen Sie durchgehend mit der Variable $g$ und setzen Sie erst am Ende Zahlenwerte ein. 
\begin{center}
    \includegraphics[width=0.5\textwidth]{Bilder/Uebungsaufgaben/schräger_Wurf_beispiel.png}
    \label{fig:schraeger_wurf}
\end{center}
\begin{enumerate}
    \item Schreiben Sie die Ortskurve $\ivec{r}(t)$ vektoriell an, wobei Sie die geeignete Formel für den Ortsvektor aus dem Skriptum nehmen können.
    \item Begründen Sie mathematisch auf Basis von 1), warum die Bewegung in $x$-Richtung (horizontale Bewegung) und die Bewegung in $z$-Richtung (vertikale Bewegung) voneinander entkoppelt also unabhängig voneinander sind. 
    \item Stellen Sie die Wurfparabel $z(x)$ auf, die in der Skizze abgebildet ist. Gehen Sie dazu wie folgt vor:
    \begin{enumerate}
        \item Stellen Sie die Bewegungsgleichung $x(t)$ nach der Zeit $t$ um, um $t(x)$ zu erhalten. 
        \item Setzen Sie diesen Ausdruck für die Zeit $t(x)$ in die Bewegungsgleichung $z(t)$ ein, um $z(t(x)) = z(x)$ zu erhalten. 
    \end{enumerate}
    \item Berechnen Sie die Formel für die $x$-Koordinate des Scheitelpunkts $x_S$ der Flugbahn. \newline
    Tipp: Im Scheitelpunkt ist die Steigung der Wurfparabel null, es gilt also $z'(x_S) = 0$. 
    \item Berechnen Sie die Wurfweite $x_W$. Tipp: Lösen Sie hierfür die quadratische Gleichung $z(x_W) = 0$. 
    \item Berechnen Sie den Scheitelpunkt $x_S$, die Wurfweite $x_W$ sowie die Flugdauer $t^*$, für die $x(t^*) = x_W$ gilt. Verwenden Sie die Werte $h=\SI{3}{\meter}$ und $\ivecS{v}{0} = (\num{2.0};\, \num{0};\, \num{1.3})\,\si{\meter\per\second}$. 
\end{enumerate}
\end{aufgabebox}



\begin{aufgabebox}{Winkelgeschwindigkeit}{winkelgeschwindigkeit}
Die abgebildete Kreisscheibe rotiert mit einer konstanten Winkelgeschwindigkeit von $\omega=\SI{3.2}{\radian\per\second}$. Der Punkt (1) hat einen Abstand von $r_1 = \SI{2}{\meter}$ und der Punkt (2) einen Abstand von $r_2 = \SI{4}{\meter}$ vom Kreismittelpunkt. 
\begin{center}
\resizebox{0.55\linewidth}{!}{
    \begin{tikzpicture}[
        >=Stealth,
        axis/.style={->, line width=1.3pt, black},
        disk/.style={fill=lightgray!70, draw=black, very thick},
        point/.style={circle, fill=black, inner sep=1.9pt},
        label/.style={font=\sffamily\bfseries, black},
        dimension/.style={thick, blue,<|-|>},
        arrow/.style={line width=2.5pt, black, -{Stealth[length=4mm, width=3mm]}}
    ]
        % Kreismittelpunkt (Ursprung)
        \node[circle, draw, fill=white, inner sep=1.5pt] (center) at (0,0) {};
        % Kreisscheibe
        \draw[disk] (center) circle (4cm); % Radius der Scheibe ist 5cm, um die Punkte zu umschließen
    
        \node[point, label=punkt1] (P1) at (0, 2) {};
        \node[above right=1mm of P1, label] {\large (1)};
        \node[point, label=punkt2] (P2) at (4, 0) {};
        \node[below right=1mm of P2, label] {\large (2)};
    
        % Dimensionen für Punkte (1,2)
        \draw[dimension] (-0.2, 0) -- (-0.2, 2) node[midway, left=2mm, blue] {$\SI{2}{\metre}$};
        \draw[dimension] (0, -0.2) -- (4, -0.2) node[midway, below=2mm, blue] {$\SI{4}{\metre}$};
    
        % Winkelgeschwindigkeit Pfeil
        \draw[arrow] (5,0) arc [start angle=0, end angle=50, radius=5cm] node[midway, above right=1mm, label] {\Large $\omega$}; 
    
        % Koordinatensystem
        \draw[axis] (-5.2,0) -- (5.6,0) node[right] {\Large $x$};
        \draw[axis] (0,-5.2) -- (0,5.2) node[above] {\Large $y$};
    \end{tikzpicture}
    }
    \label{fig:winkgesch}
\end{center}

\begin{enumerate}
    \item Wie groß ist der Betrag der Bahngeschwindigkeit $v$ für den Punkt (1) und für den Punkt (2)? 
    \item Wie groß sind die Beträge der Zentripetalbeschleunigungen $a$ der beiden Punkte? 
    \item Wie viele Umdrehungen pro Sekunde absolvieren die beiden Punkte? \newline
    Tipp: Der Zusammenhang zwischen der Frequenz $f$ und der Winkelgeschwindigkeit lautet $\omega = 2\pi f$. 
    \item Der Ortsvektor des Punktes (2) lautet 
    $$\ivecS{r}{2}(t) = r_2 \cdot \icolTwo{\cos(\omega t)}{\sin(\omega t)} = 4 \cdot \icolTwo{\cos(\num{3.2} t)}{\sin(\num{3.2} t)} = \mDot$$ 
    Wie lautet der Ortsvektor $\ivecS{r}{1}(t)$ des Punktes (1)? Beachten Sie den Startwinkel $\varphi_0$ zur Zeit $t=0$. 
\end{enumerate}
\end{aufgabebox}



\begin{aufgabebox}{Künstliche Schwerkraft in einer Raumstation}{raumstation}
Sie entwerfen eine Raumstation, die aus einem großen, ringförmigen Habitat mit einem Durchmesser von \SI{300}{\metre} besteht. Der Ring rotiert, um durch die Zentripetalkraft eine künstliche Schwerkraft zu erzeugen. Die Bewohner sollen auf der Innenseite des Rings eine Beschleunigung erfahren, die der Erdbeschleunigung $g=\SI{9.81}{\metre\per\second\squared}$ entspricht.
\begin{center}
     \includegraphics[width=0.42\textwidth]{Bilder/Uebungsaufgaben/raumstation.png}
\end{center}
\begin{enumerate}
    \item Berechnen Sie die erforderliche Winkelgeschwindigkeit $\omega$ (in \si{\radian\per\second}), mit der sich der Ring drehen muss.
    \item Geben Sie die Rotationsgeschwindigkeit auch in Umdrehungen pro Minute an.\\
    \textit{Tipp:} $\omega = 2\pi f$ und $[f] = \si{\per\second}$.
\end{enumerate}
\end{aufgabebox}




\begin{aufgabebox}{Idealer Überhöhungswinkel}{ueberhoehungswinkel}
Eine Autobahnausfahrt mit einem Kurvenradius von $r=\SI{120}{\metre}$ wird als überhöhte Kurve für eine Geschwindigkeit von \SI{80}{\kilo\metre\per\hour} gebaut. Die Kurvenüberhöhung sorgt dafür, dass ein Teil der Normalkraft als Zentripetalkraft wirkt. Beim idealen Überhöhungswinkel ist keine Reibungskraft notwendig, um das Fahrzeug bei der vorgesehenen Geschwindigkeit in der Spur zu halten.
\begin{center}
    \includegraphics[width=0.5\textwidth]{Bilder/Uebungsaufgaben/idealerÜberhöhungswinkel.png}
\end{center}
\begin{enumerate}
    \item Zeichnen Sie alle wirkenden Kräfte für den Fall des idealen Überhöhungswinkels in die Skizze ein.
    \item Stellen Sie das Kräftegleichgewicht in vertikaler ($y$-) und horizontaler ($x$-) Richtung auf. Berücksichtigen Sie, dass in horizontaler Richtung eine Netto-Zentripetalkraft wirken muss.
    \item Leiten Sie aus dem Kräftegleichgewicht die Formel für den idealen Überhöhungswinkel $\theta_{\text{opt}}$ her, indem Sie die beiden Gleichungen geeignet dividieren. \\
    \textit{Lösung: } $\theta_\text{opt} = \arctan(v^2/(r\cdot g))$.
    \item Berechnen Sie den optimalen Überhöhungswinkel für die gegebene Kurve in Grad.
    \item Erklären Sie, welche Kraft das Auto in der Spur halten würde, wenn es deutlich langsamer als die ideale Geschwindigkeit führe. Zeichnen Sie die Richtung dieser zusätzlichen Kraft in einer neuen Skizze ein.
\end{enumerate}
\end{aufgabebox}





\begin{aufgabebox}{Looping}{looping_geschwindigkeit}
Ein Achterbahnwagen mit einer Gesamtmasse von $M = \SI{1.5}{t}$ durchfährt einen vertikalen, kreisförmigen Looping mit einem Durchmesser von $d = \SI{30}{\metre}$. Die Reibung soll vernachlässigt werden.
\begin{center}
    \includegraphics[width=0.42\textwidth]{Bilder/Uebungsaufgaben/looping.png}
\end{center}
\begin{enumerate}
    \item Leiten Sie eine Formel für die Mindestgeschwindigkeit $v_{\text{oben}}$ her, die der Wagen am höchsten Punkt des Loopings haben muss, damit er nicht von der Schiene fällt. Betrachten Sie hierfür den Grenzfall, bei dem die Normalkraft gerade null wird und die Gewichtskraft allein die notwendige Zentripetalkraft aufbringt.
    \item Nutzen Sie den Energieerhaltungssatz, um die minimale Anfangsgeschwindigkeit $v_{0,\text{min}}$ am Fuße des Loopings ($h=0$) zu bestimmen, die erforderlich ist, um den höchsten Punkt mit der berechneten Geschwindigkeit $v_{\text{oben}}$ zu erreichen.
    \item Berechnen Sie den numerischen Wert für $v_{0,\text{min}}$ in \si{\meter\per\second} sowie in \si{\kilo\meter\per\hour}. (Gegeben: $g = \SI{9.81}{\meter\per\second\squared}$)
\end{enumerate}
\end{aufgabebox}






\newpage
\section{Lösungen}\label{sec:kinematik_loesungen}

\begin{loesungbox}{Lösung zu \Cref{aufg:rakete}}
Gegeben: $x(t) = (\SI{2,10}{\meter\per\second\squared}) \cdot t^2 + \SI{3,0}{\meter}$.
\begin{enumerate}
    \item \textbf{Durchschnittsgeschwindigkeit:}
    \begin{multline}
        \overline{v} = \frac{\Delta x}{\Delta t} = \frac{x(\num{5,0})-x(\num{3,0})}{\num{5,0}-\num{3,0}} = \frac{(\num{2,1}\cdot 5^2+\SI{3}) - (\num{2,1}\cdot 3^2 + \num{3})}{\num{2}} = \\
        = \frac{\num{55,5} - \num{21,9}}{\num{2}} = \SI{16,8}{\meter\per\second} \mDot
    \end{multline}
    
    \item \textbf{Momentangeschwindigkeit:}
    Zuerst wird die Geschwindigkeitsfunktion durch Ableiten von $x(t)$ bestimmt:
    \begin{equation}
        v(t) = \frac{\dd x(t)}{\dd t} = 2A\cdot t \mDot
    \end{equation}
    Zum Zeitpunkt $t = \SI{4,2}{\second}$ ergibt sich somit die Geschwindigkeit:
    \begin{equation}
        v(\num{4,2}) = 2 \cdot \num{2,1} \cdot \num{4,20} = \SI{17,64}{\meter\per\second} \mDot
    \end{equation}
    
    \item \textbf{Zurückgelegter Weg:}
    \begin{multline}
        \Delta x = x(\num{4,0}) - x(\num{2,5}) = (\num{2,1} \cdot 4^2 + \num{3}) - (\num{2,1}\cdot 2,5^2 + \num{3}) = \\
        = \SI{36,6}{\meter} - \SI{16,125}{\meter} = \SI{20,475}{\meter} \mDot
    \end{multline}
    
    \item \textbf{Beschleunigung:}
    Die Beschleunigungsfunktion ist die Ableitung der Geschwindigkeit:
    \begin{equation}
        a(t) = \frac{\dd v(t)}{\dd t} = 2A = \SI{4,20}{\meter\per\second\squared} = \const \mDot
    \end{equation}
    Da die Beschleunigung konstant ist, ist die Momentanbeschleunigung zu jedem Zeitpunkt gleich der Durchschnittsbeschleunigung in jedem beliebigen Intervall.
    \begin{equation}
        \overline{a} = \frac{\Delta v}{\Delta t} = \frac{v(\num{5}) - v(\num{0})}{\num{5} - \num{0}} = \frac{\num{21} - \num{0}}{\num{5}} = \SI{4,20}{\meter\per\second\squared} \mDot
    \end{equation}
\end{enumerate}
\end{loesungbox}

\begin{loesungbox}{Lösung zu \Cref{aufg:fadenpendel}}
\begin{enumerate}
    \item \textbf{Zeitpunkt des Nulldurchgangs:}
    Gesucht ist die Zeit $t_1>0$, für die $\phi(t_1) \eqexcl 0$ gilt:
    \begin{equation}
        \phi(t_1) = \phi_{\max}\cdot \cos\left(\sqrt{\frac{g}{l}}t_1\right) \eqexcl 0 \implies \cos\left(\sqrt{\frac{g}{l}}t_1\right) = 0 \mDot
    \end{equation}
    Die Kosinusfunktion wird zum ersten Mal bei $\pi/2$ null:
    \begin{equation}
        \sqrt{\frac{g}{l}}t_1 = \frac{\pi}{2} \implies t_1 = \frac{\pi}{2}\sqrt{\frac{l}{g}} \approx \SI{1,004}{\second} \mDot
    \end{equation}
    
    \item \textbf{Momentangeschwindigkeit:}
    Mit der Kettenregel $\frac{\dd\ivec{r}}{\dd t} = \frac{\dd\ivec{r}}{\dd\phi} \frac{\dd\phi}{\dd t}$:
    \begin{align}
        \frac{\dd\ivec{r}}{\dd\phi} &= \icolTwo{l\cdot \cos(\phi)}{l\cdot \sin(\phi)} \\
        \frac{\dd\phi}{\dd t} &= -\phi_{\max} \sin\left(\sqrt{\frac{g}{l}}t\right) \cdot \sqrt{\frac{g}{l}}
    \end{align}
    Zusammengesetzt ergibt sich der Geschwindigkeitsvektor:
    \begin{equation}\label{eq: v_vek_fadenpendel}\begin{aligned}
        \ivec{v}(t) &= \frac{\dd\ivec{r}}{\dd\phi} \frac{\dd\phi}{\dd t} \\
        &= -\phi_{\max} \sqrt{\frac{g}{l}} \sin\left(\sqrt{\frac{g}{l}}t\right) \cdot \icolTwo{l\cdot \cos(\phi(t))}{l\cdot \sin(\phi(t))} \\
        &= \underbrace{\phi_{\max} \sqrt{g\cdot l} \sin\left(\sqrt{\frac{g}{l}}t\right)}_{v(t)} \cdot \underbrace{\icolTwo{-\cos(\phi(t))}{-\sin(\phi(t))}}_{\ivecS{e}{v}(t)} \mDot
    \end{aligned}\end{equation}
    
    \item \textbf{Maximalgeschwindigkeit:}
    Der Betrag der Geschwindigkeit ist 
    $$v(t) = |\ivec{v}(t)| = \left| \phi_{\max} \sqrt{gl} \sin\left(\sqrt{\frac{g}{l}}t\right) \right| \mComma$$ laut \cref{eq: v_vek_fadenpendel} ist. Die Geschwindigkeit wird maximal, wenn der Sinus-Term den Wert $\pm 1$ annimmt.
    \begin{equation}
        v_{\max} = \phi_{\max}\sqrt{gl} = \SI{0,5}{} \cdot \sqrt{\SI{9,81}{\meter\per\second\squared} \cdot \SI{4}{\meter}} \approx \SI{3,13}{\meter\per\second} \mDot
    \end{equation}
    Dies geschieht, wenn die Kugel durch die Mittellage schwingt ($t=t_1$).
    
    \item \textbf{Graphische Überprüfung:}
    Die Steigung der Tangente im Weg-Zeit-Diagramm $r_x(t)$ entspricht dem Wert der Geschwindigkeit im Geschwindigkeits-Zeit-Diagramm $v_x(t)$ zum selben Zeitpunkt.
    \begin{itemize}
        \item Bei $t=\SI{1,0}{\second}$: Die Tangente an $r_x(t)$ hat eine stark negative Steigung. Im Graphen für $v_x(t)$ lesen wir einen Wert von ca. $\SI{-3,1}{\meter\per\second}$ ab.
        \item Bei $t=\SI{2,0}{\second}$: Die Tangente an $r_x(t)$ ist horizontal, die Steigung ist also null. Im Graphen für $v_x(t)$ ist der Wert exakt $\SI{0}{\meter\per\second}$.
        \item Bei $t=\SI{3,5}{\second}$: Die Tangente an $r_x(t)$ hat eine positive Steigung. Im Graphen für $v_x(t)$ lesen wir einen Wert von ca. $\SI{2,2}{\meter\per\second}$ ab.
    \end{itemize}
    Die graphisch ermittelten Steigungen stimmen gut mit den Werten der Geschwindigkeitskurve überein.
    \begin{center}
        \centering
        \includegraphics[width=0.6\textwidth]{Bilder/Uebungsaufgaben/fadenpendel_loesung_rx_vx.png}
        \refstepcounter{figure} % Erhöht den Abbildungszähler, ohne etwas auszugeben
        \label{fig: fadenpendel_loesung_rx_vx_von_t}
    \end{center}
\end{enumerate}
\end{loesungbox}


\begin{loesungbox}{Lösung zu \Cref{aufg:laeufer}}
\begin{enumerate}
    \item \textbf{Durchschnittsgeschwindigkeit über die Gesamtstrecke:}
    Die Durchschnittsgeschwindigkeit $\bar{v}$ ist die Gesamtstrecke $\Delta s$ geteilt durch die Gesamtzeit $\Delta t$
    \begin{equation}
        \bar{v} = \frac{\Delta s}{\Delta t} = \frac{0-(-100)}{9,58-0} = \frac{\SI{100}{\meter}}{\SI{9.58}{\second}} \approx \SI{10.44}{\meter\per\second} \approx \SI{37.58}{\kilo\meter\per\hour} \mDot
    \end{equation}
    
    \item \textbf{Durchschnittsgeschwindigkeiten auf Teilstrecken:}
    \begin{itemize}
        \item \textbf{Erste \SI{40}{\meter}:}
        \begin{equation}
            \bar{v}_{[0,40]} = \frac{\SI{40}{\meter}}{\SI{4.64}{\second}} \approx \SI{8.62}{\meter\per\second}
        \end{equation}
        \item \textbf{Restliche \SI{60}{\meter}:} Die Strecke ist $\SI{100}{\meter} - \SI{40}{\meter} = \SI{60}{\meter}$. Die Zeit dafür ist $\SI{9.58}{\second} - \SI{4.64}{\second} = \SI{4.94}{\second}$.
        \begin{equation}
            \bar{v}_{[40,100]} = \frac{\SI{60}{\meter}}{\SI{4.94}{\second}} \approx \SI{12.15}{\meter\per\second}
        \end{equation}
        Man sieht deutlich, dass er in der zweiten Hälfte des Rennens viel schneller war.
    \end{itemize}
    
    \item \textbf{Gewichteter Mittelwert:}
    Die Gesamtdurchschnittsgeschwindigkeit ist der \textit{zeitlich} gewichtete Mittelwert der Teilgeschwindigkeiten. Die Gewichte sind die Anteile der Zeitintervalle an der Gesamtzeit.
    \begin{equation}
        \bar{v} = \frac{\Delta t_1}{\Delta t_{\text{ges}}} \cdot \bar{v}_1 + \frac{\Delta t_2}{\Delta t_{\text{ges}}} \cdot \bar{v}_2
    \end{equation}
    \begin{equation}
        \bar{v} = \frac{\SI{4.64}{\second}}{\SI{9.58}{\second}} \cdot \SI{8.62}{\meter\per\second} + \frac{\SI{4.94}{\second}}{\SI{9.58}{\second}} \cdot \SI{12.15}{\meter\per\second} \approx \SI{10.44}{\meter\per\second}
    \end{equation}
    Das Ergebnis stimmt mit der Berechnung aus 1) überein.
    
    \item \textbf{Weg-Zeit-Diagramm:}
    Trägt man den Weg $s$ über der Zeit $t$ auf, entspricht die Durchschnittsgeschwindigkeit in einem Intervall der \textbf{Steigung der Sekante} zwischen dem Anfangs- und Endpunkt des Intervalls.
    \begin{equation*}
        \bar{v}_i = \frac{\Delta s_i}{\Delta t_i}
    \end{equation*}
    Die Reaktionszeit von \SI{0.146}{\second} ist der $t$-Achsenabschnitt, \gDh in dieser Zeit bewegt sich der Läufer nicht ($s=0$). Der Lauf selbst beginnt erst bei $t=\SI{0.146}{\second}$. Für die Berechnung der Geschwindigkeiten ist jedoch die gestoppte Zeit ab dem Startschuss relevant.
    
    \begin{center}
        \begin{tikzpicture}
            \begin{axis}[
                xlabel={Zeit $t$ [\si{\second}]},
                ylabel={Weg $s$ [\si{\meter}]},
                xmin=0, xmax=10,
                ymin=0, ymax=105,
                grid=major,
                legend pos=north west
            ]
            \addplot[
                color=blue,
                mark=*,
                mark size=1.5pt
            ] coordinates {
                (0,0)
                (0.146, 0)
                (2.89, 20)
                (4.64, 40)
                (6.31, 60)
                (7.92, 80)
                (9.58, 100)
            };
            \legend{Usain Bolt (2009)}
            
            % Steigungsdreieck als Beispiel
            \draw[red, thick] (axis cs:2.89,20) -- (axis cs:4.64,20);
            \draw[red, thick] (axis cs:4.64,20) -- (axis cs:4.64,40);
            \node[red, anchor=west] at (axis cs:4.8, 30) {$\Delta s_i$};
            \node[red, anchor=north] at (axis cs:3.7, 18) {$\Delta t_i$};
            \node[red, anchor=south,fill=white] at (axis cs:3.0, 42) {$\bar{v}_i = \frac{\Delta s_i}{\Delta t_i}$};
            \end{axis}
        \end{tikzpicture}
        \captionof{figure}{Weg-Zeit-Diagramm von Usain Bolts Weltrekordlauf.}
        \label{fig:bolt_lauf}
    \end{center}
\end{enumerate}
\end{loesungbox}



\begin{loesungbox}{Lösung zu \Cref{aufg:freier_fall}}
\begin{enumerate}
    \item \textbf{Momentangeschwindigkeit $v_x(t)$:} \\
    Die Geschwindigkeit ergibt sich aus dem Integral der Beschleunigung $a_x = -g$ über die Zeit. Wir verwenden ein bestimmtes Integral mit den Grenzen von $t'=0$ bis $t'=t$:
    \begin{equation}\begin{gathered}
        a_x = -g \\
        \int_{0}^{t} a_x \dd t' = \int_{0}^{t} g \dd t' 
    \end{gathered}\end{equation}
    Integration der linken Seite ergibt 
    \begin{equation}
        \int_{0}^{t} a_x \dd t' = \int_{0}^{t} \frac{\dd v'_x}{\dd t'} \dd t' = \int_{v_{0,x}}^{v_x(t)} \dd v'_x = v'_x \; \bigg|_{v_{0,x}}^{v_x(t)} = v_x(t) - v_{0,x} \mDot
    \end{equation}
    Die rechte Seite ergibt hingegegen 
    \begin{equation}
        \int_{0}^{t} (-g) \dd t' = -g\cdot  t' \; \bigg|_{0}^{t} = -g\cdot  t \mDot
    \end{equation}
    Mit der Anfangsbedingung $v_{0,x} = \SI{0}{\meter\per\second}$ folgt durch das Gleichsetzen von linker und rechter Seite 
    \begin{equation}\label{eq:freier_fall_v}
        v_x(t) = v_{0,x} - g\cdot t = -g\cdot  t \mDot
    \end{equation}

    \item \textbf{Geschwindigkeit auf Höhe des Vogels:} \\
    Zuerst benötigen wir die Ortskurve $x(t)$, die wir durch Integration von $v_x(t)$ erhalten: 
    \begin{equation}
        \int_{x_0}^{x(t)} \dd x' = \int_{0}^{t} v_x(t') \dd t' \implies x(t) - x_0 = \int_{0}^{t} (-g t') \dd t' = -\frac{1}{2}g t^2 \mDot
    \end{equation}
    Wir bringen die Anfangsbedingung $x_0$ auf die andere Seite und erhalten die Ortskurve
    \begin{equation}\label{eq:freier_fall_x}
        x(t) = x_0 - \frac{1}{2}g t^2 \mDot
    \end{equation}
    Wir berechnen den Zeitpunkt $t_V$, zu dem der Apfel die Höhe des Vogels $x(t_V) \eqexcl \SI{2.5}{\meter}$ erreicht: 
    \begin{equation}\begin{aligned}
        x(t_V) = x_0 - \frac{1}{2}gt^2 \implies t_V^2 = \frac{2}{g}(x_0 - x(t_V)) \\
        \implies t_V = \sqrt{\frac{2(\SI{4.0}{\meter} - \SI{2.5}{\meter})}{\SI{9.81}{\meter\per\second\squared}}} \approx \SI{0.553}{\second}
    \end{aligned}\end{equation}
    
    Die Geschwindigkeit zu diesem Zeitpunkt beträgt laut \cref{eq:freier_fall_v}: 
    \begin{equation}
        v_x(t_V) = -g\cdot t_V =-\left(\SI{9.81}{\meter\per\second\squared}\right) \cdot \SI{0.553}{\second} \approx \SI{-5.42}{\meter\per\second}
    \end{equation}
    
    \item \textbf{Aufprallgeschwindigkeit am Boden:} \\
    Am Boden gilt $x(t_E) = 0$. Wir berechnen die Fallzeit $t_E$: 
    \begin{equation}
        0 = x_0 - \frac{1}{2}g t_E^2 \implies t_E = \sqrt{\frac{2x_0}{g}} = \sqrt{\frac{2 \cdot \SI{4.0}{\meter}}{\SI{9.81}{\meter\per\second\squared}}} \approx \SI{0.903}{\second}
    \end{equation}
    Die Geschwindigkeit beim Aufprall ist:
    \begin{equation}
        v_x(t_E) = -g t_E = -\left(\SI{9.81}{\meter\per\second\squared}\right) \cdot \SI{0.903}{\second} \approx \SI{-8.86}{\meter\per\second}
    \end{equation}
    
    \item \textbf{Ortskurve $x(t)$:} \\
    Diese wurde bereits in Aufgabenteil 2 hergeleitet, siehe \Cref{eq:freier_fall_x}:
    \begin{equation}
        x(t) = x_0 - \frac{1}{2}gt^2 = \num{4.0} - \frac{\num{9.81}}{2} t^2
    \end{equation}
    
    \item \textbf{Dauer des Falls:} \\
    Diese wurde in Aufgabenteil 3 als $t_E$ berechnet und beträgt ca. $\SI{0.90}{\second}$. 
\end{enumerate}
\end{loesungbox}




\begin{loesungbox}{Lösung zu \Cref{aufg:schraeger_wurf}}
\begin{enumerate}
    \item \textbf{Vektorielle Ortskurve $\ivec{r}(t)$:} \\
    Die allgemeine Formel für eine Bewegung mit konstanter Beschleunigung $\ivec{a}$ lautet 
    $$\ivec{r}(t) = \frac{1}{2}\ivec{a}t^2 + \ivecS{v}{0}t + \ivecS{r}{0} \mDot$$
    Mit $\ivec{a} = \icolThree{0}{0}{-g}$, $\ivecS{v}{0} = \icolThree{v_{0,x}}{0}{v_{0,z}}$ und $\ivecS{r}{0} = \icolThree{0}{0}{h}$ erhalten wir für die Ortskurve 
    \begin{equation}\label{eq:wurf_r_t}
        \ivec{r}(t) = \frac{1}{2}\icolThree{0}{0}{-g}t^2 + \icolThree{v_{0,x}}{0}{v_{0,z}}t + \icolThree{0}{0}{h} = \icolThree{v_{0,x}t}{0}{h + v_{0,z}t - \frac{1}{2}gt^2}
    \end{equation}
    
    \item \textbf{Unabhängigkeit der Bewegungen:} \\
    Die Bewegungsgleichungen für die einzelnen Komponenten lauten:
    \begin{align}
        x(t) &= v_{0,x}t \mComma \label{eq: x_t_schraeg_wurf}\\
        y(t) &= 0 \mComma  \nonumber\\
        z(t) &= h + v_{0,z}t - \frac{1}{2}gt^2 \mDot \label{eq: z_t_schraeg_wurf}
    \end{align}
    Man erkennt, dass die Gleichung für $x(t)$ nur von $x$-Komponenten der Anfangsbedingungen abhängt und die Gleichung für $z(t)$ nur von $z$-Komponenten (und der Erdbeschleunigung, die nur eine $z$-Komponente hat). Es gibt keine Terme, die die Bewegungen in $x$- und $z$-Richtung koppeln. Die Zeit $t$ ist ein gemeinsamer Parameter, der die beiden Bewegungen synchronisiert.
    
    \item \textbf{Wurfparabel $z(x)$:} \\
    Wir stellen die \cref{eq: x_t_schraeg_wurf} für $x(t)$ nach der Zeit $t$ um:
    \begin{equation}
        t(x) = \frac{x}{v_{0,x}} \label{eq:t_von_x}
    \end{equation}
    Dies setzen wir in die \cref{eq: z_t_schraeg_wurf} für $z(t)$ ein: 
    \begin{equation}\label{eq:wurfparabel}
        z(x) = h + v_{0,z}\left(\frac{x}{v_{0,x}}\right) - \frac{1}{2}g\left(\frac{x}{v_{0,x}}\right)^2 = -\frac{g}{2v_{0,x}^2}x^2 + \frac{v_{0,z}}{v_{0,x}}x + h
    \end{equation}
    Dies ist die sogenannte Wurfparabel $z(x)$. 
    
    \item \textbf{Scheitelpunkt $x_S$:} \\
    Wir leiten $z(x)$ nach $x$ ab und setzen die Ableitung gleich null: 
    \begin{equation}
        z'(x) = \frac{\dd z}{\dd x} = -\frac{g}{v_{0,x}^2}x + \frac{v_{0,z}}{v_{0,x}} \label{eq:wurfparabel_abl}
    \end{equation}
    \begin{equation}
        z'(x_S) \eqexcl 0 \implies \frac{g}{v_{0,x}^2}x_S = \frac{v_{0,z}}{v_{0,x}} \implies x_S = \frac{v_{0,x}v_{0,z}}{g} \label{eq:scheitelpunkt_xs}
    \end{equation}
    
    \item \textbf{Wurfweite $x_W$:} \\
    Wir lösen die quadratische Gleichung $z(x_W)=0$ für $x_W$:
    \begin{equation}
        -\frac{g}{2v_{0,x}^2}x_W^2 + \frac{v_{0,z}}{v_{0,x}}x_W + h = 0
    \end{equation}
    Die Lösungsformel $x_{1,2} = \frac{-b \pm \sqrt{b^2-4ac}}{2a}$ mit $a=-\frac{g}{2v_{0,x}^2}$, $b=\frac{v_{0,z}}{v_{0,x}}$ und $c=h$ ergibt nach ein paar Umformungen:
    \begin{equation}\label{eq:wurfweite_xw}
        x_W = \frac{v_{0,x}}{g}\left(v_{0,z} + \sqrt{v_{0,z}^2 + 2gh}\right) \mDot
    \end{equation}
    
    \item \textbf{Numerische Berechnung:} \\
    Mit den gegebenen Werten $h=\SI{3}{\meter}$, $v_{0,x}=\SI{2.0}{\meter\per\second}$, $v_{0,z}=\SI{1.3}{\meter\per\second}$ und $g=\SI{9.81}{\meter\per\second\squared}$:
    \begin{align*}
        x_S &= \frac{v_{0,x}v_{0,z}}{g} = \frac{\SI{2.0}{\meter\per\second} \cdot \SI{1.3}{\meter\per\second}}{\SI{9.81}{\meter\per\second\squared}} \approx \SI{0.265}{\meter} \\ 
        x_W &= \frac{v_{0,x}}{g}\left(v_{0,z} + \sqrt{v_{0,z}^2 + 2gh}\right) \\
        & = \frac{\SI{2.0}{\meter\per\second}}{\SI{9.81}{\meter\per\second\squared}}\left(\SI{1.3}{\meter\per\second} + \sqrt{(\SI{1.3}{\meter\per\second})^2 + 2 \cdot \SI{9.81}{\meter\per\second\squared} \cdot \SI{3}{\meter}}\right) \approx \SI{1.85}{\meter} \\ 
        t^* &= \frac{x_W}{v_{0,x}} = \frac{\SI{1.85}{\meter}}{\SI{2.0}{\meter\per\second}} \approx \SI{0.925}{\second} 
    \end{align*}
    Die Flugdauer kommt aus der Formel $t = t(x) = \frac{x}{v_{0,x}}$, die für $x = x_W$ (Wurfweite) zur Flugdauer $t^*$ wird.
\end{enumerate}
\end{loesungbox}



\begin{loesungbox}{Lösung zu \Cref{aufg:winkelgeschwindigkeit}}
\begin{enumerate}
    \item \textbf{Betrag der Bahngeschwindigkeit:} \\
    Der Zusammenhang zwischen Bahngeschwindigkeit, Winkelgeschwindigkeit und Radius lautet $v = \omega \cdot r$. 
    \begin{align}
        v_1 &= \omega \cdot r_1 = \SI{3.2}{\radian\per\second} \cdot \SI{2}{\meter} = \SI{6.4}{\meter\per\second} \\ 
        v_2 &= \omega \cdot r_2 = \SI{3.2}{\radian\per\second} \cdot \SI{4}{\meter} = \SI{12.8}{\meter\per\second} 
    \end{align}
    
    \item \textbf{Betrag der Zentripetalbeschleunigung:} \\
    Die Zentripetalbeschleunigung ist gegeben durch $a = \omega^2 \cdot r = v^2/r$. 
    \begin{align}
        a_1 &= \omega^2 \cdot r_1 = \left(\SI{3.2}{\radian\per\second}\right)^2 \cdot \SI{2}{\meter} = \SI{20.48}{\meter\per\second\squared} \\ 
        a_2 &= \omega^2 \cdot r_2 = \left(\SI{3.2}{\radian\per\second}\right)^2 \cdot \SI{4}{\meter} = \SI{40.96}{\meter\per\second\squared} 
    \end{align}
    
    \item \textbf{Umdrehungen pro Sekunde (Frequenz):} \\
    Alle Punkte auf der Scheibe haben dieselbe Frequenz. Wir stellen die Formel $\omega = 2\pi f$ nach $f$ um:
    \begin{equation}
        f = \frac{\omega}{2\pi} = \frac{\SI{3.2}{\radian\per\second}}{2\pi} \approx \SI{0.509}{\hertz}
    \end{equation}
    
    \item \textbf{Ortsvektor $\ivecS{r}{1}(t)$:} \\
    Die allgemeine Form für den Ortsvektor bei einer Kreisbewegung lautet:
    \begin{equation}
        \ivec{r}(t) = \icolTwo{r \cdot \cos(\omega t + \phi_0)}{r \cdot \sin(\omega t + \phi_0)}
    \end{equation}
    wobei $\phi_0$ der Phasenwinkel zur Zeit $t=0$ ist. Für Punkt (2) auf der $x$-Achse ist $\phi_0=\SI{0}{\degree} = \SI{0}{\radian}$ und sein Ortsvektor lautet 
    $$ \ivecS{r}{2}(t) = r \cdot  \icolTwo{\cos(\omega t)}{\sin(\omega t)} =  4\cdot  \icolTwo{\cos(\num{3.2} t)}{\sin(\num{3.2} t)}$$
    Punkt (1) befindet sich zur Zeit $t=0$ auf der positiven $y$-Achse, sein Startwinkel beträgt also $\phi_0 = \SI{90}{\degree} = \pi/2\,\si{\radian}$. 
    Damit lautet der Ortsvektor für Punkt (1) mit $r_1=\SI{2}{\meter}$:
    \begin{equation}
        \ivecS{r}{1}(t) = 2 \cdot \icolTwo{\cos(3.2 \cdot t + \frac{\pi}{2})}{\sin(3.2 \cdot t + \frac{\pi}{2})}
    \end{equation}
\end{enumerate}
\end{loesungbox}



\begin{loesungbox}{Lösung zu \Cref{aufg:raumstation}}
\begin{enumerate}
    \item \textbf{Winkelgeschwindigkeit:}
    Der Radius des Rings beträgt $r = d/2 = \SI{300}{\metre} / 2 = \SI{150}{\metre}$. Die künstliche Schwerkraft wird durch die Zentripetalbeschleunigung erzeugt, die gleich $g$ sein soll.
    \begin{equation}\label{eq:raumstation_azp}
        a_{Zp} = \omega^2 \cdot r \eqexcl g
    \end{equation}
    Umstellen nach der Winkelgeschwindigkeit $\omega$:
    \begin{equation}\label{eq:raumstation_omega}
        \omega = \sqrt{\frac{g}{r}} = \sqrt{\frac{\SI{9.81}{\metre\per\second\squared}}{\SI{150}{\metre}}} \approx \SI{0.2557}{\radian\per\second} \mDot
    \end{equation}
    \item \textbf{Umdrehungen pro Minute:}
    Zuerst wird die Frequenz $f$ in Hertz (\si{\per\second}) berechnet:
    \begin{equation}\label{eq:raumstation_frequenz}
        f = \frac{\omega}{2\pi} = \frac{\SI{0.2557}{\radian\per\second}}{2\pi} \approx \SI{0.0407}{\per\second}
    \end{equation}
    Umrechnung in Umdrehungen pro Minute (rpm):
    \begin{equation}\label{eq:raumstation_rpm}
        f_{\text{rpm}} = f \cdot \SI{60}{\second\per\minute} = \SI{0.0407}{\per\second} \cdot \SI{60}{\second\per\minute} \approx \SI{2.44}{\per\minute} \mDot
    \end{equation}
    Die Raumstation muss sich also etwa $2,44$ Mal pro Minute drehen.
\end{enumerate}
\end{loesungbox}




\begin{loesungbox}{Lösung zu \Cref{aufg:ueberhoehungswinkel}}
\begin{enumerate}
    \item \textbf{Kräfteskizze:} Die wirkenden Kräfte sind die Gewichtskraft $\ivecS{F}{G}$ (vertikal nach unten) und die Normalkraft $\ivecS{F}{N}$ (senkrecht zur Fahrbahn). Bei idealer Geschwindigkeit gibt es keine Reibungskraft.
    \begin{center}
        \includegraphics[width=0.5\textwidth]{Bilder/Uebungsaufgaben/lösung_überhöhung.png}
    \end{center}
    \item \textbf{Kräftegleichgewicht:}
    In $y$-Richtung (vertikal) heben sich die Kräfte auf, es gibt keine Beschleunigung.
    \begin{equation}\label{eq:ueberhoehung_y}
        \sum F_{y} = F_{N} \cos(\theta) - m g = 0 \implies F_{N} \cos(\theta) = mg
    \end{equation}
    In $x$-Richtung (horizontal) wirkt die resultierende Kraft als Zentripetalkraft $\ivecS{F}{\text{Zp}}$.
    \begin{equation}\label{eq:ueberhoehung_x}
        \sum F_{x} = F_{N} \sin(\theta) = F_{\text{Zp}} = \frac{mv^{2}}{r}
    \end{equation}
    \item \textbf{Herleitung des Winkels:}
    Dividiert man \Cref{eq:ueberhoehung_x} durch \Cref{eq:ueberhoehung_y}, erhält man:
    \begin{equation}\label{eq:ueberhoehung_tan}
        \frac{F_{N}\sin(\theta)}{F_{N}\cos(\theta)} = \frac{mv^{2}/r}{mg} \implies \tan(\theta) = \frac{v^{2}}{r g}
    \end{equation}
    Daraus folgt für den idealen Überhöhungswinkel:
    \begin{equation}\label{eq:ueberhoehung_arctan}
        \theta_{\text{opt}} = \arctan\left(\frac{v^{2}}{r g}\right) \mDot
    \end{equation}
    \item \textbf{Berechnung des Winkels:}
    Zuerst wird die Geschwindigkeit in SI-Einheiten umgerechnet: $v = \SI{80}{\kilo\metre\per\hour} \approx \SI{22.22}{\metre\per\second}$.
    \begin{equation}\label{eq:ueberhoehung_wert}
        \theta_{\text{opt}} = \arctan\left(\frac{(\SI{22.22}{\m\per\s})^{2}}{\SI{120}{\m} \cdot \SI{9.81}{\m\per\s\squared}}\right) = \arctan(0,4194) \approx \SI{22.76}{\degree} \mDot
    \end{equation}
    \item \textbf{Fahrt mit geringerer Geschwindigkeit:}
    Fährt das Auto langsamer, ist die benötigte Zentripetalkraft geringer, die horizontale Komponente der Normalkraft ist jedoch unverändert und somit zu groß. Das Auto würde daher nach innen den Hang hinabrutschen. Die \textbf{Haftreibungskraft} wirkt dieser Tendenz entgegen, also parallel zur Fahrbahn nach außen (den Hang hinauf; siehe oranger Pfeil in Grafik oben).
\end{enumerate}
\end{loesungbox}




\begin{loesungbox}{Lösung zu \Cref{aufg:looping_geschwindigkeit}}
\begin{enumerate}
    \item \textbf{Geschwindigkeit am höchsten Punkt} \\
    Am höchsten Punkt des Loopings muss die nach unten wirkende Zentripetalkraft mindestens so groß sein wie die ebenfalls nach unten wirkende Gewichtskraft, damit der Wagen nicht herunterfällt. Die Zentripetalkraft wird durch die Normalkraft $F_N$ und die Gewichtskraft $F_G$ aufgebracht: $F_{\text{Zp}} = F_N + F_G$. Im Grenzfall, dass der Wagen gerade noch Kontakt hält, ist die Normalkraft $F_N=0$.
    \begin{align}\label{eq:looping_kraftansatz}
        F_{\text{Zp,min}} & \eqexcl F_G  \nonumber \\
        \frac{m v_{\text{oben}}^2}{r} &= m g
    \end{align}
    Daraus folgt für die Geschwindigkeit am höchsten Punkt (mit dem Radius $r = d/2 = \SI{15}{\m}$):
    \begin{equation}\label{eq:v_oben_looping}
        v_{\text{oben}} = \sqrt{gr} = \sqrt{\SI{9.81}{\meter\per\second\squared} \cdot \SI{15}{\meter}} \approx \SI{12.13}{\meter\per\second}
    \end{equation}

    \item \textbf{Anfangsgeschwindigkeit mittels Energieerhaltung} \\
    Die mechanische Gesamtenergie (kinetische plus potentielle Energie) bleibt ohne Reibung erhalten. Wir setzen die Energie am Startpunkt ($h=0$) gleich der Energie am höchsten Punkt ($h=d=2r$).
    \begin{align}\label{eq:energieerhaltung_looping}
        E_{\text{unten}} &= E_{\text{oben}} \nonumber \\
        E_{\text{kin,unten}} + E_{\text{pot,unten}} &= E_{\text{kin,oben}} + E_{\text{pot,oben}} \nonumber \\
        \frac{1}{2}m v_{0,\text{min}}^2 + 0 &= \frac{1}{2}m v_{\text{oben}}^2 + mg(2r)
    \end{align}
    Wir setzen $v_{\text{oben}}^2 = gr$ aus \Cref{eq:v_oben_looping} ein:
    \begin{align}
        \frac{1}{2}m v_{0,\text{min}}^2 &= \frac{1}{2}m(gr) + 2mgr = \frac{5}{2}mgr \nonumber \\
        v_{0,\text{min}}^2 &= 5gr \implies v_{0,\text{min}} = \sqrt{5gr}
    \end{align}

    \item \textbf{Numerische Berechnung} \\
    Wir setzen die gegebenen Werte ein:
    \begin{equation}\label{eq:v0_min_looping_ms}
        v_{0,\text{min}} = \sqrt{5 \cdot \SI{9.81}{\meter\per\second\squared} \cdot \SI{15}{\meter}} = \sqrt{\SI{735.75}{\square\meter\per\square\second}} \approx \SI{27.12}{\meter\per\second}
    \end{equation}
    Umrechnung in \si{\kilo\meter\per\hour}:
    \begin{equation}\label{eq:v0_min_looping_kmh}
        v_{0,\text{min}} \approx \SI{27.12}{\meter\per\second} \cdot \num{3.6} \approx \SI{97.65}{\kilo\meter\per\hour}
    \end{equation}
\end{enumerate}
\end{loesungbox}




\chapter{Dynamik}
\section{Aufgaben}\label{sec:anwendungen_aufgaben}
\begin{aufgabebox}{Gravitationskraft}{gravitation}
In diesem Beispiel betrachten wir die Größenordnungen der Gravitationskraft zwischen verschiedenen Objekten. Verwenden Sie für die Berechnungen die folgenden Werte:
\begin{center}
\begin{tabular}{l l  S[table-format=1.2e2]}
\toprule
\textbf{Größe} & \textbf{Symbol} & \textbf{Wert} \\
\midrule
Gravitationskonstante & $G$ & \SI{6.67e-11}{\newton\meter\squared\per\kilo\gram\squared} \\ 
Masse der Erde & $m_{\text{Erde}}$ & \SI{5.972e24}{\kilo\gram} \\ 
Radius der Erde & $r_{\text{Erde}}$ & \SI{6372}{\kilo\meter} \\ 
Masse des Mondes & $m_{\text{Mond}}$ & \SI{7.346e22}{\kilo\gram} \\ 
Radius des Mondes & $r_{\text{Mond}}$ & \SI{1737.5}{\kilo\meter} \\ 
Abstand Erde-Mond & $d_{\text{EM}}$ & \SI{384400}{\kilo\meter} \\ 
Masse eines Menschen & $m_{\text{Mensch}}$ & \SI{80}{\kilo\gram} \\ 
\bottomrule
\end{tabular}
\end{center}
\begin{center}
    \includegraphics[width=0.5\textwidth]{Bilder/Uebungsaufgaben/gravity_erde-mond.png}
    \label{fig:gravitation}
\end{center}
\begin{enumerate}
    \item Wie groß ist die Gravitationskraft $\ivecS{F}{G}$, welche die Erde auf einen Menschen auf der Erdoberfläche ausübt? 
    \item Wie groß ist die Gravitationskraft $\ivecS{F}{G}$, die der Mond auf einen Menschen auf der Erdoberfläche ausübt? 
    \item Wie groß ist die gravitative Anziehungskraft zwischen zwei Menschen mit je einer Masse von $m_{\text{Mensch}}$ in einem Abstand von $\SI{0.5}{\meter}$? 
    \item Berechnen Sie mit den obigen Werten die Fallbeschleunigung $g_{\text{Erde}}$ auf der Erdoberfläche und die Beschleunigung $g_{\text{Mond}}$, die durch den Mond auf der Erdoberfläche verursacht wird. 
\end{enumerate}
\end{aufgabebox}


\begin{aufgabebox}{Konisches Pendel}{konisches_pendel}
Ein Ball der Masse $m$ ist an einem Seil der Länge $l$ aufgehängt und rotiert mit einer konstanten Winkelgeschwindigkeit $\omega$ um die $z$-Achse. Auf den Ball wirken die Gewichtskraft $\ivecS{F}{G} = m\ivec{g}$ und die Seilkraft $\ivecS{F}{S}$. Dabei stellt sich ein konstanter Winkel $\theta$ zur Vertikalen ein, den wir berechnen wollen. 
\begin{center}
    \includegraphics[width=0.53\textwidth]{Bilder/Uebungsaufgaben/thetherbal_angabe.png}
    \hfill
    \includegraphics[width=0.38\textwidth]{Bilder/Uebungsaufgaben/thetherbal_Kraftschnitt.png}
    \label{fig:konisches_pendel_skizze}
\end{center}
\begin{enumerate}
    \item Beschriften Sie im Freikörperdiagramm (rechts) die wirkenden Kräfte. 
    \item Drücken Sie die $x$- und $z$-Komponenten der Gewichtskraft ($\ivecS{F}{G}$) und der Seilkraft ($\ivecS{F}{S}$) mithilfe des Winkels $\theta$ aus. 
    $F_{G,x} = ?$, $F_{G,z} = ?$, $F_{S,x} = ?$, $F_{S,z} = ?$.
    \item Stellen Sie die Bewegungsgleichungen nach dem zweiten Newtonschen Gesetz für die $x$- und $z$-Richtung auf. Die Summe der Kräfte in jeder Richtung entspricht dem Produkt aus Masse und Beschleunigung in dieser Richtung. 
    \begin{align*}
        \sum_{i} F_{x,i} &= m a_x \\
        \sum_{i} F_{z,i} &= m a_z
    \end{align*}
    \item Der Ball führt eine Kreisbewegung mit konstantem Winkel $\theta$ aus. Argumentieren Sie, welche der folgenden Optionen für die Beschleunigungskomponenten $a_x$ und $a_z$ zutrifft:
    \begin{itemize}
        \item Null Beschleunigung ($a=0$)
        \item Zentripetalbeschleunigung ($a_{\text{Zp}} = \omega^2 r$)
    \end{itemize}
    Ersetzen Sie anschließend $a_x$ und $a_z$ in Ihren Gleichungen aus 3) durch die korrekten Ausdrücke. 
    \item Schreiben Sie das fertige Gleichungssystem explizit auf. 
    \item Lösen Sie das Gleichungssystem nach dem Winkel $\theta$. Gehen Sie dazu wie folgt vor: Lösen Sie die $z$-Gleichung nach dem Betrag der Seilkraft $F_S = |\ivecS{F}{S}|$ auf und setzen Sie das Ergebnis in die $x$-Gleichung ein. 
    \textit{Lösung:} $g \cdot \tan(\theta) = \omega^2 r$.
    \item Ersetzen Sie den Radius $r$ durch einen Ausdruck, der die Seillänge $l$ und den Winkel $\theta$ enthält ($r = l \sin(\theta)$), und lösen Sie die Gleichung nach $\theta$. Was fällt Ihnen auf, wenn Sie den resultierenden Zusammenhang für $\theta$ als Funktion von $\omega$ betrachten (\zB für $l = \SI{1}{\meter}$)? 
\end{enumerate}
\end{aufgabebox}

\begin{aufgabebox}{Reibungsfreies Gleiten}{schiefe_ebene}
Ein Block der Masse $m$ gleitet reibungsfrei über einen horizontalen Boden und anschließend eine schiefe Ebene mit Neigungswinkel $\theta$ hinauf. Die Geschwindigkeit des Blocks kurz vor der Rampe beträgt $v_0$. Der Block erreicht eine maximale Höhe $h$, bevor er zur Ruhe kommt. Ziel ist es zu zeigen, dass diese Höhe $h$ unabhängig von der Masse $m$ und dem Winkel $\theta$ ist. Verwenden Sie unbedingt das angegebene, geneigte Koordinatensystem (rechte Abbildung). 
\begin{center}
    \includegraphics[width=0.5\textwidth]{Bilder/Uebungsaufgaben/reibungsfreies_gleiten_angabe.png}
    \hfill
    \includegraphics[width=0.42\textwidth]{Bilder/Uebungsaufgaben/reibungsfreies_gleiten_details.png}
    \label{fig:schiefe_ebene_skizze}
\end{center}
\begin{enumerate}
    \item Zerlegen Sie den Vektor der Erdbeschleunigung $\ivec{g}$ in seine Komponenten $g_x$ und $g_y$ im geneigten Koordinatensystem und tragen Sie diese in die Skizze ein. Geben Sie die Formeln für $g_x$ und $g_y$ als Funktion von $\theta$ an. 
    \item Die Bewegung findet nur entlang der $x$-Achse des geneigten Systems statt. Stellen Sie die Bewegungsgleichung $x(t)$ für den Block auf der Rampe auf. Verwenden Sie hierfür die allgemeine Formel für die Bahnkurve $\ivec{r}(t) = \ivecS{r}{0} + \ivecS{v}{0} t + \frac{1}{2}\ivec{a}t^2$ und setzen Sie die korrekten Anfangsbedingungen ein. \newline
    \textit{Tipp: Der Koordinatenursprung kann frei gewählt werden.}
    \item Leiten Sie die Geschwindigkeitsfunktion $v_x(t)$ durch Ableiten von $x(t)$ her. Berechnen Sie den Zeitpunkt $t^*$, zu dem der Block zum Stillstand kommt $[v_x(t^*) = 0]$. 
    \item Der zurückgelegte Weg entlang der Rampe ist $s = x(t^*)$. Berechnen Sie $s$ und verwenden Sie das Ergebnis, um die maximale Höhe $h$ zu bestimmen. \newline
    \textit{Tipp:} $h = s\cdot \sin(\theta)$
    \item \textbf{Bonusfrage:} Warum ist die erreichte Höhe $h$ unabhängig vom Winkel $\theta$? Würde dieser Zusammenhang auch gelten, wenn Reibung vorhanden wäre? 
\end{enumerate}
\end{aufgabebox}




\begin{aufgabebox}{Rutsche mit Reibung}{rutsche}
Ein Kind mit einer Masse von $m = \SI{35}{\kilogram}$ startet aus dem Stillstand ($v_0 = \SI{0}{\meter\per\second}$) am oberen Ende einer geraden Rutsche. Die Rutsche hat eine Höhe von $h = \SI{3}{\metre}$ und eine Länge von $L = \SI{5}{\metre}$. Der Gleitreibungskoeffizient zwischen Kind und Rutsche beträgt $\mu_{\text{Gleit}} = 0.2$. Die Erdbeschleunigung ist $g = \SI{9.81}{\metre\per\second\squared}$.
\begin{center}
    \includegraphics[width=0.49\textwidth]{Bilder/Uebungsaufgaben/kind_rutsche.png}
\end{center}
\begin{enumerate}
    \item Zeichnen Sie alle auf das Kind wirkenden Kräfte (Gewichtskraft, Normalkraft, Gleitreibungskraft) in der Skizze ein.
    \item Berechnen Sie den Neigungswinkel $\theta$ der Rutsche.
    \item Zerlegen Sie die Gewichtskraft in eine Komponente senkrecht ($F_{G, \perp}$) und eine parallel ($F_{G, \parallel}$) zur Rutschfläche. Geben Sie die Formeln und Zahlenwerte an.
    \item Berechnen Sie die Normalkraft $F_N$ und daraus die Gleitreibungskraft $F_{R, \text{Gleit}}$.
    \item Berechnen Sie die resultierende Nettokraft in Bewegungsrichtung und daraus die konstante Beschleunigung des Kindes entlang der Rutsche.
    \item Die kinematischen Gleichungen für die gleichförmig beschleunigte Bewegung lauten $s(t) = s_0 + v_0 t + \frac{1}{2} a t^2$ und $v(t) = v_0 + at$. Leiten Sie aus diesen eine zeitunabhängige Formel für die Geschwindigkeit $v$ nach einer zurückgelegten Strecke $L$ her (also $v(L)$), wenn die Bewegung aus der Ruhe startet. \\
    \textit{Lösung: } $v = \sqrt{2aL}$.
    \item Berechnen Sie mit der hergeleiteten Formel die Endgeschwindigkeit des Kindes am unteren Ende der Rutsche.
\end{enumerate}
\end{aufgabebox}




\begin{aufgabebox}{Reibungsfreies Schieben}{schiefe_ebene_schieben}
Ein kleiner Block mit der Masse $m$ ruht auf der geneigten Seite eines dreieckigen Blocks mit der Masse $M$, der selbst wiederum auf einem reibungsfreien Tisch steht.
\begin{center}
    \includegraphics[width=0.46\linewidth]{Bilder/Uebungsaufgaben/schieben_schiefe_ebene_reibfrei.png}
\end{center}
Bestimmen Sie die Horizontalkraft $\ivec{F}$, die auf den großen Block $M$ ausgeübt werden muss, damit der kleine Block $m$ relativ zu $M$ in seiner Position verharrt (d.h., $m$ rutscht auf der schiefen Ebene weder nach oben noch nach unten). Nehmen Sie an, dass alle Flächen reibungsfrei sind.

\begin{enumerate}
    \item Zeichnen Sie ein Freikörperbild für die Masse $m$ mit allen auf sie wirkenden Kräften.
    \item Stellen Sie die Kräftegleichgewichte in $x$- und $y$-Richtung auf. Beachten Sie dabei, dass sich das gesamte System mit $a_{x}$ beschleunigt, aber in $y$-Richtung keine Beschleunigung stattfindet ($\sum F_{y} = 0$).
    \item Leiten Sie aus den Gleichungen die benötigte Beschleunigung $a_x$ her.
    \item Bestimmen Sie die Gesamtkraft $F$, die auf das System mit der Gesamtmasse $m+M$ wirken muss, um diese Beschleunigung zu erzeugen.
\end{enumerate}
\end{aufgabebox}



\begin{aufgabebox}{Rollreibung}{rollreibung}
Wir analysieren den Ausrollweg eines Autos, das ausschließlich durch Rollreibung abgebremst wird. Es soll die Abhängigkeit von der Masse des Autos und der Gummimischung untersucht werden. Setzen Sie erst am Ende der Rechnung Zahlenwerte ein!
\begin{center}
    \includegraphics[width=0.5\textwidth]{Bilder/Uebungsaufgaben/rollreibung_auto.png}
    \label{fig:rollreibung_skizze}
\end{center}
\begin{enumerate}
    \item Das Auto hat eine Masse von $m_{\text{Auto}} = \SI{1500}{\kilo\gram}$. Der Rollreibungskoeffizient beträgt $\mu_{R,r}^S = \num{0.013}$ für Sommerreifen und $\mu_{R,r}^W = \num{0.015}$ für Winterreifen. Berechnen Sie die jeweilige Rollreibungskraft. 
    \item Das Auto hat eine Anfangsgeschwindigkeit von $v_0 = \SI{70}{\kilo\meter\per\hour}$ und ist mit Sommerreifen ausgestattet. Berechnen Sie, wie weit das Auto rollt, bis es zum Stillstand kommt. Gehen Sie dazu schrittweise vor:
    \begin{enumerate}
        \item Bestimmen Sie die (negative) Beschleunigung $a_x$, die durch die Rollreibung verursacht wird. 
        \item Setzen Sie $a_x$ in die Geschwindigkeitsgleichung $v_x(t) = v_0 + a_x t$ ein. 
        \item Berechnen Sie den Zeitpunkt $t^*$, zu dem das Auto anhält $[v_x(t^*) = 0]$. 
        \item Setzen Sie $t^*$ in die Ortsgleichung $x(t) = x_0 + v_0 t + \frac{1}{2}a_x t^2$ ein, um den Ausrollweg $x_W = x(t^*) - x_0$ zu erhalten. \newline
        \textit{Lösung: } $x_W = v_0^2 /(2 \mu_{R,r} \cdot g)$
        \item Berechnen Sie den numerischen Wert für den Ausrollweg. 
    \end{enumerate}
    \item Die Formel für den Ausrollweg ist unabhängig von der Masse des Autos. Warum verbraucht ein schwereres Auto dennoch mehr Treibstoff, um eine konstante Geschwindigkeit zu halten? \newline
    \textit{Tipp:} Betrachten Sie die Arbeit (Energie = Kraft $\times$ Weg). 
\end{enumerate}
\end{aufgabebox}

\begin{aufgabebox}{Haftreibung}{haftreibung}
Eine Masse $m_1$ liegt auf einer Tischplatte und ist über eine reibungsfreie Rolle mit einer Masse $m_2 = \SI{2.0}{\kilo\gram}$ verbunden. Der Haftreibungskoeffizient zwischen $m_1$ und dem Tisch beträgt $\mu_{R,h} = \num{0.4}$. Wir wollen die minimale Masse $m_1$ bestimmen, die erforderlich ist, damit das System in Ruhe verharrt. 
\begin{center}
    \includegraphics[width=0.5\textwidth]{Bilder/Uebungsaufgaben/haftreibung_zwei-massen_seil.png}
    \label{fig:haftreibung_skizze}
\end{center}
\begin{enumerate}
    \item Zeichnen Sie ein geeignetes Koordinatensystem und tragen Sie alle (für die Dynamik) relevanten Kräfte für beide Massen als Vektoren in die Skizze ein. 
    \item Damit sich die Masse $m_1$ nicht bewegt, muss die Summe aller horizontalen Kräfte null sein ($\sum F_{\text{horiz}} = 0$). Stellen Sie diese Gleichung auf und lösen Sie sie nach der Haftreibungskraft $F_{R,h}$ auf. 
    \item Das System ruht, solange die wirkende Haftreibungskraft nicht die maximale Haftreibungskraft überschreitet: $F_{R,h} \le F_{R,h,\text{max}} = \mu_{R,h} |\ivec{F}_N|$. Ersetzen Sie in dieser Ungleichung die Haftreibungskraft $F_{R,h}$ und die Normalkraft $|\ivecS{F}{N}|$ durch die entsprechenden Ausdrücke aus Ihren vorherigen Überlegungen. 
    \item Formen Sie die Ungleichung nach der gesuchten Masse $m_1$ um. Wie groß muss $m_1$ mindestens sein, damit das System nicht in Bewegung gerät? 
\end{enumerate}
\end{aufgabebox}








\newpage
\section{Lösungen}\label{sec:dynamik_loesungen}

\begin{loesungbox}{Lösung zu \Cref{aufg:gravitation}}
\begin{enumerate}
    \item \textbf{Kraft der Erde auf einen Menschen:} \\
    Der Betrag der Gravitationskraft ist $F_G = G \frac{m_1 m_2}{r^2}$. Hier ist $m_1 = m_{\text{Mensch}}$, $m_2 = m_{\text{Erde}}$ und der Abstand ist der Erdradius $r = r_{\text{Erde}}$.
    \begin{multline}
        F_{G, \text{Erde}} = G \frac{m_{\text{Mensch}} \cdot m_{\text{Erde}}}{r_{\text{Erde}}^2} = \\
        \SI{6.67e-11}{\frac{\newton\meter^2}{\kilo\gram^2}} \cdot \frac{\SI{80}{\kilo\gram} \cdot \SI{5.972e24}{\kilo\gram}}{(\SI{6.372e6}{\meter})^2} \approx \SI{784.8}{\newton}
    \end{multline}
    Dies entspricht der üblichen Gewichtskraft $F_G = mg = \SI{80}{\kilo\gram} \cdot \SI{9.81}{\meter\per\second\squared} \approx \SI{785}{\newton}$.
    
    \item \textbf{Kraft des Mondes auf einen Menschen:} \\
    Der Abstand $r$ ist hier der Abstand vom Mondmittelpunkt zum Menschen auf der Erdoberfläche. Wir nehmen vereinfacht den Abstand der Mittelpunkte minus dem Erdradius: $r = d_{\text{EM}} - r_{\text{Erde}}$. 
    \begin{multline}
        F_{G, \text{Mond}} = G \frac{m_{\text{Mensch}} \cdot m_{\text{Mond}}}{(d_{\text{EM}} - r_{\text{Erde}})^2} = \\
        \SI{6.67e-11}{\frac{\newton\meter^2}{\kilo\gram^2}} \cdot \frac{\SI{80}{\kilo\gram} \cdot \SI{7.346e22}{\kilo\gram}}{(\SI{3.844e8}{\meter} - \SI{6.372e6}{\meter})^2} \approx \SI{2.74e-3}{\newton}
    \end{multline}
    
    \item \textbf{Kraft zwischen zwei Menschen:} \\
    Hier ist $m_1 = m_2 = \SI{80}{\kilo\gram}$ und der Abstand $r=\SI{0.5}{\meter}$.
    \begin{equation}
        F_{G, \text{Mensch-Mensch}} = G \frac{m_{\text{Mensch}}^2}{r^2} = \SI{6.67e-11}{\frac{\newton\meter^2}{\kilo\gram^2}} \cdot \frac{(\SI{80}{\kilo\gram})^2}{(\SI{0.5}{\meter})^2} \approx \SI{1.71e-6}{\newton}
    \end{equation}
    
    \item \textbf{Fallbeschleunigungen:} \\
    Aus $F_G = m \cdot g$ folgt $g = F_G/m$. Wir können die Formel für $g$ direkt aus dem Gravitationsgesetz ableiten, indem wir die Masse des Testkörpers (hier des Menschen) kürzen: $g = G \frac{M}{r^2}$. 
    \begin{align}
        g_{\text{Erde}} &= G \frac{m_{\text{Erde}}}{r_{\text{Erde}}^2} = \SI{6.67e-11}{\frac{\newton\meter^2}{\kilo\gram^2}} \cdot \frac{\SI{5.972e24}{\kilo\gram}}{(\SI{6.372e6}{\meter})^2} \approx \SI{9.81}{\meter\per\second\squared} \\ 
        g_{\text{Mond}} &= G \frac{m_{\text{Mond}}}{(d_{\text{EM}} - r_{\text{Erde}})^2} = \SI{6.67e-11}{\frac{\newton\meter^2}{\kilo\gram^2}} \cdot \frac{\SI{7.346e22}{\kilo\gram}}{(\SI{3.78e8}{\meter})^2} \approx \SI{3.43e-5}{\meter\per\second\squared} 
    \end{align}
\end{enumerate}
\end{loesungbox}



\begin{loesungbox}{Lösung zu \Cref{aufg:konisches_pendel}}
\begin{enumerate}
    \item \textbf{Freikörperdiagramm:} Die auf den Ball wirkenden Kräfte sind die Gewichtskraft $\ivecS{F}{G}$, die senkrecht nach unten zeigt, und die Seilkraft $\ivecS{F}{S}$, die entlang des Seils nach oben zum Aufhängepunkt gerichtet ist. 
    \begin{center}
        \includegraphics[width=0.34\textwidth]{Bilder/Uebungsaufgaben/thetherbal_kraefte.png}
        \label{fig:thetherball_kraefte}
    \end{center}
    \item \textbf{Komponenten der Kräfte:}
    \begin{itemize}
        \item Die Gewichtskraft $\ivecS{F}{G}$ wirkt nur in negativer $z$-Richtung, daher ist
        $$F_{G,x} = 0 \quad \text{und} \quad F_{G,z} = -mg \mDot$$ 
        \item Die Seilkraft $\ivecS{F}{S}$ wird zerlegt in
        $$F_{S,x} = F_S \sin(\theta) \quad \text{und} \quad F_{S,z} = F_S \cos(\theta) \mComma$$
        wobei $F_S = |\ivecS{F}{S}|$.
    \end{itemize}

    \item \textbf{Bewegungsgleichungen:}
    Wir setzen die Komponenten in die Newtonschen Gleichungen ein: 
    \begin{align}
        &x\text{-Richtung:} &F_{G,x} + F_{S,x} = 0 + F_S \sin(\theta) = m a_x \label{eq:konisch_x} \\
        &z\text{-Richtung:} &F_{G,y} + F_{S,y} = -mg + F_S \cos(\theta) = m a_z \label{eq:konisch_z}
    \end{align}

    \item \textbf{Beschleunigungskomponenten:}
    Da der Ball auf einer konstanten Höhe fliegt, ändert sich seine $z$-Koordinate nicht. Daher ist die Beschleunigung in $z$-Richtung null: $a_z = 0$. Die Bewegung in der $(x,y)$-Ebene ist eine Kreisbewegung, also muss eine Zentripetalkraft zum Mittelpunkt der Kreisbahn wirken. Diese wird durch die Zentripetalbeschleunigung $a_x = a_{\text{Zp}} = \omega^2 r$ verursacht. 
    
    \item \textbf{Fertiges Gleichungssystem:}
    Wir setzen die Beschleunigungen in \Cref{eq:konisch_x,eq:konisch_z} ein: 
    \begin{align}
        &x\text{-Koordinate:} &F_S \sin(\theta) = m \omega^2 r \label{eq:konisch_final_x} \\
        &z\text{-Koordinate:} &F_S \cos(\theta) - mg = 0 \label{eq:konisch_final_z}
    \end{align}
    
    \item \textbf{Auflösen nach $\theta$ (Schritt 1):}
    Wir lösen \Cref{eq:konisch_final_z} nach $F_S$ auf:
    \begin{equation}
        F_S = \frac{mg}{\cos(\theta)} \label{eq:seilkraft}
    \end{equation}
    Dies setzen wir in \Cref{eq:konisch_final_x} ein: 
    \begin{equation}
        \frac{mg}{\cos(\theta)} \sin(\theta) = m \omega^2 r \implies g \tan(\theta) = \omega^2 r \label{eq:g_tan_theta}
    \end{equation}
    
    \item \textbf{Auflösen nach $\theta$ (Schritt 2):}
    Mit dem geometrischen Zusammenhang $r = l \sin(\theta)$ wird \Cref{eq:g_tan_theta} zu: 
    \begin{equation}
        g \tan(\theta) = g \frac{\sin(\theta)}{\cos(\theta)} = \omega^2 l \sin(\theta)
    \end{equation}
    Für $\theta \neq 0$ können wir $\sin(\theta)$ kürzen und erhalten:
    \begin{equation}\label{eq:theta_omega}
        \cos(\theta) = \frac{g}{\omega^2\cdot l} \implies \theta(\omega) = \arccos\left(\frac{g}{\omega^2\cdot l}\right) 
    \end{equation}
    Diese Gleichung hat nur eine reelle Lösung, wenn das Argument des Arkuskosinus zwischen -1 und 1 liegt, also wenn $\omega^2 l \ge g$. Das bedeutet, es gibt eine kritische Winkelgeschwindigkeit $\omega_{\text{crit}} = \sqrt{g/l}$. Unterhalb dieser Winkelgeschwindigkeit ist der einzige stabile Zustand $\theta=0$. Erst wenn das Pendel schnell genug rotiert, lenkt es aus, und nimmt den Winkel $\theta$ aus \cref{eq:theta_omega} ein. 
    \begin{center}
        \includegraphics[width=0.65\textwidth]{Bilder/Uebungsaufgaben/thetherbal_omega_crit.png}
        \label{fig:thetherball_omega_crit}
    \end{center}
\end{enumerate}

\paragraph{Zusatzinformation:}
Wir haben uns hier die stationären Zustände des konischen Pendels berechnet, für die $\theta =\const$ gilt. Das sind nicht die allgemeinen Bewegungen, die so ein Pendel durchführen kann. Im Allgemeinen führt das Pendel zwei überlagerte Bewegungen aus: Eine Pendelbewegung und eine Rotation um die $z$-Achse. Erst wenn $\omega_{\text{rot}} \ge \omega_{\text{crit}}$, \gDh die Winkelgeschwindigkeit der Rotation ist größer als die Winkelgeschwindigkeit der Pendelbewegung ($\omega_{\text{crit}} = \omega_{\text{pend}} = \sqrt{g/l}$), dann kann sich ein stationärer Zustand einstellen. Salopp formuliert: So lange das Pendel schneller fällt, als es sich dreht, kann es keinen stationären Zustand geben. 
\begin{center}
    \includegraphics[width=0.99\linewidth]{Bilder/Uebungsaufgaben/thetherbal_beispiele_bewegung.png}
    \label{fig:thetherball_beispiele}
\end{center}
\end{loesungbox}


\begin{loesungbox}{Lösung zu \Cref{aufg:schiefe_ebene}}
\begin{enumerate}
    \item \textbf{Komponenten der Erdbeschleunigung:}
    Im geneigten Koordinatensystem wird der Vektor $\ivec{g}$ zerlegt. Aus der Geometrie ergibt sich:
    \begin{align}
        g_x &= -g \sin(\theta) \label{eq:g_x_schief} \mComma\\
        g_y &= -g \cos(\theta) \label{eq:g_y_schief} \mComma
    \end{align}
    wobei $g = \SI{9.81}{\meter\per\second\squared} > 0$. Die Vorzeichen sind negativ, da beide Komponenten in die negative Richtung der jeweiligen Achse zeigen. 
    \begin{center}
        \includegraphics[width=0.50\linewidth]{Bilder/Uebungsaufgaben/reibungsfreies_gleiten_g_aufteilung.png}
        \label{fig:reibungsfreies_gleiten_g_auft}
    \end{center}

    \item \textbf{Bewegungsgleichung $x(t)$:}
    Wir wählen den Ursprung am Beginn der Rampe, also $\ivecS{r}{0} = \inlrowTwo{x_0}{y_0} = \inlrowTwo{0}{0}$. Die Anfangsgeschwindigkeit $\ivecS{v}{0}$ zeigt in die positive $x$-Richtung, also $\ivecS{v}{0} = \inlrowTwo{v_0}{0}$. Die Beschleunigung ist konstant $\ivec{a} = \inlrowTwo{g_x}{g_y} = \inlrowTwo{-g\sin(\theta)}{-g\cos(\theta)}$. Die Bewegungsgleichung für die $x$-Koordinate lautet somit:
    \begin{equation}\label{eq:x_t_schief}
        x(t) = x_0 + v_0 t + \frac{1}{2}g_x t^2 = v_0 t - \frac{1}{2}g \sin(\theta) t^2 
    \end{equation}
    
    \item \textbf{Zeitpunkt des Stillstands $t^*$:}
    Wir leiten $x(t)$ nach der Zeit ab, um $v_x(t)$ zu erhalten: 
    \begin{equation}
        v_x(t) = \frac{\dd x(t)}{\dd t} = v_0 - g \sin(\theta) t \mDot
    \end{equation}
    Der Block kommt zum Stillstand, wenn $v_x(t^*) = 0$: 
    \begin{equation}
        0 = v_x(t^*) = v_0 - g \sin(\theta) t^* \implies t^* = \frac{v_0}{g \sin(\theta)} \label{eq:t_stern_schief}
    \end{equation}

    \item \textbf{Maximale Höhe $h$:}
    Wir setzen $t^*$ in $x(t)$ ein, um den Weg $s$ zu berechnen: 
    \begin{align}
        s = x(t^*) &= v_0 \left(\frac{v_0}{g \sin(\theta)}\right) - \frac{1}{2}g \sin(\theta) \left(\frac{v_0}{g \sin(\theta)}\right)^2 \\
        &= \frac{v_0^2}{g \sin(\theta)} - \frac{v_0^2}{2g \sin(\theta)} = \frac{v_0^2}{2g \sin(\theta)} \label{eq:s_schief}
    \end{align}
    Aus der Geometrie der Rampe wissen wir $h = s \sin(\theta)$. Setzen wir \Cref{eq:s_schief} ein, erhalten wir: 
    \begin{equation}
        h = s \sin(\theta) = \left(\frac{v_0^2}{2g \sin(\theta)}\right) \sin(\theta) = \frac{v_0^2}{2g} \label{eq:h_schief}
    \end{equation}
    Die Höhe ist tatsächlich unabhängig von der Masse $m$ und dem Winkel $\theta$.
    
    \item \textbf{Bonusfrage:}
    Die Unabhängigkeit vom Winkel $\theta$ ist eine Folge der \textbf{Energieerhaltung}. Die anfängliche kinetische Energie ($E_{\text{kin}} = \frac{1}{2}mv_0^2$) wird vollständig in potenzielle Energie ($E_{\text{pot}} = mgh$) umgewandelt. Der Weg $s$ ist zwar vom Winkel abhängig, die dafür benötigte Arbeit gegen die Schwerkraft aber nur von der erreichten Höhe $h$. Bei Vorhandensein von Reibung würde die Energieerhaltung in dieser Form nicht mehr gelten, da ein Teil der Energie durch die Reibungsarbeit in Wärme umgewandelt würde. Der Reibungsweg $s$ wäre entscheidend, und da dieser von $\theta$ abhängt, wäre auch die erreichte Höhe $h$ vom Winkel abhängig. 
\end{enumerate}
\end{loesungbox}


\begin{loesungbox}{Lösung zu \Cref{aufg:rutsche}}
\begin{enumerate}
    \item \textbf{Kräfteskizze:} Die wirkenden Kräfte sind die Gewichtskraft $\ivecS{F}{G}$, die Normalkraft $\ivecS{F}{N}$ und die Gleitreibungskraft $\ivecS{F}{R,\text{Gleit}}$.
    \begin{center}
        \includegraphics[width=0.5\textwidth]{Bilder/Uebungsaufgaben/kind_rutsche_lösung.png}
    \end{center}
    \item \textbf{Neigungswinkel:} Der Winkel $\theta$ kann über die Geometrie der Rutsche bestimmt werden:
    \begin{equation}\label{eq:rutsche_winkel}
        \sin(\theta) = \frac{h}{L} = \frac{\SI{3}{\metre}}{\SI{5}{\metre}} = \num{0.6} \implies \theta = \arcsin(\num{0.6}) \approx \SI{36.87}{\degree} \mDot
    \end{equation}
    \item \textbf{Zerlegung der Gewichtskraft:} $F_G = m \cdot g = \SI{35}{\kg} \cdot \SI{9.81}{\m\per\s\squared} = \SI{343.35}{\newton}$.
    \begin{align}
        F_{G,x} = F_{G, \parallel} &= F_G \sin(\theta) = \SI{343.35}{\newton} \cdot \sin(\SI{36.87}{\degree}) = \SI{206.01}{\newton} \label{eq:rutsche_fg_para} \\
        F_{G,y} = F_{G, \perp} &= F_G \cos(\theta) = \SI{343.35}{\newton} \cdot \cos(\SI{36.87}{\degree}) = \SI{274.68}{\newton} \label{eq:rutsche_fg_perp}
    \end{align}
    \item \textbf{Normal- und Reibungskraft:} Die Normalkraft ist betragsmäßig gleich der senkrechten Komponente der Gewichtskraft.
    \begin{equation}\label{eq:rutsche_fn}
        F_N = F_{G, \perp} = F_G \cdot \cos(\theta) = \SI{274.68}{\newton}
    \end{equation}
    Daraus ergibt sich die Gleitreibungskraft:
    \begin{equation}\label{eq:rutsche_fr}
        F_{R, \text{Gleit}} = \mu_{\text{Gleit}} \cdot F_N = \num{0.2} \cdot \SI{274.68}{\newton} = \SI{54.936}{\newton}
    \end{equation}
    \item \textbf{Nettokraft und Beschleunigung:} Die Nettokraft in Bewegungsrichtung ist die Differenz aus der Hangabtriebskraft ($F_{G, \parallel}$) und der Gleitreibungskraft.
    \begin{equation}\label{eq:rutsche_fnet}
        F_{\text{netto}} = F_{G, \parallel} - F_{R, \text{Gleit}} = \SI{206.01}{\newton} - \SI{54.936}{\newton} = \SI{151.074}{\newton}
    \end{equation}
    Nach dem 2. Newtonschen Axiom ($F=ma$) ergibt sich die Beschleunigung:
    \begin{equation}\label{eq:rutsche_a}
        a = \frac{F_{\text{netto}}}{m} = \frac{\SI{151.074}{\newton}}{\SI{35}{\kg}} \approx \SI{4.316}{\metre\per\second\squared}
    \end{equation}
    \item \textbf{Herleitung der zeitunabhängigen Formel:} Aus dem Stillstand gilt $s_0 = 0$ und $v_0 = 0$, weshalb sich die Gleichungen zu 
    \begin{align}
        s(t) &= \frac{1}{2}a\cdot t^2 \\
        v(t) &= a\cdot t    
    \end{align}  
    
    vereinfachen. Aus der ersten Gleichung erhält man für $s = L$ 
    \begin{equation}
        L = \frac{1}{2}at^2 \Rightarrow t = \sqrt{\frac{2L}{a}} \mDot
    \end{equation}
    Einsetzen in die zweite Gleichung ergibt:
    \begin{equation}\label{eq:rutsche_v_von_s}
        v = a \cdot \sqrt{\frac{2L}{a}} = \sqrt{\frac{a^2 2L}{a}} = \sqrt{2aL}
    \end{equation}
    \item \textbf{Endgeschwindigkeit:} Mit der in Schritt (6) berechneten Beschleunigung $a$ und der Streckenlänge $L$ erhalten wir:
    \begin{equation}\label{eq:rutsche_vend}
        v = \sqrt{2 a L} = \sqrt{2 \cdot \SI{4.316}{\m\per\s\squared} \cdot \SI{5}{\m}} \approx \SI{6.57}{\metre\per\second} \mDot
    \end{equation}
    Die Endgeschwindigkeit des Kindes beträgt somit ca. $\SI{6.57}{\metre\per\second}$.
\end{enumerate}
\end{loesungbox}





\begin{loesungbox}{Lösung zu \Cref{aufg:schiefe_ebene_schieben}}
\begin{enumerate}
    \item \textbf{Freikörperbild für Masse m} \\
    Auf die Masse $m$ wirken die Gewichtskraft $\ivecS{F}{G} = m\ivec{g}$ (senkrecht nach unten) und die Normalkraft $\ivecS{F}{N}$ (senkrecht zur Auflagefläche des Keils). Das System wird mit einer Beschleunigung $\ivecS{a}{x}$ nach rechts bewegt.
    \begin{center}
        \includegraphics[width=0.4\textwidth]{Bilder/Uebungsaufgaben/schieben_schiefe_ebene_reibfrei_kraefte.png}
    \end{center}

    \item \textbf{Kräftegleichgewicht in $x$- und $y$-Richtung} \\
    Wir zerlegen die Normalkraft $\ivecS{F}{N}$ in ihre $x$- und $y$-Komponenten. In $y$-Richtung herrscht Kräftegleichgewicht, da keine vertikale Bewegung stattfindet.
    \begin{equation}\label{eq:schiefe_ebene_Fy}
        \sum F_y = F_N \cos(\theta) - mg \stackrel{!}{=} 0 \quad \Rightarrow \quad F_N \cos(\theta) = mg
    \end{equation}
    In $x$-Richtung bewirkt die $x$-Komponente der Normalkraft die Beschleunigung $a_x$:
    \begin{equation}\label{eq:schiefe_ebene_Fx}
        \sum F_x = F_N \sin(\theta) \eqexcl m a_x
    \end{equation}

    \item \textbf{Herleitung der Beschleunigung $a_x$} \\
    Aus \Cref{eq:schiefe_ebene_Fy} erhalten wir für die Normalkraft:
    \begin{equation}
        F_N = \frac{mg}{\cos(\theta)}
    \end{equation}
    Dies setzen wir in \Cref{eq:schiefe_ebene_Fx} ein:
    \begin{align}\label{eq:ax_schiefe_ebene}
        F_N \sin(\theta) = \left(\frac{mg}{\cos(\theta)}\right) \sin(\theta) &= m a_x \nonumber \\
        mg \tan(\theta) &= m a_x \nonumber \\
        \implies a_x &= g \tan(\theta)
    \end{align}

    \item \textbf{Berechnung der Gesamtkraft $F$} \\
    Die externe Kraft $F$ muss die Gesamtmasse des Systems, $M+m$, mit der berechneten Beschleunigung $a_x$ bewegen:
    \begin{equation}\label{eq:F_ges_schiefe_ebene}
        F = (M+m) a_x = (M+m) g \tan(\theta) \mDot
    \end{equation}
\end{enumerate}
\end{loesungbox}



\begin{loesungbox}{Lösung zu \Cref{aufg:rollreibung}}
\begin{enumerate}
    \item \textbf{Rollreibungskraft:}
    Die Normalkraft auf das Auto ist gleich seiner Gewichtskraft, $|\ivecS{F}{N}| = mg$. Der Betrag der Rollreibungskraft ist 
    $$F_{R,r} = \mu_{R,r} |\ivecS{F}{N}| = \mu_{R,r} m g \mDot$$ 
    \begin{itemize}
        \item \textbf{Sommerreifen:} $F_{R,r}^S = \num{0.013} \cdot \SI{1500}{\kilo\gram} \cdot \SI{9.81}{\meter\per\second\squared} \approx \SI{191.3}{\newton}$ 
        \item \textbf{Winterreifen:} $F_{R,r}^W = \num{0.015} \cdot \SI{1500}{\kilo\gram} \cdot \SI{9.81}{\meter\per\second\squared} \approx \SI{220.7}{\newton}$ 
    \end{itemize}
    
    \item \textbf{Ausrollweg:}
    \begin{enumerate}
        \item \textbf{Beschleunigung:} Die einzige wirkende Kraft in Bewegungsrichtung ist die Rollreibungskraft, die entgegen der Bewegung zeigt. Nach Newtons zweitem Gesetz gilt:
        \begin{equation}\label{eq:a_x_roll}
            \sum F_x = -F_{R,r} = m a_x \implies -\mu_{R,r} \cdot m g = m a_x \implies a_x = -\mu_{R,r}\cdot g 
        \end{equation}
        Die Beschleunigung ist also konstant und unabhängig von der Masse. 
        
        \item \textbf{Geschwindigkeit:} Wir setzen $a_x$ in die Geschwindigkeitsgleichung ein: 
        \begin{equation}
            v_x(t) = v_0 - \mu_{R,r} \cdot g \cdot t
        \end{equation}
        
        \item \textbf{Zeitpunkt des Stillstands $t^*$:} Der Stillstand tritt ein bei $v_x(t^*) = 0$: 
        \begin{equation}\label{eq:t_stern_roll}
            0 = v_0 - \mu_{R,r}\cdot g\cdot t^* \implies t^* = \frac{v_0}{\mu_{R,r} \cdot g} 
        \end{equation}
        
        \item \textbf{Ausrollweg $x_W$:} Wir setzen $t^*$ und $a_x$ in die Ortsgleichung ein (mit $x_0 = 0$): 
        \begin{align}
            x_W = x(t^*) &= v_0 t^* + \frac{1}{2} a_x (t^*)^2 = v_0 \left(\frac{v_0}{\mu_{R,r} g}\right) + \frac{1}{2}(-\mu_{R,r} g)\left(\frac{v_0}{\mu_{R,r} g}\right)^2 \\
            &= \frac{v_0^2}{\mu_{R,r} g} - \frac{v_0^2}{2\mu_{R,r} g} = \frac{v_0^2}{2\mu_{R,r} g} \label{eq:x_w_roll}
        \end{align}
        
        \item \textbf{Numerischer Wert:} Zuerst rechnen wir die Geschwindigkeit um: $v_0 = \SI{70}{\kilo\meter\per\hour} = \frac{70}{3.6} \si{\meter\per\second} \approx \SI{19.44}{\meter\per\second}$.
        \begin{equation}
            x_W = \frac{(\SI{19.44}{\meter\per\second})^2}{2 \cdot \num{0.013} \cdot \SI{9.81}{\meter\per\second\squared}} \approx \SI{1482}{\meter}
        \end{equation}
    \end{enumerate}
    
    \item \textbf{Treibstoffverbrauch:}
    Obwohl der Ausrollweg massenunabhängig ist, ist die Rollreibungskraft $F_{R,r} = \mu_{R,r} m g$ direkt proportional zur Masse. Um eine konstante Geschwindigkeit beizubehalten, muss der Motor eine Vorwärtskraft aufbringen, die genau diese Rollreibungskraft kompensiert ($F_{\text{Motor}} = F_{R,r}$). Die dafür pro Strecke $s$ benötigte Energie (Arbeit) ist $E = F_{\text{Motor}} \cdot s = (\mu_{R,r} m g) \cdot s$. Diese Energie ist nun wie erwartet abhängig von der Masse $m$ des Autos. Ein schwereres Auto benötigt also bei gleicher Geschwindigkeit und gleicher Strecke mehr Energie und somit mehr Treibstoff. 
\end{enumerate}
\end{loesungbox}


\begin{loesungbox}{Lösung zu \Cref{aufg:haftreibung}}
\begin{enumerate}
    \item \textbf{Kräfteplan:}
    \begin{center}
        \includegraphics[width=0.50\linewidth]{Bilder/Uebungsaufgaben/haftreibung_zwei-massen_seil_kraefte.png}
        \label{fig:haftreibung_zwei_massen_kraefte}
    \end{center}
    \begin{itemize}
        \item \textbf{Masse $m_1$:} Es wirken die Gewichtskraft $\ivecS{F}{G,1}$ nach unten, die Normalkraft $\ivecS{F}{N}$ vom Tisch nach oben, die Zugkraft durch das Seil $\ivecS{F}{A}$ nach rechts und die Haftreibungskraft $\ivecS{F}{R,h}$ nach links. 
        \item \textbf{Masse $m_2$:} Es wirken die Gewichtskraft $\ivecS{F}{G,2}$ nach unten und die Zugkraft des Seils nach oben ($\ivecS{F}{A}$).
    \end{itemize}
    Es sind nur jene Kräfte in der Skizze eingezeichnet, die für die Dynamik eine Relevanz haben.

    \item \textbf{Horizontales Kräftegleichgewicht für $m_1$:}
    Im Ruhezustand ist die Beschleunigung null. Die Summe der horizontalen Kräfte auf $m_1$ ist:
    \begin{equation}\label{eq: haftreib_kraftgleichgew_x}
        \sum F_x = F_A - F_{R,h} = 0
    \end{equation}
    Die Zugkraft $F_A$ wird durch das Gewicht der Masse $m_2$ verursacht, also $F_A = m_2 g$. Aus \cref{eq: haftreib_kraftgleichgew_x} folgt:
    \begin{equation}\label{eq:haft_frh}
        F_{R,h} = F_A = m_2 g 
    \end{equation}

    \item \textbf{Bedingung für die Haftung:}
    Das System bleibt in Ruhe, solange die Haftreibungskraft kleiner oder gleich der maximalen Haftreibungskraft ist. 
    \begin{equation}
        F_{R,h} \le F_{R,h,\text{max}} =  \mu_{R,h} |\ivecS{F}{N}|
    \end{equation}
    Die Normalkraft auf $m_1$ ist gleich der Gewichtskraft von $m_1$, also $|\ivecS{F}{N}| = m_1 g$. Wir setzen dies und \Cref{eq:haft_frh} in die Ungleichung ein: 
    \begin{equation}\label{eq:haft_ungleichung}
        m_2 g \le \mu_{R,h} m_1 g 
    \end{equation}

    \item \textbf{Minimale Masse $m_1$:}
    Wir kürzen $g$ aus \Cref{eq:haft_ungleichung} und formen nach $m_1$ um: 
    \begin{equation}
        m_1 \ge \frac{m_2}{\mu_{R,h}}
    \end{equation}
    Wir setzen die gegebenen Werte ein, um die Mindestmasse zu finden: 
    \begin{equation}
        m_1 \ge \frac{\SI{2.0}{\kilo\gram}}{\num{0.4}} = \SI{5.0}{\kilo\gram}
    \end{equation}
    Die Masse $m_1$ muss also mindestens $\SI{5.0}{\kilo\gram}$ betragen, damit das System nicht ins Gleiten kommt. 
\end{enumerate}
\end{loesungbox}






\chapter{Arbeit, Energie und Leistung}\label{chap:arbeit_energie}

\section{Aufgaben}\label{sec:arbeit_energie_aufgaben}

\begin{aufgabebox}{Erste kosmische Geschwindigkeit}{erste_kosmische}
Die erste kosmische Geschwindigkeit $v_I$ ist die Mindestgeschwindigkeit, die ein Körper tangential zur Erdoberfläche haben muss, um eine stabile, niedrige Kreisbahn um die Erde einzunehmen. 
Leiten Sie eine Formel für $v_I$ für eine Punktmasse $m$ her. Nehmen Sie an, dass die Erde eine perfekte Kugel mit Masse $M$ und Radius $R$ ist und nur die Gravitationskraft $F_G$ wirkt. 
\begin{center}
    \includegraphics[width=0.40\linewidth]{Bilder/Uebungsaufgaben/erste_kosmische_geschwindigkeit.png}
    \label{fig:erste_kosmische_geschw}
\end{center}
\textbf{Tipp:} Für eine stabile Kreisbahn muss die Gravitationskraft genau der Zentripetalkraft entsprechen. \\
\textbf{Gegebene Werte:} $G \approx \SI{6.674e-11}{\newton\meter\squared\per\kilo\gram\squared}$, $R \approx \SI{6371}{\kilo\meter}$, $M \approx \SI{5.972e24}{\kilo\gram}$. 
\end{aufgabebox}

\begin{aufgabebox}{Arbeit und Leistung eines PKW}{pkw_leistung}
Berechnen Sie die Leistungsanteile, die ein PKW-Motor aufbringen muss, um ein Fahrzeug der Masse $m=\SI{1500}{\kilo\gram}$ mit konstanter Geschwindigkeit $v$ gegen den Luftwiderstand $\ivecS{F}{LW}$ und die Rollreibung $\ivecS{F}{R,r}$ zu bewegen. Berechnen Sie außerdem die benötigte Zeit und den Energieverbrauch für eine Distanz von $\SI{100}{\kilo\meter}$. 
\begin{itemize}
    \item Der Betrag des \textbf{Luftwiderstands} ist 
    $$F_{LW} = \frac{1}{2}\rho_{\text{Luft}} c_w A v^2 \mComma$$
    mit $\rho_{\text{Luft}}=\SI{1.2}{\kilo\gram\per\meter\cubed}$, $A=\SI{2.1}{\meter\squared}$ und $c_w = \num{0.26}$. 
    \item Der \textbf{Rollreibungskoeffizient} ist geschwindigkeitsabhängig: 
    $$\mu_{R,r}(v) = \num{0.01} + \tau v^2 \mComma$$
    mit $\tau=\SI{5.6e-6}{\second\squared\per\meter\squared}$. Die Rollreibungskraft ist $F_{R,r} = \mu_{R,r} F_N$. 
\end{itemize}
Füllen Sie die folgende Tabelle für die Geschwindigkeiten $v \in \{50, 100, 130, 150, 200\}\,\si{\kilo\meter\per\hour}$ aus. 
\begin{center}
\begin{tabular}{l|ccccc}
\toprule
\thead{$v$ [\si{\kilo\meter\per\hour}]} & \thead{50} & \thead{100} & \thead{130} & \thead{150} & \thead{200} \\
\midrule
$F_{LW}$ [\si{\newton}] & & & & & \\
$F_{R,r}$ [\si{\newton}] & & & & & \\
$P_{LW}$ [\si{\kilo\watt}] & & & & & \\
$P_{R,r}$ [\si{\kilo\watt}] & & & & & \\
$\Delta t/\SI{100}{\kilo\meter}$ [\si{\minute}] & & & & & \\
$E/\SI{100}{\kilo\meter}$ [\si{\kilo\watt\hour}] & & & & & \\
\bottomrule
\end{tabular}
\end{center}
\end{aufgabebox}




\begin{aufgabebox}{Jump-Bag}{jump_bag}
Zwei Männer mit jeweils \SI{80}{\kilogram} Körpermasse springen gemeinsam von einem Sprungturm mit \SI{6}{\metre} Höhe auf einen „Jump-Bag". Dort übertragen sie ihre gesamte potentielle Energie verlustfrei auf eine Frau am anderen Ende, die eine Masse von $m_{\text{Frau}}=\SI{65}{\kilogram}$ hat. Reibungsverluste können vernachlässigt werden.

\begin{center}
    \includegraphics[width=0.7\textwidth]{Bilder/Uebungsaufgaben/jump_bag.png}
\end{center}

\begin{enumerate}
    \item Beschreiben Sie die Bewegungsform der Frau, nachdem sie das Kissen verlassen hat. Begründen Sie Ihre Entscheidung!
    \item Berechnen Sie die maximale Flughöhe $h_{\text{max}}$, die die Frau erreicht!
    \item Mit welcher maximalen Geschwindigkeit $v_{\text{max}}$ kommt sie auf der Wasseroberfläche bei $h=0$ m auf?
\end{enumerate}
\end{aufgabebox}





\begin{aufgabebox}{Trebuchet}{trebuchet}
Ein Trebuchet ist ein Katapult, das ein schweres Gegengewicht nutzt, um ein Projektil abzufeuern. Das Gegengewicht habe eine Masse von $M=\SI{9000}{\kilo\gram}$ und kann um eine Höhe $h=\SI{5}{\meter}$ angehoben werden. Abgefeuert wird ein Projektil der Masse $m=\SI{50}{\kilo\gram}$. Das Projektil verlässt die Schleuder auf einer Höhe von $H=\SI{16}{\meter}$ über dem Boden. 
\begin{center}
    \includegraphics[width=0.50\linewidth]{Bilder/Uebungsaufgaben/trebuchet.png}
    \label{fig:trebuchet}
\end{center}
\begin{enumerate}
    \item Wie groß ist die potenzielle Energie, die im angehobenen Gegengewicht gespeichert ist? 
    \item Nehmen Sie an, dass die Hälfte der potenziellen Energie des Gegengewichts in kinetische Energie des Projektils umgewandelt wird. Berechnen Sie die Abwurfgeschwindigkeit $v_0$ des Projektils, wenn es die Schleuder verlässt.
    \item Das Projektil wird unter einem Winkel von $\alpha = \SI{25}{\degree}$ zur Horizontalen abgeschossen. Berechnen Sie die Wurfweite. 
\end{enumerate}
\end{aufgabebox}

\begin{aufgabebox}{Schlittenfahren am Iglu}{iglu_rutsche}
Ein Schlittenfahrer der Masse $m$ startet aus der Ruhe am höchsten Punkt ($\alpha=0$) eines halbkugelförmigen Iglus mit Radius $R$. Wir wollen den Winkel $\alpha_{\text{max}}$ berechnen, bei dem der Schlittenfahrer den Kontakt zum Iglu verliert und abhebt. Verwenden Sie das mitgeführte Koordinatensystem $(\ivecS{e}{r}, \ivecS{e}{t})$. 
\begin{center}
    \includegraphics[width=0.50\linewidth]{Bilder/Uebungsaufgaben/schlittenfahren_iglu.png}
    \label{fig:iglu}
\end{center}
\begin{enumerate}
    \item Zeichnen Sie die auf den Schlittenfahrer wirkenden Kräfte (Gewichtskraft $\ivecS{F}{G}$ und Normalkraft $\ivecS{F}{N}$) in die Skizze ein. 
    \item Zerlegen Sie diese Kräfte in ihre radialen und tangentialen Komponenten und schreiben Sie $F_{G,r}$, $F_{G,t}$, $F_{N,r}$ und $F_{N,t}$ an. 
    \item Stellen Sie die Bewegungsgleichungen für die radiale und tangentiale Richtung mit dem zweiten Newtonschen Gesetz auf. 
    \begin{align*}
        \sum_{i} F_{i,r} &= m a_r \\
        \sum_{i} F_{i,t} &= m a_t
    \end{align*}
    \item Im Moment des Abhebens bei $\alpha_{\text{max}}$ wird die Normalkraft null ($F_N=0$). Die Bewegung bis zu diesem Punkt ist eine Kreisbahn, daher ist die Radialbeschleunigung die Zentripetalbeschleunigung ($a_r = -v^2/R$). Setzen Sie diese beiden Bedingungen in Ihre radiale Bewegungsgleichung ein, um einen Zusammenhang zwischen der Geschwindigkeit $v$ und dem Winkel $\alpha_{\text{max}}$ zu finden. \newline
    \textit{Lösung:} $g \cos(\alpha_{\text{max}}) = v^2 /R$.
    \item Nutzen Sie den Energieerhaltungssatz, um die Geschwindigkeit $v$ als Funktion des Winkels $\alpha$ auszudrücken. Die Abnahme der potenziellen Energie vom Startpunkt bis zur momentanen Höhe $h$ wird komplett in kinetische Energie umgewandelt. 
    \item Setzen Sie den Ausdruck für die Geschwindigkeit aus 5) in die Gleichung aus 4) ein, um $\alpha_{\text{max}}$ zu bestimmen. 
\end{enumerate}
\end{aufgabebox}




\newpage
\section{Lösungen}\label{sec:arbeit_energie_loesungen}

\begin{loesungbox}{Lösung zu \Cref{aufg:erste_kosmische}}
Für eine stabile Kreisbahn muss die anziehende Gravitationskraft $\ivecS{F}{G}$ genau die erforderliche Zentripetalkraft $\ivecS{F}{\text{Zp}}$ aufbringen. Wir setzen die Beträge der beiden Kräfte gleich:
\begin{equation}
    F_{\text{Zp}} = F_G
\end{equation}
Die Zentripetalkraft ist $F_{\text{Zp}} = mv_I^2 /R$, und die Gravitationskraft auf der Erdoberfläche ist $F_G = G (M\cdot m)/R^2$. Einsetzen ergibt:
\begin{equation}
    \frac{m v_I^2}{R} = G \frac{Mm}{R^2}
\end{equation}
Wir können die Masse des Körpers $m$ und einen Faktor $R$ kürzen und erhalten:
\begin{equation}
    v_I^2 = \frac{GM}{R}
\end{equation}
Daraus folgt für die erste kosmische Geschwindigkeit:
\begin{equation}\label{eq:v_kosmisch_1}
    v_I = \sqrt{\frac{GM}{R}}
\end{equation}
Einsetzen der gegebenen Werte ergibt
\begin{equation}
    v_I = \sqrt{\frac{(\SI{6.674e-11}{\newton\meter\squared\per\kilo\gram\squared}) \cdot (\SI{5.972e24}{\kilo\gram})}{\SI{6.371e6}{\meter}}} \approx \SI{7909}{\meter\per\second} \approx \SI{7.91}{\kilo\meter\per\second} \mDot
\end{equation}
\end{loesungbox}

\begin{loesungbox}{Lösung zu \Cref{aufg:pkw_leistung}}
Zuerst müssen alle Geschwindigkeiten von \si{\kilo\meter\per\hour} in \si{\meter\per\second} umgerechnet werden ($v[\si{\meter\per\second}] = v[\si{\kilo\meter\per\hour}] / 3.6$). Die Normalkraft entspricht der Gewichtskraft $F_N = mg = \SI{1500}{\kilo\gram} \cdot \SI{9.81}{\meter\per\second\squared} = \SI{14715}{\newton}$. \\

\textbf{Formeln}:
\begin{itemize}
    \item Luftwiderstand: $F_{LW}(v) = \frac{1}{2} (\SI{1.2}{\kilo\gram\per\meter\cubed}) (\num{0.26}) (\SI{2.1}{\meter\squared}) \cdot v^2 = \num{0.3276} \cdot v^2$ 
    \item Rollreibung: $F_{R,r}(v) = (\num{0.01} + (\SI{5.6e-6}{\second\squared\per\meter\squared})\cdot  v^2) F_N = \num{147.15} + \num{0.0824} \cdot v^2$ 
    \item Leistung: $P(v) = F \cdot v$. \\
    Daraus folgt $P_{LW} = F_{LW} \cdot v$ und $P_{RR} = F_{R,r} \cdot v$. 
    \item Zeit für 100 km: $\Delta t = \frac{\SI{100}{\kilo\meter}}{v}$ 
    \item Energie für 100 km: $E = (P_{LW} + P_{R,r}) \cdot \Delta t = (F_{LW} + F_{R,r}) \cdot \SI{100}{\kilo\meter}$. \\
    Umrechnung von Joule in kWh: $E[\si{\kilo\watt\hour}] = E[\si{\joule}] / (3.6 \cdot 10^6)$. 
\end{itemize}

\textbf{Ergebnistabelle}:
\begin{center}
\begin{tabular}{l|ccccc}
\toprule
\thead{$v$ [\si{\kilo\meter\per\hour}]} & \thead{50} & \thead{100} & \thead{130} & \thead{150} & \thead{200} \\
\midrule
$F_{LW}$ [\si{\newton}] & \num{63.2} & \num{252.8} & \num{427.2} & \num{568.8} & \num{1011.1} \\ 
$F_{RR}$ [\si{\newton}] & \num{163.0} & \num{210.7} & \num{254.6} & \num{290.2} & \num{401.5} \\ 
$P_{LW}$ [\si{\kilo\watt}] & \num{0.88} &\num{7.02} & \num{15.43} & \num{23.70} & \num{56.17} \\ 
$P_{RR}$ [\si{\kilo\watt}] & \num{2.26} & \num{5.85} & \num{9.19} & \num{12.09} &\num{22.31} \\ 
$\Delta t/\SI{100}{\kilo\meter}$ [\si{\minute}] & \num{120} & \num{60} & \num{46.2} & \num{40} & \num{30} \\ 
$E/\SI{100}{\kilo\meter}$ [\si{\kilo\watt\hour}] & \num{6.3} & \num{12.9} & \num{18.9} & \num{23.9} & \num{39.2} \\ 
\bottomrule
\end{tabular}
\end{center}
Man erkennt, dass der Luftwiderstand bei höheren Geschwindigkeiten dominiert und der Energieverbrauch pro Kilometer stark ansteigt.
\begin{center}
    \includegraphics[width=0.85\linewidth]{Bilder/Uebungsaufgaben/zeit_energie_arbeit_pkw.png}
    \label{fig:resultat_pkw_energieverbrauch}
\end{center}

\end{loesungbox}




\begin{loesungbox}{Lösung zu \Cref{aufg:jump_bag}}
\begin{enumerate}
    \item \textbf{Bewegungsform:} Die Frau führt nach dem Verlassen des Kissens eine \\ \textbf{gleichförmig beschleunigte Bewegung} aus (genauer: einen vertikalen Wurf). \\
    Begründung: Sobald sie in der Luft ist, wirkt als einzige wesentliche Kraft die konstante Erdanziehungskraft nach unten. Dies führt zu einer konstanten Beschleunigung ($g \approx \SI{9.81}{\metre\per\second\squared}$).
    
    \item \textbf{Maximale Flughöhe:}
    Die potentielle Energie der beiden Männer wird vollständig in die potentielle Energie der Frau an ihrem höchsten Punkt umgewandelt (Energieerhaltungssatz).
    \begin{equation}\label{eq:jumpbag_epot_maenner}
        E_{\text{pot, Männer}} = (2 \cdot m_{\text{Mann}}) \cdot g \cdot H = (2 \cdot \SI{80}{\kg}) \cdot \SI{9.81}{\m\per\s\squared} \cdot \SI{6}{\m} = \SI{9417.6}{\joule}
    \end{equation}
    Am höchsten Punkt ihrer Flugbahn gilt $E_{\text{pot, Frau}} = m_{\text{Frau}} \cdot g \cdot h_{\text{max}}$. Gleichsetzen der Energien liefert:
    \begin{gather}\label{eq:jumpbag_hmax}
        E_\text{Frau} = E_\text{Männer} \\
        h_{\text{max}} = \frac{E_{\text{pot, Männer}}}{m_{\text{Frau}} \cdot g} = \frac{\SI{9417.6}{\joule}}{\SI{65}{\kg} \cdot \SI{9.81}{\m\per\s\squared}} \approx \SI{14.77}{\metre} \mDot
    \end{gather}

    \item \textbf{Maximale Geschwindigkeit:}
    Beim Aufprall auf die Wasseroberfläche ($h=0$) ist ihre gesamte potentielle Energie in kinetische Energie umgewandelt.
    \begin{equation}\label{eq:jumpbag_ekin_frau}
        E_{\text{kin, Frau}} = \frac{1}{2} m_{\text{Frau}} v_{\text{max}}^2 = E_{\text{pot, Männer}} = \SI{9417.6}{\joule}
    \end{equation}
    Auflösen nach der maximalen Geschwindigkeit:
    \begin{equation}\label{eq:jumpbag_vmax}
        v_{\text{max}} = \sqrt{\frac{2 \cdot E_{\text{pot, Männer}}}{m_{\text{Frau}}}} = \sqrt{\frac{2 \cdot \SI{9417.6}{\joule}}{\SI{65}{\kg}}} \approx \SI{17.02}{\metre\per\second} \approx \SI{61.27}{\kilo\meter\per\hour} \mDot
    \end{equation}
\end{enumerate}
\end{loesungbox}





\begin{loesungbox}{Lösung zu \Cref{aufg:trebuchet}}
\begin{enumerate}
    \item \textbf{Potentielle Energie des Gegengewichts:}
    Die potentielle Energie wird mit der Formel $E_{\text{pot}} = mgh$ berechnet.
    \begin{equation}
        E_{\text{pot}} = Mgh = (\SI{9000}{\kilo\gram}) \cdot (\SI{9.81}{\meter\per\second\squared}) \cdot (\SI{5}{\meter}) = \SI{441450}{\joule} = \SI{441.45}{\kilo\joule}
    \end{equation}
    
    \item \textbf{Abwurfgeschwindigkeit des Projektils:}
    Die Hälfte der potentiellen Energie wird in kinetische Energie $E_{\text{kin}} = \frac{1}{2}mv_0^2$ umgewandelt. 
    \begin{equation}
        E_{\text{kin}} = 0.5 \cdot E_{\text{pot}} = 0.5 \cdot \SI{441450}{\joule} = \SI{220725}{\joule}
    \end{equation}
    Wir lösen nach der Geschwindigkeit $v_0$ auf:
    \begin{equation}
        v_0 = \sqrt{\frac{2 E_{\text{kin}}}{m}} = \sqrt{\frac{2 \cdot \SI{220725}{\joule}}{\SI{50}{\kilo\gram}}} \approx \SI{93.96}{\meter\per\second}
    \end{equation}
    
    \item \textbf{Wurfweite:}
    Wir verwenden die Formel für die Wurfweite des schrägen Wurfs mit Abwurfhöhe $H$ (siehe \Cref{eq:wurfweite_xw}):
    \begin{equation}
        x_W = \frac{v_{0x}}{g}\left(v_{0z} + \sqrt{v_{0z}^2 + 2gH}\right)
    \end{equation}
    Zuerst berechnen wir die Geschwindigkeitskomponenten:
    \begin{align}
        v_{0x} &= v_0 \cos(\alpha) = \SI{93.96}{\meter\per\second} \cdot \cos(\ang{25}) \approx \SI{85.16}{\meter\per\second} \\ 
        v_{0z} &= v_0 \sin(\alpha) = \SI{93.96}{\meter\per\second} \cdot \sin(\ang{25}) \approx \SI{39.71}{\meter\per\second} 
    \end{align}
    Einsetzen in die Formel für die Wurfweite:
    \begin{equation}
        x_W = \frac{\SI{85.16}{m/s}}{\SI{9.81}{m/s^2}}\left(\SI{39.71}{m/s} + \sqrt{(\SI{39.71}{m/s})^2 + 2(\SI{9.81}{m/s^2})(\SI{16}{m})}\right) \approx \SI{722.2}{\meter}
    \end{equation}
\end{enumerate}
\end{loesungbox}




\begin{loesungbox}{Lösung zu \Cref{aufg:iglu_rutsche}}
\begin{enumerate}
    \item \textbf{Kräfteplan:} Es wirken die Gewichtskraft $\ivecS{F}{G}$ senkrecht nach unten und die Normalkraft $\ivecS{F}{N}$ senkrecht zur Oberfläche des Iglus nach außen (in $\ivecS{e}{r}$-Richtung).
    \begin{center}
        \includegraphics[width=0.50\linewidth]{Bilder/Uebungsaufgaben/schlittenfahren_iglu_kraefte.png}
        \label{fig:iglu_kraefte_loesung}
    \end{center}

    \item \textbf{Komponentenzerlegung:}
    \begin{itemize}
        \item Gewichtskraft: Der Winkel $\alpha$ ist zwischen der Vertikalen und dem Ortsvektor. Die Zerlegung ergibt 
        \begin{align}
            &F_{G,r} = -mg\cos(\alpha) & &\text{(zeigt entgegen $\ivecS{e}{r}$)} \\  &F_{G,t} = mg\sin(\alpha) & &\text{(zeigt in Richtung von $\ivecS{e}{t}$)} \mDot
        \end{align}
        \item Normalkraft: Wirkt nur in radialer Richtung
        \begin{align}
            &F_{N,r} = F_N &\text{(zeigt in Richtung $\ivecS{e}{r}$)}\\
            &F_{N,t} = 0 & \mDot
        \end{align}
    \end{itemize}

    \item \textbf{Bewegungsgleichungen:}
    \begin{align}
        &\text{radial:} &\sum F_r = F_N - mg\cos(\alpha) &= m a_r \label{eq:iglu_radial} \\
        &\text{tangential:} &\sum F_t = mg\sin(\alpha) &= m a_t \label{eq:iglu_tangential}
    \end{align}

    \item \textbf{Bedingung für das Abheben:}
    Im Moment des Abhebens ist $F_N=0$. Die radiale Beschleunigung ist die Zentripetalbeschleunigung, die zum Kreismittelpunkt zeigt, also $a_r = -v^2/R$. Eingesetzt in \Cref{eq:iglu_radial}:
    \begin{equation}
        0 - mg\cos(\alpha_{\text{max}}) = m \left(-\frac{v^2}{R}\right) \implies g\cos(\alpha_{\text{max}}) = \frac{v^2}{R} \label{eq:iglu_abheben}
    \end{equation}
    
    \item \textbf{Geschwindigkeit aus Energieerhaltung:}
    Die potentielle Energie am Startpunkt (Höhe $R$) ist $E_{\text{pot,start}} = mgR$. Bei einem Winkel $\alpha$ ist die Höhe $h(\alpha) = R\cos(\alpha)$, also $E_{\text{pot}}(\alpha) = mgR\cos(\alpha)$. Die Differenz wird in kinetische Energie umgewandelt: 
    \begin{equation}
        \Delta E_{\text{pot}} = mgR - mgR\cos(\alpha) = \frac{1}{2}mv^2 = E_{\text{kin}}
    \end{equation}
    Auflösen nach $v^2$ ergibt:
    \begin{equation}
        v^2 = 2gR(1-\cos(\alpha)) \label{eq:iglu_v2}
    \end{equation}
    
    \item \textbf{Bestimmung von $\alpha_{\text{max}}$:}
    Wir setzen \Cref{eq:iglu_v2} in \Cref{eq:iglu_abheben} ein: 
    \begin{equation}
        g\cos(\alpha_{\text{max}}) = \frac{2gR(1-\cos(\alpha_{\text{max}}))}{R}
    \end{equation}
    Wir kürzen $g$ und $R$ und lösen nach $\cos(\alpha_{\text{max}})$ auf:
    \begin{align}
        \cos(\alpha_{\text{max}}) &= 2(1-\cos(\alpha_{\text{max}})) \nonumber\\
        \cos(\alpha_{\text{max}}) &= 2 - 2\cos(\alpha_{\text{max}}) \nonumber\\
        3\cos(\alpha_{\text{max}}) &= 2 \nonumber\\
        \cos(\alpha_{\text{max}}) &= \frac{2}{3}
    \end{align}
    Der maximale Winkel ist somit:
    \begin{equation}
        \alpha_{\text{max}} = \arccos\left(\frac{2}{3}\right) \approx \ang{48.2}
    \end{equation}
\end{enumerate}
Man sieht, dass der maximale Winkel unabhängig vom Radius $R$, der Masse des Schlittenfahrers $m$ und sogar der Erdbeschleunigung $g$ ist. 
\end{loesungbox}







\chapter{Thermodynamische Grundlagen und Auftrieb}\label{chap:thermo_auftrieb}

\section{Aufgaben}\label{sec:thermo_auftrieb_aufgaben}

\begin{aufgabebox}{Wärmeausdehnung einer Pipeline}{pipeline_ausdehnung}
Ein $L = \SI{260}{\meter}$ langes Stück einer Stahlpipeline in Alaska kühlt im Winter auf eine Temperatur von $\SI{-50}{\celsius}$ ab. Wenn Öl hindurchfließt, erwärmt sich die Pipeline auf $\SI{+60}{\celsius}$. Der Längenausdehnungskoeffizient von Stahl sei $\alpha = \SI{11e-6}{\kelvin^{-1}}$. 
\begin{center}
    \includegraphics[width=0.25\linewidth]{Bilder/Uebungsaufgaben/pipeline_ausdehnung.png}
    \label{fig:pipeline}
\end{center}
\begin{enumerate}
    \item Um welche Länge $\Delta L$ dehnt sich der Pipelineabschnitt bei dieser Erwärmung aus? 
    \item Die gesamte oberirdische Länge der Pipeline beträgt $\SI{670}{\kilo\meter}$. Wie groß ist die Längenänderung der Gesamtstrecke bei derselben Temperaturänderung? 
    \item Wie würden sich die Ergebnisse aus 1) und 2) ändern, wenn man einen nichtlinearen Ausdehnungsterm mit $\beta = \SI{1.5e-8}{\kelvin^{-2}}$ berücksichtigt? 
\end{enumerate}
\end{aufgabebox}




\begin{aufgabebox}{Warme Autoreifen}{autoreifen_druck}
Ein Autoreifen wird nach einer längeren Fahrt an einer Tankstelle bis zu einem Druck von \SI{2.2}{\bar} bei einer Reifentemperatur von \SI{40}{\degreeCelsius} mit Luft gefüllt. Am nächsten Tag ist die Außentemperatur auf \SI{10}{\degreeCelsius} gesunken, und die Reifen haben sich entsprechend abgekühlt.
\begin{center}
    \includegraphics[width=0.35\linewidth]{Bilder/Uebungsaufgaben/reifen_erwärmung.png}
    \includegraphics[width=0.35\linewidth]{Bilder/Uebungsaufgaben/phasendiagramm_luft_reifen.png}
\end{center}
Nehmen Sie für die folgenden Berechnungen an, dass das Volumen der Reifen konstant bleibt und sich die Luft annähernd wie ein ideales Gas verhält.

\begin{enumerate}
    \item Berechnen Sie den Druck im Reifen bei der neuen Temperatur von \SI{10}{\celsius}.
    \item Wie viele Mol Luft befinden sich in einem Autoreifen mit einem Volumen von $V = \SI{40}{\litre}$ bei den ursprünglichen \SI{2.2}{\bar} und \SI{40}{\degreeCelsius}? (Gegeben: $R = \SI{8.314}{\joule\per(\mole\cdot\kelvin)}$)
    \item Skizzieren Sie den typischen Betriebsbereich eines Autoreifens (Drücke um 2 bis 3 \si{\bar}, Temperaturen von ca. -20 bis 50\si{\celsius}) qualitativ in dem Phasendiagramm für Luft ein. Ist unter normalen Bedingungen eine Verflüssigung der Luft im Reifen zu befürchten?
\end{enumerate}
\end{aufgabebox}





\begin{aufgabebox}{Phasendiagramm von Wasser}{phasendiagramm}
In der Abbildung sehen Sie das Phasendiagramm von Wasser ($H_{2}O$).
\begin{center}
    \includegraphics[width=0.44\textwidth]{Bilder/Uebungsaufgaben/phasendiagramm_wasser.png}
\end{center}
\begin{enumerate}
    \item Beschriften Sie die Schmelzkurve, die Dampfdruckkurve und die Sublimationskurve in der Grafik!
    \item Welches Vorzeichen hat die Steigung $\frac{\dd p}{\dd T}$ entlang dieser drei Kurven?
    \item Die Clausius-Clapeyron-Gleichung für den Schmelzprozess lautet:
    \begin{equation}\label{eq:clausius_clapeyron_schmelzen}
        \Lambda_{\text{Schmelz}} = T \frac{\dd p}{\dd T}(V_{\text{Fl}} - V_{\text{Fest}})
    \end{equation}
    Erklären Sie anhand dieser Formel und der Grafik die Dichteanomalie des Wassers.
    \item Erklären Sie, was passieren würde, wenn Sie eine Glasflasche bis zum Rand mit Wasser bei $T=\SI{20}{\degreeCelsius}$ füllen, sie fest verschließen und in den Gefrierschrank legen.
\end{enumerate}
\end{aufgabebox}




\begin{aufgabebox}{Thermische Ausdehnung}{thermische_ausdehnung_flaeche}
Eine rechteckige Platte mit den Seitenlängen $l$ und $b$ und besitzt einen linearen thermischen Ausdehnungskoeffizienten $\alpha$. Zeigen Sie, dass die Flächenänderung der Platte aufgrund der Temperaturänderung $\Delta T$ gegeben ist durch 
$$ \Delta A = 2\alpha A_0 \Delta T \mComma$$
wenn Änderungen proportional zu $(\alpha \cdot \Delta T)^2 \ll 1 $ vernachlässigt werden. 
\begin{center}
    \includegraphics[width=0.6\linewidth]{Bilder/Uebungsaufgaben/ausdehnung_platte.png}
\end{center}
\begin{enumerate}
    \item Geben Sie die Formeln für die Längenänderungen $\Delta l$ und $\Delta b$ infolge einer Temperaturänderung $\Delta T$ an.
    \item Die neue Fläche $A'$ nach der Erwärmung ist $A'=(l+\Delta l)(b+\Delta b)$. Leiten Sie daraus eine Formel für die Flächenänderung $\Delta A = A' - A$ her.
    \item Zeigen Sie, dass für kleine Änderungen die Flächenänderung näherungsweise durch
    \begin{equation}
        \Delta A \approx 2\alpha A \Delta T
    \end{equation}
    gegeben ist. Vernachlässigen Sie dabei Terme, die proportional zu $(\alpha \Delta T)^2$ sind.
    \item Berechnen Sie die Flächenänderung $\Delta A$ für eine Eisenplatte ($\alpha = \SI{12e-6}{\kelvin^{-1}}$) mit den Maßen $l = \SI{2.3}{\metre}$ und $b = \SI{1.2}{\metre}$, die um $\Delta T = \SI{40}{\degreeCelsius}$ erwärmt wird.
\end{enumerate}
\end{aufgabebox}



\begin{aufgabebox}{Statischer Auftrieb}{auftrieb_herleitung}
Leiten Sie das Archimedische Prinzip her. Es besagt, dass die Auftriebskraft $F_{\text{Auftrieb}}$, die auf einen Körper mit Volumen $V_\text{K}$ in einer Flüssigkeit der Dichte $\rho_{\text{Fl}}$ wirkt, gleich der Gewichtskraft der verdrängten Flüssigkeit ist: 
\begin{equation}
    F_{\text{Auftrieb}} = F_{G, \text{verdr. Fl.}} = \rho_{\text{Fl}} \cdot V_\text{K} \cdot g
\end{equation}
\begin{center}
    \includegraphics[width=0.40\linewidth]{Bilder/Uebungsaufgaben/statischer_auftrieb_kein_koerper.png}
    \hfill
    \includegraphics[width=0.40\linewidth]{Bilder/Uebungsaufgaben/statischer_auftrieb_mit_koerper.png}
    \label{fig:auftrieb_ohne_koerper}
\end{center}
\textbf{Vorgehen:} Betrachten Sie zunächst nur die Flüssigkeit ohne den eingetauchten Körper (linkes Bild). 
\begin{enumerate}
    \item Eine Flüssigkeitsschicht in der Tiefe $h_2$ muss das Gewicht der gesamten Flüssigkeitssäule über ihr tragen. Geben Sie die Drücke $p_1$ in der Tiefe $h_1$ und $p_2$ in Tiefe $h_2$ an, wenn der Druck $p$ auf die Fläche $A_\text{K}$ die Gewichtskraft der Flüssigkeitssäule ausgleicht
    \begin{equation}
        p = \frac{F_\text{G,Fl.säule}}{A_\text{K}} \mDot
    \end{equation}
    Welche Drücke $p_1$ bzw. $p_2$ stellen sich daher ein? \\
    \textit{Tipp: } Die Drücke sollen von $\rho_\text{Fl}$, $g$ und $h$ abhängen.
    \item Wie groß ist der Differenzdruck $\Delta p = p_2 - p_1$ in Abhängigkeit der Höhe des gedachten Körpers $\Delta h = h_2 - h_1$? 
    \item Nun wird ein realer Körper mit Volumen $V_\text{K}$ und Grundfläche $A_\text{K}$ in die Flüssigkeit eingebracht (rechtes Bild). Der Differenzdruck $\Delta p$ erzeugt eine nach oben gerichtete Kraftdifferenz $\Delta F$ auf die Ober- und Unterseite des Körpers. Geben Sie die Kraftdifferenz $\Delta F = \Delta p \cdot A_\text{K}$ zwischen Ober- und Unterseite an!
    \item Diese Kraftdifferenz ist die Auftriebskraft, $\Delta F = F_{\text{Auftrieb}}$. Formen Sie den Ausdruck aus 3) um, indem Sie das Volumen des Körpers $V_\text{K} = A_\text{K} \cdot \Delta h$ verwenden, um die finale Formel des Archimedischen Prinzips zu erhalten. 
\end{enumerate}
\end{aufgabebox}

\begin{aufgabebox}{Ballonfahrt}{ballonfahrt}
Ein Heißluftballon mit einem Volumen von $V_B = \SI{4500}{\meter\cubed}$ hat ein Leergewicht (Hülle, Korb, Passagiere) von $m_{\text{Leer}} = \SI{850}{\kilo\gram}$. Die Außentemperatur beträgt $T_{\text{Außen}} = \SI{20}{\celsius}$. Der Luftdruck auf Meereshöhe ist $p_0 = \SI{101325}{\pascal}$, die molare Masse von Luft $M_{\text{mol}} = \SI{28.96e-3}{\kilo\gram\per\mol}$, und die allgemeine Gaskonstante $R = \SI{8.314}{\joule\per(\mol\cdot\kelvin)}$. \\
Vorsicht: Achten Sie auf die Verwendung von SI-Einheiten (Kelvin statt Celsius). 
\begin{center}
    \includegraphics[width=0.32\linewidth]{Bilder/Uebungsaufgaben/ballonfahrt.png}
    \label{fig:ballonfahrt}
\end{center}
\begin{enumerate}
    \item Leiten Sie aus der idealen Gasgleichung $p V = n R T$ eine Formel für die Dichte der Luft $\rho(p, T)$ als Funktion von Druck und Temperatur her. \\
    \textit{Tipp: } Die Masse $M$ hängt mit der Molmasse über $M = n\cdot M_{\text{mol}}$ zusammen.
    \item Wie groß ist die gesamte Gewichtskraft $F_{G, \text{Ballon}}$ des Ballons (Leergewicht + Füllung), wenn die Luft im Inneren eine Temperatur von $T_{\text{Innen}} = \SI{60}{\celsius}$ bei $p = \SI{1}{\atm}$ hat? 
    \item Berechnen Sie die Auftriebskraft $F_{\text{Auftrieb}}$ des Ballons auf Meereshöhe bei $T_{\text{Innen}} = \SI{60}{\celsius}$. Hebt der Ballon ab? 
    \item Welche Mindesttemperatur $T_{\text{Innen,min}}$ muss die Luft im Ballon haben, damit er gerade abhebt (Schwebezustand)? 
    \item \textbf{Bonus:} Die maximale Innentemperatur beträgt $T_{\text{Innen,max}} = \SI{125}{\celsius}$. Berechnen Sie die maximale Steighöhe $h_{\text{max}}$. Verwenden Sie dafür die barometrische Höhenformel $p(h) = p_0 \exp(-h/H)$ mit einer Skalenhöhe von $H \approx \SI{8.3}{\kilo\meter}$ für eine isotherme Atmosphäre ($T_{\text{Atmosphäre}} = \SI{20}{\celsius}$). 
\end{enumerate}
\end{aufgabebox}




\begin{aufgabebox}{Holzklotz im Wasser}{auftrieb_holzklotz}
Ein würfelförmiger Holzklotz mit einer Kantenlänge von $L = \SI{20}{\centi\metre}$ und einer Dichte von $\rho_{\text{Holz}} = \SI{750}{\kilogram\per\metre\cubed}$ wird in ein Wasserbecken gelegt. Die Dichte von Wasser beträgt $\rho_{\text{Wasser}} = \SI{1000}{\kilogram\per\metre\cubed}$.
\begin{center}
    \includegraphics[width=0.4\textwidth]{Bilder/Uebungsaufgaben/holzklotz_eintauchen.png}
\end{center}
Berechnen Sie eine Formel für die Eintauchtiefe $h$ des Holzklotzes! Wie tief taucht der Holzklotz in der Angabe in das Wasser ein (in $\si{\centi\meter}$)? 
\begin{enumerate}
    \item Formulieren Sie die Gleichgewichtsbedingung für das Schwimmen des Körpers. Welche beiden Kräfte müssen sich die Waage halten?
    \item Berechnen Sie das Gesamtvolumen $V_\text{ges}$ des Holzklotzes und daraus seine Gewichtskraft $F_G$ (als Funktion der Kantenlänge $L$).
    \item Stellen Sie eine allgemeine Formel für die Auftriebskraft $F_A$ auf, die vom eingetauchten Volumen $V_{\text{ein}}$ abhängt.
    \item Drücken Sie das eingetauchte Volumen $V_{\text{ein}}$ durch die Kantenlänge $L$ und die gesuchte Eintauchtiefe $h$ aus. Setzen Sie dies in die Formel für die Auftriebskraft aus (3) ein.
    \item Setzen Sie die Formel für die Gewichtskraft aus (2) und die der Auftriebskraft aus (4) in (1) ein und lösen Sie die Gleichung nach der Eintauchtiefe $h$ auf.
    \item Berechnen Sie, wie tief der Holzklotz in das Wasser eintaucht.
\end{enumerate}
\end{aufgabebox}




\begin{aufgabebox}{Statischer Auftrieb mit Heliumballons}{heliumballons}
Sie möchten mithilfe von Helium-Ballonen eine Person zum Schweben bringen. Ein einzelner Ballon wiegt \SI{3}{\gram} und hat ein Volumen von \SI{4.5}{\litre}. Die Dichte der Umgebungsluft beträgt $\rho_{\text{Luft}}=\SI{1.23}{\kilogram\per\metre\cubed}$, die von Helium $\rho_{\text{He}}=\SI{0.18}{\kilogram\per\metre\cubed}$. Die Erdbeschleunigung ist $g \approx \SI{9.81}{\metre\per\second\squared}$.
\begin{center}
    \includegraphics[height=2.5cm]{Bilder/Uebungsaufgaben/balloons.png}
\end{center}
\begin{enumerate}
    \item Wie groß ist die Nettoauftriebskraft eines einzelnen Heliumballons? Berechnen Sie dazu zuerst die Auftriebskraft nach dem Archimedischen Prinzip und ziehen Sie dann die Gewichtskräfte der Ballonhülle und der Heliumfüllung ab.
    \item Wie viele solcher Ballons werden benötigt, um eine Person mit einer Masse von \SI{80}{\kilogram} schweben zu lassen?
    \item Angenommen, Sie verwenden mehr Ballons als zum Schweben nötig sind. Erklären Sie, warum die Ballons trotzdem nur eine maximale Steighöhe erreichen.
    \item Vergleichen Sie den Auftrieb in der Atmosphäre mit dem Auftrieb unter Wasser. Welcher wesentliche Unterschied besteht bezüglich der Dichteänderung des umgebenden Mediums als Funktion der Höhe bzw. Tiefe? Begründen Sie Ihre Antwort.
    \item Stellen Sie sich vor, Sie füllen bei einem Tauchgang in \SI{100}{\metre} Tiefe einen Ballon mit Luft aus Ihrer Druckluftflasche, verknoten ihn und lassen ihn los. Beschreiben Sie, was mit dem Ballon geschieht, während er aufsteigt.
\end{enumerate}
\end{aufgabebox}




\begin{aufgabebox}{Eigenschaften idealer Gase}{ideale_gase}
Die ideale Gasgleichung lautet $$p V = n R T \mComma$$ 
und beschreibt den Zusammenhang zwischen Druck $p$, Volumen $V$, Temperatur $T$ und Stoffmenge $n$ mit $R = \SI{8.314}{\joule\per(\mol\cdot\kelvin)}$.
\begin{center}
    \includegraphics[width=0.35\linewidth]{Bilder/Uebungsaufgaben/ideales_gas_teilchen.png}
    \label{fig:ideales_gas_teilchen}
\end{center}

\begin{enumerate}
    \item Welches Volumen nimmt $n = \SI{1}{\mol}$ eines idealen Gases unter Normalbedingungen ($p=\SI{1}{\atm}$, $T=\SI{0}{\celsius}$) ein? 
    \item Wie viele Teilchen $N$ enthält dieses Volumen? Wie nennt man diese fundamentale Konstante? Verwenden Sie dazu die alternative Form der Gasgleichung $p V = N k_B T$ mit der Boltzmann-Konstante $k_B \approx \SI{1.38054e-23}{\joule\per\kelvin}$. 
    \item Die mittlere kinetische Energie eines Gasteilchens ist $\frac{1}{2}m\overline{v^2} = \frac{3}{2}k_B T$. Berechnen Sie die Wurzel aus dem mittleren Geschwindigkeitsquadrat ($\sqrt{\overline{v^2}}$) für:
    \begin{itemize}
        \item Heliumatome ($m_{\text{He}} \approx \SI{6.65e-27}{\kilo\gram}$) 
        \item Stickstoffmoleküle ($m_{\text{N}_2} \approx \SI{4.65e-26}{\kilo\gram}$) 
        \item Sauerstoffmoleküle ($m_{\text{O}_2} \approx \SI{5.31e-26}{\kilo\gram}$) 
    \end{itemize}
    jeweils bei einer Raumtemperatur von $T=\SI{20}{\celsius}$. 
\end{enumerate}
\end{aufgabebox}




\newpage
\section{Lösungen}\label{sec:thermo_auftrieb_loesungen}

\begin{loesungbox}{Lösung zu \Cref{aufg:pipeline_ausdehnung}}
Die Temperaturänderung beträgt $\Delta T = \SI{60}{\celsius} - (\SI{-50}{\celsius}) = \SI{110}{\celsius} = \SI{110}{\kelvin}$. 
\begin{enumerate}
    \item \textbf{Längenänderung des Pipelineabschnitts:}
    Die lineare Längenänderung ist $\Delta L = \alpha L \Delta T$. 
    \begin{equation}
        \Delta L = (\SI{11e-6}{\per\kelvin}) \cdot (\SI{260}{\meter}) \cdot (\SI{110}{\kelvin}) \approx \SI{0.315}{\meter}
    \end{equation}
    
    \item \textbf{Längenänderung der Gesamtstrecke:}
    Wir verwenden dieselbe Formel für die Gesamtlänge $L_{\text{Ges}} = \SI{670}{\kilo\meter} = \SI{670000}{\meter}$.
    \begin{equation}
        \Delta L_{\text{Ges}} = \alpha L_{\text{Ges}} \Delta T = (\SI{11e-6}{\per\kelvin}) \cdot (\SI{670000}{\meter}) \cdot (\SI{110}{\kelvin}) \approx \SI{810.7}{\meter}
    \end{equation}
    
    \item \textbf{Längenänderung mit nicht-linearem Term:}
    Die Formel für die nicht-lineare Längenänderung lautet 
    $$\Delta L = L(\alpha\Delta T + \beta (\Delta T)^2) = L\alpha\Delta T + L\beta(\Delta T)^2 \mDot $$ 
    Die beiden Pipelineabschnitte dehnen sich unter dem Einfluss einer nicht-linearen Ausdehnung wie folgt:
    \begin{itemize}
        \item \textbf{Pipelineabschnitt (260 m):}
        \begin{multline}
            \Delta L = \SI{0.315}{\meter} + (\SI{260}{\meter}) \cdot (\SI{1.5e-8}{\per\kelvin\squared}) \cdot (\SI{110}{\kelvin})^2 \\
            \approx \SI{0.315}{\meter} + \SI{0.047}{\meter} \approx \SI{0.362}{\meter}
        \end{multline}
        \item \textbf{Gesamtstrecke (670 km):}
        \begin{multline}
            \Delta L_{\text{Ges}} = \SI{810.7}{\meter} + (\SI{670000}{\meter}) \cdot (\SI{1.5e-8}{\per\kelvin\squared}) \cdot (\SI{110}{\kelvin})^2 \\
            \approx \SI{810.7}{\meter} + \SI{121.8}{\meter} \approx \SI{932.5}{\meter}
        \end{multline}
    \end{itemize}
\end{enumerate}
\end{loesungbox}




\begin{loesungbox}{Lösung zu \Cref{aufg:autoreifen_druck}}
\begin{enumerate}
    \item \textbf{Druck bei \SI{10}{\degreeCelsius}} \\
    Das das Volumen gleich bleibt, können wir die beiden Volumina gleichsetzen
    \begin{align}
        \frac{n R T_1}{p_1} = V_1 &= V_2 = \frac{n R T_2}{p_2} \nonumber\\
        \implies \frac{p_1}{T_1} &= \frac{p_2}{T_2} \mDot
    \end{align}
    Die Temperaturen müssen in der absoluten Temperaturskala (Kelvin) angegeben werden:
    \begin{itemize}
        \item $T_1 = \SI{40}{\degreeCelsius} = (40 + 273,15)\,\si{\kelvin} = \SI{313.15}{\kelvin}$
        \item $T_2 = \SI{10}{\degreeCelsius} = (10 + 273,15)\,\si{\kelvin} = \SI{283.15}{\kelvin}$
    \end{itemize}
    Nun lösen wir nach dem neuen Druck $p_2$ auf:
    \begin{equation}\label{eq:druck_reifen_neu}
        p_2 = p_1 \cdot \frac{T_2}{T_1} = \SI{2.2}{\bar} \cdot \frac{\SI{283.15}{\kelvin}}{\SI{313.15}{\kelvin}} \approx \SI{1.99}{\bar}
    \end{equation}
    Der Druck im Reifen beträgt bei \SI{10}{\degreeCelsius} ca. \SI{1.99}{\bar}.

    \item \textbf{Anzahl der Mol im Reifen} \\
    Wir verwenden die ideale Gasgleichung und lösen nach der Stoffmenge $n$ auf.
    \begin{equation}\label{eq:ideale_gasgleichung_autoreifen}
        pV = nRT \quad \Rightarrow \quad n = \frac{pV}{RT}
    \end{equation}
    Dafür müssen die Einheiten SI-konform sein:
    \begin{itemize}
        \item $p_1 = \SI{2.2}{\bar} = \SI{2.2e5}{\pascal}$
        \item $V = \SI{40}{\litre} = \SI{0.04}{\cubic\meter}$
        \item $T_1 = \SI{313.15}{\kelvin}$
    \end{itemize}
    Eingesetzt ergibt dies:
    \begin{equation}\label{eq:mol_reifen}
        n = \frac{\SI{2.2e5}{\pascal} \cdot \SI{0.04}{\cubic\meter}}{\SI{8.314}{\joule\per(\mole\cdot\kelvin)} \cdot \SI{313.15}{\kelvin}} \approx \SI{3.38}{mol}
    \end{equation}

    \item \textbf{Phasendiagramm} \\
    Der typische Betriebsbereich eines Autoreifens liegt bei Drücken um \SIrange{2}{3}{\bar} und Temperaturen von ca. \SIrange{-20}{50}{\celsius}. Dieser Bereich befindet sich im Phasendiagramm von Luft (hauptsächlich Stickstoff und Sauerstoff) weit oberhalb der Siedetemperatur und weit entfernt von der Sublimations- oder Schmelzkurve. Eine Verflüssigung oder Erstarrung ist unter normalen Fahrbedingungen ausgeschlossen.
    \begin{center}
        \includegraphics[width=0.6\textwidth]{Bilder/Uebungsaufgaben/phasendiagramm_luft_loesung.png}
    \end{center}
\end{enumerate}
\end{loesungbox}



\begin{loesungbox}{Lösung zu \Cref{aufg:phasendiagramm}}
\begin{enumerate}
    \item \textbf{Beschriftung der Kurven:} Siehe Grafik.
    \begin{center}
        \includegraphics[width=0.5\textwidth]{Bilder/Uebungsaufgaben/phasendiagramm_wasser_loesung.png}
    \end{center}
    \item \textbf{Vorzeichen der Steigung:}
    \begin{itemize}
        \item \textbf{Sublimationskurve} ($p_{\text{Sub}}(T)$): $\frac{\dd p}{\dd T} > 0$ (positive Steigung)
        \item \textbf{Dampfdruckkurve} ($p_{\text{Dampf}}(T)$): $\frac{\dd p}{\dd T} > 0$ (positive Steigung)
        \item \textbf{Schmelzkurve} ($p_{\text{Schm}}(T)$): $\frac{\dd p}{\dd T} < 0$ (negative Steigung)
    \end{itemize}
    \item \textbf{Anomalie des Wassers:}
    Die Schmelzwärme $\Lambda_{\text{Schmelz}}$ und die absolute Temperatur $T$ sind stets positiv. 
    \begin{equation}
        \underbrace{\Lambda_{\text{Schmelz}}}_{>0} = \underbrace{T}_{>0} \frac{\dd p}{\dd T}(V_{\text{Fl}} - V_{\text{Fest}})
    \end{equation}
    Wie oben festgestellt, ist die Steigung der Schmelzkurve $\frac{\dd p}{\dd T}$ für Wasser negativ. Damit die Gleichung erfüllt ist, muss der Term $(V_{\text{Fl}} - V_{\text{Fest}})$ ebenfalls negativ sein. Das bedeutet $V_{\text{Fl}} < V_{\text{Fest}}$. Festes Wasser (Eis) hat also ein größeres Volumen (und damit eine geringere Dichte) als flüssiges Wasser bei gleicher Masse. Dies ist die Dichteanomalie des Wassers.
    
    \item \textbf{Glasflasche im Gefrierschrank:}
    Da sich Wasser beim Gefrieren ausdehnt (Anomalie), vergrößert sich sein Volumen. In einer verschlossenen, vollen Flasche kann sich das Eis nicht ungehindert ausdehnen. Der dadurch entstehende Druck auf die Flaschenwände wird so groß, dass die Glasflasche platzt.
\end{enumerate}
\end{loesungbox}





\begin{loesungbox}{Lösung zu \Cref{aufg:thermische_ausdehnung_flaeche}}
\begin{enumerate}
    \item \textbf{Längenänderungen $\Delta l$ und $\Delta b$} \\
    Gemäß der Formel für die lineare thermische Ausdehnung gilt:
    \begin{equation}
        \Delta l = \alpha \cdot l \cdot \Delta T \quad \text{und} \quad \Delta b = \alpha \cdot b \cdot \Delta T \mDot
    \end{equation}

    \item \textbf{Formel für die Flächenänderung $\Delta A$} \\
    Die neue Fläche $A'$ ist das Produkt der neuen Seitenlängen $l' = l + \Delta l$ und $b' = b + \Delta b$. Die Flächenänderung ist $\Delta A = A' - A$:
    \begin{equation}\label{eq:delta_A_exact}
        \Delta A = (l + \Delta l)(b + \Delta b) - (l\cdot b) = l \Delta b + b \Delta l + \Delta l \Delta b
    \end{equation}

    \item \textbf{Vereinfachung} \\
    Wir ersetzen $\Delta l$ und $\Delta b$ in \Cref{eq:delta_A_exact} durch die Ausdrücke aus (1):
    \begin{align}
        \Delta A &= l (\alpha b \Delta T) + b (\alpha l \Delta T) + (\alpha l \Delta T)(\alpha b \Delta T) \nonumber \\
        &= \alpha l b \Delta T + \alpha b l \Delta T + \alpha^2 l b (\Delta T)^2 \nonumber \\
        &= 2\alpha (l b) \Delta T + (l b) (\alpha \Delta T)^2
    \end{align}
    Da $A = l b$ und der Term $(\alpha \Delta T)^2$ für typische Materialien und Temperaturänderungen sehr klein und somit laut Angabe vernachlässigbar ist, ergibt sich:
    \begin{equation}\label{eq:delta_A_approx}
        \Delta A = A(2\alpha \Delta T + (\alpha\Delta T)^2) \approx 2\alpha A \Delta T \mDot
    \end{equation}
    Der Flächenausdehnungskoeffizient ist also unter dieser Näherung das Doppelte des Längenausdehnungskoeffizienten.

    \item \textbf{Numerisches Beispiel} \\
    Gegeben sind $\alpha = \SI{12e-6}{\per\kelvin}$, $l = \SI{2.3}{\meter}$, $b = \SI{1.2}{\meter}$ und $\Delta T = \SI{40}{\kelvin}$. Zuerst berechnen wir die Anfangsfläche $A$:
    \begin{equation}
        A = l \cdot b = \SI{2.3}{\m} \cdot \SI{1.2}{\m} = \SI{2.76}{\square\meter}
    \end{equation}
    Nun setzen wir die Werte in die hergeleitete Formel \Cref{eq:delta_A_approx} ein:
    \begin{equation}
        \Delta A = 2 \cdot (\SI{12e-6}{\kelvin^{-1}}) \cdot \SI{2.76}{\square\meter} \cdot \SI{40}{\kelvin} \approx \SI{0.00265}{\square\meter} = \SI{26.5}{\square\centi\meter}
    \end{equation}
\end{enumerate}
\end{loesungbox}




\begin{loesungbox}{Lösung zu \Cref{aufg:auftrieb_herleitung}}
\begin{enumerate}
    \item \textbf{Hydrostatischer Druck:}
    Der Druck $p_2$ hält die Flüssigkeitssäule mit einem Gewicht 
    $$F_{\text{G,Fl.säule}} = M_{\text{Fl},h_2} \cdot g = \rho_\text{Fl} \cdot V_{\text{Fl},h_2} \cdot g \mDot$$
    Daher ergibt sich ein Druck von 
    $$p_2 = \frac{F_{\text{G,Fl.säule}}}{A_{\text{K}}} = \frac{\rho_{\text{Fl}} \cdot V_{\text{Fl},h_2} \cdot g}{A_\text{K}} = \rho_{\text{Fl}}\cdot g \cdot h_2 \mDot$$
    Der Druck $p_1$ ergibt sich durch die gleiche Überlegung zu 
    $$p_1 = \rho_{\text{Fl}}\cdot g \cdot h_1 \mDot$$
    Dieser Druck nennt sich hydrostatischer Druck $p(h) = \rho_\text{Fl} \cdot g \cdot h$.

    \item \textbf{Differenzdruck:}
    Die Differenz der beiden Drücke ist:
    \begin{equation}
        \Delta p = p_2 - p_1 = \rho_{\text{Fl}}\cdot g \cdot h_2 - \rho_{\text{Fl}}\cdot g\cdot h_1 = \rho_{\text{Fl}}\cdot g\cdot (h_2 - h_1) = \rho_{\text{Fl}}\cdot g\cdot \Delta h \mDot
    \end{equation}
    
    \item \textbf{Kraftdifferenz:}
    Die aus dem Differenzdruck resultierende Kraft auf die Fläche $A_\text{K}$ ist: 
    \begin{equation}
        \Delta F = \Delta p \cdot A_\text{K} = (\rho_{\text{Fl}} g \Delta h) \cdot A_\text{K} \label{eq:deltaF_auftrieb}
    \end{equation}
    
    \item \textbf{Herleitung des Archimedischen Prinzips:}
    Die Kraftdifferenz $\Delta F$ ist die Auftriebskraft $F_{\text{Auftrieb}}$. Der Term $\Delta h \cdot A_\text{K}$ in \Cref{eq:deltaF_auftrieb} entspricht genau dem Volumen des eingetauchten Körpers, $V_\text{K}$. Somit folgt:
    \begin{equation}
        F_{\text{Auftrieb}} = \rho_{\text{Fl}} g \cdot (\underbrace{A_\text{K}\cdot \Delta h}_{V_\text{K}}) = \rho_{\text{Fl}} g V_\text{K} \mDot
    \end{equation}
    Der Term $\rho_{\text{Fl}} V_\text{K}$ ist die Masse der Flüssigkeit, die durch das Volumen des Körpers verdrängt wurde. Die Auftriebskraft entspricht also genau der Gewichtskraft der verdrängten Flüssigkeit. 
\end{enumerate}
\end{loesungbox}


\begin{loesungbox}{Lösung zu \Cref{aufg:ballonfahrt}}
Zuerst rechnen wir alle Temperaturen in die SI-Einheit Kelvin um:
\begin{align*}
    T_{\text{Außen}} &= \SI{20}{\celsius} = \SI{293.15}{\kelvin} \\
    T_{\text{Innen}} &= \SI{60}{\celsius} = \SI{333.15}{\kelvin} \\
    T_{\text{Innen,max}} &= \SI{125}{\celsius} = \SI{398.15}{\kelvin}
\end{align*}
Der Druck auf Meereshöhe ist $p_0 = \SI{101325}{\pascal} \approx \SI{1}{\atm}$.

\begin{enumerate}
    \item \textbf{Herleitung der Dichteformel:}
    Ausgangspunkt ist die ideale Gasgleichung:
    \begin{equation}
        p V = n R T
    \end{equation}
    Die Stoffmenge $n$ kann durch die Gesamtmasse $m$ und die molare Masse $M_{\text{mol}}$ ausgedrückt werden: $n = m/M_{\text{mol}}$. Eingesetzt ergibt das:
    \begin{equation}
        p V = \frac{m}{M_{\text{mol}}} R T
    \end{equation}
    Die Dichte ist definiert als $\rho = m/V$. Wir formen die Gleichung um, um diesen Ausdruck zu erhalten:
    \begin{equation}
        p \cdot M_{\text{mol}} = \frac{m}{V} R T = \rho R T
    \end{equation}
    Daraus folgt die gesuchte Formel für die Dichte:
    \begin{equation} \label{eq:dichte_gas}
        \rho(p, T) = \frac{p \cdot M_{\text{mol}}}{R \cdot T}
    \end{equation}

    \item \textbf{Gesamte Gewichtskraft:}
    Die Gesamtmasse $m_{\text{Gesamt}}$ setzt sich aus der Leermasse $m_{\text{Leer}}$ und der Masse der heißen Luft im Ballon $m_{\text{Innen}}$ zusammen. Zuerst berechnen wir die Dichte der Innenluft bei $T_{\text{Innen}} = \SI{333.15}{\kelvin}$:
    \begin{equation}
        \rho_{\text{Innen}} = \frac{\SI{101325}{\pascal} \cdot \SI{28.96e-3}{\kilo\gram\per\mol}}{\SI{8.314}{\joule\per(\mol\cdot\kelvin)} \cdot \SI{333.15}{\kelvin}} \approx \SI{1.059}{\kilo\gram\per\meter\cubed}
    \end{equation}
    Die Masse der Füllung ist $m_{\text{Innen}} = \rho_{\text{Innen}} \cdot V_B = \SI{1.059}{\kilo\gram\per\meter\cubed} \cdot \SI{4500}{\meter\cubed} \approx \SI{4765.5}{\kilo\gram}$.
    Die Gesamtmasse beträgt:
    \begin{equation}
        m_{\text{Gesamt}} = m_{\text{Leer}} + m_{\text{Innen}} = \SI{850}{\kilo\gram} + \SI{4765.5}{\kilo\gram} = \SI{5615.5}{\kilo\gram}
    \end{equation}
    Die Gewichtskraft ist somit:
    \begin{equation}
        F_{G, \text{Ballon}} = m_{\text{Gesamt}} \cdot g = \SI{5615.5}{\kilo\gram} \cdot \SI{9.81}{\meter\per\second\squared} \approx \SI{55088}{\newton}
    \end{equation}
    
    \item \textbf{Auftriebskraft und Abheben:}
    Die Auftriebskraft entspricht dem Gewicht der verdrängten Außenluft. Wir benötigen die Dichte der Außenluft bei $T_{\text{Außen}} = \SI{293.15}{\kelvin}$:
    \begin{equation}
        \rho_{\text{Außen}} = \frac{\SI{101325}{\pascal} \cdot \SI{28.96e-3}{\kilo\gram\per\mol}}{\SI{8.314}{\joule\per(\mol\cdot\kelvin)} \cdot \SI{293.15}{\kelvin}} \approx \SI{1.205}{\kilo\gram\per\meter\cubed}
    \end{equation}
    Die Auftriebskraft ist nach dem Archimedischen Prinzip:
    \begin{equation}
        F_{\text{Auftrieb}} = \rho_{\text{Außen}} \cdot V_B \cdot g = \SI{1.205}{\kilo\gram\per\meter\cubed} \cdot \SI{4500}{\meter\cubed} \cdot \SI{9.81}{\meter\per\second\squared} \approx \SI{53195}{\newton}
    \end{equation}
    Vergleich: $F_{\text{Auftrieb}} (\approx \SI{53.2}{\kilo\newton}) < F_{G, \text{Ballon}} (\approx \SI{55.1}{\kilo\newton})$.
    Die Auftriebskraft ist geringer als die Gewichtskraft. \textbf{Der Ballon hebt nicht ab.}

    \item \textbf{Mindesttemperatur zum Abheben:}
    Im Schwebezustand sind Auftriebskraft und Gewichtskraft gleich: $F_{\text{Auftrieb}} = F_{G, \text{Ballon}}$.
    \begin{equation}
        \rho_{\text{Außen}} \cdot V_B \cdot g = (m_{\text{Leer}} + m_{\text{Innen}}) \cdot g = (m_{\text{Leer}} + \rho_{\text{Innen,min}} \cdot V_B) \cdot g
    \end{equation}
    Wir setzen die Dichteformel \Cref{eq:dichte_gas} für $\rho_{\text{Innen,min}}$ ein:
    \begin{equation}
        \rho_{\text{Außen}} \cdot V_B = m_{\text{Leer}} + \frac{p_0 \cdot M_{\text{mol}}}{R \cdot T_{\text{Innen,min}}} \cdot V_B
    \end{equation}
    Nun lösen wir nach $T_{\text{Innen,min}}$ auf:
    \begin{align}
        (\rho_{\text{Außen}} \cdot V_B - m_{\text{Leer}}) &= \frac{p_0 M_{\text{mol}} V_B}{R \cdot T_{\text{Innen,min}}} \\
        T_{\text{Innen,min}} &= \frac{p_0 M_{\text{mol}} V_B}{R \cdot (\rho_{\text{Außen}} V_B - m_{\text{Leer}})}
    \end{align}
    Einsetzen der Werte:
    \begin{equation}
        T_{\text{Innen,min}} = \frac{\SI{101325}{\pascal} \cdot \SI{28.96e-3}{\kilo\gram\per\mol} \cdot \SI{4500}{\meter\cubed}}{\SI{8.314}{\frac{\joule}{\mol\kelvin}} \cdot (\SI{1.205}{\frac{\kilo\gram}{\meter\cubed}} \cdot \SI{4500}{\meter\cubed} - \SI{850}{\kilo\gram})} \approx \SI{347.1}{\kelvin}
    \end{equation}
    Das entspricht einer Temperatur von $T_{\text{Innen,min}} \approx \SI{73.95}{\celsius}$.

    \item \textbf{Bonus: Maximale Steighöhe:}
    Der Ballon steigt, bis seine durchschnittliche Dichte der Dichte der umgebenden Luft entspricht. In der Schwebe auf Höhe $h$ gilt:
    \begin{equation}
        \rho_{\text{Ballon}}(h) = \frac{m_{\text{Leer}} + m_{\text{Innen}}(h)}{V_B} = \rho_{\text{Außen}}(h)
    \end{equation}
    Die Massen der Luftfüllung $m_{\text{Innen}}$ und der verdrängten Luft $\rho_{\text{Außen}}$ hängen beide vom Umgebungsdruck $p(h)$ ab.
    \begin{equation}
        \frac{m_{\text{Leer}}}{V_B} + \frac{p(h) M_{\text{mol}}}{R T_{\text{Innen,max}}} = \frac{p(h) M_{\text{mol}}}{R T_{\text{Atmosphäre}}}
    \end{equation}
    Wir lösen nach dem Druck $p(h)$ auf, bei dem der Schwebezustand erreicht wird:
    \begin{align}
        \frac{m_{\text{Leer}}}{V_B} &= p(h) \frac{M_{\text{mol}}}{R} \left( \frac{1}{T_{\text{Atmosphäre}}} - \frac{1}{T_{\text{Innen,max}}} \right) \\
        p(h_{\text{max}}) &= \frac{m_{\text{Leer}} R}{V_B M_{\text{mol}}} \left( \frac{T_{\text{Innen,max}} - T_{\text{Atmosphäre}}}{T_{\text{Atmosphäre}} T_{\text{Innen,max}}} \right)^{-1}
    \end{align}
    Wir berechnen diesen Druck:
    \begin{equation}
        p(h_{\text{max}}) = \frac{\SI{850}{\kilo\gram} \cdot \SI{8.314}{\frac{\joule}{\mol\kelvin}}}{\SI{4500}{\meter\cubed} \cdot \SI{28.96e-3}{\frac{\kilo\gram}{\mol}}} \left( \frac{\SI{398.15}{\kelvin} - \SI{293.15}{\kelvin}}{\SI{293.15}{\kelvin} \cdot \SI{398.15}{\kelvin}} \right)^{-1} \approx \SI{61274}{\pascal}
    \end{equation}
    Mit der barometrischen Höhenformel $p(h) = p_0 e^{-h/H}$ finden wir die Höhe $h_{\text{max}}$:
    \begin{align}
        \frac{p(h_{\text{max}})}{p_0} &= e^{-h_{\text{max}}/H} \\
        h_{\text{max}} &= -H \cdot \ln\left(\frac{p(h_{\text{max}})}{p_0}\right)
    \end{align}
    Einsetzen der Werte:
    \begin{equation}
        h_{\text{max}} = -\SI{8.3}{\kilo\meter} \cdot \ln\left(\frac{\SI{61274}{\pascal}}{\SI{101325}{\pascal}}\right) \approx -\SI{8.3}{\kilo\meter} \cdot (-0.504) \approx \SI{4.18}{\kilo\meter}
    \end{equation}
    Die maximale Steighöhe beträgt ca. \SI{4180}{\meter}.
\end{enumerate}
\end{loesungbox}


\begin{loesungbox}{Lösung zu \Cref{aufg:ballonfahrt}}
Zuerst die Temperaturen in Kelvin: \\
$T_{\text{Außen}} = 20 + 273.15 = \SI{293.15}{\kelvin}$. \\
$T_{\text{Innen}} = 60 + 273.15 = \SI{333.15}{\kelvin}$. 
\begin{enumerate}
    \item \textbf{Dichte aus der idealen Gasgleichung:}
    Wir ersetzen die Stoffmenge $n = M / M_{\text{mol}}$ und das Volumen $V = M / \rho$. 
    \begin{gather}
        p \left(\frac{M}{\rho}\right) = p\cdot V = n \cdot R \cdot T = \left(\frac{M}{M_{\text{mol}}}\right) R T \nonumber\\
        \implies \frac{p}{\rho} = \frac{RT}{M_{\text{mol}}}
    \end{gather}
    Umgeformt nach der Dichte ergibt sich:
    \begin{equation} \label{eq:dichte_gas}
        \rho(p, T) = \frac{p M_{\text{mol}}}{R T} \mDot
    \end{equation}
    
    \item \textbf{Gesamtgewicht des Ballons:}
    Das Gesamtgewicht ist die Summe aus Leergewicht und dem Gewicht der heißen Luft im Inneren. Die Masse der Luft im Inneren ist $m_{\text{Innen}} = \rho(p_0, T_{\text{Innen}}) \cdot V_B$. 
    \begin{align}
        F_{G, \text{Ballon}} &= (m_{\text{Leer}} + m_{\text{Innen}})g = \big(m_{\text{Leer}} + \underbrace{\frac{p_0 M_{\text{mol}}}{R T_{\text{Innen}}}}_{\rho(p_0, T_{\text{Innen}})} V_B\big)g \nonumber\\
        &= \left(\SI{850}{\kilo\gram} + \frac{(\SI{101325}{\pascal})(\SI{28.96e-3}{\kilo\gram\per\mole})}{(\SI{8.314}{\joule\per(\mole\cdot\kelvin)})(\SI{333.15}{\kelvin})} \SI{4500}{\meter^3}\right) \SI{9.81}{m/s^2} \nonumber\\
        &\approx \SI{55106}{\newton}
    \end{align}
    
    \item \textbf{Auftriebskraft:}
    Die Auftriebskraft ist das Gewicht der verdrängten Außenluft. 
    \begin{equation}
        F_{\text{Auftrieb}} = \rho(p_0, T_{\text{Außen}}) \cdot V_B \cdot g = \frac{p_0 M_{\text{mol}}}{R T_{\text{Außen}}} \cdot V_B \cdot g \approx \SI{53149}{\newton}
    \end{equation}
    Da $F_{\text{Auftrieb}} < F_{G, \text{Ballon}}$ ($\SI{53149}{N} < \SI{55106}{N}$), hebt der Ballon nicht ab. 
    
    \item \textbf{Mindesttemperatur für das Abheben:}
    Der Ballon hebt ab, wenn $F_{\text{Auftrieb}} = F_{G, \text{Ballon}}$. 
    \begin{align}
        \rho(p_0, T_{\text{Außen}}) V_B g &= (m_{\text{Leer}} + \rho(p_0, T_{\text{Innen,min}}) V_B) g \\
        \frac{p_0 M_{\text{mol}}}{R T_{\text{Außen}}} V_B g &= \left(m_\text{Leer} + \frac{p_0 M_{\text{mol}}}{R T_{\text{Innen,min}}} V_B\right) g \mDot
    \end{align}
    Wir lösen nach $T_{\text{Innen,min}}$ auf:
    \begin{align}
        T_{\text{Innen,min}} &= \frac{p_0 M_{\text{mol}} V_B T_\text{Außen}}{p_0 M_{\text{mol}} V_B - m_{\text{Leer}} R T_\text{Außen}} = \\
        &= \frac{(101325\cdot \num{28.96e-3} \cdot 4500 \cdot \num{293.15} }{101325 \cdot \num{28.96e-3} \cdot 4500 - 850 \cdot \num{8.314} \cdot \num{293.15})} \\
        &\approx \SI{347.78}{\kelvin} \approx \SI{74.6}{\celsius}
    \end{align}
    
    \item \textbf{Bonus: Maximale Steighöhe:}
    Auf der Höhe $h$ muss gelten: $F_{\text{Auftrieb}}(h) = F_{G, \text{Ballon}}(h)$. Der Druck $p(h)$ und die Dichten ändern sich mit der Höhe.
    \begin{equation}
        \rho(p(h), T_{\text{Außen}}) V_B = m_{\text{Leer}} + \rho(p(h), T_{\text{Innen,max}}) V_B
    \end{equation}
    Einsetzen der Dichteformel und $p(h)$ führt nach Umformung zu: 
    \begin{equation}
        e^x = \frac{M_{\text{mol}} p_0 V_B}{m_{\text{Leer}} R} \left(\frac{1}{T_{\text{Außen}}} - \frac{1}{T_{\text{Innen,max}}}\right) \quad \text{mit } x = \frac{h}{H}
    \end{equation}
    Mit $T_{\text{Innen,max}} = 125 + \num{273.15} = \SI{398.15}{\kelvin}$:
    \begin{equation}
        e^x \approx 1.68 \implies x = \ln(\num{1.68}) \approx \num{0.519}
    \end{equation}
    Die maximale Höhe ist $h_{\text{max}} = x \cdot H = \num{0.519} \cdot \SI{8.3}{\kilo\meter} \approx \SI{4.3}{\kilo\meter}$. 
\end{enumerate}
\end{loesungbox}



\begin{loesungbox}{Lösung zu \Cref{aufg:auftrieb_holzklotz}}
\begin{enumerate}
    \item \textbf{Schwimmbedingung:} Ein Körper schwimmt, wenn seine abwärts gerichtete Gewichtskraft $\ivecS{F}{G}$ und die aufwärts gerichtete Auftriebskraft $\ivecS{F}{A}$ im Gleichgewicht sind. Es gilt also betragsmäßig
    \begin{equation}\label{eq:holz_gleichgewicht}
        F_G = F_A \mDot
    \end{equation}
    \item \textbf{Gewichtskraft:} Das Gesamtvolumen des Würfels ist $V_{\text{ges}} = L^3 = (\SI{0.2}{\metre})^3 = \SI{0.008}{\metre\cubed}$.
    \begin{equation}\label{eq:holz_fg}
        F_G = \rho_{\text{Holz}} \cdot V_{\text{ges}} \cdot g = \SI{750}{\kg\per\m\cubed} \cdot \SI{0.008}{\m\cubed} \cdot \SI{9.81}{\m\per\s\squared} = \SI{58.86}{\newton} \mDot
    \end{equation}
    \item \textbf{Auftriebskraft (allgemein):} Nach dem Archimedischen Prinzip entspricht die Auftriebskraft dem Gewicht der verdrängten Flüssigkeit.
    \begin{equation}\label{eq:holz_fa_allg}
        F_A = \rho_{\text{Wasser}} \cdot V_{\text{ein}} \cdot g
    \end{equation}
    Die Auftriebskraft hängt vom eingetauchten Volumen $V_\text{ein}$ ab.
    \item \textbf{Auftriebskraft (spezifisch):} Das eingetauchte Volumen ist das Produkt aus der Grundfläche $A = L^2$ und der Eintauchtiefe $h$, also $V_{\text{ein}} = L^2 \cdot h$. Eingesetzt in die Formel für die Auftriebskraft:
    \begin{equation}\label{eq:holz_fa_spez}
        F_A = \rho_{\text{Wasser}} \cdot (L^2 \cdot h) \cdot g
    \end{equation}
    \item \textbf{Herleitung der Eintauchtiefe:} Wir setzen die Terme in die Gleichgewichtsbedingung ein:
    \begin{align}
        F_G &= F_A \\
        \rho_{\text{Holz}} \cdot L^3 \cdot g &= \rho_{\text{Wasser}} \cdot L^2 \cdot h \cdot g
    \end{align}
    Wir kürzen auf beiden Seiten durch $g$ und $L^2$:
    \begin{equation}\label{eq:holz_h_formel}
        \rho_{\text{Holz}} \cdot L = \rho_{\text{Wasser}} \cdot h \implies h = L \cdot \frac{\rho_{\text{Holz}}}{\rho_{\text{Wasser}}}
    \end{equation}
    \item \textbf{Berechnung der Eintauchtiefe:}
    \begin{equation}\label{eq:holz_h_wert}
        h = \SI{0.2}{\metre} \cdot \frac{\SI{750}{\kg\per\m\cubed}}{\SI{1000}{\kg\per\m\cubed}} = \SI{0.2}{\metre} \cdot 0.75 = \SI{0.15}{\metre} = \SI{15}{\centi\metre} \mDot
    \end{equation}
    Die Eintauchtiefe des Holzklotzes beträgt $\SI{15}{\centi\metre}$.
\end{enumerate}
\end{loesungbox}





\begin{loesungbox}{Lösung zu \Cref{aufg:heliumballons}}
\begin{enumerate}
    \item \textbf{Nettoauftriebskraft eines Ballons:}
    \begin{itemize}
        \item Auftriebskraft nach Archimedes: $V_{\text{Ballon}} = \SI{4.5}{\litre} = \SI{0.0045}{\metre\cubed}$.
        \begin{equation}\label{eq:ballon_auftrieb}
            F_A = \rho_{\text{Luft}} \cdot V_{\text{Ballon}} \cdot g = \SI{1.23}{\kg\per\m\cubed} \cdot \SI{0.0045}{\m\cubed} \cdot \SI{9.81}{\m\per\s\squared} \approx \SI{0.0543}{\newton}
        \end{equation}
        \item Gewichtskraft des gefüllten Ballons (Hülle + Helium):
        \begin{multline}\label{eq:ballon_gewicht}
            F_{G,\text{ges}} = (m_{\text{Hülle}} + \rho_{\text{He}} \cdot V_{\text{Ballon}}) \cdot g = \\
            =(\SI{0.003}{\kg} + \SI{0.18}{\kg\per\m\cubed} \cdot \SI{0.0045}{\m\cubed}) \cdot g \approx \SI{0.0374}{\newton}
        \end{multline}
        \item Nettoauftriebskraft:
        \begin{equation}\label{eq:ballon_netto}
            F_{A,\text{netto}} = F_A - F_{G,\text{ges}} = \SI{0.0543}{\newton} - \SI{0.0374}{\newton} = \SI{0.0169}{\newton}
        \end{equation}
    \end{itemize}
    \item \textbf{Anzahl der Ballons:} Die gesamte Nettoauftriebskraft muss der Gewichtskraft der Person entsprechen.
    \begin{equation}\label{eq:ballon_person_gewicht}
        F_{G,\text{Person}} = m_{\text{Person}} \cdot g = \SI{80}{\kg} \cdot \SI{9.81}{\m\per\s\squared} = \SI{784.8}{\newton}
    \end{equation}
    \begin{equation}\label{eq:ballon_anzahl}
        N = \frac{F_{G,\text{Person}}}{F_{A,\text{netto}}} = \frac{\SI{784.8}{\newton}}{\SI{0.0169}{\newton}} \approx 46438
    \end{equation}
    Man benötigt also rund 46438 Ballons.
    \item \textbf{Maximale Steighöhe:} Die Dichte der Atmosphäre nimmt mit zunehmender Höhe ab. Dadurch sinkt die Auftriebskraft ($F_A = \rho_{\text{Luft}}(h) \cdot V \cdot g$). Die maximale Höhe ist erreicht, wenn die Auftriebskraft genau der konstanten Gesamtgewichtskraft (Person + Ballons) entspricht.
    \item \textbf{Vergleich Auftrieb Luft vs. Wasser:} Die Dichte von Luft nimmt mit zunehmender Höhe signifikant ab (Luft ist kompressibel). Die Dichte von Wasser hingegen bleibt mit zunehmender Tiefe nahezu konstant (Wasser ist annähernd inkompressibel). Dies führt zu fundamental unterschiedlichem Verhalten: Ein aufsteigender Ballon erfährt eine immer geringere Auftriebskraft, während ein aufsteigendes Objekt im Wasser (z.B. ein U-Boot) eine fast konstante Auftriebskraft erfährt.
    \item \textbf{Ballon unter Wasser:} Beim Aufsteigen des Ballons nimmt der hydrostatische Umgebungsdruck ab. Nach dem Gesetz von Boyle-Mariotte ($p \cdot V = \const$) dehnt sich die kompressible Luft im Ballon bei sinkendem Außendruck stark aus, sein Volumen vergrößert sich. Da die Auftriebskraft direkt vom verdrängten Volumen abhängt ($F_A = \rho_{\text{Wasser}} \cdot V_{\text{Ballon}}(h) \cdot g$), nimmt sie während des Aufstiegs kontinuierlich zu. Dies führt zu einer immer stärker werdenden Beschleunigung nach oben. Hält das Material der Ausdehnung nicht stand, platzt der Ballon.
\end{enumerate}
\end{loesungbox}




\begin{loesungbox}{Lösung zu \Cref{aufg:ideale_gase}}
\begin{enumerate}
    \item \textbf{Molares Volumen bei Normalbedingungen:} \\
    Normalbedingungen: $p=\SI{101325}{\pascal}$ und $T = \SI{0}{\celsius} = \SI{273.15}{\kelvin}$. 
    \begin{equation}
        V = \frac{n R T}{p} = \frac{(\SI{1}{\mol}) \cdot (\SI{8.314}{\joule\per(\mol\per\kelvin)}) \cdot (\SI{273.15}{\kelvin})}{\SI{101325}{\pascal}} \approx \SI{0.0224}{\meter\cubed} = \SI{22.4}{\litre}
    \end{equation}
    
    \item \textbf{Anzahl der Teilchen und Avogadro-Konstante:}
    Wir formen $p V = N k_B T$ nach $N$ um: 
    \begin{equation}
        N = \frac{p V}{k_B T} = \frac{(\SI{101325}{\pascal}) \cdot (\SI{0.0224}{\meter\cubed})}{(\SI{1.38e-23}{\joule\per\kelvin}) \cdot (\SI{273.15}{\kelvin})} \approx \SI{6.022e23}{}
    \end{equation}
    Diese Zahl ist die \textbf{Avogadro-Konstante} $N_A$. Sie gibt die Anzahl der Teilchen pro Mol an. 
    
    \item \textbf{Wurzel des mittleren Geschwindigkeitsquadrats (RMS-Geschwindigkeit):}
    Aus $\frac{1}{2}m\overline{v^2} = \frac{3}{2}k_B T$ folgt 
    $$\sqrt{\overline{v^2}} = \sqrt{\frac{3 k_B T}{m}} \mDot$$
    Wir verwenden $T = \SI{20}{\celsius} = \SI{293.15}{\kelvin}$.
    \begin{itemize}
        \item \textbf{Helium:} $\sqrt{\overline{v^2}_{\text{He}}} = \sqrt{\frac{3 \cdot (\SI{1.38e-23}{J/K}) \cdot (\SI{293.15}{K})}{\SI{6.65e-27}{kg}}} \approx \SI{1352}{\meter\per\second}$ 
        \item \textbf{Stickstoff:} $\sqrt{\overline{v^2}_{\text{N}_2}} = \sqrt{\frac{3 \cdot (\SI{1.38e-23}{J/K}) \cdot (\SI{293.15}{K})}{\SI{4.65e-26}{kg}}} \approx \SI{511}{\meter\per\second}$ 
        \item \textbf{Sauerstoff:} $\sqrt{\overline{v^2}_{\text{O}_2}} = \sqrt{\frac{3 \cdot (\SI{1.38e-23}{J/K}) \cdot (\SI{293.15}{K})}{\SI{5.31e-26}{kg}}} \approx \SI{478}{\meter\per\second}$ 
    \end{itemize}
\end{enumerate}
\end{loesungbox}


%\end{comment}


\chapter{Thermodynamik}\label{chap:thermodynamik}

\section{Aufgaben}\label{sec:thermodynamik_aufgaben}

\begin{aufgabebox}{Druck durch Kraftübertragung}{druck_kraftuebertragung}
In einem kugelförmigen Volumen mit Radius $R$ sei ein ideales Gas enthalten. Die Geschwindigkeitsverteilung der Teilchen sei isotrop, weshalb $\overline{v_{x}^{2}}=\overline{v_{y}^{2}}=\overline{v_{z}^{2}}=\frac{1}{3}\overline{v^{2}}$ gilt. Berechnen Sie den Druck $p_{(N)}$ auf die Wand, der durch die Kraftübertragung von $N$ Teilchen mit Masse $m$ bei den elastischen Stößen mit der Wand entsteht.

Die Herleitung wird sehr einfach, wenn man sich ein Teilchen herausnimmt, das die mittlere Position und die mittlere Geschwindigkeit aller $N$ Teilchen hat.

\begin{center}
    \includegraphics[width=0.35\linewidth]{Bilder/Uebungsaufgaben/druck_kraftuebertrag_kugel.png}
    \label{fig:druck_kraftuebertr_kugel}
\end{center}

\textbf{Vorgehen:} Betrachten Sie das Teilchen mit Geschwindigkeit $\ivec{v}$ im Mittelpunkt der Kugel. Die Geschwindigkeit soll genau die mittlere Geschwindigkeit aller Teilchen sein.

\begin{enumerate}
    \item Wie groß ist der Impulsübertrag $\Delta \ivec{P}_{(1)}$ des Teilchens beim Stoß mit der Wand bei der Geschwindigkeit $\ivec{v}$?
    
    \item In welchem Zeitintervall $\Delta t$ findet im Schnitt ein Stoßprozess statt? Überlegen Sie, welche Strecke das Teilchen bei der Geschwindigkeit $v$ zurücklegen muss, damit es im selben Zeitintervall $\Delta t$ den nächsten Stoß absolvieren kann?
    
    \item Die Kraft für 1 Teilchen ist dann $\ivec{F}_{(1)}=\Delta \ivec{P}_{(1)}/\Delta t$ und der Druck ist $p_{(1)}=|\ivec{F}_{(1)}|/A$. Welchen Druck übt also 1 Teilchen auf die Innenfläche der Kugel aus?
    
    \item Das Volumen der Kugel ist $V=\frac{4}{3}\pi R^{3}$. Setzen Sie das Volumen in den Druck passend ein und ersetzen Sie anschließend $mv^{2}$ durch $2 E_{\kin,(1)}$. Der Druck durch $N$ Teilchen ist dann $p_{(N)}=N\cdot p_{(1)}$. Wie lautet dieser Druck und was ergibt das Produkt $p_{(N)}\cdot V$?
    
    \item Was für ein Zusammenhang ergibt sich also zwischen der kinetischen Energie des Durchschnittteilchens $E_{\kin,(1)}$ und der Temperatur eines idealen Gases? Tipp: $pV=NkT$.
    
    \item Welche Temperatur haben demnach Sauerstoffmoleküle $O_2$ mit Masse $m_{O_2}=\SI{5.31e-26}{\kilo\gram}$ und einer mittleren Geschwindigkeit von $v=\SI{478}{\meter\per\second}$? ($k=\SI{1.38054e-23}{\joule\per\kelvin}$)
\end{enumerate}
\end{aufgabebox}



\begin{aufgabebox}{Wärmekapazitäten}{waermekapazitaet}
\begin{enumerate}
    \item Ein Juwelier stellt Amulette aus Gold her. Um das Gold in die Gussform gießen zu können, muss er es schmelzen. Welche Wärmemenge ist nötig, um \SI{3}{\kilo\gram} Gold von Raumtemperatur $T_0 = \SI{22}{\celsius}$ auf den Schmelzpunkt von Gold $T_{\text{Schm}} = \SI{1063}{\celsius}$ zu erwärmen? Die spezifische Wärmekapazität von Gold ist $c_{\text{Au}} = \SI{0.126}{\kilo\joule\per(\kilo\gram\cdot\kelvin)}$.
    
    \item Ein \SI{2}{\kilo\gram} Block Aluminium hat anfangs eine Temperatur von $T=\SI{10}{\celsius}$. Dann werden ihm \SI{36}{\kilo\joule} zugeführt. Welche Temperatur hat er danach? Die spezifische Wärmekapazität von Aluminium ist $c_{\text{Al}} = \SI{0.888}{\kilo\joule\per(\kilo\gram\cdot\kelvin)}$.
    
    \item Ende des Sommers hat die Betonmasse eines Hauses mit $M_{\text{Beton}}=\SI{10000}{\kilo\gram}$ eine Temperatur von $\SI{25}{\celsius}$. Im Herbst kühlt das Haus langsam auf $\SI{20}{\celsius}$ ab. Wie viel Wärmemenge hat das Haus abgegeben, wenn $c_{\text{Beton}}=\SI{0.920}{\kilo\joule\per(\kilo\gram\cdot\kelvin)}$?
\end{enumerate}
\end{aufgabebox}

\begin{aufgabebox}{Eis zu Wasserdampf}{eis_zu_dampf}
Sie haben einen Block aus Eis mit $m_{\text{Eis}}=\SI{1.5}{\kilo\gram}$ bei einer Temperatur von $T=\SI{-20}{\celsius}$. Sie möchten den Block aus Eis komplett verdampfen. Wie viel Wärmemenge ist dazu nötig? \\
Gegeben: $c_{\text{Eis}} = \SI{2.1}{\kilo\joule\per(\kilo\gram\cdot\kelvin)}$, $c_{\text{Wasser}} = \SI{4.18}{\kilo\joule\per(\kilo\gram\cdot\kelvin)}$, $\lambda_{\text{Schm}} = \SI{332.8}{\kilo\joule\per\kilo\gram}$, $\lambda_{\text{Verd}} = \SI{2256.0}{\kilo\joule\per\kilo\gram}$.
\begin{center}
    \includegraphics[width=0.6\textwidth]{Bilder/Uebungsaufgaben/eis_zu_wasserdampf.png}
\end{center}
\textbf{Vorgehen:} Teilen Sie den Vorgang in 4 Teilschritte auf und fassen Sie anschließend die gesamte Wärmemenge zusammen.
\begin{enumerate}
    \item Ermitteln Sie die Wärmemenge $Q_1$ zum Erwärmen des Eisblocks auf $\SI{0}{\celsius}$.
    \item Verwenden Sie die spezifische Schmelzwärme $\lambda_{\text{Schm}}$, um die Wärmemenge $Q_2$ zu berechnen, die zum Schmelzen des Eises nötig ist!
    \item Berechnen Sie nun die Wärmemenge $Q_3$ zum Erhöhen der Temperatur des Wassers von $\SI{0}{\celsius}$ auf $\SI{100}{\celsius}$!
    \item Wie viel Wärmemenge $Q_4$ ist nun erforderlich, um das Wasser mit der spezifischen Verdampfungswärme $\lambda_{\text{Verd}}$ zu verdampfen?
    \item Wie viel Wärmemenge ist demnach gesamt zum Verdampfen von \SI{1.5}{\kilo\gram} Eis notwendig?
\end{enumerate}
\end{aufgabebox}




\begin{aufgabebox}{Saft im Tiefkühler}{saft_tk}
Sie haben um 22:00 Uhr \SI{3.5}{\litre} Saft ($\rho_{\text{Saft}}=\SI{1.1}{\kilogram\per\deci\metre\cubed}$) eingekocht und stellen ihn mit einer Temperatur von $T_{\text{Saft}}=\SI{85}{\degreeCelsius}$ in den Tiefkühlschrank, damit er schneller abkühlt. Der Tiefkühlschrank hat eine konstante (temperaturunabhängige) Kühlleistung von \SI{120}{\watt}. Sie vergessen die Flasche über Nacht und schauen um 07:00 Uhr morgens wieder danach.
Der Saft hat eine spezifische Wärmekapazität von $c_{\text{Saft}}=\SI{4.2}{\kilo\joule\per\kilogram\per\kelvin}$ und eine spezifische Schmelzwärme von $\lambda_{\text{Schmelz}}=\SI{333.5}{\kilo\joule\per\kilogram}$.
\begin{enumerate}
    \item Skizzieren Sie den Abkühlprozess in einem Temperatur-Zeit-Diagramm.
    \item Berechnen Sie die Energie, die abgeführt werden muss, um den Saft auf \SI{0}{\degreeCelsius} abzukühlen.
    \item Berechnen Sie die Energie, die abgeführt werden muss, damit der gesamte Saft bei \SI{0}{\degreeCelsius} gefriert.
    \item Berechnen Sie die Gesamtzeit (in \si{\hour}), die das Gefrierfach für den gesamten Prozess (Abkühlen und Frieren) benötigt.
    \item Ziehen Sie ein Fazit: Ist der Saft um 07:00 Uhr morgens bereits vollständig gefroren?
\end{enumerate}
\end{aufgabebox}




\begin{aufgabebox}{Wasserkocher vergessen}{wasserkocher}
Ein Wasserkocher habe eine Heizleistung von \SI{2000}{\watt}. Sie haben \SI{1}{\litre} Wasser ($\rho_{W}=\SI{1}{\kilogram\per\deci\metre\cubed}$) mit einer Anfangstemperatur von $T=\SI{20}{\degreeCelsius}$ eingefüllt und den Wasserkocher eingeschaltet. Ihr Wasserkocher hat \textbf{keine Selbstabschaltung} und Sie gehen für \SI{25}{\minute} einkaufen.

Gegeben sind die spezifische Wärmekapazität $c_{W}=\SI{4.18}{\kilo\joule\per(\kilogram\cdot\kelvin)}$ und die spezifische Verdampfungswärme $\lambda_{\text{verd}}=\SI{2256}{\kilo\joule\per\kilogram}$. Kommen Sie rechtzeitig vom Einkaufen zurück, bevor das gesamte Wasser verdampft ist und der Kocher zu verschmoren droht?

\begin{center}
    \includegraphics[width=0.3\textwidth]{Bilder/Uebungsaufgaben/wasserkocher.png}
\end{center}

\begin{enumerate}
    \item Skizzieren Sie den zeitlichen Ablauf in einem Temperatur-Zeit-Diagramm! Beschriften Sie die einzelnen Abschnitte des Prozesses!
    \item Berechnen Sie die Energie, die benötigt wird, um das Wasser auf \SI{100}{\degreeCelsius} zu erhitzen.
    \item Berechnen Sie die Energie, die benötigt wird, um das gesamte Wasser bei \SI{100}{\degreeCelsius} zu verdampfen.
    \item Berechnen Sie, wie viel Energie der Wasserkocher in \SI{25}{\minute} liefert.
    \item Ziehen Sie ein Fazit: Kommen Sie rechtzeitig zurück?
\end{enumerate}
\end{aufgabebox}




\begin{aufgabebox}{Heißes Metall in Wasser}{mischtemperatur}
Eine massive Eisenkugel mit einer Masse von $m_{\text{Kugel}} = \SI{500}{\gram}$ und einer Temperatur von $T_{\text{Kugel}} = \SI{250}{\celsius}$ wird in ein wärmeisoliertes Gefäß mit \SI{5}{\litre} Wasser gegeben, das eine Anfangstemperatur von $T_{\text{Wasser}} = \SI{20}{\degreeCelsius}$ hat. \\
Gegeben sind die spezifische Wärmekapazität von Wasser $c_{\text{Wasser}} = \SI{4.18}{\kilo\joule\per(\kilogram\cdot\kelvin)}$, die spezifische Wärmekapazität von Eisen $c_{\text{Eisen}} = \SI{0.45}{\kilo\joule\per(\kilogram\cdot\kelvin)}$ und die Dichte von Wasser $\rho_{\text{Wasser}} = \SI{1}{\kilogram\per\deci\meter\cubed}$.\\

Berechnen Sie die Endtemperatur (Mischtemperatur) $T_{\text{Misch}}$, die sich im thermischen Gleichgewicht einstellt.
\begin{center}
    \includegraphics[width=0.25\textwidth]{Bilder/Uebungsaufgaben/kugel_in_wasser_3.jpeg}
\end{center}
\begin{enumerate}
    \item Stellen Sie die Formel für die vom Wasser aufgenommene Wärme $Q_{\text{Wasser,auf}}$ als Funktion der gesuchten Mischtemperatur $T_{\text{Misch}}$ auf.
    \item Stellen Sie die Formel für die von der Eisenkugel abgegebene Wärme $Q_{\text{Eisen,ab}}$ als Funktion der gesuchten Mischtemperatur $T_{\text{Misch}}$ auf.
    \item Nutzen Sie den Energieerhaltungssatz für ein abgeschlossenes System ($Q_{\text{Wasser,auf}} = Q_{\text{Eisen,ab}}$). Lösen Sie die resultierende Gleichung nach der Mischtemperatur $T_{\text{Misch}}$ auf und berechnen Sie ihren Wert.
\end{enumerate}
\end{aufgabebox}






\begin{aufgabebox}{Eiswürfel in Limonade}{eiswuerfel_limo}
Ihr Krug mit Limonade stand bei einer Außentemperatur von \SI{33}{\degreeCelsius} den ganzen Tag auf einem Gartentisch im Freien. Sie gießen nun \SI{0.24}{\kilogram} der Limonade in einen Styroporbecher und geben zwei Eiswürfel hinein, die jeweils \SI{25}{\gram} schwer sind und eine Temperatur von \SI{0}{\degreeCelsius} haben.

\begin{center}
    \includegraphics[width=0.27\linewidth]{Bilder/Uebungsaufgaben/eiswuerfel_limo.png}
\end{center}

\noindent
Gegeben sind die spezifische Wärmekapazität $c_{\text{Wasser}} = c_{\text{Limonade}} = \SI{4.18}{\kilo\joule\per(\kilogram\cdot\kelvin)}$ und die spezifische Schmelzwärme von Eis $\lambda_{\text{Schmelz}} = \SI{332.8}{\kilo\joule\per\kilogram}$.

\bigskip
\noindent
Welche Temperatur hat die Limonade im Becher, nachdem sich ein thermisches Gleichgewicht eingestellt hat? Nehmen Sie dabei an, dass keine Wärme an die Umgebung abgegeben wird.

\begin{enumerate}
    \item Bestimmen Sie zunächst, ob die in der Limonade gespeicherte Wärme ausreicht, um die beiden Eiswürfel vollständig zu schmelzen.
    \item Berechnen Sie die gesuchte Endtemperatur $T_{\text{Ende}}$, die sich einstellt, indem Sie die von der Limonade abgegebene Wärme $Q_\text{Limo,ab}$ mit der vom Eis (bzw. dem geschmolzenen Wasser) aufgenommenen Wärme $Q_\text{Eis,auf}$ gleichsetzen.
\end{enumerate}
\end{aufgabebox}




\begin{aufgabebox}{Ideales Gas}{ideales_gas}
Die Zustandsgleichung eines idealen Gases mit Volumen $V$ [\si{\metre\cubed}], Druck p [\si{\pascal}] und der Temperatur T [\si{\kelvin}] lautet
\begin{equation}\label{eq:ideale_gasgleichung}
    p \cdot V = n \cdot R \cdot T \mComma
\end{equation}
wobei $n$ [\si{\mole}] die Stoffmenge des Gases ist und $R = \SI{8.314}{\joule\per(\mole\cdot\kelvin)}$ die universelle Gaskonstante. Betrachten Sie ein ideales Gas mit $n=1$ mol.
\begin{center}
    \includegraphics[width=0.8\textwidth]{Bilder/Uebungsaufgaben/ideales_gas.png}
\end{center}
\begin{enumerate}
    \item In der Grafik ist eine Isotherme eingezeichnet und mit (I) markiert. Welche Temperatur hat diese Isotherme?
    \item Zeichnen Sie eine weitere Isotherme für eine Temperatur von $T=\SI{250}{\kelvin}$ in das Diagramm ein und markieren Sie sie mit (II).
    \item Zeichnen Sie eine beliebige Isobare ein und beschriften Sie sie mit (III).
    \item Zeichnen Sie eine beliebige Isochore ein und beschriften Sie sie mit (IV).
    \item Berechnen Sie das Volumen von einem Mol eines idealen Gases unter Normalbedingungen ($p=\SI{1.013}{\bar}$, $T=\SI{0}{\degreeCelsius}$)!
\end{enumerate}
\end{aufgabebox}




\begin{aufgabebox}{Volumenarbeit am idealen Gas}{volumenarbeit}
Ein ideales Gas wird dem zyklischen Prozess unterzogen, der in der Abbildung dargestellt ist.
\begin{center}
    \includegraphics[width=0.6\textwidth]{Bilder/Uebungsaufgaben/volumenarbeit_ideales_gas.png}
\end{center}
Der Prozess durchläuft die folgenden Zustandsänderungen:
\begin{itemize}
    \item \textbf{A $\rightarrow$ B:} Der Zyklus beginnt bei Zustand A ($V_A = \SI{1}{\litre}$, $p_A = \SI{2}{\bar}$) mit einer isobaren Expansion (konstanter Druck) bis $V_B = \SI{2.5}{\litre}$.
    \item \textbf{B $\rightarrow$ C:} Anschließend wird das Gas bei konstantem Volumen (isochor) abgekühlt, bis ein Druck von $p_C = \SI{1}{\bar}$ erreicht ist.
    \item \textbf{C $\rightarrow$ D:} Nun wird das Gas bei konstantem Druck (isobar) auf ein Volumen von $V_D = \SI{1}{\litre}$ komprimiert.
    \item \textbf{D $\rightarrow$ A:} Schließlich wird das Gas bei konstantem Volumen (isochor) erwärmt, bis der Anfangszustand mit $p_A = \SI{2}{\bar}$ wieder erreicht ist.
\end{itemize}
Wie groß ist die dem Gas während des gesamten Zyklus zugeführte Arbeit und die ihm netto zugeführte Wärmemenge?

\textbf{Hinweise:} Die differentielle Volumenarbeit ist $\dd W = -p \cdot \dd V$. \\
Der erste Hauptsatz der Thermodynamik lautet $\Delta U = \Delta Q + \Delta W$. \\

\textbf{Vorgehen:}
\begin{enumerate}
    \item Berechnen Sie zunächst für die vier Teilschritte die verrichtete Arbeit und daraus die Gesamtarbeit des Zyklus $\Delta W_{\text{ges}}$.
    \item Für einen zyklischen Prozess muss die Änderung der inneren Energie $\Delta U_{\text{ges}}=0$ sein. Die netto zugeführte Wärmemenge kann daher über $\Delta Q_{\text{ges}} = -\Delta W_{\text{ges}}$ berechnet werden.
\end{enumerate}
\end{aufgabebox}





\begin{aufgabebox}{Idealer Kreisprozess}{kreisprozess}
Eine Wärmekraftmaschine arbeitet mit $n=\SI{1}{\mole}$ eines einatomigen idealen Gases, das den skizzierten Kreisprozess aus zwei isobaren und zwei isochoren Zustandsänderungen durchläuft.
Gegeben sind die universelle Gaskonstante $R = \SI{8.314}{\joule\per(\mole\cdot\kelvin)}$, die molare Wärmekapazität bei konstantem Volumen $C_{V}=3R/2$ und die molare Wärmekapazität bei konstantem Druck $C_{p}=5R/2$.\\
Die Zustandsgleichung eines idealen Gases lautet 
$$ p\cdot V = n\cdot R \cdot T \mDot$$
\begin{center}
    \begin{tikzpicture}
        % Place the image in a node named "image"
        % The anchor is set to the bottom-left corner and inner sep is removed
        \node[anchor=south west, inner sep=0] (image) {\includegraphics[width=0.8\textwidth]{Bilder/Uebungsaufgaben/idealerKreisprozess.png}};

        % Start a new scope, defining a relative coordinate system on the image
        % x-vector goes from origin to bottom-right, y-vector goes from origin to top-left
        \begin{scope}[x={(image.south east)},y={(image.north west)}]
            \node[black] at (0.28, 0.83) {\textbf{(1)}};
            \node[black] at (0.78, 0.83) {\textbf{(2)}};
            \node[black] at (0.78, 0.25) {\textbf{(3)}};
            \node[black] at (0.28, 0.25) {\textbf{(4)}};
        \end{scope}
    \end{tikzpicture}
\end{center}
\begin{enumerate}
    \item Bestimmen Sie die Zustandsvariablen (Druck $p$, Volumen $V$, Temperatur $T$) an den vier Eckpunkten (1), (2), (3) und (4) des Kreisprozesses.
    \item Beschreiben Sie die vier Teilprozesse ($1\to2$, $2\to3$, $3\to4$, $4\to1$) jeweils hinsichtlich der Prozessart (isobar/isochor), des volumetrischen Vorgangs (Expansion/Kompression) und des thermischen Vorgangs (Erhitzung/Abkühlung).
    \item Berechnen Sie die in den einzelnen Teilschritten aufgenommene bzw. abgegebene Arbeit.
    \item Formulieren Sie den 1. Hauptsatz der Thermodynamik und benennen Sie die vorkommenden Größen.
    \item Berechnen Sie die in den einzelnen Teilschritten zugeführte bzw. abgegebene Wärme.
    \item Überprüfen Sie anhand Ihrer Ergebnisse aus (3) und (5), dass die Änderung der inneren Energie $\Delta U$ über einen gesamten Zyklus null ist.
    \item Die gesamte zugeführte Wärme beträgt $\Delta Q_{\text{zu}}=\SI{24.5}{\kilo\joule}$, die gesamte abgeführte Wärme $\Delta Q_{\text{ab}}=\SI{-18.5}{\kilo\joule}$, und die verrichtete Nettoarbeit der Maschine beträgt $\Delta W=\SI{-6}{\kilo\joule}$. Berechnen Sie den Wirkungsgrad der Maschine.
\end{enumerate}
\end{aufgabebox}




\begin{aufgabebox}{Carnot'scher Kreisprozess}{carnot}
Der Carnot'sche Kreisprozess beschreibt eine ideale Wärmekraftmaschine, die einen Zyklus aus vier reversiblen Zustandsänderungen durchläuft: isotherme Expansion, adiabatische Expansion, isotherme Kompression und adiabatische Kompression. Der Wirkungsgrad $\eta$ der Maschine ist definiert durch
\begin{equation}\label{eq:carnot_wirkungsgrad}
    \eta = \left|\frac{\Delta W}{\Delta Q_{\text{zu}}}\right| = \frac{T_{1}-T_{2}}{T_{1}} = 1 - \frac{T_2}{T_1}\mComma
\end{equation}
wobei $T_1$ die Temperatur des wärmeren und $T_2$ die des kälteren Reservoirs ist ($T_1 > T_2$).

\begin{center}
    \includegraphics[width=0.4\textwidth]{Bilder/Uebungsaufgaben/kreisprozess_waermekraftmaschine.png}
\end{center}

\begin{enumerate}
    \item Ordnen Sie die vier Prozessschritte den entsprechenden Kurvenabschnitten im $(p,V)$-Diagramm zu.
    \item In welchen Schritten verrichtet die Maschine Arbeit ($\Delta W_{ij}<0$) und in welchen wird Arbeit am System verrichtet ($\Delta W_{ij}>0$)? Zeichnen Sie die Arbeitsflüsse als Pfeile an den jeweiligen Kurvenabschnitten ein.
    \item Woran erkennt man im Diagramm, dass eine Nettoarbeit $\Delta W < 0$ von der Maschine verrichtet wird? Kennzeichnen Sie diese Nettoarbeit in der Grafik.
    \item Wie kann der Wirkungsgrad dieser Wärmekraftmaschine maximiert werden? Welche technischen Schwierigkeiten ergeben sich bei der praktischen Umsetzung?
\end{enumerate}
\end{aufgabebox}







\newpage
\section{Lösungen}\label{sec:thermodynamik_loesungen}

\begin{loesungbox}{Lösung zu \Cref{aufg:druck_kraftuebertragung}}
\begin{enumerate}
    \item Bei einem elastischen Stoß kehrt sich die zur Wand normale Impulskomponente $P_\perp$ um, \gDh $P_\perp \rightarrow -P_\perp$. Für das Teilchen im Zentrum, das sich senkrecht auf die Wand zubewegt, ist der Gesamtimpuls gleich der Normalkomponente ($|\ivec{P}| = P_\perp$ und daher auch $v = v_\perp$). Die Impulsänderung des Teilchens beim Aufprall auf die Wand ist daher     \begin{equation}
        \Delta P_{\text{Teilchen}} = P_{\text{nachher}} - P_{\text{vorher}} = (m(-v)) - (mv) = -2mv \mDot
    \end{equation}
    Der Impulsübertrag auf die Wand ist $\Delta P_{(1)} = - \Delta P_\text{Teilchen} = 2mv$.
    \item Das Teilchen startet im Mittelpunkt, legt die Strecke $R$ zur Wand zurück, wird reflektiert und legt wieder die Strecke $R$ zurück, um den Mittelpunkt zu durchqueren. Die Gesamstrecke für einen Zyklus, nach dem sich der Prozess wiederholt, ist demnach $2R$. Die Zeit für einen Stoßprozess ist somit
    \begin{equation}
        \Delta t = \frac{\Delta s}{v} = \frac{2R}{v} \mDot
    \end{equation}
    \item Die durchschnittliche Kraft, die ein Teilchen auf die Wand ausübt, ist der Impulsübertrag pro Zeitintervall:
    \begin{equation}
        F_{(1)} = \frac{\Delta P_{(1)}}{\Delta t} = \frac{2mv}{2R/v} = \frac{mv^2}{R} \mDot
    \end{equation}
    Der Druck ist die Kraft pro Fläche. Die Oberfläche der Kugel ist $A=4\pi R^2$.
    \begin{equation}
        p_{(1)} = \frac{F_{(1)}}{A} = \frac{mv^2/R}{4\pi R^2} = \frac{mv^2}{4\pi R^3} \mDot
    \end{equation}
    
    \item Das Volumen der Kugel ist $V = \frac{4}{3}\pi R^3$, also ist $4\pi R^3 = 3V$. Wir setzen dies in die Druckformel ein. Mit der kinetischen Energie $E_{\kin,(1)} = \frac{1}{2}mv^2 \implies mv^2 = 2E_{\kin,(1)}$ erhalten wir:
    \begin{equation}
        p_{(1)} = \frac{mv^2}{3V} = \frac{2E_{\kin,(1)}}{3V} \mDot
    \end{equation}
    Für $N$ Teilchen ist der Gesamtdruck $p_{(N)} = N \cdot p_{(1)}$ und damit ergibt sich für den Gesamtdruck
    \begin{equation}\label{eq:druck_N_teilchen}
        p_{(N)} = \frac{2}{3}\frac{N}{V}E_{\kin,(1)} \mDot
    \end{equation}
    Das Produkt aus Druck und Volumen ist:
    \begin{equation}\label{eq:druck_volumen_produkt}
        p_{(N)} \cdot V = \frac{2}{3} N E_{\kin,(1)} \mDot
    \end{equation}
    
    \item Wir vergleichen das Ergebnis aus \cref{eq:druck_volumen_produkt} mit der idealen Gasgleichung $pV = NkT$ -- $p_{(N)} = p$ in der idealen Gasgleichung --
    \begin{equation}
        \frac{2}{3} N E_{\kin,(1)} = NkT \implies E_{\kin,(1)} = \frac{3}{2}kT \mDot
    \end{equation}
    Die mittlere kinetische Energie eines Teilchens ist direkt proportional zur absoluten Temperatur.
    
    \item Zuerst berechnen wir die kinetische Energie des $O_2$-Moleküls:
    \begin{equation}
        E_{\kin, (1)} = \frac{1}{2}m v^2 = \frac{1}{2}(\SI{5.31e-26}{\kilo\gram})(\SI{478}{\meter\per\second})^2 \approx \SI{6.074e-21}{\joule}
    \end{equation}
    Nun stellen wir die Formel nach $T$ um:
    \begin{equation}
        T = \frac{2}{3}\frac{E_{\kin,(1)}}{k} = \frac{mv^2}{3k} = \frac{(\SI{5.31e-26}{\kilo\gram})(\SI{478}{\meter\per\second})^2}{3\cdot \SI{1.38054e-23}{\joule\per\kelvin}} \approx \SI{292.94}{\kelvin} \mDot
    \end{equation}
    Das entspricht einer Temperatur von ca. $\SI{19.79}{\celsius}$.
\end{enumerate}
\end{loesungbox}




\begin{loesungbox}{Lösung zu \Cref{aufg:waermekapazitaet}}
\begin{enumerate}
    \item Die benötigte Wärmemenge $Q$ wird mit der Formel $Q = c \cdot m \cdot \Delta T$ berechnet. Die Temperaturdifferenz ist $\Delta T = \SI{1063}{\celsius} - \SI{22}{\celsius} = \SI{1041}{\kelvin}$.
    \begin{equation}
        Q = \SI{0.126}{\kilo\joule\per(\kilo\gram\per\kelvin)} \cdot \SI{3}{\kilo\gram} \cdot \SI{1041}{\kelvin} = \SI{393.546}{\kilo\joule} \mDot
    \end{equation}
    
    \item Wir stellen die Formel nach der Temperaturänderung $\Delta T$ um: $\Delta T = \frac{Q}{c \cdot m}$.
    \begin{equation}
        \Delta T = \frac{\SI{36}{\kilo\joule}}{\SI{0.888}{\kilo\joule\per\kilo\gram\per\kelvin} \cdot \SI{2}{\kilo\gram}} \approx \SI{20.27}{\kelvin} \mDot
    \end{equation}
    Die Endtemperatur ist $T_{\text{End}} = T_{\text{Anfang}} + \Delta T = \SI{10}{\celsius} + \SI{20.27}{\celsius} = \SI{30.27}{\celsius}$.
    
    \item Die abgegebene Wärmemenge wird wieder mit $Q = c \cdot m \cdot \Delta T$ berechnet. Die Temperaturdifferenz ist $\Delta T = \SI{25}{\celsius} - \SI{20}{\celsius} = \SI{5}{\kelvin}$.
    \begin{equation}
        Q = \SI{0.920}{\kilo\joule\per\kilo\gram\per\kelvin} \cdot \SI{10000}{\kilo\gram} \cdot \SI{5}{\kelvin} = \SI{46000}{\kilo\joule} \mDot
    \end{equation}
\end{enumerate}
\end{loesungbox}


\begin{loesungbox}{Lösung zu \Cref{aufg:eis_zu_dampf}}
Die Gesamtwärmemenge $Q_{\text{ges}}$ ist die Summe der Wärmemengen der einzelnen Prozessschritte: $Q_{\text{ges}} = Q_1 + Q_2 + Q_3 + Q_4$.
\begin{enumerate}
    \item \textbf{Erwärmen des Eises von \SI{-20}{\celsius} auf \SI{0}{\celsius}:}
    Die Temperaturdifferenz ist $\Delta T_1 = \SI{0}{\celsius} - (\SI{-20}{\celsius}) = \SI{20}{\kelvin}$.
    \begin{equation}
        Q_1 = c_{\text{Eis}} \cdot m \cdot \Delta T_1 = \SI{2.1}{\kilo\joule\per\kilo\gram\per\kelvin} \cdot \SI{1.5}{\kilo\gram} \cdot \SI{20}{\kelvin} = \SI{63.0}{\kilo\joule} \mDot
    \end{equation}
    
    \item \textbf{Schmelzen des Eises bei \SI{0}{\celsius}:}
    Die hierfür benötigte Energie ist die Schmelzwärme $Q_2 = \lambda_{\text{Schm}} \cdot m$.
    \begin{equation}
        Q_2 = \SI{332.8}{\kilo\joule\per\kilo\gram} \cdot \SI{1.5}{\kilo\gram} = \SI{499.2}{\kilo\joule} \mDot
    \end{equation}
    
    \item \textbf{Erwärmen des Wassers von \SI{0}{\celsius} auf \SI{100}{\celsius}:}
    Die Temperaturdifferenz ist $\Delta T_3 = \SI{100}{\celsius} - \SI{0}{\celsius} = \SI{100}{\kelvin}$.
    \begin{equation}
        Q_3 = c_{\text{Wasser}} \cdot m \cdot \Delta T_3 = \SI{4.18}{\kilo\joule\per\kilo\gram\per\kelvin} \cdot \SI{1.5}{\kilo\gram} \cdot \SI{100}{\kelvin} = \SI{627.0}{\kilo\joule} \mDot
    \end{equation}
    
    \item \textbf{Verdampfen des Wassers bei \SI{100}{\celsius}:}
    Die hierfür benötigte Energie ist die Verdampfungswärme $Q_4 = \lambda_{\text{Verd}} \cdot m$.
    \begin{equation}
        Q_4 = \SI{2256.0}{\kilo\joule\per\kilo\gram} \cdot \SI{1.5}{\kilo\gram} = \SI{3384.0}{\kilo\joule} \mDot
    \end{equation}
    
    \item \textbf{Gesamte Wärmemenge} die zum Verdampfen von $\SI{1.5}{\kilo\gram}$ Eis benötigt wird, ist 
    \begin{equation}
        Q_{\text{ges}} = Q_1 + Q_2 + Q_3 + Q_4 = (63.0 + 499.2 + 627.0 + 3384.0)\,\si{\kilo\joule} = \SI{4573.2}{\kilo\joule} \mDot
    \end{equation}
\end{enumerate}
\end{loesungbox}





\begin{loesungbox}{Lösung zu \Cref{aufg:saft_tk}}
\begin{enumerate}
    \item \textbf{Temperatur-Zeit-Diagramm:}
    \begin{center}
        \begin{tikzpicture}[
                axis/.style={-{Stealth}, line width=2.0pt},
                box/.style={draw, fill=blue!15, minimum height=1.0cm, align=center},
                dashed_line/.style={dashed, line width=1.0pt}
            ]
            \def\yInitial{5}   % Starttemperatur 85 °C
            \def\yFreeze{1.5}    % Gefriertemperatur 0 °C
            \def\xTzero{0.2}     % Startzeit 0
            \def\xTone{4}      % Zeitpunkt t1
            \def\xTtwo{7}      % Zeitpunkt t2
            \def\xEnd{10}       % Endpunkt für die gestrichelte Linie
            \def\yBoxCenter{5.7}
            \def\boxHeight{0.8cm}
            
            % Achsen zeichnen
            \draw[axis] (-0.5, 0) -- (11, 0) node[right] {\Large $t$};
            \draw[axis] (0, -0.5) -- (0, 6) node[left] {\Large $T_{\text{Saft}}$};
            \foreach \x in {\xTzero, \xTone, \xTtwo} {
                    \draw[line width=1.2pt] (\x, -0.15) -- (\x, 0.15);
                }
            \draw[line width=1.2pt] (-0.15, \yInitial) -- (0.15, \yInitial);
            \draw[line width=1.2pt] (-0.15, \yFreeze) -- (0.15, \yFreeze);
            
            % Achsenbeschriftungen
            \node[left] at (-0.15, \yInitial) {\large $\SI{85}{\celsius}$};
            \node[left] at (-0.15, \yFreeze) {\large $\SI{0}{\celsius}$};
            \node[below] at (\xTzero, -0.15) {\large $0$};
            \node[below] at (\xTone, -0.15) {\large $t_1$};
            \node[below] at (\xTtwo, -0.15) {\large $t_2$};
            
            \draw[line width=1.4pt, black] (\xTzero, \yInitial) -- (\xTone, \yFreeze) -- (\xTtwo, \yFreeze);
            \draw[dashed_line, black] (\xTtwo, \yFreeze) -- (\xEnd, \yFreeze-2.);
            
            % Gestrichelte Hilfslinien
            \draw[dotted] (\xTone, 0) -- (\xTone, \yFreeze);
            \draw[dotted] (\xTtwo, 0) -- (\xTtwo, \yFreeze);
            
            % Markierungspunkt "gesamter Saft ist gefroren"
            \filldraw[black] (\xTtwo, \yFreeze) circle (2pt);
            
            % Anmerkung mit Pfeil
            \draw[->, >=stealth] (\xTtwo+1, \yFreeze+1) -- (\xTtwo+0.1, \yFreeze+0.1);
            \node[above right, align=left] at (\xTtwo+1, \yFreeze+1) {gesamter Saft ist gefroren};
            
            \pgfmathsetmacro{\widthPhaseOne}{\xTone - \xTzero}
            \pgfmathsetmacro{\widthPhaseTwo}{\xTtwo - \xTone}
            \pgfmathsetmacro{\widthPhaseThree}{\xEnd - \xTtwo} 
            
            \node[box, text width=\widthPhaseOne cm-0.2cm] at ($(\xTzero, \yBoxCenter) + (0.5*\widthPhaseOne, 0)$) {Abkühlen (flüssig)};
            \node[box, text width=\widthPhaseTwo cm-0.2cm] at ($(\xTone, \yBoxCenter) + (0.5*\widthPhaseTwo, 0)$) {Gefrieren};
            \node[box, text width=\widthPhaseThree cm-0.2cm] at ($(\xTtwo, \yBoxCenter) + (0.5*\widthPhaseThree, 0)$) {Abkühlen (fest)};
        \end{tikzpicture}
    \end{center}
    \item \textbf{Energie zum Abkühlen:}
    Zuerst wird die Masse des Saftes berechnet:
    \begin{equation}\label{eq:saft_masse}
        m_{\text{Saft}} = \rho_{\text{Saft}} \cdot V_{\text{Saft}} = \SI{1.1}{\kilogram\per\deci\metre\cubed} \cdot \SI{3.5}{\litre} = \SI{3.85}{\kilogram} \mDot
    \end{equation}
    Die Wärme, die für die Abkühlung auf \SI{0}{\degreeCelsius} abgeführt werden muss, ist:
    \begin{equation}\label{eq:saft_q1}
        Q_{1} = m_{\text{Saft}} \cdot c_{\text{Saft}} \cdot \Delta T = \SI{3.85}{\kg} \cdot \SI{4.2}{\kilo\joule\per\kilogram\per\kelvin} \cdot \SI{85}{\kelvin} = \SI{1374.45}{\kilo\joule} \mDot
    \end{equation}
    \item \textbf{Energie zum Gefrieren:}
    Damit der Saft gefriert, muss die latente Schmelzwärme abgeführt werden:
    \begin{equation}\label{eq:saft_q2}
        Q_{2} = m_{\text{Saft}} \cdot \lambda_{\text{Schmelz}} = \SI{3.85}{\kg} \cdot \SI{333.5}{\kilo\joule\per\kilogram} = \SI{1283.98}{\kilo\joule} \mDot
    \end{equation}
    \item \textbf{Gesamtzeit für den Prozess:}
    Die gesamte abzuführende Energie ist die Summe aus Abkühlung und Gefrieren:
    \begin{equation}\label{eq:saft_q_ges}
        Q_{\text{ges}} = Q_{1} + Q_{2} = \SI{1374.45}{\kilo\joule} + \SI{1283.98}{\kilo\joule} = \SI{2658.43}{\kilo\joule}\mDot
    \end{equation}
    Die benötigte Zeit dafür ergibt sich aus der Leistung $P = E/t$:
    \begin{equation}\label{eq:saft_zeit}
        t = \frac{Q_{\text{ges}}}{P} = \frac{\SI{2658430}{\joule}}{\SI{120}{\watt}} = \SI{22153.5}{\second} \approx \SI{6.15}{\hour} \mDot
    \end{equation}
    \item \textbf{Fazit:} Der gesamte Prozess dauert ca. \SI{6.15}{\hour}. Von 22:00 Uhr bis 07:00 Uhr morgens sind 9 Stunden vergangen. Da $\SI{9}{\hour} > \SI{6.15}{\hour}$, ist der Saft nicht nur vollständig gefroren, sondern bereits im festen Zustand weiter abgekühlt.
\end{enumerate}
\end{loesungbox}




\begin{loesungbox}{Lösung zu \Cref{aufg:wasserkocher}}
\begin{enumerate}
    \item \textbf{Temperatur-Zeit-Diagramm:}
    Das Diagramm zeigt zunächst einen linearen Anstieg der Temperatur von \SI{20}{\degreeCelsius} auf \SI{100}{\degreeCelsius} (Erhitzen). Anschließend bleibt die Temperatur konstant bei \SI{100}{\degreeCelsius}, während das Wasser verdampft.
    \begin{center}
    \resizebox{0.80\linewidth}{!}{
        \begin{tikzpicture}[
                axis/.style={-{Stealth}, line width=1.2pt},
                box/.style={draw, fill=blue!15, minimum height=0.8cm, align=center}
            ]
            
            \def\xTzero{0.3}      % Startzeit 0 s
            \def\xTone{2.5}    % Zeitpunkt t1 (skaliert)
            \def\xTtwo{10}      % Zeitpunkt t2 (skaliert)
            
            \def\yInitial{1.5}  % Starttemperatur 20 °C
            \def\yBoil{5.5}     % Siedetemperatur 100 °C
            \def\yBoxCenter{6.5} % y-Position der Boxen
            
            % Achsen zeichnen
            \draw[axis,line width=1.8pt] (-0.5, 0) -- (12, 0) node[right] {\Large $t \,[\si{\second}]$};
            \draw[axis,line width=1.8pt] (0, -0.5) -- (0, 7) node[above] {\Large $T\, [\si{\celsius}]$};
            
            % Ticks auf den Achsen
            \draw[line width=1pt] (\xTzero, -0.15) -- (\xTzero, 0.15); % Tick bei 0
            \draw[line width=1pt] (\xTone, -0.15) -- (\xTone, 0.15); % Tick bei 0
            \draw[line width=1pt] (\xTtwo, -0.15) -- (\xTtwo, 0.15); % Tick bei 0
            \draw[line width=1pt] (-0.15, \yInitial) -- (0.15, \yInitial);
            \draw[line width=1pt] (-0.15, \yBoil) -- (0.15, \yBoil);
            
            % Achsenbeschriftungen
            \node[left] at (-0.15, \yInitial) {\large $\SI{20}{\celsius}$};
            \node[left] at (-0.15, \yBoil) {\large $\SI{100}{\celsius}$};
            \node[below] at (\xTzero, -0.15) {\large $0$};
            \node[below] at (\xTone, -0.15) {\large $t_1$};
            \node[below] at (\xTtwo, -0.15) {\large $t_2$};
            
            % Temperaturverlauf
            \draw[line width=1.4pt, black] (\xTzero, \yInitial) -- (\xTone, \yBoil) -- (\xTtwo, \yBoil);
            % Gestrichelte Hilfslinien
            \draw[dotted] (\xTone, 0) -- (\xTone, \yBoil);
            \draw[dotted] (\xTtwo, 0) -- (\xTtwo, \yBoil);
            % Markierungspunkt am Ende
            \filldraw[black] (\xTtwo, \yBoil) circle (2.5pt);
            
            % Anmerkung mit Pfeil
            \draw[-{Stealth},, thick] (\xTtwo-1, \yBoil-1.1) -- (\xTtwo-0.1, \yBoil-0.1);
            \node[above left, align=left] at (\xTtwo-1, \yBoil-1.6) {Wasser komplett verdampft};
            
            % Boxen für die Phasen
            % Berechnung der Breiten und Positionen
            \pgfmathsetmacro{\widthPhaseOne}{\xTone - \xTzero}
            \pgfmathsetmacro{\widthPhaseTwo}{\xTtwo - \xTone}
            \pgfmathsetmacro{\centerPhaseOne}{(\xTone + \xTzero) / 2}
            \pgfmathsetmacro{\centerPhaseTwo}{(\xTtwo + \xTone) / 2}
            
            \node[box, text width=\widthPhaseOne cm-0.2cm] at (\centerPhaseOne, \yBoxCenter) {Erwärmen};
            \node[box, text width=\widthPhaseTwo cm-0.2cm] at (\centerPhaseTwo, \yBoxCenter) {Sieden / Verdampfen};
            
        \end{tikzpicture}
    }
    \end{center}
    \item \textbf{Energie zum Erhitzen:}
    Zuerst berechnen wir die Masse des Wassers: $m_W = \rho_W \cdot V = \SI{1}{\kilogram\per\deci\metre\cubed} \cdot \SI{1}{\litre} = \SI{1}{\kilogram}$. Die benötigte Wärme zum Erhitzen ist:
    \begin{equation}\label{eq:wasserkocher_erhitzen}
        Q_{1} = m_{W} \cdot c_{W} \cdot \Delta T = \SI{1}{\kg} \cdot \SI{4.18}{\kilo\joule\per\kilogram\per\kelvin} \cdot (\SI{100}{\celsius} - \SI{20}{\celsius}) = \SI{334.4}{\kilo\joule} \mDot
    \end{equation}
    \item \textbf{Energie zum Verdampfen:}
    Die Verdampfungswärme berechnet sich zu:
    \begin{equation}\label{eq:wasserkocher_verdampfen}
        Q_{2} = m_{W} \cdot \lambda_{\text{verd}} = \SI{1}{\kg} \cdot \SI{2256}{\kilo\joule\per\kilogram} = \SI{2256}{\kilo\joule} \mDot
    \end{equation}
    \item \textbf{Energie des Wasserkochers:}
    Die Zeit in SI-Einheiten beträgt 
    $$t = \SI{25}{\minute} = 25 \cdot 60\,\si{\second} = \SI{1500}{\second} \mDot$$
    Die vom Kocher in dieser Zeit gelieferte Energie ist:
    \begin{equation}\label{eq:wasserkocher_energie}
        E_{\text{Kocher}} = P \cdot t = \SI{2000}{\watt} \cdot \SI{1500}{\second} = \SI{3000000}{\joule} = \SI{3000}{\kilo\joule} \mDot
    \end{equation}
    \item \textbf{Fazit:}
    Die insgesamt benötigte Energie, um das Wasser zu erhitzen und vollständig zu verdampfen, beträgt:
    \begin{equation}\label{eq:wasserkocher_gesamtenergie}
        Q_{\text{ges}} = Q_1 + Q_2 = \SI{334.4}{\kilo\joule} + \SI{2256}{\kilo\joule} = \SI{2590.4}{\kilo\joule} \mDot
    \end{equation}
    Da der Wasserkocher mit \SI{3000}{\kilo\joule} mehr Energie liefert als die benötigten \SI{2590.4}{\kilo\joule}, kommen Sie nicht rechtzeitig zurück. Das gesamte Wasser ist bereits verdampft.
\end{enumerate}
\end{loesungbox}



\begin{loesungbox}{Lösung zu \Cref{aufg:mischtemperatur}}
Zuerst wird die Masse des Wassers berechnet:
\begin{equation}\label{eq:mischtemperatur_mw}
     m_{\text{Wasser}} = \rho_{\text{Wasser}} \cdot V_{\text{Wasser}} = \SI{1}{\kilogram\per\deci\metre\cubed} \cdot \SI{5}{\litre} = \SI{5}{\kilogram} \mDot
\end{equation}
\begin{enumerate}
    \item \textbf{Vom Wasser aufgenommene Wärme:} Das Wasser erwärmt sich von $T_{\text{Wasser}}$ auf $T_{\text{Misch}}$.
    \begin{equation}\label{eq:mischtemperatur_q_auf}
        Q_{\text{Wasser,auf}} = m_{\text{Wasser}} \cdot c_{\text{Wasser}} \cdot (T_{\text{Misch}} - T_{\text{Wasser}})
    \end{equation}
    \item \textbf{Von der Eisenkugel abgegebene Wärme:} Die Eisenkugel kühlt sich von $T_{\text{Kugel}}$ auf $T_{\text{Misch}}$ ab.
    \begin{equation}\label{eq:mischtemperatur_q_ab}
        Q_{\text{Eisen,ab}} = m_{\text{Kugel}} \cdot c_{\text{Eisen}} \cdot (T_{\text{Kugel}} - T_{\text{Misch}})
    \end{equation}
    \item \textbf{Thermisches Gleichgewicht und Mischtemperatur:} Im thermischen Gleichgewicht ist die aufgenommene Wärme gleich der abgegebenen Wärme:
    \begin{align}
        Q_{\text{Wasser,auf}} &= Q_{\text{Eisen,ab}} \nonumber\\
        m_{\text{W}} c_{\text{W}} (T_{\text{Misch}} - T_{\text{W}}) &= m_{\text{E}} c_{\text{E}} (T_{\text{E}} - T_{\text{Misch}}) \nonumber\\
        m_{\text{W}} c_{\text{W}} T_{\text{Misch}} - m_{\text{W}} c_{\text{W}} T_{\text{W}} &= m_{\text{E}} c_{\text{E}} T_{\text{E}} - m_{\text{E}} c_{\text{E}} T_{\text{Misch}}\nonumber \\
        (m_{\text{W}} c_{\text{W}} + m_{\text{E}} c_{\text{E}}) T_{\text{Misch}} &= m_{\text{E}} c_{\text{E}} T_{\text{E}} + m_{\text{W}} c_{\text{W}} T_{\text{W}} \nonumber\\
        T_{\text{Misch}} &= \frac{m_{\text{E}} c_{\text{E}} T_{\text{E}} + m_{\text{W}} c_{\text{W}} T_{\text{W}}}{m_{\text{W}} c_{\text{W}} + m_{\text{E}} c_{\text{E}}} \label{eq:mischtemperatur_formel}
    \end{align}
    Einsetzen der gegebenen Werte (Temperaturen können in \si{\celsius} bleiben, da nur Differenzen relevant sind):
    \begin{equation}
        T_{\text{Misch}} = \frac{ \SI{0.5}{\kg} \cdot \SI{0.45}{\kilo\joule\per\kg\per\kelvin} \cdot \SI{250}{\celsius} + \SI{5}{\kg} \cdot \SI{4.18}{\kilo\joule\per\kg\per\kelvin} \cdot \SI{20}{\celsius}}{ \SI{5}{\kg} \cdot \SI{4.18}{\kilo\joule\per\kg\per\kelvin} + \SI{0.5}{\kg} \cdot \SI{0.45}{\kilo\joule\per\kg\per\kelvin}} \approx \SI{22.45}{\celsius} \mDot
    \end{equation}
\end{enumerate}
\end{loesungbox}




\begin{loesungbox}{Lösung zu \Cref{aufg:eiswuerfel_limo}}
\begin{enumerate}
    \item \textbf{Prüfung, ob das Eis vollständig schmilzt} \\
    Zuerst berechnen wir die Gesamtmasse der Eiswürfel:
    \begin{equation}
        m_{\text{Eis}} = 2 \cdot \SI{25}{\gram} = \SI{50}{\gram} = \SI{0.05}{\kilogram} \mDot
    \end{equation}
    Die zum vollständigen Schmelzen des Eises bei \SI{0}{\degreeCelsius} benötigte Energie \\ (Schmelzwärme) ist:
    \begin{equation}\label{eq:Q_schmelz_limo}
        Q_{\text{Schmelz}} = m_{\text{Eis}} \cdot \lambda_{\text{Schmelz}} = \SI{0.05}{\kg} \cdot \SI{332.8}{\kilo\joule\per\kilogram} = \SI{16.64}{\kilo\joule} \mDot
    \end{equation}
    Die maximal verfügbare Wärme, welche die Limonade abgeben kann, wenn sie von \SI{33}{\degreeCelsius} auf \SI{0}{\degreeCelsius} abkühlt, beträgt:
    \begin{equation}\label{eq:Q_limo_max}
        Q_{\text{Limo, max}} = m_{\text{Limo}} \cdot c_{\text{Limo}} \cdot \Delta T = \SI{0.24}{\kg} \cdot \SI{4.18}{\kilo\joule\per\kilogram\per\kelvin} \cdot (\SI{33}{\kelvin}) \approx \SI{33.1}{\kilo\joule} \mDot
    \end{equation}
    Da $Q_{\text{Limo, max}} > Q_{\text{Schmelz}}$ $(\SI{33.1}{\kilo\joule} > \SI{16.64}{\kilo\joule})$, schmelzen die Eiswürfel vollständig.

    \item \textbf{Berechnung der Endtemperatur $T_{\text{Ende}}$} \\
    Im thermischen Gleichgewicht ist die von der Limonade abgegebene Wärme gleich der vom Eis aufgenommenen Wärme. Die vom Eis aufgenommene Wärme setzt sich aus der Schmelzwärme und der Wärme zur Erwärmung des geschmolzenen Wassers von \SI{0}{\celsius} auf $T_{\text{Ende}}$ zusammen.
    \begin{equation}\label{eq:waermebilanz_limo}
        Q_{\text{abgegeben}} = Q_{\text{aufgenommen}}
    \end{equation}
    \begin{equation}
        m_{\text{L}} c_{\text{L}} (T_{\text{L,Anfang}} - T_{\text{Ende}}) = m_{\text{E}} \lambda_{\text{Schmelz}} + m_{\text{E}} c_{\text{W}} (T_{\text{Ende}} - T_{\text{E,Anfang}})
    \end{equation}
    Wir lösen die Gleichung nach $T_\text{Ende}$ auf:
    \begin{gather}
        m_{\text{L}} c_{\text{L}} T_{\text{L,Anfang}} - m_{\text{E}} \lambda_{\text{Schmelz}} + m_{\text{E}} c_{\text{W}} T_{\text{E,Anfang}} = m_{\text{L}} c_{\text{L}} T_{\text{Ende}} + m_{\text{E}} c_{\text{W}} T_{\text{Ende}} \nonumber\\
        m_{\text{L}} c_{\text{L}} T_{\text{L,Anfang}} + m_{\text{E}} c_{\text{W}} T_{\text{E,Anfang}} - m_{\text{E}} \lambda_{\text{Schmelz}} = T_{\text{Ende}} (m_{\text{L}} c_{\text{L}} + m_{\text{E}} c_{\text{W}}) \nonumber\\
        T_{\text{Ende}} = \frac{m_{\text{L}} c_{\text{L}} T_{\text{L,Anfang}} + m_{\text{E}} c_{\text{W}} T_{\text{E,Anfang}} - m_{\text{E}} \lambda_{\text{Schmelz}}}{m_{\text{L}} c_{\text{L}} + m_{\text{E}} c_{\text{W}}} \mDot
    \end{gather}
    Setzt man die numerischen Werte ein, ergibt sich die Endtemperatur der Mischung
    \begin{multline*}
        T_\text{Ende} = \frac{0,24 \cdot 4,18 \cdot 306,15 + 0,05 \cdot 4,18 \cdot 273,15 - 0,05\cdot 332,8}{0,24\cdot 4,18 + 0,05\cdot 4,18} \\
        =\SI{286.73}{\kelvin} = \SI{13.58}{\celsius}.
    \end{multline*}
\end{enumerate}
\end{loesungbox}



\begin{loesungbox}{Lösung zu \Cref{aufg:ideales_gas}}
\begin{enumerate}
    \item \textbf{Temperatur der Isotherme (I):}
    Wir lesen einen Punkt von der Isotherme ab, z.B. $(p_1, V_1) = (\SI{100000}{\pascal}, \SI{0.04}{\metre\cubed})$. Die Temperatur berechnet sich mit der idealen Gasgleichung:
    \begin{equation}\label{eq:isotherme_temp1}
        T_1 = \frac{p_1 V_1}{n R} = \frac{\SI{100000}{\pascal} \cdot \SI{0.04}{\metre\cubed}}{\SI{1}{\mole} \cdot \SI{8.314}{\joule\per\mole\per\kelvin}} \approx \SI{481.1}{\kelvin} \mDot
    \end{equation}

    \item \textbf{Isotherme (II) bei $T=\SI{250}{\kelvin}$:}
    Die Isotherme (II) verläuft unterhalb der Isotherme (I), da die Temperatur geringer ist. Die Punkte der Isotherme müssen über 
    $$ p(V) = \frac{n R T}{V} = \frac{1 \cdot 8,314 \cdot 250}{V}$$ 
    bzw. 
    $$ V(p) = \frac{n R T}{p} = \frac{1 \cdot 8,314 \cdot 250}{p}$$ 
    berechnet werden:
    \begin{center}
        \begin{tabular}{c|c}
            $V$   & $p$  \\
            \hline
            0,005 & 415.700 \\
            0,01 & 207.850 \\
            0,02 & 103.925 \\
            0,03 & 69.283 \\
            0,04 & 51.963 \\
            0,05 & 41.570 \\
        \end{tabular}
    \end{center}

    \item \textbf{Isobare (III):} Eine Isobare ist eine horizontale Linie im $(p,V)$-Diagramm (konstanter Druck).

    \item \textbf{Isochore (IV):} Eine Isochore ist eine vertikale Linie im $(p,V)$-Diagramm (konstantes Volumen).
    
    \begin{center}
        \includegraphics[width=0.78\textwidth]{Bilder/Uebungsaufgaben/ideales_gas_loesung.png}
    \end{center}

    \item \textbf{Volumen unter Normalbedingungen:}
    Normalbedingungen: $p = \SI{1.013}{\bar} = \SI{101300}{\pascal}$, $T = \SI{0}{\degreeCelsius} = \SI{273.15}{\kelvin}$.
    \begin{equation}\label{eq:molarvolumen_normal}
        V = \frac{n R T}{p} = \frac{\SI{1}{\mole} \cdot \SI{8.314}{\joule\per\mole\per\kelvin} \cdot \SI{273.15}{\kelvin}}{\SI{101300}{\pascal}} \approx \SI{0.0224}{\metre\cubed} = \SI{22.4}{\litre} \mDot
    \end{equation}
\end{enumerate}
\end{loesungbox}



\begin{loesungbox}{Lösung zu \Cref{aufg:volumenarbeit}}
\begin{enumerate}
    \item \textbf{Berechnung der Arbeit in den Teilschritten:}
    \begin{itemize}
        \item \textbf{A $\rightarrow$ B (Isobare Expansion):}
        Der Druck ist konstant bei $p_A = \SI{2}{\bar} = \SI{2e5}{\pascal}$. Das Volumen ändert sich von $V_A = \SI{1}{\litre} = \SI{e-3}{\meter\cubed}$ auf $V_B = \SI{2.5}{\litre} = \SI{2.5e-3}{\meter\cubed}$.
        \begin{multline}
            W_{AB} = -\int p_A \dd V = -p_A \cdot (V_B - V_A) = \\ -\SI{2e5}{\pascal} \cdot (\SI{2.5e-3} {\meter\cubed} - \SI{1e-3}{\meter\cubed}) = -\SI{300}{\joule} \mDot
        \end{multline}
        Die Arbeit ist negativ, da das Gas Arbeit an der Umgebung verrichtet (Expansion).
        
        \item \textbf{B $\rightarrow$ C (Isochore Abkühlung):}
        Das Volumen ist konstant, daher ist $\dd V = 0$. Es wird keine Volumenarbeit verrichtet.
        \begin{equation}
            W_{BC} = 0 \mDot
        \end{equation}
        
        \item \textbf{C $\rightarrow$ D (Isobare Kompression):}
        Der Druck ist konstant bei $p_C = \SI{1}{\bar} = \SI{1e5}{\pascal}$. Das Volumen ändert sich von $V_C = \SI{2.5}{\litre}$ auf $V_D = \SI{1}{\litre}$.
        \begin{multline}
            W_{CD} = -\int p_A \dd V = -p_C \cdot (V_D - V_C) =\\
            -\SI{1e5}{\pascal} \cdot (\SI{1e-3}{\meter\cubed} - \SI{2.5e-3}{\meter\cubed}) = +\SI{150}{\joule} \mDot
        \end{multline}
        Die Arbeit ist positiv, da am Gas Arbeit verrichtet wird (Kompression).
        
        \item \textbf{D $\rightarrow$ A (Isochore Erwärmung):}
        Das Volumen ist konstant, daher ist $\dd V = 0$. Es wird keine Volumenarbeit verrichtet.
        \begin{equation}
            W_{DA} = 0 \mDot
        \end{equation}
    \end{itemize}
    
    \textbf{Gesamtarbeit des Zyklus:}
    Die Gesamtarbeit ist die Summe der Arbeiten der Teilschritte:
    \begin{equation}
        \Delta W_{\text{ges}} = W_{AB} + W_{BC} + W_{CD} + W_{DA} = -300 + 0 + 150 + 0 = \SI{-150}{\joule} \mDot
    \end{equation}
    
    \item \textbf{Berechnung der netto zugeführten Wärmemenge:}
    Bei einem vollständigen Kreisprozess kehrt das System in seinen Ausgangszustand zurück. Daher ist die Änderung der inneren Energie $\Delta U$ null.
    \begin{equation}
        \Delta U_{\text{ges}} = 0 \mDot
    \end{equation}
    Nach dem ersten Hauptsatz der Thermodynamik gilt:
    \begin{equation}
        \Delta U_{\text{ges}} = \Delta Q_{\text{ges}} + \Delta W_{\text{ges}} = 0 \implies \Delta Q_{\text{ges}} = -\Delta W_{\text{ges}} \mDot
    \end{equation}
    Setzen wir den Wert für die Gesamtarbeit ein:
    \begin{equation}
        \Delta Q_{\text{ges}} = -(\SI{-150}{\joule}) = +\SI{150}{\joule} \mDot
    \end{equation}
    Dem Gas wird netto eine Wärmemenge von \SI{150}{\joule} zugeführt, und es verrichtet eine Netto-Arbeit von \SI{150}{\joule} an der Umgebung.
\end{enumerate}
\end{loesungbox}




\begin{loesungbox}{Lösung zu \Cref{aufg:kreisprozess}}
\begin{enumerate}
    \item \textbf{Zustandsvariablen an den Eckpunkten:}
    Aus dem Diagramm abgelesen und mit der idealen Gasgleichung 
    $$T_i = \frac{p_i V_i}{n R}$$
    berechnet:
    \begin{itemize}
        \item Punkt (1): $p_1 = \SI{400}{\kilo\pascal}$, $V_1 = \SI{0.01}{\metre\cubed}$, $T_1 = \SI{481.1}{\kelvin}$
        \item Punkt (2): $p_2 = \SI{400}{\kilo\pascal}$, $V_2 = \SI{0.03}{\metre\cubed}$, $T_2 = \SI{1443.4}{\kelvin}$
        \item Punkt (3): $p_3 = \SI{100}{\kilo\pascal}$, $V_3 = \SI{0.03}{\metre\cubed}$, $T_3 = \SI{360.8}{\kelvin}$
        \item Punkt (4): $p_4 = \SI{100}{\kilo\pascal}$, $V_4 = \SI{0.01}{\metre\cubed}$, $T_4 = \SI{120.3}{\kelvin}$
    \end{itemize}
    \item \textbf{Beschreibung der Teilprozesse:}
    \begin{center}
        \includegraphics[width=0.65\textwidth]{Bilder/Uebungsaufgaben/lösung_kreisprozess.png}
    \end{center}
    \item \textbf{Berechnung der Arbeit:} \\
    Bei isochoren Prozessen ist die Arbeit Null: $W_{23} = 0$ und $W_{41} = 0$. \\
    Bei isobaren Prozessen gilt $W = -\int p \dd V = -p \Delta V$:
    \begin{align}
        W_{12} &= -p_1(V_2 - V_1) = -\SI{400e3}{\pascal}(\SI{0.03}{\m\cubed} - \SI{0.01}{\m\cubed}) = \SI{-8000}{\joule} \label{eq:kreisprozess_w12}\\
        W_{34} &= -p_3(V_4 - V_3) = -\SI{100e3}{\pascal}(\SI{0.01}{\m\cubed} - \SI{0.03}{\m\cubed}) = \SI{+2000}{\joule} \label{eq:kreisprozess_w34}
    \end{align}
    Im Schritt $1\rightarrow 2$ wird Arbeit vom System geleistet/abgegeben ($W < 0$), während im Schritt $3 \rightarrow 4$ vom System Arbeit aufgenommen wird ($W > 0$).
    \item \textbf{1. Hauptsatz der Thermodynamik:}
    \begin{equation}\label{eq:erster_hauptsatz}
        \Delta U = \Delta Q + \Delta W
    \end{equation}
    Dabei ist $\Delta U$ die Änderung der inneren Energie eines Systems, $\Delta Q$ die ihm zu- oder abgeführte Wärme und $\Delta W$ die am System verrichtete oder vom System geleistete Arbeit.
    \item \textbf{Berechnung der Wärme:} Für isobare Prozesse gilt 
    $$Q = n C_p \Delta T$$ 
    und für isochore Prozesse gilt 
    $$Q = n C_V \Delta T \mDot $$
    \begin{align}
        Q_{12} &= n C_p (T_2 - T_1) = \SI{1}{\mole} \cdot (\tfrac{5}{2} R) \cdot (\SI{1443.4}{K}-\SI{481.1}{K}) \approx \SI{20.0}{\kilo\joule} \label{eq:kreisprozess_q12}\\
        Q_{23} &= n C_V (T_3 - T_2) = \SI{1}{\mole} \cdot (\tfrac{3}{2} R) \cdot (\SI{360.8}{K}-\SI{1443.4}{K}) \approx \SI{-13.5}{\kilo\joule} \label{eq:kreisprozess_q23}\\
        Q_{34} &= n C_p (T_4 - T_3) = \SI{1}{\mole} \cdot (\tfrac{5}{2} R) \cdot (\SI{120.3}{K}-\SI{360.8}{K}) \approx \SI{-5.0}{\kilo\joule} \label{eq:kreisprozess_q34}\\
        Q_{41} &= n C_V (T_1 - T_4) = \SI{1}{\mole} \cdot (\tfrac{3}{2} R) \cdot (\SI{481.1}{K}-\SI{120.3}{K}) \approx \SI{4.5}{\kilo\joule} \label{eq:kreisprozess_q41}
    \end{align}
    \item \textbf{Innere Energie im Kreisprozess:} Für einen vollständigen Kreisprozess muss die Gesamtänderung der inneren Energie Null sein, da sie eine Zustandsgröße ist.
    \begin{multline}
        \Delta U_{\text{ges}} = \sum Q_{ij} + \sum W_{ij} = (20,0 - 13,5 - 5,0 + 4,5)\,\si{\kilo\joule} + (-8,0 + 2,0)\,\si{\kilo\joule} \\
        = (6,0) +(-6,0) = \SI{0}{\kilo\joule} \mDot
    \end{multline}
    \item \textbf{Wirkungsgrad der Maschine:} Der Wirkungsgrad ist das Verhältnis von verrichteter Nettoarbeit zu zugeführter Wärme:
    \begin{equation}\label{eq:wirkungsgrad_allg}
        \eta = \left| \frac{\Delta W}{\Delta Q_{\text{zu}}} \right| = \frac{\SI{6}{\kilo\joule}}{\SI{24.5}{\kilo\joule}} \approx 0,245 = \SI{24.5}{\percent}
    \end{equation}
\end{enumerate}
\end{loesungbox}





\begin{loesungbox}{Lösung zu \Cref{aufg:carnot}}
\begin{enumerate}
    \item \textbf{Zuordnung der Prozessschritte:} Adiabaten verlaufen im p-V-Diagramm steiler als Isothermen.
    \begin{itemize}
        \item $1 \rightarrow 2$: Isotherme Expansion (Volumen nimmt zu; Wärme $\Delta Q_{1}$ wird bei $T_1$ zugeführt)
        \item $2 \rightarrow 3$: Adiabatische Expansion (Volumen nimmt zu; Keine Wärmezufuhr; Temperatur sinkt von $T_1$ auf $T_2$)
        \item $3 \rightarrow 4$: Isotherme Kompression (Volumen nimmt ab; Wärme $\Delta Q_{2}$ wird bei $T_2$ abgeführt)
        \item $4 \rightarrow 1$: Adiabatische Kompression (Volumen nimmt ab; Keine Wärmezufuhr; Temperatur steigt von $T_2$ auf $T_1$)
    \end{itemize}
    \begin{center}
        \includegraphics[width=0.6\textwidth]{Bilder/Uebungsaufgaben/carnot_prozess_loesung.png}
    \end{center}
    \item \textbf{Arbeit:}
    Bei jeder Expansion ($1 \rightarrow 2$ und $2 \rightarrow 3$) verrichtet das Gas Arbeit ($\Delta W < 0$). Bei jeder Kompression ($3 \rightarrow 4$ und $4 \rightarrow 1$) wird Arbeit am Gas verrichtet ($\Delta W > 0$). Die blauen Pfeile symbolisieren die Zu- bzw. Abfuhr von Arbeit in der Grafik.
    
    \item \textbf{Nettoarbeit:} Die vom System verrichtete Nettoarbeit $\Delta W$ entspricht der von den Kurven umschlossenen Fläche im $(p,V)$-Diagramm. Da der Prozess im Uhrzeigersinn durchlaufen wird, ist die Fläche positiv, was einer vom System abgegebenen (negativen Konvention) Arbeit entspricht.

    \item \textbf{Maximierung des Wirkungsgrads:}
    Der Wirkungsgrad $\eta = 1 - T_2/T_1$ wird maximiert, indem die Temperatur $T_1$ des warmen Reservoirs so hoch wie möglich und die Temperatur $T_2$ des kalten Reservoirs so niedrig wie möglich ist. \\
    \textbf{Technische Schwierigkeiten:}
    \begin{itemize}
        \item \textbf{Hohe Temperatur $T_1$:} Die maximale Temperatur ist durch die Hitzebeständigkeit der Materialien des Motors begrenzt.
        \item \textbf{Niedrige Temperatur $T_2$:} Die niedrigste praktisch erreichbare Temperatur ist meist die Umgebungstemperatur. Eine aktive Kühlung darunter wäre sehr energieaufwändig.
        \item \textbf{Prozessführung:} Isotherme und adiabatische Zustandsänderungen sind in der Realität nur annähernd und sehr langsam realisierbar, was der geforderten Leistung von Motoren widerspricht.
    \end{itemize}
\end{enumerate}
\end{loesungbox}






\end{document}