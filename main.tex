\documentclass[ngerman]{scrreprt}
\usepackage[utf8]{inputenc}
\usepackage[ngerman]{babel}
\usepackage[colorlinks=true,citecolor=blue,linkcolor=blue,urlcolor=blue]{hyperref}
\usepackage{lmodern} % Sorgt für eine schönere Schrift
\usepackage[dvipsnames, table]{xcolor}
\usepackage[most]{tcolorbox}
\usepackage{colortbl}
\usepackage{enumitem} % for lists
\usepackage[framemethod=tikz]{mdframed}
\usepackage{cancel}

\usepackage{caption}
\captionsetup{
  format=plain, % Kein Einzug
  labelfont=it, % Kursiv für "Abbildung X"
  textfont=normal, % Normaler Text für die restliche Caption
}

\usepackage{tikz}
\usetikzlibrary{positioning,arrows.meta, angles, calc, quotes, decorations.pathmorphing, decorations.pathreplacing, decorations.markings, patterns}

\usepackage{subcaption} % needed for labelphantom
% \usepackage{subfigure} % outdated and replaced by subcaption

\usepackage[table]{xcolor}
\usepackage{array}
\usepackage{booktabs}
\usepackage{tabularx}
\usepackage{nicematrix} % nicematrix statt tabularx laden

\usepackage{multicol}
\usepackage{multirow}

\usepackage{makecell} 
\renewcommand\theadfont{\bfseries} 

\usepackage{graphicx}
% \usepackage{float}
\usepackage{epstopdf}

\usepackage{eurosym}  % euro symbol € 
\usepackage{amsmath,amssymb,latexsym,amsfonts,bbm, bm, bbold}
\usepackage{mathtools}
\usepackage[b]{esvect} % For \vv (option [d] is a nice default)

% \usepackage{unicode-math} % clashes with si-unit x
\usepackage{dsfont,relsize}

\usepackage{siunitx}

\sisetup{
    locale = DE, 
    angle-symbol-degree = ^\circ,
    separate-uncertainty, 
    inter-unit-product = \,,
    range-units = single, 
    list-units = single, 
    per-mode=symbol-or-fraction,
    range-phrase = { bis }
}
\DeclareSIUnit{\litre}{l}
\DeclareSIUnit{\calorie}{cal}
\DeclareSIUnit{\barPr}{bar}
\DeclareSIUnit{\fahrenheit}{\SIUnitSymbolDegree F}

%\crefname{subsection}{subsection}{subsections}
\definecolor{mygreen}{rgb}{0.831, 0.929, 0.855}
\definecolor{myred}{rgb}{0.973, 0.843, 0.855}
\definecolor{mygray}{rgb}{0.902, 0.902, 0.902}
\definecolor{boxcol_title_navyblue}{RGB}{20, 50, 120} 
%tikz colors
\definecolor{pointblue}{HTML}{004A8F}
\definecolor{darkyellow}{HTML}{ffd900}
\definecolor{purpleCol}{HTML}{a834eb}
\definecolor{purpleColFade}{HTML}{ccb2db}
% integral
\definecolor{figOrange}{RGB}{230, 145, 56}
\definecolor{figGreen}{RGB}{39, 174, 96}
\definecolor{figRed}{RGB}{192, 57, 43}
\definecolor{figGrayFill}{RGB}{245, 245, 245}
\definecolor{figGrayBorder}{RGB}{170, 170, 170}
% colors for the cref package
\definecolor{myblue}{RGB}{51,51,255}
\definecolor{myRefcolor}{RGB}{51,51,200}
\definecolor{myCiteColor}{RGB}{33,140,33}


% the next 4 lines are needed to be able to reference subfigures properly
\newcommand{\labelphantom}[1]{%
  \parbox{0pt}{\phantomsubcaption\label{#1}}%
}

\usepackage[nameinlink]{cleveref}
\hypersetup{
    unicode=false,          % non-Latin characters in Acrobat’s bookmarks
    pdftoolbar=true,        % show Acrobat’s toolbar?
    pdfmenubar=true,        % show Acrobat’s menu?
    pdffitwindow=false,     % window fit to page when opened
    pdfstartview={FitH},    % fits the width of the page to the window
    pdftitle={Physikalische Grundlagen},    % title
    pdfauthor={Schumer Alexander},     % author
    pdfnewwindow=false,      % links in new window
    colorlinks=true,       % false: boxed links; true: colored links
    linktoc=page,
    linkcolor=blue,  %Violet % color of internal links (change box color with linkbordercolor)
    citecolor=myCiteColor, %ForestGreen % color of links to bibliography
    filecolor=blue,      % color of file links
    urlcolor=blue,           % color of external links
    pdfborder={0 0 0}
}	
% \creflabelformat{subfigure}{#1~(#2)}


% --- EIGENE BOX-UMGEBUNG DEFINIEREN ---
\definecolor{boxcol_back_lorange}{RGB}{255, 232, 201} % Gelb/Orange-Tons
\definecolor{boxcol_title_orange}{RGB}{254, 204, 75} % Gelb/Orange-Tons
\definecolor{boxcol_frame_orange}{RGB}{245, 196, 75} % Gelb/Orange-Tons
\newtcolorbox{rememberbox}[2][]{
  colback=boxcol_back_lorange,   % Hintergrundfarbe der Box
  colframe=boxcol_frame_orange,         % Rahmenfarbe
  coltitle=black,                % Farbe des Titels
  colbacktitle=boxcol_title_orange,   % Hintergrundfarbe der Box
  fonttitle=\bfseries,
  leftrule=4mm,
  boxsep=6pt,         % Innerer Abstand
  arc=1mm,            % Scharfe Ecken (keine Rundung)
  enhanced, 
  before skip balanced=0.5cm,
  after skip balanced=0.5cm,
  % drop shadow,
  % sharp corners,      % Alternative für scharfe Ecken
  fontupper=\normalsize, % Schriftgröße im Inneren
  title={#2}, #1
}

\definecolor{boxcol_back_lblue}{RGB}{239, 247, 255} % Blau (hell)-Ton
\definecolor{boxcol_title_blue}{RGB}{167, 209, 255} % Blau (dunkel)-Ton
\newtcolorbox{examplebox}[2][]{
  colback=boxcol_back_lblue,   % Hintergrundfarbe
  colframe=boxcol_title_blue,  % Rahmenfarbe (hier identisch zum Hintergrund)
  fonttitle=\bfseries,
  coltitle=black,
  boxsep=5pt,         % Innerer Abstand
  arc=1mm,            % Scharfe Ecken (keine Rundung)
  enhanced,
  before skip balanced=0.5cm,
  after skip balanced=0.5cm,
  sharp corners,      % Alternative für scharfe Ecken
  fontupper=\normalsize, % Schriftgröße im Inneren
  title={#2}, #1
}

\definecolor{boxcol_back_lred}{RGB}{255, 239, 239} % rot (hell)-Ton
\definecolor{boxcol_title_red}{RGB}{255, 167, 167} % rot (dunkel)-Ton
\definecolor{boxcol_frame_red}{RGB}{245, 98, 75} % rot-Tons
\newtcolorbox{importantbox}[2][]{
  colback=boxcol_back_lred,   % Hintergrundfarbe der Box
  colframe=boxcol_frame_red,         % Rahmenfarbe
  coltitle=black,                % Farbe des Titels
  colbacktitle=boxcol_title_red,   % Hintergrundfarbe der Box
  fonttitle=\bfseries,
  leftrule=4mm,
  boxsep=6pt,         % Innerer Abstand
  arc=1mm,            % Scharfe Ecken (keine Rundung)
  enhanced, 
  before skip balanced=0.5cm,
  after skip balanced=0.5cm,
  % drop shadow,
  % sharp corners,      % Alternative für scharfe Ecken
  fontupper=\normalsize, % Schriftgröße im Inneren
  title={#2}, #1
}

\newcolumntype{C}[1]{>{\centering\arraybackslash}p{#1}}

% Needed for the bibliography style (.bst files)
\providecommand{\Verfuegbar}{Verf{\"u}gbar}


%defs for the paper
\newcommand{\mDot}{\,.}
\newcommand{\mComma}{\,{,}\,}

% text subscripts
\newcommand{\kin}{\mathrm{kin}}
\newcommand{\pot}{\mathrm{pot}}
\newcommand{\rot}{\mathrm{rot}}
\newcommand{\trans}{\mathrm{trans}}
\newcommand{\atm}{\mathrm{atm}}
\newcommand{\minText}{\mathrm{min}}
\newcommand{\maxText}{\mathrm{max}}
% quantities with text subscripts
\newcommand{\Ekin}{E_{\kin}}
\newcommand{\Epot}{E_{\pot}}
\newcommand{\Erot}{E_{\rot}}
\newcommand{\Etrans}{E_{\trans}}
\newcommand{\kB}{k_{\mathrm{B}}}

% Text 
\newcommand{\Schro}{Schr\"o\-din\-ger }

% vectors 
\newcommand{\ivec}[1]{\vv{#1}} % using esvect package
\newcommand{\ivecS}[2]{\vv*{#1}{\!#2}} % using esvect package
% column vector
\newcommand{\icolTwo}[2]{\begin{pmatrix} #1 \\ #2 \end{pmatrix}}
\newcommand{\icolThree}[3]{\begin{pmatrix} #1 \\ #2 \\ #3 \end{pmatrix}}
% row vector (INLINE)
\newcommand{\inlrowTwo}[2]{(#1, #2)}
\newcommand{\inlrowThree}[3]{(#1, #2, #3)}

% point 
\newcommand{\ipTwo}[2]{(#1\!\mid\! #2)}
\newcommand{\ipThree}[3]{(#1\!\mid\! #2\!\mid\! #3)}

% rangle, langle 
\newcommand{\lrangle}[1]{{\langle{#1}\rangle}}
% measurement units
\newcommand{\Unit}[1]{\,\mathrm{#1}}

\newcommand{\msSp}{\;}
\newcommand{\mdSp}{\;\;}
\newcommand{\mtSp}{\;\;\;}
\newcommand{\mqSp}{\;\;\;\;}

  
%mathematical symbols
\newcommand{\defeq}{\vcentcolon=}
\newcommand{\eqdef}{=\vcentcolon}
\newcommand*\conj[1]{\bar{#1}}
\newcommand{\eqexcl}{\stackrel{!}{=}}
\newcommand{\eqquestion}{\stackrel{?}{=}}
\newcommand\equalhatInl{\mathrel{\stackon[1.0pt]{=}{\stretchto{%
    \scalerel*[\widthof{=}]{\wedge}{\rule{1ex}{3ex}}}{0.45ex}}}}
\newcommand\equalhat{\mathrel{\stackon[4.8pt]{=}{\stretchto{%
    \scalerel*[\widthof{=}]{\wedge}{\rule{1ex}{3ex}}}{0.45ex}}}}
\newcommand{\mAND}{\land}
\newcommand{\mOR}{\lor}
\newcommand{\mNOT}{\lnot}

% text editing
\newcommand{\textunderscript}[1]{$_{\text{#1}}$}
\newcommand{\textupperscript}[1]{$^{\text{#1}}$}
\newcommand{\eqqref}[1]{eq.\!~(\ref{#1})}
\newcommand{\Eqqref}[1]{Eq.\!~(\ref{#1})}
\newcommand{\figref}[1]{fig.\!~(\ref{#1})}
\newcommand{\Figref}[1]{Fig.\!~(\ref{#1})}
\newcommand{\secref}[1]{sec.\!~(\ref{#1})}
\newcommand{\Secref}[1]{Sec.\!~(\ref{#1})}

%general abbreviations (in German)
\newcommand{\wA}{\mbox{w.\,A.\ }}
\newcommand{\fA}{\mbox{f.\,A.\ }}
\newcommand{\zB}{\mbox{z.\,B.\ }}
\newcommand{\bzw}{\mbox{bzw.\ }}
\newcommand{\gDh}{\mbox{d.\,h.\ }}
\newcommand{\gDQ}[1]{\glqq #1\grqq}
\newcommand{\oBdA}{\mbox{o.\,B.\,d.\,A.\ }}
\newcommand{\sEUR}{\text{\euro}}

%latin abbreviations
\newcommand{\etal}{\mbox{\emph{et al.\ }}}
\newcommand{\exgrat}{\mbox{e.g.\ }}
\newcommand{\idest}{\mbox{i.e.\ }}

%general math terms
\newcommand{\const}{\mathrm{const}}
\newcommand{\bigO}{\mathcal{O}}

% Lorem ipsum
\newcommand*{\QEDA}{\hfill\ensuremath{\blacksquare}}%
\newcommand*{\QEDB}{\hfill\ensuremath{\square}}%

%  ------------------ abbreviations

%matrix operations
\newcommand{\T}{T}
\DeclareMathOperator{\arcsinh}{arcsinh}
\DeclareMathOperator{\Tr}{Tr}
\DeclareMathOperator{\argg}{arg}
\DeclareMathOperator{\Arg}{arg}
\DeclareMathOperator{\codim}{codim}
\DeclareMathOperator{\atanTwo}{atan2}
\DeclareMathOperator{\diag}{diag}

%real and complex numbers latin Letters
\newcommand{\Real}{\mathbb{R}}
\newcommand{\Complex}{\mathbb{C}}
\newcommand{\Integer}{\mathbb{N}}

% differentials 
\newcommand{\dd}{\mathrm{d}}



\title{Physikalische Grundlagen}
\subtitle{HKLS ILV| SS2025}
\author{Dr.\ Dipl.-Ing. Alexander Schumer}
\date{Juli 2025}

% ----------------------------------------------------------------
% ----------------------------------------------------------------
% -------------   BEGIN DOCUMENT    --------------------------
% ----------------------------------------------------------------
% ----------------------------------------------------------------
% \includeonly{chapter_Allgemeines, chapter_Grundlagen}
% \includeonly{chapter_0_titlepage}

% \includeonly{chapter_0_titlepage, chapter_1-Allgemeines, chapter_2-Grundlagen, chapter_3-MaßeinheitenMessverfahren, chapter_4-MessgenauigkeitMessfehler, chapter_App-MathematischeEinschübe}
% \includeonly{chapter_0_titlepage, chapter_2-Grundlagen}
\includeonly{chapter_0_titlepage}

\begin{document}
\begin{titlepage}
    % Logo (optional, but recommended)
    % \hfill \includegraphics[width=0.3\textwidth]{Bilder/Kapitel_Titlepage/LogoFHCampus.png}
    \includegraphics[width=0.3\textwidth]{Bilder/Kapitel_Titlepage/LogoFHCampus.png}
    
    \centering % Center everything on the page

    % Use \vfill to push content vertically
    % Top of page
    \vfill
    \vspace{1.0cm}

    % Title
    {\Huge \bfseries Physikalische Grundlagen\par}
    \vspace{0.5cm}
    {\Large Skriptum\par}
    \vspace{1cm}
    % University Name
    {\large FH Campus Wien\par}

    % Subtitle
    {\Large HKLS ILV | SS2025\par}

    \vfill % Fills the space in the middle of the page

    % Author
    {\large Dr. Dipl.-Ing. Alexander Schumer\par}
    \vspace{1cm}

    % Date
    {\large Juli 2025\par}

    % Bottom of page
    \vfill

\end{titlepage}
\tableofcontents

% \chapter*{Allgemeines}\label{chap: Allgemeines}
\addcontentsline{toc}{chapter}{Allgemeines}

\section*{Das griechische Alphabet}\label{sec: griechisches_Alphabet}
\addcontentsline{toc}{section}{Das griechische Alphabet} 
Das griechische Alphabet wird in den Naturwissenschaften und der Technik häufig für Variablennamen verwendet. Beachten Sie, dass manche griechische Buchstaben zwei Arten der Notation haben. \\
\begin{center}
\definecolor{lightgray}{gray}{0.9}
\rowcolors{1}{white}{lightgray}
\begin{tabular}{l c c @{\quad\quad\vrule\quad\quad} l c c} 
%\hline
\textbf{Alpha} & A & $\alpha$ & \textbf{Ny} & N & $\nu$ \\
\textbf{Beta} & B & $\beta$ & \textbf{Xi} & $\Xi$ & $\xi$ \\
\textbf{Gamma} & $\Gamma$ & $\gamma$ & \textbf{Omikron} & O & $o$ \\
\textbf{Delta} & $\Delta$ & $\delta$ & \textbf{Pi} & $\Pi$ & $\pi$ \\
\textbf{Epsilon} & E & $\epsilon$, $\varepsilon$ & \textbf{Rho} & P & $\rho$ \\
\textbf{Zeta} & Z & $\zeta$ & \textbf{Sigma} & $\Sigma$ & $\sigma$ \\
\textbf{Eta} & H & $\eta$ & \textbf{Tau} & T & $\tau$ \\
\textbf{Theta} & $\Theta$ & $\theta$, $\vartheta$ & \textbf{Ypsilon} & Y & $\upsilon$ \\
\textbf{Iota} & I & $\iota$ & \textbf{Phi} & $\Phi$ & $\phi$, $\varphi$ \\
\textbf{Kappa} & K & $\kappa$ & \textbf{Chi} & X & $\chi$ \\
\textbf{Lambda} & $\Lambda$ & $\lambda$ & \textbf{Psi} & $\Psi$ & $\psi$ \\
\textbf{My} & M & $\mu$ & \textbf{Omega} & $\Omega$ & $\omega$ \\
%\hline 
\end{tabular}
\end{center}

\section*{Mathematische Symbole und Formeln}\label{sec: mathematische_Symbole-Formeln}
\addcontentsline{toc}{section}{Mathematische Symbole und Formeln} 
Um die Sprache der Mathematik zu verstehen, ist es wichtig, die Bedeutung der einzelnen Symbole zu kennen. In der Mathematik haben sich unzählige Symbole und Abkürzungen eingebürgert, die allerdings wie \gDQ{Wörter} zu lesen sind. Die meisten Formeln lesen sich daher wie ein Satz. Die gängigsten Symbole und Formeln, die zum Verständnis dieses Skriptums notwendig sind, sind hier aufgelistet.

\begin{center}
\definecolor{lightgray}{gray}{0.9}
\rowcolors{1}{lightgray}{white} 
\begin{tabular}{c l @{\qquad\vrule\qquad} c l}
$=$ & ist gleich & $\Delta x$ & Differenz \\
$\neq$ & ist ungleich & $dx$ & Differenzial \\
$\equiv$ & äquivalent & $x'$ & 1. Ableitung \\
$\approx$ & ungefähr gleich & $\dot{x}$ & 1. Ableitung nach der Zeit \\
$\propto$ & proportional zu & $\int$ & Integral \\
$\gg$ & viel größer als & $\Sigma$ & Summe \\
$\ll$ & viel kleiner als & $:=$ & Definition \\
$\hat{=}$ & entspricht & $\stackrel{?}{=}$ & Hypothese/Frage \\
$\wedge$ & mathematisches „Und“ & $\stackrel{!}{=}$ & bekannte Gleichheit \\
$\vee$ & mathematisches „Oder“ & $\infty$ & unendlich \\
\oBdA & ohne Beschränkung der Allgemeinheit & & \\
\end{tabular}
\end{center}
Der mathematische Begriff \textbf{ohne Beschränkung der Allgemeinheit}, oft mit \oBdA abgekürzt, ist eine gängige Formulierung in Beweisen. Sie signalisiert, dass der Beweis für einen speziellen Fall geführt wird, dieser aber so gewählt ist, dass er alle anderen möglichen Fälle repräsentiert. Die Gültigkeit des Beweises für diesen einen Fall überträgt sich somit auf alle anderen Fälle, ohne dass die Allgemeingültigkeit der Aussage eingeschränkt wird.

\section*{Häufig vorkommende Begriffe in der Physik}\label{sec: häufige_Begriffe_Physik}
\addcontentsline{toc}{section}{Physikalische Begriffe} 
Manchmal kommt es vor, dass man spezielle (physikalische) Begriffe zwar schon häufig gehört hat, deren Bedeutung aber nicht gänzlich verstanden wird. Diese Liste dient als Nachschlagewerk, um Begriffe, die nicht im allgemeinen Sprachgebrauch vorkommen, zu definieren. 
\vspace{0.3cm}
\begin{center}
\definecolor{lightgray}{gray}{0.9}
\rowcolors{2}{white}{lightgray} % Ab der zweiten Zeile abwechselnde Farben
\begin{tabularx}{0.9\textwidth}{l X} 
\toprule
\textbf{Begriff} & \textbf{Erklärung} \\
\midrule
quantitativ & zahlenmäßig \\
qualitativ & beschreibend, interpretierend \\
monochromatisch & aus einer einzelnen Wellenlänge oder Frequenz bestehend \\
homogen & einheitlich; gleichartig aufgebaut/zusammengesetzt \\
heterogen & uneinheitlich; nicht gleichartig aufgebaut/zusammengesetzt \\
Abszisse & „x-Achse“ \\
Ordinate & „y-Achse“ \\
normal, orthogonal & rechtwinklig \\
orthonormal & orthogonal und normiert \\
normiert & Länge ist 1 (bei Vektoren) \\
isotrop & unabhängig von der Richtung \\
isotherm & gleiche Temperatur \\
isobar & gleicher Druck \\
isochor & gleiches Volumen \\
isentrop & gleiche Entropie \\
Isotop & gleiche Protonenzahl \\
\bottomrule
\end{tabularx}
\end{center}

\newpage
\section*{Physikalische Größen}\label{sec: Physikalische_Größen}
\addcontentsline{toc}{section}{Physikalische Größen} 
Die folgende Tabelle stellt die gängigsten physikalischen Größen, die in diesem Skriptum vorkommen, übersichtlich dar. Die Basiseinheiten sind allesamt SI-Einheiten. Für diejenigen Größen, für die kein Einheitensymbol angegeben ist, verwendet man üblicherweise die Basiseinheiten. 
\begin{center}
\definecolor{lightgray}{gray}{0.9}
\rowcolors{2}{white}{lightgray}
\begin{tabularx}{\textwidth}{l c c c c}
\toprule
% \textbf{Größe} & \textbf{Einheit} & \textbf{\shortstack{Einheiten-\\symbol}} & \textbf{\shortstack{Basis-\\einheiten}} & \textbf{\shortstack{Formel-\\symbol}} \\
\thead{Größe} & \thead{Einheit} & \thead{Einheiten-\\symbol} & \thead{Basis-\\einheiten} & \thead{Formel-\\symbol} \\
\midrule
Zeit & Sekunde & \si{\second} & \si{\second} & $t$ \\
Ort, Weg, Länge & Meter & \si{\meter} & \si{\meter} & $\vec{s}$ \\
Masse & Kilogramm & \si{\kilogram} & \si{\kilogram} & $m$ \\
Trägheitsmoment & & & \si{\kilogram\,\meter^2} & $I, J$ \\
Temperatur & Kelvin & \si{\kelvin} & \si{\kelvin} & $T$ \\
Winkel & Radiant & \si{\radian} & 1 & $\alpha, \beta, \gamma, \varphi, \theta, \dots$ \\
Raumwinkel & Steradiant & \si{\steradian} & 1 & $\Omega$ \\
Geschwindigkeit & & & \si{\meter\per\second} & $\vec{v}$ \\
Beschleunigung & & & \si{\meter\per\second^2} & $\vec{a}$ \\
Winkelgeschwindigkeit & & & \si{\radian\per\second} & $\vec{\omega}$ \\
Winkelbeschleunigung & & & \si{\radian\per\second^2} & $\vec{\alpha}$ \\
Impuls & & & \si{\kilogram\meter\per\second} & $\vec{p}$ \\
Kraft & Newton & \si{\newton} & \si{\kilogram\meter\per\second^2} & $\vec{F}$ \\
Drehimpuls & & & \si{\kilogram\meter^2\per\second} & $\vec{L}$ \\
Drehmoment & & \si{\newton\,\meter} & \si{\kilogram\meter^2\per\second^2} & $\vec{M}$ \\
Wirkung & & \si{\joule\second} & \si{\kilogram\meter^2\per\second} & $S$ \\
Energie, Arbeit & Joule & \si{\joule} & \si{\kilogram\meter^2\per\second^2} & $E, W$ \\
Leistung & Watt & \si{\watt} & \si{\kilogram\meter^2\per\second^3} & $P$ \\
Frequenz & Hertz & \si{\hertz} & \si{\per\second} & $f, \nu$ \\
Wellenlänge & & & \si{\meter} & $\lambda$ \\
Lichtgeschwindigkeit & & & \si{\meter\per\second} & $c$ \\
Druck & Pascal & \si{\pascal} & \si{\kilogram\per(\meter\,\second^2)} & $p$ \\
Dichte & & & \si{\kilogram\per\meter^3} & $\rho$ \\
Volumen & & & \si{\meter^3} & $V$\\
Spannung & Volt & \si{\volt} & \si{\kilogram\meter^2\per(\second^3\,\ampere)} & $U$ \\
Stromstärke & Ampere & \si{\ampere} & \si{\ampere} & $I$ \\
Elektrischer Widerstand & Ohm & \si{\ohm} & \si{\kilogram\meter^2\per(\second^3\,\ampere^2)} & $R$ \\
Ladung & Coulomb & \si{\coulomb} & \si{\ampere\,\second} & $Q, q$ \\
\bottomrule
\end{tabularx}
\end{center}
\newpage


\section*{Einheitenpräfixe}\label{sec: Einheitenpräfixe}
\addcontentsline{toc}{section}{Einheitenpräfixe} 
Einheitenpräfixe, auch Vorsilben genannt, vereinfachen den Umgang mit (sehr) großen oder (sehr) kleinen Zahlenwerten. Ihre Aufgabe ist es, eine Basiseinheit (wie Meter, Gramm oder Sekunde) mit einer festen Zehnerpotenz zu skalieren (multiplizieren).

Man verwendet sie, um lange Ziffernfolgen zu vermeiden und Zahlen lesbarer und verständlicher zu machen. Anstatt zum Beispiel $\SI{1000}{\meter}$ zu schreiben, verwendet man das Präfix \textit{Kilo} und schreibt \SI{1}{\kilo\meter}. Genauso ist es bei sehr kleinen Werten praktischer, \SI{1}{\nano\meter} (Nanometer) statt \SI{0,000000001}{\meter} zu schreiben. Diese kompakte Schreibweise reduziert die Fehleranfälligkeit und erleichtert das alltägliche Arbeiten mit physikalischen Größen erheblich.

% \begin{center}
% \definecolor{lightgray}{gray}{0.9}
% \rowcolors{2}{white}{lightgray} % Abwechselnde Farben ab der 2. Datenzeile
% \begin{tabular}{l l l @{\qquad\vrule\qquad} l l l}
% \toprule
% \textbf{Präfix} & \textbf{Symbol} & \textbf{Faktor} & \textbf{Präfix} & \textbf{Symbol} & \textbf{Faktor} \\
% \midrule
% atto   & a      & $10^{-18}$ & exa    & E  & $10^{18}$ \\
% femto  & f      & $10^{-15}$ & peta   & P  & $10^{15}$ \\
% pico   & p      & $10^{-12}$ & tera   & T  & $10^{12}$ \\
% nano   & n      & $10^{-9}$  & giga   & G  & $10^{9}$  \\
% mikro  & \textmu & $10^{-6}$  & mega   & M  & $10^{6}$  \\
% milli  & m      & $10^{-3}$  & kilo   & k  & $10^{3}$  \\
% zenti  & c      & $10^{-2}$  & hekto  & h  & $10^{2}$  \\
% dezi   & d      & $10^{-1}$  & deka   & da & $10^{1}$  \\
% \bottomrule
% \end{tabular}
% \end{center}

\begin{center}
% Die Farbe wird wie gewohnt definiert
\definecolor{lightgray}{gray}{0.9}

% Verwende die NiceTabular-Umgebung
\begin{NiceTabular}{l c c @{\qquad\vrule\qquad} l c c}
\CodeBefore
  \rowcolors{2}{white}{lightgray}
\Body
\toprule
\textbf{Präfix} & \textbf{Symbol} & \textbf{Faktor} & \textbf{Präfix} & \textbf{Symbol} & \textbf{Faktor} \\
\midrule
atto    & a       & $10^{-18}$ & exa    & E  & $10^{18}$ \\
femto   & f       & $10^{-15}$ & peta   & P  & $10^{15}$ \\
pico    & p       & $10^{-12}$ & tera   & T  & $10^{12}$ \\
nano    & n       & $10^{-9}$  & giga   & G  & $10^{9}$  \\
mikro   & \textmu & $10^{-6}$  & mega   & M  & $10^{6}$  \\
milli   & m       & $10^{-3}$  & kilo   & k  & $10^{3}$  \\
zenti   & c       & $10^{-2}$  & hekto  & h  & $10^{2}$  \\
dezi    & d       & $10^{-1}$  & deka   & da & $10^{1}$  \\
\bottomrule
\end{NiceTabular}
\end{center}

\vspace{1em} % Ein kleiner vertikaler Abstand

Jede dieser Vorsilben kann mit beliebigen Einheiten kombiniert werden. Das Präfix steht dabei immer direkt vor der Einheit, die skaliert werden soll: Die Einheit $\si{\kilo\gram\meter\second^{-1}}$ ist demnach nicht dasselbe wie $\si{\gram\kilo\meter\second^{-1}}$.

\subsubsection*{Beispiele:}
\begin{multicols}{3}
\begin{itemize}
    \item \si{\kilo\gram}
    \item \si{\hecto\liter}
    \item \si{\mega\hertz}
    \item \si{\kilo\watt\hour}
    \item \si{\giga\byte}
    \item \si{\nano\second}
    \item \si{\micro\meter}
    \item \si{\milli\farad}
    \item \si{\kilo\ohm}
    \item \si{\peta\joule}
\end{itemize}
\end{multicols}
% \chapter{Grundlagen}\label{chap: Grundlagen}
Die Physik ist die Naturwissenschaft, die sich mit den Bausteinen der uns umgebenden Welt und deren gegenseitigen Wechselwirkungen beschäftigt. Es soll ein grundlegendes Verständnis auch komplizierter Körper aus ihrem Aufbau aus elementaren Teilchen und deren Wechselwirkungen geschaffen werden.\newline
Komplexe Naturvorgänge sollen auf einfache Gesetzmäßigkeiten zurückgeführt, quantifiziert und voraussagbar werden.
\begin{examplebox}[sidebyside, sidebyside align=top seam, lower separated=false, righthand width=7.5cm]{Beispiel}
    Radioaktivität
    \begin{itemize}
        \item Spontaner Zerfall der Atomkerne führt zu Emission von Teilchen oder Strahlung
        \item Quantifizierung dieses Prozesses über Zerfallsreihen 
        \item Voraussagbar über Halbwertszeit
    \end{itemize}
     \tcblower
     \begin{center}
     \includegraphics[width=0.65\linewidth]{Bilder//Kapitel_Grundlagen/Zerfallsreihe_Uran235.png}
     \end{center}
     Die Zerfallsreihe von $^{235}\text{U}$. $Z$ ist die Anzahl der Protonen, $N$ die Anzahl der Neutronen und $A$ die Gesamtzahl der Kernteilchen.
\end{examplebox}

\section{Das Experiment}\label{sec: das_Experiment}
\begin{rememberbox}[]{}
    \textbf{Das Experiment} ist eine gezielte Frage an die Natur, auf die bei geeigneter experimenteller Anordnung eine eindeutige Antwort erhalten werden kann.
\end{rememberbox}
Das Ziel des Experimentes ist es einen Naturvorgang in einem kontrollierbaren und beliebig oft wiederholbaren Vorgang zu untersuchen. Die Bedingungen sind dabei strikt vorgegeben. 
Dabei wird besonderen Wert auf die Eliminierung aller störenden Einflüsse gelegt, die den untersuchten Effekt überlagern könnten. Im Sinne der Physik sollen damit Gesetzmäßigkeiten gefunden werden und eine Fülle von Beobachtungen in physikalische Gesetze überführt werden. 


\begin{examplebox}[sidebyside, sidebyside align=top seam, lower separated=false, righthand width=4.5cm]{Beispiel}
    Pendel-Experiment zur Bestimmung der Gravitationsbeschleunigung $\vec{g}$.
    
\textbf{Hypothese:} Die Schwingungsdauer eines Pendels hängt von der Länge des Pendels und der Erdbeschleunigung ab, aber nicht von der Masse des Pendels.

\textbf{Störende Einflüsse:} Das Experiment wird in einem Raum ohne Luftzug durchgeführt. Die Startauslenkung kann mit bekannter Genauigkeit eingestellt werden.

\textbf{Reproduzierbarkeit:} Was muss dokumentiert werden, damit andere Forschende zu denselben Ergebnissen kommen? 
     \tcblower
     \begin{center}
     \includegraphics[width=0.95\linewidth]{Bilder//Kapitel_Grundlagen/Pendel_Experiment.png}
     \end{center}
\end{examplebox}


\section{Bausteine der wissenschaftlichen Erkenntnis}\label{sec: Wissenschaftliche_Erkenntnis}
In den Wissenschaften, insbesondere in der Mathematik und Physik, wird Wissen nicht willkürlich gesammelt, sondern systematisch aufgebaut. Die Basis dieses Gebäudes bilden verschiedene Arten von Aussagen und Annahmen, die sich in ihrem Status, ihrer Funktion und ihrem Grad an Sicherheit fundamental unterscheiden. Von der ersten vagen Vermutung bis hin zum umfassenden Erklärungsmodell durchläuft eine wissenschaftliche Idee verschiedene Stufen der Prüfung und Etablierung. Die Grundbegriffe der Wissenschaftstheorie – Hypothese, Axiom, Satz, Gesetz und Theorie – sind die wesentlichen Werkzeuge, um die Welt zu beschreiben, zu verstehen und sich der Wahrheit schrittweise anzunähern. Sie bilden eine Hierarchie des Wissens, die von vorläufigen Annahmen bis zu in sich geschlossenen, widerspruchsfreien Gedankengebäuden reicht.
\subsection{Hypothese}\label{subsec: Hypothese}
Eine \textit{Hypothese} (Unterstellung, Annahme) ist der wissenschaftliche Ausgangspunkt für einen Erkenntnisgewinn. Sie ist eine vorläufige, testbare Annahme oder eine Vermutung, die ein Phänomen zu erklären versucht. Sie muss so formuliert sein, dass sie durch Beobachtungen oder Experimente prinzipiell widerlegt (falsifiziert) werden kann. Eine Hypothese wartet auf ihre Überprüfung.
\begin{rememberbox}[]{Hypothese}
    Eine \textbf{Hypothese} ist eine vorläufige Annahme oder ein Erklärungsversuch, die getestet werden soll. Sie ist eine Aussage, die durch Experimente oder Beobachtungen überprüft wird. 
\end{rememberbox}

\textit{Beispiele:}\newline
Pflanzen wachsen bei Bestrahlung mit rotem Licht schneller als bei Bestrahlung mit blauem Licht.\newline
Wenn man Salz in Wasser auflöst, dann erhöht sich der Siedepunkt des Wassers. 


\subsection{Axiom}\label{subsec: Axiom}
\textit{Axiome} sind die unbewiesenen Fundamente einer Theorie. Es sind grundlegende Annahmen oder Prinzipien, die innerhalb eines Systems als wahr, allgemein akzeptiert oder offensichtlich angesehen werden. Man kann sie nicht aus anderen Sätzen ableiten, sondern sie dienen als Basis, um weitere Sätze (Theoreme) logisch zu beweisen.
\begin{rememberbox}[]{Axiom}
    \textbf{Axiome} sind grundlegende Annahmen oder Prinzipien, die als bekannt, allgemein akzeptiert oder eindeutig wahr aufgefasst werden. Sie dienen als Basis für weitere Theorien und Beweise. 
\end{rememberbox}

\textit{Beispiele:} \newline
Durch zwei Punkte kann genau eine Gerade gelegt werden.\newline
Zwei parallele Linien schneiden sich nie.


\subsection{Satz (Theorem)}\label{subsec: Theorem}
Ein Satz, in der Wissenschaft oft als Theorem bezeichnet, ist eine logische Schlussfolgerung, die aus Axiomen und bereits bewiesenen Sätzen abgeleitet wird. Ein Satz muss erst streng bewiesen werden, bevor er als wahr akzeptiert wird. Er ist das Ergebnis eines logisch-deduktiven Prozesses.
\begin{rememberbox}[]{Satz}
    \textbf{Sätze} sind in der Mathematik und Logik Aussagen, die aus Axiomen und anderen bewiesenen Sätzen abgeleitet werden. Ein Satz muss erst bewiesen werden, um als wahr akzeptiert zu werden.
\end{rememberbox}

\textit{Beispiele:} \newline 
[Satz von Thales] Alle von einem Halbkreis umschriebenen Dreiecke sind rechtwinklig. \newline
[Satz von Fermat] Für $n > 2$ gibt es keine positiven ganzen Zahlen $a, b, c$ die die Gleichung $a^n + b^n = c^n$ erfüllen.


\subsection{Gesetz}\label{subsec: Gesetz}
Ein physikalisches Gesetz beschreibt eine grundlegende und regelmäßig wiederkehrende Beziehung in der Natur unter bestimmten Bedingungen. Es verknüpft messbare Größen miteinander, oft in Form einer präzisen mathematischen Gleichung. Ein Gesetz beschreibt, was passiert, liefert aber nicht zwingend eine Erklärung dafür, warum es passiert.
\begin{rememberbox}[]{Gesetz}
    Als \textbf{Gesetze} bezeichnet man Aussagen, die bestimmte Phänomene unter bestimmten Bedingungen beschreiben (meist durch Formeln). Anders als Theorien beschreiben Gesetze meist nur und liefern keine Erklärungen.
\end{rememberbox}

\textit{Beispiele:} \newline
[Ohm'sche Gesetz] ($U = R\cdot I$) beschreibt den Zusammenhang zwischen Spannung, Widerstand und Stromstärke, ohne die mikroskopischen Ursachen des elektrischen Widerstands zu erklären.\newline
[Gravitationsgesetz] Jede Masse zieht jede andere Masse mit einer Kraft an, die direkt proportional zum Produkt ihrer Massen und umgekehrt proportional zum Quadrat des Abstands zwischen ihren Schwerpunkten ist, $F = G \frac{M_1 \cdot M_2}{r^2}$.


\subsection{Theorie}\label{subsec: Theorie}
Eine physikalische Theorie ist die höchste Stufe wissenschaftlicher Erklärung. Sie ist ein umfassendes, in sich widerspruchsfreies System, das mehrere Gesetze und Prinzipien zusammenfasst. Eine Theorie liefert eine tiefgehende Erklärung für eine ganze Reihe von Naturvorgängen und erklärt beispielsweise die Vorgänge hinter den Gesetzen. Der Gültigkeitsbereich einer Theorie wird durch Experimente ständig überprüft, bestätigt und bei Bedarf eingeschränkt oder erweitert. Eine wissenschaftliche Theorie ist das bestmögliche Erklärungsmodell, das es gibt und unterscheidet sich daher radikal von der missbräuchlichen Verwendung im allgemeinen Sprachgebrauch -- hier wird ein unbewiesener Gedanke oder eine Hypothese oftmals als Theorie bezeichnet.
\begin{rememberbox}[]{Theorie}
    Eine physikalische \textbf{Theorie} ist die Zusammenfassung mehrerer physikalischer Gesetze und Prinzipien zu einem geschlossenen und in sich widerspruchsfreien Aufbau. Sie liefert eine Erklärung für Naturvorgänge. 
Der Gültigkeitsbereich einer physikalischen Theorie wird durch Experimente geprüft und eingeschränkt.

\end{rememberbox}

\textit{Beispiel:} \newline
Die Relativitätstheorie von Albert Einstein ist eine umfassende Theorie, die das Gesetz der Schwerkraft und die Bewegung von Objekten bei hohen Geschwindigkeiten erklärt und dabei auf grundlegenden Prinzipien aufbaut.

% Chapter end - always start new page after chapter
\newpage
%\chapter{Maßeinheiten und Messverfahren}\label{chap: MasseinheitenMessverfahren}
Eine objektive Naturbeschreibung soll durch quantitative (zahlenmäßige) Zusammenhänge dargestellt werden. Dies wird durch das Messen physikalischer Größen erreicht. Dies bedarf Einigkeit über die Maßeinheit von physikalischen Größen. Grundlage dafür ist ein international einheitliches System von Maßeinheiten, das \textit{Internationale Einheitensystem} (SI).
\subsubsection{Was ist eine Messung?}
Eine Messung ist fundamental ein Vergleich. Man vergleicht die zu messende Eigenschaft eines Objekts (\zB die Länge eines Tisches) mit einer bekannten Größe derselben Art (\zB dem Urmeter). Das Ergebnis der Messung, der Messwert, gibt an, wie oft die Maßeinheit in der zu messenden Größe enthalten ist.
\begin{importantbox}[]{}
    Eine Messung ist immer ein Vergleich zweier Größen. Jede Messung besteht aus zwei Teilen: dem \textbf{Zahlenwert} und der \textbf{Einheit} (Maßeinheit).
\end{importantbox}
Um die Wiederholbarkeit einer Messung zu garantieren, definiert man Normale (Standards), mit denen eine physikalische Größe verglichen werden sol, wie beispielsweise das Urmeter. Da nicht jede Messung mit diesem ausgezeichneten Normal verglichen werden kann, werden Kopien angefertigt, die als Messstandards ausgegeben werden. Damit Messgeräte vergleichbar und korrekt eingestellt sind, müssen Messgeräte regelmäßig überprüft werden. Hierbei unterscheidet man drei wesentliche Verfahren:

\begin{rememberbox}[]{}
Der Vergleich mit dem Normal ist die \textbf{Eichung}. Die Eichung entspricht einer Qualitätsprüfung des Messgeräts (keine Manipulation des Geräts). 

Die \textbf{Kalibrierung} ist die Ermittlung des Zusammenhangs zwischen Ausgabewerten eines Messgeräts und den zugehörigen Werten des Normals. 

Die \textbf{Justierung} ist ein Eingriff am Messgerät, bei dem es auf einen Sollwert eingestellt wird.
\end{rememberbox}
Eine Eichung ist eine (meist rechtlich verpflichtende) Feststellung der Abweichung des Messwerts vom gültigen Normal (Standard). Bei der Kalibrierung wird der Zusammenhang zwischen den angezeigten Messwerten und den Sollwerten (des Normals) eruiert. Schließlich wird ein Messgerät mittels Justierung bestmöglich auf die Sollwerte eingestellt. \\

Jede physikalische Größe kann prinzipiell nicht genauer gemessen werden, als das entsprechende Normal definiert ist. \\
Es stellt sich nun die Frage, wie viele Grundgrößen und damit Normale braucht man in der Physik? 

\section{SI-Basiseinheiten}\label{sec: SI-Basiseinheiten}
Die gesamte moderne Physik und Technik basiert auf einem System von nur sieben fundamentalen Grundgrößen, den sogenannten SI-Basiseinheiten (Le Système International d'Unités). Alle anderen physikalischen Einheiten (wie Geschwindigkeit, Kraft oder Energie) können von diesen sieben abgeleitet werden.
\begin{rememberbox}[]{}
Es gibt 7 Basiseinheiten und daraus leitet man alle anderen Einheiten durch Multiplikation oder Division ab.
\end{rememberbox}

\begin{center}
\begin{NiceTabularX}{\textwidth}{X c c c} % NiceTabularX verwenden
\CodeBefore
  \rowcolor{boxcol_title_blue}{1} % Färbt die erste Zeile
  \rowcolors{2}{white}{boxcol_back_lblue} 
\Body
\toprule
\textbf{Größe} & \textbf{Einheit} & \textbf{Einheitensymbol} & \textbf{Formelsymbol} \\
\midrule
Zeit & Sekunde & \si{\second} & $t$ \\
Länge & Meter & \si{\meter} & $\ivec{s}$ \\
Masse & Kilogramm & \si{\kilogram} & $m$ \\
elektrische Stromstärke & Ampere & \si{\ampere} & $I$ \\
Temperatur & Kelvin & \si{\kelvin} & $T$ \\
Stoffmenge & Mol & \si{\mol} & $n$ \\
Lichtstärke & Candela & \si{\candela} & $I_V$ \\
\bottomrule
\end{NiceTabularX}
\end{center}
\vspace{1cm}

\subsection{Zeit}\label{subsec: SI-Zeit}
Im Alltag werden oft größere, von der Sekunde abgeleitete Einheiten wie die Minute oder die Stunde verwendet. Die moderne Definition der Sekunde über die Schwingungen einer Atomuhr wurde eingeführt, da die frühere astronomische Definition, die auf der Erdrotation basierte, für wissenschaftliche Zwecke nicht ausreichend konstant und präzise ist.
\begin{importantbox}[]{Sekunde}
Die Maßeinheit der \textbf{Zeit} ist die \textbf{Sekunde}. Das Formelzeichen der Zeit ist $t$ und das Einheitszeichen der Sekunde ist $\si{\second}$. 
\tcblower
Eine Sekunde ist über einen Hyperfeinstrukturübergang\footnotemark{} des $^{133}\text{Cs}$ Atoms definiert. Indem man die Frequenz dieses Übergangs mit $\nu = \SI{9 162 631 770}{\hertz}$ definiert, ist eine Sekunde die Zeitdauer von $\num{9 162 631 770}$ Schwingungen dieser Strahlung.  
\end{importantbox}
\footnotetext{Neben den Hauptquantenzahlen und der Feinstrukturaufspaltung gibt es noch die Hyperfeinstrukturaufspaltung. Sie entsteht durch die Kopplung des magnetischen Moments des Atomkerns $\mu_I$ mit dem Magnetfeld der Elektronen $B_J$.}

\subsection{Länge}\label{subsec: SI-Länge}
Die ursprüngliche Definition des Meters aus dem 18. Jahrhundert war als der zehnmillionste Teil der Entfernung vom Nordpol zum Äquator festgelegt, was die enge historische Verbindung zur Vermessung der Erde verdeutlicht. Heute ist die hochpräzise Längenmessung, etwa mittels Laser-Interferometrie, eine direkte technologische Anwendung seiner Definition über die Lichtgeschwindigkeit.
\begin{importantbox}[]{Meter}
Die Maßeinheit der \textbf{Länge} ist der \textbf{Meter}. Das Formelzeichen der Länge ist $s$ und das Einheitszeichen des Meters ist $\si{\meter}$. 
\tcblower
Indem die Lichtgeschwindigkeit im Vakuum auf $c = \SI{299 792 458}{\meter\per\second}$ festgelegt wurde\footnotemark{}, ist ein Meter genau die Länge, die Licht in $(1/\num{299 792 458})\,\si{\second}$ zurücklegt.
\end{importantbox}
\footnotetext{Das ist höchst praktikabel, da aus der Speziellen Relativitätstheorie nur hervorgeht, dass es eine konstante Lichtgeschwindigkeit im Vakuum geben muss. Ihr Zahlenwert ist aber frei wählbar.}

\subsection{Masse}\label{subsec: SI-Masse}
Es ist wichtig, die Masse als eine intrinsische Eigenschaft von Materie vom Gewicht zu unterscheiden, welches die Kraft darstellt, die durch die Gravitation auf eine Masse wirkt $F_G = m\cdot g$. Die Neudefinition von 2018 war notwendig, da sich die Masse des physischen Urkilogramms über die Zeit im Vergleich zu den Kopien veränderte und somit keine stabile Referenz mehr darstellte.
\begin{importantbox}[]{Kilogramm}
Die Maßeinheit der \textbf{Masse} ist das \textbf{Kilogramm}. Das Formelzeichen der Masse ist $m$ und das Einheitszeichen des Kilogramms ist $\si{\kilo\gram}$. 
\tcblower
Seit 2018 ist die Masse über den Zahlenwert des Planckschen Wirkungsquantums $h = \num{6.62607015e-34}$ festgelegt\footnotemark{}. Die Einheit von $h$ ist $[h] = \si{\kilo\gram \meter^2 \second^{-1}}$.
\end{importantbox}
\footnotetext{Seit 1799 (\bzw 1889) war die Masse über das Urkilogramm definiert: Ein Platin-Iridium Zylinder dessen Masse mit $\SI{1}{\kilo\gram}$ definiert wurde.}

\subsection{Stoffmenge}\label{subsec: SI-Stoffmenge}
Das Mol dient in der Chemie als unverzichtbare Brücke, um die unsichtbare mikroskopische Welt der Atome und Moleküle mit der makroskopischen Welt zu verbinden, die wir im Labor wiegen können. Es erlaubt Chemikern, Reaktionsgleichungen nicht nur qualitativ, sondern auch quantitativ zu beschreiben und umzusetzen.
\begin{importantbox}[]{Mol}
Die Maßeinheit der \textbf{Stoffmenge} ist das \textbf{Mol}. Das Formelzeichen der Stoffmenge ist $n$ und das Einheitszeichen ist \si{\mol}.
\tcblower
Ein Mol enthält genau \num{6.02214076e23} Einzelteilchen. Der Zahlenwert entspricht der Avogadro-Konstante $N_A$, welche die Einheit $\si{\mol^{-1}}$ hat.\footnotemark{}
\end{importantbox}
\footnotetext{Die Definition hat sich ebenfalls 2018 geändert. Davor war $\SI{1}{\mol}$ die Stoffmenge, die aus ebenso vielen Teilchen besteht, wie Atome in \SI{0.012}{\kilo\gram} $^{12}\text{C}$ enthalten sind.}
Die Anzahl der Teilchen $N$ in $n$ Mole ist $N = n \cdot N_A$.

\subsection{Elektrische Stromstärke}\label{subsec: SI-Stromstärke}
Ein Ampere beschreibt die Flussrate elektrischer Ladung -- es entspricht einem Fluss von etwa $\num{6.24e18}$ Elementarladungen (\zB Elektronen) pro Sekunde durch den Querschnitt eines Leiters. Diese Einheit ist die Grundlage für viele andere elektrische Maßeinheiten wie Volt oder Ohm.
\begin{importantbox}[]{Ampere}
Die Maßeinheit der \textbf{elektrischen Stromstärke} ist das \textbf{Ampere}. Das Formelzeichen der Stromstärke ist $I$ und das Einheitszeichen von Ampere ist \si{\ampere}.
\tcblower
Seit 2018 ist das Ampere über den Zahlenwert der Elementarladung $e = \num{1.602176634e-19}$ festgelegt.\footnotemark{} Die Einheit von $e$ ist eigentlich Coulomb \si{\coulomb}, dies entspricht aber $\SI{1}{\coulomb} = \SI{1}{\ampere\second}$.
\end{importantbox}
\footnotetext{Vor der Neudefinition war das Ampere höchst umständlich über die Kraft zwischen zwei stromdurchflossenen, unendlich langen dünnen Leitern definiert.}

\subsection{Temperatur}\label{subsec: SI-Temperatur}
Die Kelvin-Skala ist eine absolute Temperaturskala, die am absoluten Nullpunkt ($\SI{0}{\kelvin}$) beginnt – der theoretisch kältesten möglichen Temperatur, bei der keine thermische Bewegung der Teilchen mehr stattfindet. Eine Temperaturdifferenz von einem Kelvin entspricht exakt der von einem Grad Celsius, jedoch sind die Nullpunkte verschoben, $\SI{0}{\celsius} = \SI{273.15}{\kelvin}$.
\begin{importantbox}[]{Kelvin}
Die Maßeinheit der \textbf{Temperatur} ist das \textbf{Kelvin}. Das Formelzeichen der Temperatur ist $T$ und das Einheitszeichen ist \si{\kelvin}.
\tcblower
Die Temperatur ist ebenfalls seit 2018 über den Zahlenwert einer Naturkonstante definiert. Der Zahlenwert der Boltzmann-Konstanten wurde mit $k = \num{1.380649e-23}$ festgelegt.\footnotemark{} Die Einheit von $k$ ist $[k] = \si{\kilogram\meter^2\second^{-2}\kelvin^{-1}}$.
\end{importantbox}
\footnotetext{Von 1954 an war die Temperatur über den Tripelpunkt des Wassers definiert: \SI{1}{\kelvin} ist der $273,16$te Teil der thermodynamischen Temperatur des Tripelpunktes von Wasser.}

\subsection{Lichtstärke}\label{subsec: SI-Lichtstärke}
Die Candela ist eine photometrische Größe, das heißt, sie berücksichtigt die unterschiedliche Helligkeitsempfindlichkeit des menschlichen Auges für verschiedene Farben. Unser Auge ist für grünes Licht, nahe der Frequenz in der Definition, am empfindlichsten, weshalb eine grüne Lichtquelle bei gleicher physikalischer Leistung als heller wahrgenommen wird als etwa eine rote oder blaue.
\begin{importantbox}[]{Candela}
Die Maßeinheit der \textbf{Lichtstärke} ist die \textbf{Candela}. Das Formelzeichen der Lichtstärke ist $I_v$ und das Einheitszeichen von Candela ist \si{\candela}.
\tcblower
Die Candela ist gleich der Lichtstärke einer Strahlungsquelle in einer gegebenen Richtung, welche eine monochromatische Strahlung mit einer Frequenz von $\SI{540e12}{\hertz}$ aussendet und deren Strahlstärke $\frac{\SI{1}{\watt}}{\SI{683}{\steradian}}$ in dieser Richtung beträgt.\footnotemark{}
\end{importantbox}
\footnotetext{Das \textbf{Steradiant} ist die Einheit des Raumwinkels.}


\section{Winkeleinheiten}\label{sec: Winkeleinheiten}
Die beiden folgenden Winkeleinheiten sind keine der $7$ Basisgrößen, aber deren Verwendungen hat sich in den Naturwissenschaften universell durchgesetzt, dass sie hier behandelt werden sollen.
\subsection{Ebener Winkel}\label{subsec: ebener_Winkel}
\begin{figure}[h!]
    \centering
    \includegraphics[width=0.3\textwidth]{Bilder/Kapitel_Grundlagen/radiant_Circle.png} 
    \caption[Bogenlänge eines Kreissektors]{Die Bogenlänge $L$ eines Kreissektors mit Radius $r$ und Winkel $\alpha$.}\label{fig: bogenlaenge_kreissektor}
\end{figure}
Der \textbf{Radiant} ist ein Winkelmaß, bei dem der Winkel durch die entsprechende Länge des Kreisbogens im Einheitskreis angegeben wird. Die Bogenlänge eines Kreissektors ist $L = r \cdot \alpha$.
\begin{importantbox}[]{}
    $\SI{1}{\radian}$ ist jener Winkel, der im Einheitskreis ($r = \SI{1}{\meter}$) eine Bogenlänge von $\SI{1}{\meter}$ ergibt: $\alpha = L/r$. 
\end{importantbox}

Das Formelzeichen eines ebenen Winkels ist meist ein griechischer Buchstabe $\alpha, \beta, \gamma, \ldots$ und das Einheitszeichen ist $\si{\radian}$. Radiant ist \textbf{einheitenlos}, hat aber dennoch ein Einheitszeichen.

Die Umrechnung von Radiant auf Grad erfolgt über die Bogenlänge des gesamten Einheitskreises:
\begin{equation*}\begin{aligned}
    2\pi\,\si{\radian} &= \ang{360} \\
    \Rightarrow \SI{1}{\radian} &= \frac{\ang{360}}{2\pi} \approx 57,2958^\circ
\end{aligned}\end{equation*}

\subsection{Raumwinkel}\label{subsec: Raumwinkel}
\begin{figure}[h!]
    \centering
    \includegraphics[width=0.3\textwidth]{Bilder/Kapitel_Grundlagen/steradiant_sphere.png} 
    \caption[Raumwinkel (Steradiant) -- Verhältnis der ausgeschnittenen Fläche zur Kugeloberfläche (Quelle:~\protect\cite{WikiRaumwinkel})]{Der Kegel mit Öffnungswinkel $\Omega$ schneidet aus der Kugeloberfläche (Radius $r$) die Fläche $A$. (Quelle:~\cite{WikiRaumwinkel})}\label{fig: steradiant_kugeloberflaeche}
\end{figure}
Der Steradiant ist die Maßeinheit für den Raumwinkel. Unter dem Raumwinkel $\Omega$ versteht man das Verhältnis $\Omega = A/r^2$, wobei $r$ der Kugelradius und $A$ der Teil der Kugeloberfläche ist, der von einem Kegel mit Öffnungswinkel $\Omega$ und der Spitze im Mittelpunkt der Kugel ausgeschnitten wird.

\begin{importantbox}[]{}
    $\SI{1}{\steradian}$ ist jener Raumwinkel (eines Kegels), der auf der Oberfläche der Einheitskugel ($r = \SI{1}{\meter}$) eine Fläche von $\SI{1}{\meter^2}$ ausschneidet: $\Omega = A/r^2$. 
\end{importantbox}

Das Formelzeichen des Raumwinkels ist $\Omega$, das Einheitszeichen ist \si{\steradian}.

Steradiant ist ebenfalls \textbf{einheitenlos} hat aber dennoch ein Einheitszeichen. Die Oberfläche einer Kugel ist $S = 4\pi r^2$. Durch das Bilden des Verhältnisses $\Omega = A/r^2$ wird der Raumwinkel einheitenlos und unabhängig vom Radius der Kugel. 


\section{Darstellung von Zahlenwerten und Einheiten}\label{sec: Zahlenwerte_Einheiten} 
\begin{itemize}
    \item \textbf{Variablen und Formelsymbole} werden in der Regel \textit{kursiv} geschrieben.\newline
    \textbf{Zahlenwerte und Einheitensymbole} werden \textbf{nicht} kursiv geschrieben.

    \item Zwischen dem Zahlenwert und der Einheit wird ein Leerzeichen eingefügt. 

    \item Einheiten sind keine Abkürzungen und haben daher keinen Abkürzungspunkt.

    \item Einheitensymbole haben keine Mehrzahl.

    \item Verhältnisse werden in der Wissenschaft nicht mittels \gDQ{p} (per) in der Einheit ausgedrückt, da p für das Präfix \gDQ{pico} reserviert ist. 
    Stattdessen können Verhältnisse mit negativen Exponenten oder mit dem Geteiltzeichen (/) dargestellt werden. 
    Keine Doppelbrüche verwenden! Zur besseren Lesbarkeit darf (muss eventuell) geklammert werden.

    \item Intervalle können entweder mittels Halbgeviertstrichs (--) oder dem Zwischenwort \gDQ{bis} angegeben werden.

    \item Messtoleranzen werden mittels $\pm$ angegeben. Die Einheit kann dabei auch zweimal angegeben werden.
\end{itemize}

\subsection*{Beispiele}

\renewcommand{\arraystretch}{1.5} % Erhöht den Zeilenabstand in der Tabelle
\begin{tabular}{p{0.3\linewidth} p{0.3\linewidth} p{0.4\linewidth}}
    \cellcolor{mygreen} \textbf{Richtig} & \cellcolor{myred} \textbf{Falsch} & \cellcolor{mygray} \textbf{Begründung} \\
    \hline
    $v = \frac{s}{t}$ & v = s/t & Variablen nicht kursiv \\
    \hline
    $U = R \cdot I$ & U = R * I & Variablen nicht kursiv \\
    \hline
    $U = \SI{230}{\volt}$ & $U = 230\si{\volt}$ & Kein Leerzeichen \\
    \hline
    $R = \SI{3}{\ohm}$ & $R = 3\mathit{\Omega}$ & Einheitensymbol kursiv \\
    \hline
    $\mathit{\Omega} = \SI{3}{\steradian}$ & $\Omega = \SI{3}{\steradian}$ & Formelzeichen nicht kursiv \\
    \hline
    $s = \SI{3}{\centi\meter}$ & $s = \SI{3}{\centi\meter}$. & mit Punkt \\
    \hline
    $s = \SI{3}{\centi\meter}$ & $s = \SI{3}{cms}$ & Einheiten haben keinen Plural \\
    \hline
    $n = \SI{15000}{\minute^{-1}}$ & $n = \num{15000}\,\mathrm{rpm}$ & rpm ist keine anerkannte Einheit \\
    \hline
    $v = \SI{30}{\kilo\meter\per\hour}$ & $v = \num{30}\,\mathrm{kph}$ & kph ist keine anerkannte Einheit \\
    \hline
    $\frac{\si{\meter^2}}{\si{\second^2 \ampere}}$, \si{\meter^2 \per(\second^2\ampere)}, \si{\meter^2 \second^{-2}\ampere^{-1}} & $\frac{\si{\meter^2}}{\si{\second^2}}/\si{\ampere}$, $\si{\meter\per\second^2\per\ampere}$ & Doppelbruch, mehrdeutig \\
    \hline
    $\SIrange{1}{3}{\mega\hertz}$, $\numrange[range-phrase = \text{--}]{1}{3}\,\si{\mega\hertz}$ & \texttt{1-3 MHz} & falscher Strich/Hyphen \\
    \hline
    \SI{1}{\mega\hertz} bis \SI{3}{\mega\hertz} & $1-3 \si{\mega\hertz}$ & falscher Strich/Hyphen\\
    \hline
    $123 \pm \SI{5}{\gram}$, \SI{123(5)}{\gram} &  &  \\
\end{tabular}

\section{Umrechnung von Einheiten}\label{sec: Umrechnung_Einheiten}
Zwei physikalische Größen $A, B$ lassen sich nur dann sinnvoll addieren oder subtrahieren,  
\begin{equation}
    C = A \pm B \mComma
\end{equation}
wenn sie nicht nur dieselben Dimensionen, sondern auch dieselben Einheiten besitzen.
Oftmals werden einfache Umrechnungen im Kopf durchgeführt, wenn die Einheiten nicht ident sind, wie zum Beispiel $\SI{3}{\meter} + \SI{20}{\centi\meter} = \SI{3.2}{\meter}$. Dieses Beispiel dient lediglich der Illustration und solche Gleichungen sollten vermieden werden. \\

Das Umrechnen von Einheiten in andere Maßeinheiten ist in der Praxis häufig notwendig. Es kann zum Beispiel gewünscht oder notwendig sein, Ergebnisse in andere Einheiten umzurechnen:
\begin{gather}
    \SI{0.00176}{\second} = \SI{1.76}{\milli\second} \mComma \label{eq: Umrechnung_s_ms}\\
    \SI{3}{\kilo\watt\hour} = \SI{10.8}{\mega\joule} \mDot \label{eq: Umrechnung_kwh_mj}    
\end{gather}
Im ersten Fall, \cref{eq: Umrechnung_s_ms}, wird die Einheit mittels Präfix skaliert. Hat man noch wenig Erfahrung damit, geht man am besten Schrittweise vor:  
\begin{equation}
    \SI{0.00176}{\second} = \num{0.00176}\cdot \underbrace{10^3\cdot 10^{-3}}_{1} \,\si{\second} = \num{1.76} \cdot \underbrace{10^{-3}}_{\textrm{milli}} \,\si{\second} =\SI{1.76}{\milli\second}\mDot
\end{equation}
Im zweiten Fall, \cref{eq: Umrechnung_kwh_mj}, wird in eine gänzlich andere Einheit umgerechnet, nämlich von $\textrm{Wh}$ in $\si{\joule}$. Auch hier geht man am besten Schrittweise vor. Zunächst bestimmen wir die Umrechnung der einzelnen Teile, 
\begin{gather*}
    \SI{1}{\watt} = \SI{1}{\joule\per\second}\; , \\
    \SI{1}{\hour} = \SI{60}{\minute} = 60\cdot 60 \si{\second} = \SI{3600}{\second}\mDot\\
\end{gather*}
Damit wandeln wir nun die $\SI{3}{\kilo\watt\hour}$ um:
\begin{equation}
    3 \underbrace{\textrm{k}}_{10^3} \underbrace{\textrm{W}}_{\si{\joule/\second}} \underbrace{\textrm{h}}_{\SI{3600}{\second}} = 3\cdot \underbrace{\left(10^3 \frac{\si{\joule}}{\si{\second}} \right)}_{\si{\kilo\watt}} \cdot \underbrace{(\SI{3600}{\second})}_{\si{\hour}} = 3\cdot 3600\cdot 10^3 \si{\joule} = 10800 \cdot 10^3 \si{\joule} = \SI{10.8}{\mega\joule}\mDot
\end{equation}
Oft kann man Berechnungen einfach anhand der Einheiten durchführen, indem man mit der Einheit der gesuchten Größe beginnt. 

\begin{rememberbox}[]{}
    Es gibt nur eine Möglichkeit dieselben Einheiten auf zulässige Weise zu einer neuen physikalischen Größe zusammenzusetzen. Einheiten sind demnach eindeutig.
\end{rememberbox}

\begin{examplebox}[]{Beispiel}
    \textbf{Kosten pro gefahrenem Kilometer}
    Sie fragen sich, wie viel jeder gefahrene Kilometer mit dem Auto kostet. Das Ergebnis wird also die Einheit [€/km] haben. Sie haben aber nur den Preis pro Liter Treibstoff [$\num{1.56}$ €/l] und wissen, dass ihr Auto $\num{5.2}$ Liter pro $\SI{100}{\kilo\meter}$ verbraucht. \\
    
    \textit{Lösung:}
    \begin{gather*}
        \frac{\sEUR}{\si{\kilo\meter}} = \frac{\sEUR}{\si{\liter}} \cdot \frac{\si{\liter}}{\si{\kilo\meter}} \\
        \num{1.56}\,\frac{\sEUR}{\si{\liter}} \cdot \num{5.2}\,\frac{\si{\liter}}{\SI{100}{\kilo\meter}} = \num{8.112}\,\frac{\sEUR}{\SI{100}{\kilo\meter}} \approx \num{0.08}\,\frac{\sEUR}{\si{\kilo\meter}}
    \end{gather*}
    Bei einem Preis von $\num{1.56}$ €/l und einem Verbrauch von $\num{5.2}$ Liter pro $\SI{100}{\kilo\meter}$ kostet jeder gefahrene Kilometer $\num{0.08}$ €.
\end{examplebox}

\section{Dimensionsanalyse}\label{sec: Dimensionsanalyse}
Die Dimension einer physikalischen Größe gibt an, was gemessen wird, zum Beispiel eine \textit{Masse}, \textit{Länge} oder \textit{Zeit}. Auf der Suche nach Zusammenhängen zwischen physikalischen Größen abstrahiert man Einheiten auf ihre Dimensionen (so eliminiert man Präfixe). Es spielt zunächst keine Rolle, ob die Länge in einer Formel in $\si{\meter}$ oder $\si{\centi\meter}$ angegeben wird. Man gibt die Dimension einer Größe oft in eckigen Klammern an
\begin{center}
\begin{NiceTabularX}{\textwidth}{X c X c} % NiceTabularX verwenden
\CodeBefore
  \rowcolor{boxcol_title_blue}{1} % Färbt die erste Zeile
  \rowcolors{2}{white}{boxcol_back_lblue} 
\Body
\toprule
\textbf{Größe} & \textbf{Zeichen} & \textbf{Dimension} & \textbf{Einheit} \\ \midrule
Flächeninhalt & $A$ & $[A] = L^2$ & $\Unit{m^2}$ \\ \hline
Volumen & $V$ & $[V] = L^3$ & $\Unit{m^3}$ \\ \hline
Geschwindigkeit & $v$ & $[v] = L/T$ & $\Unit{m/s}$ \\ \hline
Beschleunigung & $a$ & $[a] = L/T^2$ & $\Unit{m/s^2}$ \\ \hline
Kraft & $F$ & $[F] = M L/T^2$ & $\Unit{kg \cdot m/s^2}$ \\ \hline
Druck & $p$ & $[p] = M/(L \cdot T^2)$ & $\Unit{kg/(m \cdot s^2)}$ \\ \hline
Dichte & $\rho$ & $[\rho] = M/L^3$ & $\Unit{kg/m^3}$ \\ \hline
Energie & $E$ & $[E] = ML^2/T^2$ & $\Unit{kg \cdot m^2/s^2}$ \\ \hline
\bottomrule
\end{NiceTabularX}
\end{center}


\begin{examplebox}[]{Beispiel}
\textbf{Die Dimension des Drucks} \\
Der Druck $p$ in einer bewegten Flüssigkeit hängt von ihrer Dichte $\rho$ und von ihrer Geschwindigkeit $v$ ab. Gesucht ist eine einfache Kombination von Dichte und Geschwindigkeit, die die richtige Dimension des Drucks ergibt.

\textbf{Problembeschreibung:} Der Druck hat die Dimension $[p] = \frac{M}{L \cdot T^2}$, die Dichte die Dimension $[\rho] = \frac{M}{L^3}$ und die Geschwindigkeit die Dimension $[v] = \frac{L}{T}$.\newline

\textbf{Lösung:}
Die Masse kommt im gesuchten Druck im Zähler mit dem Exponenten $1$ vor und ebenso in der Dichte. Demnach muss die Dichte ebenso im Nenner stehen und zwar mit dem Exponenten 1. Für die Geschwindigkeit setzen wir zunächst eine variable Hochzahl an 
\begin{equation*}\begin{aligned}
    [p] &= [\rho]\cdot {[v]}^x \mComma  \\
    \frac{M}{L T^2} &= \frac{M}{L^3} {\left(\frac{L}{T}\right)}^x \mDot \\
\end{aligned}\end{equation*} 
Im Nenner benötigen wir ein $T^2$, das wir nur von der Geschwindigkeit bekommen können und daher versuchen wir es mit $x = 2$: 
\begin{equation*}\begin{aligned}
    [p] = \frac{M}{L T^2} &= \frac{M}{L^{\cancel{3}}} \frac{\cancel{L^2}}{T^2} = [\rho]\cdot {[v]}^2\\
    \frac{M}{L T^2} &=  \frac{M}{L T^2}  
\end{aligned} \end{equation*}  \QEDA
\end{examplebox}

\begin{examplebox}[]{Beispiel}
\textbf{Die Dimension der kinetischen Energie} \\
Die kinetische Energie (Bewegungsenergie) eines Körpers $E_\kin$ ($[E_\kin] = \si{\joule}$) hängt von der Masse $m$ ($[m] = \si{\kilo\gram}$) des Körpers und dessen Geschwindigkeit $v$ ($[v] = \si{\meter/\second}$) ab. Ermitteln Sie den Zusammenhang zwischen Energie und den beiden anderen Größen!

\textbf{Lösung:}
Wie oben erwähnt, kann es nur eine Multiplikation oder Division der Größen mit unterschiedlichen Potenzen sein. Negative Hochzahlen drücken dabei Brüche aus. Joule ersetzen wir laut Tabelle durch die Basiseinheiten $\si{\kilo\gram\meter^2/\second^2}$. Wir setzen daher variable Hochzahlen an. 
\textit{Merke:} Es werden nur die Einheiten gleichgesetzt, nicht die physikalischen Größen!

\begin{equation*}\begin{gathered}
    \underbrace{\si{\joule}}_{\si{\kilogram}\cdot \frac{\si{\meter^2}}{\si{\second^2}} } = [E_\kin] = {[m]}^x \cdot {[v]}^y = \si{\kilo\gram}^x \cdot {\left( \frac{\si{\meter}}{\si{\second}}\right)}^y \\
    \si{\kilogram}\cdot \frac{\si{\meter^2}}{\si{\second^2}} = \si{\kilogram}^x\cdot \frac{\si{\meter^y}}{\si{\second^y}} \\
    \Rightarrow x = 1,\; y = 2
\end{gathered}\end{equation*} 
Damit folgt, dass 
\begin{equation*}
    E_\kin \propto m\cdot v^2 \mDot
\end{equation*}  
Wir können damit nur eine Proportionalität ableiten, da eine Multiplikation (Division) einer Zahl auf der rechten (oder linken) Seite die Einheit nicht verändert. \QEDA
\end{examplebox}

% Chapter end - always start new page after chapter
\newpage
%\chapter{Messgenauigkeit und Messfehler}\label{chap: MessgenauigkeitMessfehler}
\section{Signifikante Stellen}\label{sec: signifikante_Stellen}

\begin{rememberbox}[]{}
    Signifikante Stellen geben die Genauigkeit einer Zahl an, die aus einer Messung oder Berechnung hervorgeht. Sie zeigen an, welche Ziffern tatsächlich zur Genauigkeit eines Wertes beitragen und zudem nicht nur Platzhalter sind. Die Berücksichtigung signifikanter Stellen ist wichtig, um die Unsicherheit in Messungen und Berechnungen korrekt darzustellen.
\end{rememberbox}
Führende Nullen sind generell nicht aussagekräftig. Nullen am Ende einer Zahl können signifikant sein. 
Alle Zahlen hinter der letzten signifikanten Stelle sind bedeutungslos. Die letzte signifikante Stelle ist mit einer Unsicherheit behaftet und nur eine Schätzung innerhalb der gegebenen Präzision. 
Es kann sowohl zu einer falschen Präzisionsangabe führen, wenn man zu viele (nicht signifikante) Stellen mitnimmt, als auch zu überhaupt falschen Zahlenwerten. 
Für exakte Werte wendet man das Prinzip der signifikanten Stellen nicht an.
Signifikante Stellen sind nicht dasselbe wie Nachkommastellen.

\begin{center}
\begin{NiceTabular}{l c c}
\CodeBefore
  \rowcolor{boxcol_title_navyblue}{1}
\Body
\toprule
\color{white}\textbf{Zahl} & \color{white}\textbf{Signifikante Stellen} & \color{white}\textbf{Nachkommastellen} \\
\midrule
\num{15.23} & 4 & 2 \\
\num{0.1523} & 4 & 4 \\
\num{1.523e3} & 4 & 3 \\
\num{4500} & 2 & 0 \\
\num{4500.0} & 5 & 1 \\
\num{4.5e3} & 2 & 1 \\
\num{4.50e2} & 3 & 2 \\
\num{4.500e3} & 4 & 3 \\
\bottomrule
\end{NiceTabular}
\end{center}
\vspace{0.3cm}
Die wissenschaftliche Schreibweise als 
\begin{equation}
    \mathrm{Zahl} = \mathrm{Mantisse} \cdot 10^{\mathrm{Exponent}}
\end{equation}
erlaubt eine einfache Angabe der signifikanten Stellen, da auch endende Nullen sofort als signifikant gekennzeichnet sind. 

Vorsicht ist geboten beim Wechsel von Einheiten oder Zehnerpotenzen (die Genauigkeit darf dadurch nicht verändert werden):
\begin{equation}
    \SI{3.2}{\meter} = \SI{320}{\centi\meter}\mDot
\end{equation}
Die Umrechnung stimmt und die Meterangabe hat definitiv 2 signifikante Stellen. Die Zentimeterangabe könnte allerdings 2 oder 3 signifikante Stellen haben. Besser wäre es also $\SI{3.2e2}{\centi\meter}$ zu schreiben.

Eine Zahl in nicht wissenschaftlichem Format wird meist so interpretiert, dass die letzte Stelle gerundet ist. Die Zahl $50$ kann also zwischen $\num{49.5}$ und $\num{50.4}$ liegen. Die signifikanten Stellen enden an der letzten Stelle, die nach dem Runden noch angegeben werden kann.

\section{Messgenauigkeit und Messfehler}\label{sec: Messgenauigkeit_Messfehler}
\textbf{Jede Messung ist fehlerbehaftet}. Der Messwert muss daher mit einer Fehlerangabe versehen werden, um seine Genauigkeit feststellen zu können. 
Man unterscheidet grundlegend zwischen systematischen Fehlern und statistischen (zufälligen) Fehlern. 
\subsection{Systematische Fehler}\label{subsec: systematische_Fehler}
\begin{rememberbox}[]{}
    Ein \textbf{systematischer Fehler} ergibt eine fehlerhafte Abweichung des Messwertes vom wahren Wert, die immer gleich ist, wenn die Messung unter den gleichen Bedingungen wiederholt wird. 
\end{rememberbox}
Systematische Fehler ergeben sich durch ein falsches Messverfahren, durch die Messapparatur oder durch die Nichtberücksichtigung von äußeren Einflüssen. 
Häufig werden systematische Fehler unterschätzt!

\textit{Beispiel:} Messungen der Elektronenmasse \newline
Man sieht in \cref{fig: elektronenmasse_ungenauigkeit}, dass die Fehlerbalken allesamt den heutigen Bestwert nicht inkludieren und damit die systematischen Fehler unterschätzt werden.
 \begin{figure}
     \centering
     \includegraphics[width=0.7\linewidth]{Bilder/Kapitel_Grundlagen/Elektronenmasse_Messungenauigkeit.png}
     \caption{Historische Messwerte für die Elektronenmasse in Einheiten von $\SI{e-31}{\kilo\gram}$. Dargestellt sind die relativen Abweichungen $\Delta m/m$ vom heutigen Bestwert.}
     \label{fig: elektronenmasse_ungenauigkeit}
 \end{figure}

 \begin{examplebox}[sidebyside, sidebyside align=center, lower separated=false, righthand width=5.5cm]{Beispiel}
     \textbf{Raummaße messen mit Maßband:}
     \begin{itemize}[itemsep=1.5pt]
        \item Maßband hat einen systematischen Fehler (Kalibrierung)
        \item Schiefe Messung
        \item Parallaxenfehler (Ablesefehler)
        \item Nullpunktsverschiebung (Metallende könnte verbogen sein)
        \item Dehnung des Maßbands unter Belastung 
        \item Temperaturbedingte Ausdehnung (sehr heißer oder kalter Raum)
     \end{itemize} 
     \tcblower
     \begin{center}
     \includegraphics[width=0.9\linewidth]{Bilder/Kapitel_Grundlagen/massband_raum.png}
     \end{center}
 \end{examplebox}

\subsection{Statistische Fehler}\label{subsec: Statistische_Fehler}
Wiederholte Messungen ergeben im Allgemeinen (abseits von systematischen Fehlern) trotzdem nicht den gleichen Wert. Das kann an einer ungenauen Ablesung liegen, an Vibrationen des Messinstruments oder auch an intrinsischen Schwankungen der Messgröße selbst. 

\begin{rememberbox}[]{}
    Ein rein \textbf{statistischer Fehler} ergibt bei wiederholten Messungen eine Verteilung der Messwerte $(x_i),$ um einen Mittelwert $(\bar{x}$).
    Bei Abwesenheit von systematischen Fehlern nähert sich der Mittelwert dem wahren Wert $(x_\mathrm{W})$ für unendlich viele Messungen an. 

\end{rememberbox}
Da man nicht unendlich viele Messungen durchführen kann, bleibt der wahre Wert im Allgemeinen unbekannt. Die Breite der Messwerte-Verteilung ist ein Maß für die Güte der Messung. Das arithmetische Mittel aller Messungen, 
\begin{equation}
    \bar{x} = \frac{1}{n} \sum_{i=1}^{n} x_i \mComma
\end{equation}
nähert sich dem wahren Wert bei unendlich vielen Messungen 
\begin{equation}
    x_\mathrm{W} = \lim_{n\rightarrow \infty} \frac{1}{n} \sum_{i=1}^n x_i \mDot
\end{equation}
\begin{figure}
    \centering
    \includegraphics[width=0.5\linewidth]{Bilder/Kapitel_Grundlagen/stat_Verteilung_messwerte.png}
    \caption{Typische Verteilung von Messwerten $x_i$ um den Mittelwert $\bar{x}$ bei statistischer Fehlerverteilung.}
    \label{fig: stat_Verteilung_Messwerte}
\end{figure}
Liegen nur statistische Fehler vor, erhält man für die Verteilung der Messwerte eine Normalverteilung bei einer hohen Anzahl an Messwerten.\footnote{Vorausgesetzt die Messgröße ist kontinuierlich und die Schwankung auch.}

\begin{importantbox}[]{}
    Das \textbf{arithmetische Mittel der Einzelmessungen}
    \begin{equation}
        \bar{x} = \frac{1}{n} \sum_{i=1}^n x_i \mDot
    \end{equation}
    Die \textbf{Standardabweichung der Einzelmessungen}
    \begin{equation}
        \sigma = \sqrt{\frac{\sum \left( \bar{x} - x_i \right)^2}{n-1}} \mDot
    \end{equation}
    Mittlere Fehler des arithmetischen Mittels
    \begin{equation}
        \sigma_m = \frac{\sigma}{\sqrt{n}} = \sqrt{\frac{\sum \left(\bar{x} - x_i \right)^2}{n(n-1}} \mDot
    \end{equation}
\end{importantbox}
Bei sehr vielen Messungen nähert sich der Mittelwert dem wahren Wert
\begin{equation}
    x_\mathrm{W} = \lim_{n \rightarrow \infty} \frac{1}{n} \sum_{i=1}^n x_i
\end{equation}
Die Definition der Standardabweichung bedingt, dass bei sogenannten Normalverteilungen 
\begin{itemize}[itemsep=1.5pt]
    \item $68,3 \%$ der Werte innerhalb der \\einfachen Standardabweichung $[\bar{x} - \sigma,\, \bar{x}+\sigma]$,
    \item $95,4\%$ der Werte innerhalb der \\zweifachen Standardabweichung $[\bar{x} - 2\sigma,\, \bar{x} + 2\sigma]$, und
    \item $99,7\%$ der Werte liegen innerhalb der \\dreifachen Standardabweichung $[\bar{x} - 3\sigma,\, \bar{x} + 3\sigma]$
\end{itemize}
liegen.

\begin{examplebox}[sidebyside, sidebyside align=top, lower separated=false, righthand width=5.0cm]{Beispiel}
    Schwingungsdauer\footnotemark{} eines Fadenpendels: \newline
    Man misst 
    Daraus ergibt sich ein Mittelwert von $\SI{2.00}{\second}$ und eine Standardabweichung von $\sigma = \SI{0.05}{\second}$: 
    \begin{equation*}
        T_W = \SI{2.00 +- 0.05}{\second}
    \end{equation*}
    \tcblower
    \begin{center}
    \begin{tabular}{C{1.6cm} | C{1.6cm}}
    \hline
    % \rowcolor{boxcol_title_navyblue}
    % \multicolumn{2}{c}{\textbf{\color{white}Messwerte [s]}} \\
    \multicolumn{2}{c}{\textbf{Messwerte [s]}} \\
    \hline \num{2.08} & \num{2.02} \\
    \hline \num{2.06} & \num{1.98} \\
    \hline \num{1.96} & \num{2.00} \\
    \hline \num{2.02} & \num{1.94} \\
    \hline \num{1.98} & \num{1.96} \\
    \hline
    \end{tabular}
    \end{center}
\end{examplebox}
\footnotetext{Die Schwingungsdauer ist die Länge des Zeitabschnitts, nach der sich die Bewegung wiederholt (nicht nur derselbe Punkt erreicht wird).}

% Chapter end - always start new page after chapter
\newpage
%\chapter{Kinematik}\label{chap: Kinematik}
In diesem Kapitel möchten wir uns mit den Bewegungsformen von Körpern in Kraftfeldern befassen -- also mit den Kernthemen der Mechanik -- wobei wir uns zuerst dem Modell des Massenpunktes bedienen. Im Rahmen der Mechanik wenden wir uns zunächst der Kinematik zu, welche die Grundlage zum Verständnis der unterschiedlichen Bewegungsformen liefert.
\begin{figure}[ht]
    \centering
    \begin{tikzpicture}[
    main_node/.style={
        rectangle,
        fill=blue!60!black,
        text=white,
        font=\sffamily\bfseries\Large,
        minimum width=3.8cm,
        minimum height=1.5cm,
        text centered,
    },
    % Definiert den Stil für die Beschreibungstexte
    desc_node/.style={
        rectangle,
        draw=black, % Rand in der Farbe der Unterboxen
        font=\sffamily,
        align=center,
        text width=4cm,
        inner ysep=4pt,
        inner xsep=0pt,
    }
    ]
    \node[main_node] (mechanik) {Mechanik};
    \node[rectangle,fill=cyan!80!blue,text=white, minimum width=4.5cm,minimum height=1.2cm,text centered,font=\sffamily\bfseries\Large,below left=1.3cm and -0.3cm of mechanik] (kinematik) {Kinematik};
    \node[rectangle,draw=cyan!40!blue, line width=1pt,fill=none,text=black, minimum width=4.5cm,minimum height=1.2cm,text centered,font=\sffamily\bfseries\Large,below right=1.3cm and -0.3cm of mechanik] (dynamik) {Dynamik};
    % Beschreibungstexte über die Unterboxen legen
    \node[desc_node, minimum height=1.5cm,below=0.1cm of kinematik] {Gesetze der Bewegung \\ (ohne Kräfte)};
    \node[desc_node, minimum height=1.5cm,below=0.1cm of dynamik] {Wirkung von Kräften};
    % Verbindungslinien zeichnen
    \coordinate[below=0.75cm of mechanik] (midpoint);
    \draw[black, thick] (mechanik.south) -- (midpoint);
    \draw[black, thick] (kinematik.north) -- (midpoint) -- (dynamik.north);
    \end{tikzpicture}
\end{figure}
\vspace{0.3cm}

\begin{rememberbox}{Mechanik}
    Die Untersuchung der Bewegung von Körpern sowie die Konzepte der Kraft, Masse und Energie bilden den Bereich der Mechanik.
    Die Mechanik wird in zwei Subbereiche unterteilt: die \textbf{Kinematik} und die \textbf{Dynamik}.
    Die Kinematik charakterisiert die Bewegung eines Körpers und vernachlässigt dabei zunächst die Ausdehnung des Körpers. Die Dynamik befasst sich mit der Frage \textit{warum} ein Körper eine Bewegung ausführt und behandelt somit das Konzept von Kräften.
\end{rememberbox}
\section{Das Modell des Massenpunktes}\label{sec: Modell_Massenpunkt}
Bei vielen Problemen in der Physik kann man von der räumlichen Ausdehnung der Körper absehen und die Körper wie punktförmige Gebilde mit der Masse $m$ behandeln, die wir \textbf{Massenpunkte} nennen. Die Lage eines Massenpunktes kann mithilfe eines geeigneten Koordinatensystems durch seine Koordinaten $(x, y, z)$ in kartesischen Koordinaten oder $(r, \theta, \phi)$ in Kugelkoordinaten angegeben werden. 
Die Größenordnung der Masse eines ausgedehnten Körpers spielt dabei keine Rolle, ob das Modell des Massenpunktes ein geeignetes Modell ist: 
\begin{itemize}
    \item Die Bahnbewegungen von Planeten können sehr genau vorausgesagt werden, wenn die Planeten als Massenpunkte vereinfacht werden. 
    \item Stoßprozesse von Atomen oder deren Flugbahnen werden häufig mit Massenpunkten berechnet.
\end{itemize}

\section{Bezugssystem und Koordinatensysteme}\label{sec: Bezugssystem_Koordinatensystem}
Um über das Konzept von Ruhe oder Bewegung sprechen zu können, benötigt man zunächst ein Bezugssystem. Ein Bezugssystem ist ein räumlicher und zeitlicher Rahmen, der verwendet wird, um die Position, Bewegung und Interaktion von Objekten zu beschreiben. Es bildet die Grundlage für die Beschreibung von Phänomenen in der Physik.
\begin{figure}[tbh]
    \centering
    \resizebox{0.55\linewidth}{!}{
    \begin{tikzpicture}[
        % Stil für die Achsen
        axis/.style={-Stealth, line width=1.5pt},
        % Stil für die Hilfslinien
        grid line/.style={dashed, line width=0.6pt}
    ]
    % ========= Schwarzes Koordinatensystem (x,y) =========
    \draw[axis, black] (-1,0) -- (5,0) node[right] {\LARGE $x$};
    \draw[axis, black] (0,-1) -- (0,4) node[above left] {\LARGE $y$};
    % ========= Rotes, rotiertes Koordinatensystem (u,v) =========
    \begin{scope}[rotate=45]
        \draw[axis, red] (-1,0) -- (5,0) node[right] {\LARGE $u$};
        \draw[axis, red] (0,-1) -- (0,4) node[left] {\LARGE $v$};
    \end{scope}
    % ========= Punkt P und Projektionen =========
    \coordinate (P) at (2,3);
    \coordinate (u) at (2.5,2.5);
    \coordinate (v) at (-0.5,0.5);
    \fill[blue] (P) circle (2pt) node[above=5pt] {\color{black}\large $P_{(x,y)}(2,3)$};
    % Projektionen auf das (x,y)-System (schwarz, gestrichelt)
    \draw[grid line, black] (P) -- (2,0) node[below] {$x=2$}; % x
    \draw[grid line, black] (P) -- (0,3) node[left] {$y=3$};  % y
    % Projektionen auf das (u,v)-System (rot, gestrichelt)
    \draw[grid line, red] (P) -- (u) node[right=4pt, sloped] {$u=3,53$};
    \draw[grid line, red] (P) -- (v) node[left=19pt, below] {$v=0,71$};
    \end{tikzpicture}
    }
    \caption{Zwei (kartesische) Koordinatensysteme $S_1\mathrm{:}\,(x,y)$ und $S_2\mathrm{:}\,(u,v)$, wobei $(u,v)$ um $\pi/4$ $(=\ang{45})$ gegenüber $(x,y)$ rotiert ist. Die Koordinaten des Punktes $P$ sind in $(x,y)$-Koordinaten und $(u,v)$-Koordinaten unterschiedlich.}\label{fig: zwei_koordinatensysteme}
\end{figure}

Ein Koordinatensystem ist ein spezielles Konzept innerhalb eines Bezugssystems, das verwendet wird, um Punkte im Raum zu lokalisieren. Es besteht aus einer Reihe von Achsen, die durch eine bestimmte Anzahl von Parametern, wie \zB Koordinaten oder Winkeln, definiert sind.

Das Koordinatensystem ist somit das Werkzeug, das innerhalb des Bezugssystems verwendet wird, um die Position eines Objekts oder Punktes im Raum zu bestimmen. Ein Bezugssystem kann dabei mit unzähligen verschiedenen Koordinatensystemen überzogen werden. 
\begin{rememberbox}[]{}
    Ein Bezugssystem wird durch einen Bezugspunkt, ausgezeichnete Raumrichtungen (meist die Achsen des Koordinatensystems) und eine Zeiteinheit festgelegt.
\end{rememberbox}
Die Beschreibung von physikalischen Vorgängen hängt vom gewählten Bezugssystem ab. Ist kein Bezugssystem angegeben, geht man meist von einem relativ zur Erdoberfläche ruhenden Bezugssystem aus.
\begin{figure}[tbh]
        \centering
        \includegraphics[width=0.6\linewidth]{Bilder/Kapitel_Mechanik/Kapitel_Kinematik/bezugssysteme.png} 
        \caption{Illustration von zwei verschiedenen Bezugssystemen. Die Bewegungszustände aus der Sicht des Autofahrers und aus Sicht des Passanten unterscheiden sich.}\label{fig: bezugssystem_auto_passant}
\end{figure}
\begin{examplebox}[]{Beispiel [\Cref{fig: bezugssystem_auto_passant}]}
    \textbf{Bezugssystem 1:} Ein Beobachter in einem Auto, das sich gleichförmig bewegt (keine Beschleunigung), sieht den Baum auf sich zukommen. Das Auto ruht relativ zum Beobachter im Auto.\\
    \textbf{Bezugssystem 2:} Ein außenstehender Beobachter sieht das Auto auf den ruhenden Baum zufahren.\newline
    Beide Beschreibungen sind korrekt und führen zu richtigen Vorhersagen.
\end{examplebox}

\begin{importantbox}[]{}
    Physikalische Vorgänge können je nach Bezugssystem unterschiedliche Beschreibungen haben.
\end{importantbox}

\section{Bahnkurve}\label{sec: Bahnkurve}
Die Bewegung eines Massenpunktes wird durch die zeitliche Änderung seiner Koordinaten beschrieben.
\begin{equation}
    \ivec{r}(t) = \icolThree{x(t)}{y(t)}{z(t)} \mDot
\end{equation}
Der Ortsvektor $\ivec{r}=(x,y,z)$ fasst die drei Koordinaten zusammen. Bei einer ebenen Bewegung hat der Ortsvektor nur zwei Koordinaten $[\ivec{r}=(x,y)]$. Der Ortsvektor zeigt vom Ursprung des Koordinatensystems zum Ort des Teilchens $P(t)$.
\begin{figure}[h]
    \centering
    \includegraphics[width=0.6\textwidth]{Bilder/Kapitel_Mechanik/Kapitel_Kinematik/bahnkurzve.png}
    \caption{Darstellung einer Bahnkurve eines Massenpunktes im 3-dimensionalen Raum. Die Bahnkurve (rot) ist die Menge aller Punkte $P$, die der Körper während der Bewegung durchläuft.}\label{fig: bahnkurve_allg}
\end{figure}
Die Bewegung, die der Massenpunkt beim Durchlaufen der Bahnkurve erfährt, heißt \textbf{Translation}. Als Rotation wird die Bewegung eines Körpers um eine Rotationsachse bezeichnet. Da ein Massenpunkt keine Ausdehnung hat, kann er keine Rotation ausführen.\footnote{Kreisbewegungen eines Massenpunktes sind keine Rotation, sondern eine Translation.}
\begin{rememberbox}{Bahnkurve}
    Die Funktion $\ivec{r} = \ivec{r}(t)$ stellt eine Kurve im Raum dar, die der Massenpunkt im Laufe der Zeit durchläuft. Die Bahnkurve ist die Menge aller Punkte $P(t)$, die der Körper während der Bewegung durchläuft.
\end{rememberbox}
Die Darstellung $\ivec{r}(t)$ heißt Parameterdarstellung, da die Koordinaten des Massenpunktes $\inlrowThree{x(t)}{y(t)}{z(t)}$ vom Parameter $t$ (Zeit) abhängen.
\begin{examplebox}{Beispiele}
    \begin{enumerate}
        \item \textit{Geradlinige Bewegung}\newline
        Eine geradlinige Bewegung liegt vor, sofern alle Koordinaten maximal linear von der Zeit abhängen, \zB $x(t) = x_0 + a\cdot t$, $y(t) = b\cdot t$, $z(t) = 0$, wobei $a,b, x_0 \in \Real$.
        \item \textit{Ebene Kreisbewegung} \newline
        Eine ebene Kreisbewegung lässt sich am elegantesten in Polarkoordinaten $(r,\varphi)$ darstellen, $x(t) = r\cdot \cos(\omega t)$, $y(t) = r\cdot \sin(\omega t)$, $z(t)=0$, wobei $\omega = \dd \varphi/\dd t$. 
    \end{enumerate}
\end{examplebox}


\subsection{Die geradlinige Bewegung}\label{subsec: geradlinige_Bewegung}

\begin{figure}[h]
    \centering
    \includegraphics[width=0.6\textwidth]{Bilder/Kapitel_Mechanik/Kapitel_Kinematik/geradlinigeBewegungBahnkurve.png}
    \caption{Geradlinige Bewegung eines Massenpunktes in der $(x,y)$-Ebene. Die Bahnkurve in rot ist die Menge aller Punkte $P(t)$, die der Massenpunkt durchläuft. Der Ortsvektor (blau) ist für den Zeitpunkt $t=5$ eingezeichnet, $\protect\ivec{r}(5)$.}\label{fig: geradlinige_bewegung_bsp}
\end{figure}
Die geradlinige Bewegung ist dadurch gekennzeichnet, dass die Bahnkurve eine Gerade darstellt. In \cref{fig: geradlinige_bewegung_bsp} ist folgende Bahnkurve dargestellt
\begin{equation}\label{eq: geradlinige_bewegung_bsp}
    \ivec{r}(t) = \icolThree{x(t)}{y(t)}{z(t)} = \icolThree{4t}{8+2t}{0} = \icolThree{0}{8}{0} + \icolThree{4}{2}{0} t \mDot
\end{equation}
Es handelt sich um eine ebene Bewegung, da sie in einer Ebene verläuft -- der $(x,y)$-Ebene. Daher kann die $z$-Komponente ignoriert werden.
Zum Zeitpunkt $t=0$ befindet sich der Massenpunkt an der Stelle $P(0) = \ipThree{0}{8}{0}$ und zum Zeitpunkt $t = 5$ an der Stelle $P(5) = \ipThree{20}{18}{0}$. Der Ortsvektor zu diesem Zeitpunkt lautet
\begin{equation}
    \ivec{r}(5) = \icolThree{20}{18}{0} \mDot
\end{equation}

Die Parameterform der geradlinigen Bewegung, wie in \cref{eq: geradlinige_bewegung_bsp}, kann allgemein als
\begin{equation}\label{eq: parameterform_geradlinige_bewegung}
    \ivec{r}(t) = \icolThree{x(t)}{y(t)}{z(t)} = \icolThree{x_0}{y_0}{z_0} + \icolThree{v_x}{v_y}{v_z} t
\end{equation}
dargestellt werden. In dieser Darstellungsform ist $P_0 = \ipThree{x_0}{y_0}{z_0}$ der Startpunkt der Bewegung zum Zeitpunkt $t = 0$: $P_0 = P(t=0)$. Der Richtungsvektor der Geraden ist -- wie wir noch sehen werden -- gleich der Geschwindigkeit $\ivec{v} = \inlrowThree{v_x}{v_y}{v_z}$.\footnote{Beachte, dass in \cref{eq: parameterform_geradlinige_bewegung} sowohl der Startpunkt $P_0$ als auch der Geschwindigkeitsvektor $\ivec{v}$ als Spaltenvektor dargestellt werden, obwohl es sich um unterschiedliche mathematische Objekte handelt.}

\begin{rememberbox}{Position des Massenpunktes und Ortsvektor}
    Der Ortsvektor $\ivec{r}(t)$ fällt zahlenmäßig mit der Position des Teilchens $P(t)$ zusammen, sofern man ein standardmäßiges kartesisches Koordinatensystem verwendet.
    Die Bahnkurve einer geradlinigen Bewegung umfasst alle Punkte $P$, die der Massenpunkt durchläuft. Der Ortsvektor $\ivec{r}(t)$ zeigt vom Ursprung zur Position des Teilchens $P(t)$.
\end{rememberbox}
Obwohl der Ortsvektor $\ivec{r}(t)$ und der Ort des Teilchens $P(t)$ zahlenmäßig gleich sind, gilt es dennoch zu beachten, dass ein Punkt und ein Vektor nicht dasselbe mathematische Objekt darstellen. Während für Vektoren beispielsweise algebraische Operationen wie Addition und Subtraktion definiert sind, gilt das für einen Punkt nicht. 

\subsection{Die ebene Kreisbewegung}\label{subsec: ebene_Kreisbewegung}
Bei der ebenen Kreisbewegung beschreibt die Bahnkurve einen Kreis und damit liegt die momentane Position des Massenpunktes $P(t)$ immer auf einem Kreis. Der Ortsvektor zeigt vom Ursprung zum Punkt $P(t)$ auf dem Kreis. Die Bewegung wiederholt sich nach jeder Umdrehung.
\begin{figure}[tbh]
    \centering
    \includegraphics[width=0.6\textwidth]{Bilder/Kapitel_Mechanik/Kapitel_Kinematik/ebeneKreisbewegung.png}
    \caption{Ebene Kreisbewegung eines Massenpunktes.}\label{fig: ebene_kreisbewegung_bsp}
\end{figure}
Als Beispiel für eine ebene Kreisbewegung betrachten wir das Beispiel
\begin{equation}
    \ivec{r}(t) = \icolThree{x(t)}{y(t)}{z(t)} = \icolThree{3\cdot \cos(2t)}{3 \cdot \sin(2t)}{0} \mComma
\end{equation}
das in \cref{fig: ebene_kreisbewegung_bsp} dargestellt ist. 
Die Kreisbahn kann leicht nachgewiesen werden:
\begin{equation*}
    {x(t)}^2 + {y(t)}^2 = 3^2 \cdot (\underbrace{{\cos(2t)}^2 + {\sin(2t)}^2}_{=1}) = 3^2 \mDot
\end{equation*}
Dies ist die Gleichung eines Kreises (siehe \cref{eq: kreisgleichung_allg}) mit Radius $R = 3$. Der Ortsvektor $\ivec{r}(t)$ rotiert in diesem Fall um den Ursprung (Mittelpunkt des Kreises) mit der Kreisfrequenz $\omega = \SI{2}{\radian\per\second}$. 

Die allgemeinste Form der ebenen Kreisbewegung in kartesischen Koordinaten $x(t),y(t)$ lautet
\begin{equation}\label{eq: ebene_kreisbewegung_allg}
    \ivec{r}(t) = \icolTwo{x(t)}{y(t)} = \icolTwo{x_0}{y_0} + R \cdot \underbrace{\icolTwo{\cos(\omega\cdot t + \varphi_0)}{\sin(\omega\cdot t + \varphi_0)}}_{\ivecS{e}{r}}\mDot
\end{equation}
Hierbei ist $M = \ipTwo{x_0}{y_0}$ der Mittelpunkt des Kreises, $R$ der Radius, $\omega$ die Kreisfrequenz und $\varphi_0$ der Startwinkel. Der Richtungsvektor $\ivecS{e}{r}$ ist der radial nach außen zeigende Einheitsvektor ($|\ivecS{e}{r}| = 1$) am Einheitskreis, der mit dem Radius $R$ skaliert wird.  
\begin{rememberbox}{Bahnkurve der Kreisbewegung}
    Bei der Bahnkurve einer ebenen Kreisbewegung zeigt der Ortsvektor $\ivec{r}(t)$ vom Ursprung zur Position des Teilchens $P(t)$. Das Teilchen mit den Koordinaten $x(t), y(t)$ bewegt sich mit der Winkelgeschwindigkeit $\omega$ und der Geschwindigkeit $v = \omega\cdot R$.  
\end{rememberbox}

\subsubsection{Kreisgleichung}
\begin{rememberbox}{Kreisgleichung}
    Ein Kreis ist die Menge aller Punkte mit einem festen Abstand R (dem Radius) von einem gemeinsamen Mittelpunkt $M = \ipTwo{x_M}{y_M}$. Für jeden Punkt $P=\ipTwo{x}{y}$ auf dem Kreis muss
    \begin{equation}\label{eq: kreisgleichung_allg}
        R^2 = {(x-x_M)}^2 + {(y-y_M)}^2 
    \end{equation}
    gelten. Dies ist die sogenannte Kreisgleichung. 
\end{rememberbox}
Wenn der Mittelpunkt des Kreises im Ursprung des Koordinatensystems liegt, $M = \ipTwo{0}{0}$, vereinfacht sich die Kreisgleichung zu 
\begin{equation*}
    R^2 = x^2 + y^2 \mDot
\end{equation*}



\section{Ort und Verschiebung}\label{sec: ort_verschiebung}
Um die Bewegung eines Teilchens zu beschreiben, geben wir zunächst seinen Ort mit Hilfe von Koordinaten an. Bei einer eindimensionalen Bewegung genügt hierbei die $x$-Koordinate. 
\begin{importantbox}[]{Verschiebung (1D)}
    Die Ortsänderung eines Massenpunktes wird als Verschiebung bezeichnet. Die Verschiebung $\Delta x$ ist die Differenz zwischen dem Endort $x_E$ und dem Anfangsort $x_A$:
    \begin{equation}
        \Delta x=x_{E}-x_{A}\mDot
    \end{equation}
\end{importantbox}

\begin{figure}[htbp]
    \centering
    \includegraphics[width=0.65\textwidth]{Bilder/Kapitel_Mechanik/Kapitel_Kinematik/verschiebung_vektor_auto.png}
    \caption{Darstellung der eindimensionalen Bewegung eines Autos entlang der $x$-Achse vom Ort $x_A$ zum Ort $x_E$ und dem Verschiebungsvektor $\Delta x$, der diese Bewegung beschreibt.}\label{fig: ort_1d}
\end{figure}
Die Verschiebung $\Delta x$ gibt auch die Richtung der Ortsänderung an und kann auch negativ sein, wenn sich das Objekt in die negative Richtung der Koordinatenachse bewegt.

\begin{rememberbox}{Notation: Das Delta-Symbol}
    Änderungen physikalischer Größen (hier: Ortskoordinate) werden mit einem (nicht-kursiven) großen Delta $\Delta$ vorangestellt. Die Änderung von $x$ ist demnach $\Delta x$. Im eindimensionalen Fall verzichtet man meist auf eine Vektorschreibweise ($\Delta x$ statt $\Delta \ivec{x}$), obwohl es sich hier streng genommen um einen Vektor handelt.
\end{rememberbox}
\noindent Die Position eines Massenpunktes wird durch einen Ortsvektor $\ivec{r}$ beschrieben. Mit Hilfe von Einheitsvektoren kann man den Ortsvektor in kartesischen Koordinaten schreiben als
\begin{equation}\label{eq: vektor_aufteilung_einheitsvektoren}
    \ivec{r} = x\cdot\ivecS{e}{x} + y\cdot\ivecS{e}{y} + z\cdot\ivecS{e}{z}\mDot
\end{equation}
Hier geben der Skalar $x$ die Anzahl der \gDQ{Schritte} in Richtung von $\ivecS{e}{x} = \inlrowThree{1}{0}{0}^T$, $y$ die Anzahl der \gDQ{Schritte} in Richtung von $\ivecS{e}{y} = \inlrowThree{0}{1}{0}^T$ und $z$ die Anzahl der \gDQ{Schritte} in Richtung von $\ivecS{e}{z} = \inlrowThree{0}{0}{1}^T$ an.\footnote{Das hochgestellte $T$ steht für Transposition (siehe \cref{subsec: Spalten_vs_Zeilenvektor}) und soll verdeutlichen, dass es sich um einen Spaltenvektor handelt, der aus Platzgründen als transponierter Zeilenvektor geschrieben wird.} Damit sind $x$,$y$,$z$ klarerweise die Koordinaten des Massenpunktes. 

\begin{figure}[htbp]
    \centering
    \includegraphics[width=0.6\textwidth]{Bilder/Kapitel_Mechanik/Kapitel_Kinematik/verschiebungsvektor_bsp.png}
    \caption{Der Verschiebungsvektor $\Delta\protect\ivec{r}$ als Differenz der Ortsvektoren $\protect\ivecS{r}{2}$ und $\protect\ivecS{r}{1}$.}\label{fig: verschiebungsvektor_als_differenz}
\end{figure}

Unter Verwendung der Schreibweise der Ortsvektoren aus \cref{eq: vektor_aufteilung_einheitsvektoren} kann die Verschiebung nun auch als
\begin{equation}
    \Delta\ivec{r}=\ivecS{r}{2} - \ivecS{r}{1} = \Delta x \cdot \ivecS{e}{x} + \Delta y\cdot \ivecS{e}{y} + \Delta z\cdot \ivecS{e}{z}
\end{equation}
geschrieben werden, wobei $\Delta x = x_{2} - x_{1}$, $\Delta y = y_{2} - y_{1}$ und $\Delta z = z_{2} - z_{1}$ sind. In dieser Schreibweise sind $\Delta x$, $\Delta y$, $\Delta z$ Skalare. Die Verschiebung wird in mehreren Dimensionen konsequent als Vektor notiert ($\Delta\ivec{r} = \ivecS{r}{2} - \ivecS{r}{1}$).

\begin{rememberbox}{Verschiebungsvektor (Allgemein)}
    Der Verschiebungsvektor $\Delta\ivec{r}$ ergibt sich durch die Differenz der Ortsvektoren \begin{equation}
        \Delta\ivec{r}=\ivecS{r}{2}-\ivecS{r}{1}\mComma
    \end{equation}
    siehe \cref{fig: verschiebungsvektor_als_differenz}. Die kartesischen Koordinaten des Verschiebungsvektors sind die Verschiebungen entlang der Koordinatenachsen: $\Delta\ivec{r} = \inlrowThree{\Delta x}{\Delta y}{\Delta z}^T$. Eine Verschiebung soll genau jenen Vektor darstellen, der von $\ivecS{r}{1}$ nach $\ivecS{r}{2}$ zeigt und daher gilt $\ivecS{r}{1} + \Delta\ivec{r} = \ivecS{r}{2}$.
\end{rememberbox}


\section{Zeit und Zeitintervall}\label{sec: Zeit_Zeitintervall}
Die Bahnkurve des Massenpunktes gibt den Ort des Teilchens zu jedem Zeitpunkt $t$ an. Die Differenz zweier Zeitpunkte $t_E$ und $t_A$ wird als Zeitintervall $\Delta t$ bezeichnet:
\begin{equation}
    \Delta t=t_{E} - t_{A}\mDot
\end{equation}
Auch eine Zeitänderung hat ein Vorzeichen und somit eine Richtung. 

\subsection{Weg-Zeit-Diagramm}\label{subsec: weg-zeit-diagramm}
Ein Weg-Zeit-Diagramm ist eine weitere grafische Repräsentation der Bewegung eines Teilchens. \Cref{fig: Weg_zeit_Diagramm_Bsp} zeigt ein Weg-Zeit-Diagramm für eine eindimensionale Bewegung, bei dem üblicherweise die Zeit auf der Abszisse (horizontale Achse) und der Weg auf der Ordinate (vertikale Achse) aufgetragen wird. Während die Zeit $t$ auf der Abszisse voranschreitet, kann sich der Massenpunkt sowohl in die positive $x$-Richtung ($\Delta x > 0$) als auch in die negative $x$-Richtung ($\Delta x < 0$) bewegen. Jeder Punkt entlang der Bahnkurve repräsentiert einen Zeitpunkt und den zugehörigen Ort, $P = (t,x)$.
\begin{figure}[htbp]
    \centering
    \includegraphics[width=0.7\textwidth]{Bilder/Kapitel_Mechanik/Kapitel_Kinematik/bewegung_bahnkurve_zeitintervall_ort.png}
    \caption{Ein exemplarisches Weg-Zeit-Diagramm: Von links nach rechts liest sich ein Weg-Zeit-Diagramm wie die zeitliche Bewegung eines Massenpunktes. Jeder Punkt $P$ ist mit einer bestimmten Zeit $t$ und einem bestimmten Ort $x$ assoziiert.}\label{fig: Weg_zeit_Diagramm_Bsp}
\end{figure}


\section{Geschwindigkeit}\label{sec: Geschwindigkeit}
Bewegte Körper unterscheiden sich von ruhenden durch eine von Null verschiedene Geschwindigkeit. Die Geschwindigkeit gibt an, wie schnell (in welchem Zeitintervall $\Delta t$) eine bestimmte Verschiebung $\Delta\ivec{r}$ stattgefunden hat. Betrachten wir zunächst noch einmal \cref{fig: Weg_zeit_Diagramm_Bsp}: Die Verschiebung $\Delta x$ ist hier die Differenz zweier Orte, \zB $\Delta x_{1,2} = x_2 - x_1$. Diese Verschiebung hat im Zeitintervall $\Delta t_{1,2} = t_2-t_1$ stattgefunden. Die Geschwindigkeit dieser Verschiebung gibt das Verhältnis von zurückgelegtem Weg zu benötigter Zeit an, also 
\begin{equation}\label{eq: geschwindigkeit_def_verhältnis_bsp}
    v_{1,2} = \Delta x_{1,2}/\Delta t_{1,2}\mDot
\end{equation}
Die Geschwindigkeit wird demnach größer, wenn mehr Weg zurückgelegt wird oder die dafür benötigte Zeit sinkt. Man beachte, dass die Geschwindigkeit in \cref{eq: geschwindigkeit_def_verhältnis_bsp} eine Richtung hat und auch negativ werden kann. Die Geschwindigkeit selbst ist also ein Vektor und wird hier wieder aufgrund der eindimensionalen Beschreibung ohne Vektorpfeil notiert.
Das Prinzip der Geschwindigkeit bleibt bei mehr als einer Dimension dasselbe. Zunächst nehmen wir an, dass sich die Geschwindigkeit während des Zeitintervalls $\Delta t$ nicht ändert. Dann ist die mittlere Geschwindigkeit in jedem Intervall gleich der Momentangeschwindigkeit. 

\begin{rememberbox}{Definition: Mittlere Geschwindigkeit}
    Die mittlere Geschwindigkeit (Durchschnittsgeschwindigkeit) ist definiert als der Quotient aus Verschiebung und dem dafür benötigten Zeitintervall:
    \begin{equation}\label{eq: def_durchschnittsgeschwindigkeit}
        \overline{\ivec{v}}=\frac{\Delta\ivec{r}}{\Delta t}\mDot
    \end{equation}
    Die Durchschnittsgeschwindigkeit $\overline{\ivec{v}}$ ist proportional zum Verschiebungsvektor $\Delta \ivec{r}$ und zeigt in dieselbe Richtung wie dieser. Die Maßeinheit der Geschwindigkeit ist $[v] = \SI{1}{\meter\per\second}$.
\end{rememberbox}

Mittels Einheitsvektoren kann der Geschwindigkeitsvektor auch als
\begin{equation}
    \ivec{v} = \icolThree{v_x}{v_y}{v_z} = v_{x}\cdot\ivecS{e}{x} + v_{y} \cdot \ivecS{e}{y} + v_{z} \cdot \ivecS{e}{z}
\end{equation}
geschrieben werden.

\begin{rememberbox}{Terminologie: Geschwindigkeit und Tempo}
    Im deutschen Sprachgebrauch wird oft nicht zwischen dem Geschwindigkeitsvektor $\ivec{v}$ (Richtung und Betrag) und seinem Betrag $v=|\ivec{v}|$ (Tempo) unterschieden. Im Englischen werden dagegen die Begriffe \gDQ{velocity} (Vektor) und \gDQ{speed} (Betrag) strikt getrennt.
\end{rememberbox}

\subsection{Durchschnittsgeschwindigkeit}\label{subsec: durchschnittsgeschwindigkeit}
Für endliche Zeitintervalle ist die zuvor definierte Geschwindigkeit aus \cref{eq: def_durchschnittsgeschwindigkeit} die sogenannte \textbf{Durchschnittsgeschwindigkeit}. Sie gibt an, welche Verschiebung $\Delta\ivec{r}$ der Körper im Zeitintervall $\Delta t$ zurückgelegt hat. Die Durchschnittsgeschwindigkeit ist jene konstante Geschwindigkeit, mit der sich der Massenpunkt bewegen muss, um im selben Zeitintervall vom selben Ausgangspunkt zum selben Endpunkt zu gelangen.
\Cref{fig: durchschnittsgeschwindigkeit_bsp} zeigt ein Weg-Zeit-Diagramm für eine räumliche Dimension ($x$). In diesem eindimensionalen Fall vereinfacht sich der Quotient der Durchschnittsgeschwindigkeit zu:
\begin{equation}
    \overline{v}_{i,j}=\frac{\Delta x}{\Delta t}=\frac{x_{j}-x_{i}}{t_{j}-t_{i}}\mDot
\end{equation}
In der Abbildung sind die Durchschnittsgeschwindigkeiten in den drei Intervallen $[t_1, t_2]$, $[t_2, t_3]$, $[t_1, t_3]$
\begin{equation*}
    \overline{v}_{1,2}=\frac{\Delta x}{\Delta t}=\frac{x_{2}-x_{1}}{t_{2}-t_{1}} \mComma\,\overline{v}_{2,3}=\frac{\Delta x}{\Delta t}=\frac{x_{3}-x_{2}}{t_{3}-t_{2}} \mComma\,  \overline{v}_{1,3}=\frac{\Delta x}{\Delta t}=\frac{x_{3}-x_{1}}{t_{3}-t_{1}} \mComma 
\end{equation*}
als blaue Geraden (Sekanten) eingezeichnet. Unterschiedliche Zeitintervalle liefern unterschiedliche Durchschnittsgeschwindigkeiten.

\begin{rememberbox}{Geometrische Interpretation der Durchschnittsgeschwindigkeit}
    Die Durchschnittsgeschwindigkeit entspricht der Steigung der geraden Verbindungslinie (Sekante) zwischen zwei Punkten im Weg-Zeit-Diagramm. Je steiler die Sekante verläuft, desto größer ist die Geschwindigkeit. Eine fallende Sekante kommt einer negativen Durchschnittsgeschwindigkeit in diesem Intervall gleich.
\end{rememberbox}

\begin{figure}[htbp]
    \centering
    \includegraphics[width=0.7\textwidth]{Bilder/Kapitel_Mechanik/Kapitel_Kinematik/durchschnittsgeschwindigkeit_intervalle.png}
    \caption{Die Steigung der Verbindungslinie (blaue Linie) zwischen $(t_i, x_i)$ und $(t_j, x_j)$ ist die Durchschnittsgeschwindigkeit $\overline{v}_{i,j}$ des Massenpunktes im Intervall $[t_i, t_j]$. Beispielsweise ist $\overline{v}_{1,3} = \Delta x_{1,3}/\Delta t_{1,3} = (x_3-x_1)/(t_3-t_1)$ die Durchschnittsgeschwindigkeit im Zeitintervall $[t_1, t_3]$.}\label{fig: durchschnittsgeschwindigkeit_bsp}
\end{figure}


\subsubsection{Gleichförmig geradlinige Bewegung}\label{subsubsec: geschw_bei_gleichf_geradliniger_Bewegung}

Eine Bewegung, bei der die Geschwindigkeit nach Betrag und Richtung konstant bleibt, heißt \textbf{gleichförmig-geradlinige Bewegung}:
\begin{equation}
    \ivec{v}=(v_{x},v_{y},v_{z})= \const\mDot
\end{equation}
In kartesischen Koordinaten hat die Bahnkurve für gleichförmig-geradlinige Bewegungen die Form einer Geradengleichung in Parameterdarstellung (Parameter $t$):
\begin{equation}\label{eq: bahnkurve_parameterform_gleichf_geradl_Bew}
    \ivec{r}(t) = \ivecS{r}{0}+\ivec{v} \cdot t =
    \icolThree{x_0}{y_0}{z_0} + \icolThree{v_x}{v_y}{v_z} \cdot t\mDot
\end{equation}
Der Geschwindigkeitsvektor $\ivec{v}$ ist hier der Richtungsvektor der Geraden. Sein Betrag ist gleich der Steigung der Geraden 
\begin{equation}
    |\ivec{v}| = \left| \frac{\Delta \ivec{r}}{\Delta t}\right| = \frac{|\Delta \ivec{r}|}{|\Delta t|} \mDot
\end{equation}
Zwei beliebige Punkte der Bahnkurven reichen aus für eine vollständige Charakterisierung einer gleichförmigen geradlinigen Bewegung. Seien die Punkte $P(t_i, \ivecS{r}{i})$ und $Q(t_j, \ivecS{r}{j})$ bekannt, dann kann der Startpunkt als $\ivecS{r}{0} = \ivecS{r}{i}$ gewählt werden. Die Geschwindigkeit berechnet sich aus $\ivec{v} = (|\ivecS{r}{j} - \ivecS{r}{i}|)/(|t_j-t_i|)$, womit die Parameterform \cref{eq: bahnkurve_parameterform_gleichf_geradl_Bew} aufgestellt werden kann.
\begin{figure}[htbp]
    \centering
    \includegraphics[width=0.55\textwidth]{Bilder/Kapitel_Mechanik/Kapitel_Kinematik/bahnkurve_geradlinigeBew_Geschw.png}
    \caption{Für eine gleichförmig-geradlinige Bewegung ist die Bahnkurve (schwarzer Pfeil) eine Gerade im Weg-Zeit-Diagramm. Der Geschwindigkeitsvektor $\protect\ivec{v}$ steht parallel zur Bahnkurve.}\label{fig: r_v_geradlinige_Bew}
\end{figure}


\subsection{Momentangeschwindigkeit}\label{subsec: momentangeschwindigkeit}
Im Allgemeinen ist die Geschwindigkeit $\ivec{v}$ nicht konstant, sondern eine Funktion der Zeit $t$. In der \cref{fig: momentangeschwindigkeit_bahnkurve} sehen wir einen Massenpunkt, der sich zur Zeit $t$ im Punkt $P_1$ befindet. Zu einem späteren Zeitpunkt $t+\Delta t$ ist er zum Punkt $P_2$ vorgerückt. Wie bisher bezeichnen wir den Quotienten 
\begin{equation}\label{eq: quotient_durchschnittsgeschwindigkeit}
    \overline{\ivec{v}} = \frac{\ivec{r}(t+\Delta t) - \ivec{r}(t)}{\Delta t} = \frac{\Delta \ivec{r}}{\Delta t}
\end{equation}
als die mittlere Geschwindigkeit zwischen $P_1$ und $P_2$. 
\begin{figure}[htbp]
    \centering
    \includegraphics[width=0.55\textwidth]{Bilder/Kapitel_Mechanik/Kapitel_Kinematik/momentangeschwindigkeit.png}
    \caption{Während der Verschiebungsvektor $\Delta \protect\ivec{r}$ die Sekante zwischen $P_1$ und $P_2$ bildet und damit die Durchschnittsgeschwindigkeit $\overline{\protect\ivec{v}} \propto \Delta \protect\ivec{r}$ ebenso parallel zur Sekante ist, zeigt der Vektor der Momentangeschwindigkeit $\protect\ivec{v}(t)$ zu jedem Zeitpunkt entlang der Tangente an die Bahnkurve.}\label{fig: momentangeschwindigkeit_bahnkurve}
\end{figure}

Um eine (variable) Geschwindigkeit zu einem exakten Zeitpunkt $t$ zu bestimmen, lassen wir den Punkt $P_2$ immer näher an $P_1$ heranrücken, wodurch das Zeitintervall $\Delta t$ infinitesimal klein wird: $\Delta t \to 0$. Für sich genommen würde der Quotient \cref{eq: quotient_durchschnittsgeschwindigkeit} damit divergieren. Da aber der Zähler ebenso gegen $0$ geht, kann der Quotient einen endlichen Wert annehmen. Die \textbf{Momentangeschwindigkeit} $\ivec{v}(t)$ ist daher der Grenzwert der Durchschnittsgeschwindigkeit für ein gegen Null gehendes Zeitintervall. Mathematisch entspricht diese Grenzwertbildung der Ableitung.

\begin{importantbox}{Definition: Momentangeschwindigkeit}
    Die Momentangeschwindigkeit $\ivec{v}(t)$ ist die erste Ableitung des Ortsvektors nach der Zeit
    \begin{equation}\label{eq: def_momentangeschwindigkeit}
        \ivec{v}(t) = \lim_{\Delta t\to0}\frac{\ivec{r}(t+\Delta t)-\ivec{r}(t)}{\Delta t} = \frac{d\ivec{r}(t)}{dt} = \dot{\ivec{r}}(t)\mDot
    \end{equation}
    Die Ableitung eines Vektors erfolgt dabei komponentenweise:
    \begin{equation}
        \ivec{v}(t) = \frac{\dd\ivec{r}}{\dd t} = \frac{\dd}{\dd t}\icolThree{x(t)}{y(t)}{z(t)} = \begin{pmatrix}
            \frac{\dd x}{\dd t} \\[2.5pt]
            \frac{\dd y}{\dd t} \\[2.5pt]
            \frac{\dd z}{\dd t}
        \end{pmatrix}\mDot
    \end{equation}
    Die Geschwindigkeit ist ein Vektor mit einer Maßeinheit von $[v] = \SI{1}{\meter\per\second}$. 
\end{importantbox}
\begin{figure}
    \centering
    \includegraphics[width=0.7\textwidth]{Bilder/Kapitel_Mechanik/Kapitel_Kinematik/momentangeschwindigkeit_tangenten.png}
    \caption{Die Momentangeschwindigkeit $\protect\ivec{v}(t)$ zeigt zu jedem Zeitpunkt entlang der Tangente an die Bahnkurve.}\label{fig: momentangeschwindigkeit_tangente}
\end{figure}
Die Momentangeschwindigkeit ist also die zeitliche Ableitung des Ortsvektors $\ivec{r}$ nach der Zeit (nicht die Ableitung des Verschiebungsvektors $\Delta \ivec{r}$). Der Zähler $\dd \ivec{r}(t)$ in der Definition der Momentangeschwindigkeit (\cref{eq: def_momentangeschwindigkeit}) entspricht jedoch genau dem infinitesimalen Verschiebungsvektor. 
Die erste Ableitung einer Funktion ist äquivalent zur Steigung der Tangente an die Funktion, daher zeigt der Vektor der Momentangeschwindigkeit immer entlang der Tangente an die Bahnkurve. In \cref{fig: momentangeschwindigkeit_tangente} ist die Momentangeschwindigkeit für drei Zeitpunkte aufgetragen. Man sieht, dass die Geschwindigkeit immer entlang der Tangente an die Bahnkurve zeigt und somit in Richtung des infinitesimalen Verschiebungsvektors $\dd \ivec{r}(t)$.

\begin{rememberbox}[]{Geometrische Interpretation der Momentangeschwindigkeit}
    Der Vektor der Momentangeschwindigkeit $\ivec{v}(t)$ zeigt immer entlang der Tangente an die Bahnkurve am jeweiligen Punkt. Ist die Geschwindigkeit konstant, entspricht die Momentangeschwindigkeit der Durchschnittsgeschwindigkeit.
\end{rememberbox}

\begin{rememberbox}[]{Notation: Zeitliche Ableitung}
    In der Physik notiert man zeitliche Ableitungen oftmals mit einem Punkt über der Variablen und räumliche Ableitungen mit einem Strich:
    \begin{align*}
        \dot{f}(t) &:= \frac{df(t)}{dt} \\
        f'(x) &:= \frac{df(x)}{dx}
    \end{align*}
\end{rememberbox}

\section{Beschleunigung}\label{sec: beschleunigung}
Zuvor haben wir gesehen, dass die Änderung des Ortes als Funktion der Zeit die Geschwindigkeit ergeben hat. Nun wollen wir die Änderung der Geschwindigkeit quantitativ erfassen. In \cref{fig: momentangeschwindigkeit_bahnkurve} ist die Bahnkurve eines Massenpunktes dargestellt, der sich zum Zeitpunkt $t$ an der Stelle $P_1$ befindet und dort die Geschwindigkeit $\ivec{v}(t)$ besitzt. Zum späteren Zeitpunkt $t+\Delta t$ ist der Massenpunkt an die Stelle $P_2$ vorgerückt und hat dort die Geschwindigkeit $\ivec{v}(t+\Delta t)$ -- die Geschwindigkeit kann sich während der Bewegung im Allgemeinen ändern. Die quantitative Beschreibung der \textit{Änderung der Geschwindigkeit} führt zum Begriff der Beschleunigung. Analog zur Geschwindigkeit definieren wir zunächst die \textbf{mittlere Beschleunigung} als die Änderung der Geschwindigkeit pro endlichem Zeitintervall $\Delta t$:
\begin{equation}\label{eq: definition_durchschnittsbeschleunigung}
    \overline{\ivec{a}} = \frac{\ivec{v}(t+\Delta t)-\ivec{v}(t)}{\Delta t} = \frac{\Delta\ivec{v}}{\Delta t} \mDot
\end{equation}

Die \textbf{Momentanbeschleunigung} erhält man wiederum aus der Durchschnittsbeschleunigung durch Grenzwertbildung $\lim_{\Delta t \to 0}$:

\begin{importantbox}{Definition: Momentanbeschleunigung}
    Die Momentanbeschleunigung ist die erste zeitliche Ableitung des Geschwindigkeitsvektors und damit die zweite zeitliche Ableitung des Ortsvektors
    \begin{equation}\label{eq: definition_momentanbeschleunigung}
        \ivec{a}(t) = \lim_{\Delta t\to 0}\frac{\ivec{v}(t+\Delta t)-\ivec{v}(t)}{\Delta t} = \frac{\dd \ivec{v}(t)}{\dd t} = \dot{\ivec{v}}(t) = \frac{\dd^2 \ivec{r}(t)}{\dd t^2} = \ddot{\ivec{r}}(t)\mDot
    \end{equation}
\end{importantbox}

\begin{rememberbox}{Eigenschaften der Beschleunigung}
    Die Beschleunigung $\ivec{a}(t)={(a_{x}(t),a_{y}(t),a_{z}(t))}^\T$ ist ein Vektor. Sie zeigt entlang der Tangente an die Geschwindigkeitskurve $\ivec{v}(t)$ und ist gleich der Krümmung (zweite Ableitung) der Weg-Zeit-Funktion $\ivec{r}(t)$. Die Maßeinheit der Beschleunigung ist $[a] = \SI{1}{\meter\per\second\squared}$.
\end{rememberbox}
Ganz allgemein hat eine physikalische Größe und deren Differenzen und Differentiale immer dieselbe Einheit, \zB $[\ivec{r}] = [\Delta \ivec{r}] = [\dd \ivec{r}] = \SI{1}{\meter}$.

\section{Gleichförmig beschleunigte Bewegung}\label{sec: gleichf_beschl_bewegung}
Eine Bewegung, bei der der Beschleunigungsvektor $\ivec{a}$ konstant bleibt -- hierbei bleibt sowohl der Betrag als auch die Richtung konstant -- heißt \textbf{gleichförmig beschleunigte Bewegung}.
\begin{equation}\label{eq: def_glg_gleichf_beschl_Bewegung}
    \ivec{a} = \const \mDot
\end{equation}
Die Differentialgleichung (siehe \cref{sec: differentialgleichungen}) der gleichförmig beschleunigten Bewegung lautet:
\begin{equation}\label{eq: diff_glg_gleichf_beschl_Bewegung}
    \ddot{\ivec{r}}(t) = \frac{\dd^2 \ivec{r}(t)}{\dd t^2}=\ivec{a}=\const \mDot
\end{equation}
Die Vektorgleichung $\ddot{\ivec{r}}(t) = \ivec{a}$ liest sich aufgeschlüsselt in Komponentenschreibweise als 
\begin{equation}\begin{aligned}
    \ddot{x}(t) &= a_x\mComma \\
    \ddot{y}(t) &= a_y\mComma \\
    \ddot{z}(t) &= a_z\mDot
\end{aligned}\end{equation}
Die \cref{eq: diff_glg_gleichf_beschl_Bewegung} lässt sich durch zweimalige Integration lösen. Die erste Integration liefert die Geschwindigkeit (wobei $\ivecS{v}{0}$ die Anfangsgeschwindigkeit zum Zeitpunkt $t=0$ ist):
\begin{equation}\label{eq: integral_momentantgeschwindigkeit}
    \ivec{v}(t) = \dot{\ivec{r}}(t) = \int \ivec{a} \,\dd t = \ivec{a} \cdot t + \ivecS{C}{1} \mDot
\end{equation}
Die Integrationskonstante $\ivecS{C}{1}$ ist ein Vektor mit konstanten Komponenten und wird durch die Anfangsbedingungen festgelegt. Da die Beschleunigung in \cref{eq: integral_momentantgeschwindigkeit} nur die Änderungsrate der Geschwindigkeit angibt, benötigt man zur Angabe der exakten Momentangeschwindigkeit $\ivec{v}(t)$ noch einen Ausgangswert, der üblicherweise zum Zeitpunkt $t = 0$ angegeben wird (Anfangsbedingung). Zum Beispiel definieren wir, dass zum Zeitpunkt $t = 0$ die Geschwindigkeit $\ivec{v}(t=0) \eqexcl \ivecS{v}{0}$ ist. Damit wird die Integrationskonstante \begin{equation}
    \ivecS{v}{0} \eqexcl \ivec{v}(0) = \underbrace{\ivec{a} \cdot 0}_{=0} + \ivecS{C}{1} = \ivecS{C}{1}  \implies \ivecS{C}{1} = \ivecS{v}{0} \mDot
\end{equation}
Die Momentangeschwindigkeit in \cref{eq: integral_momentantgeschwindigkeit} wird unter Verwendung der Anfangsbedingung $\ivec{v}(t=0) \eqexcl \ivecS{v}{0}$ nun 
\begin{equation}
    \ivec{v}(t) = \ivec{a} \cdot t + \ivecS{v}{0} \mDot
\end{equation}
Eine weitere Integration liefert die Bahnkurve \bzw den Ortsvektor -- also den Ort als Funktion der Zeit
\begin{equation}
    \ivec{r}(t) = \int \ivec{v}(t)\,\dd t = \int \left(\ivec{a}\cdot t + \ivecS{v}{0} \right) \,\dd t = \frac{1}{2} \ivec{a} \cdot t^2 + \ivecS{v}{0}\cdot t + \ivecS{C}{2} \mDot
\end{equation}
Die Integrationskonstante $\ivecS{C}{2}$ ergibt sich hier durch die Festlegung des Anfangsortes (Anfangsbedingung) zu 
\begin{equation}
    \ivecS{r}{0} \eqexcl \ivec{r}(0) = \frac{1}{2} \ivec{a} \cdot 0 + \ivecS{v}{0}\cdot 0 + \ivecS{C}{2} = \ivecS{C}{2} \implies \ivecS{C}{2} = \ivecS{r}{0}\mDot
\end{equation}
\begin{importantbox}{Bewegungsgleichung der gleichförmig beschleunigten Bewegung}
    Die allgemeine Bewegungsgleichung für eine konstante Beschleunigung $\ivec{a} = \const$ lautet:
    \begin{equation}\label{eq: formel_r_gleichf_beschl_bewegung}
        \ivec{r}(t) = \frac{1}{2}\ivec{a}\cdot t^{2}+\ivec{v_{0}} \cdot t + \ivecS{r}{0}
    \end{equation}
    Durch Auswahl von Anfangs- oder Randbedingungen wird aus dieser unendlichen Kurvenschar eine bestimmte Trajektorie ausgewählt. Die zugehörige Geschwindigkeit der gleichförmig beschleunigten Bewegung lautet 
    \begin{equation}\label{eq: formel_v_gleichf_beschl_bewegung}
        \ivec{v}(t) = \ivec{a} \cdot t + \ivecS{v}{0}\mDot
    \end{equation}
\end{importantbox}

\begin{rememberbox}{Anfangsbedingungen}
    Zusammen mit der konstanten Beschleunigung $\ivec{a}$, repräsentieren die Vektoren $\ivecS{r}{0}$ und $\ivecS{v}{0}$ die Anfangsbedingungen der Bewegung zum Zeitpunkt $t=0$:
    \begin{itemize}[itemsep=1.5pt]
        \item $\ivecS{r}{0} = {(x(0),y(0),z(0))}^\T$: Anfangsort,
        \item $\ivecS{v}{0} = {(v_{x}(0),v_{y}(0),v_{z}(0))}^\T$: Anfangsgeschwindigkeit,
        \item $\ivec{a} = {(a_{x},a_{y},a_{z})}^\T$: konstante Beschleunigung.
    \end{itemize}
\end{rememberbox}
Die beiden gängigsten Beispiele für eine gleichförmig beschleunigte Bewegung sind \textit{der freie Fall} und \textit{der schräge Wurf}, die als nächstes behandelt werden. 

\subsection{Der freie Fall}\label{subsec: freier_Fall}
Der freie Fall ist eine Art der eindimensionalen gleichförmig beschleunigten Bewegung, bei der die konstante Beschleunigung entlang der negativen $z$-Achse wirkt und durch die Gravitationsbeschleunigung gegeben ist, $\ivec{a} = \ivec{g} = \inlrowThree{0}{0}{-9.81}\,\si{\meter\per\second\squared}$ ($|\ivec{g}| = \SI{9.81}{\meter\per\second\squared}$). Der Massenpunkt habe keine Anfangsgeschwindigkeit, $\ivecS{v}{0} = \inlrowThree{0}{0}{0}$\,\si{\meter\per\second}, und die Anfangsposition $\ivecS{r}{0}$ sei durch die Anfangshöhe $h > 0$ vollständig definiert, $\ivecS{r}{0} = \inlrowThree{0}{0}{h}$. 

Die gesamte Bewegung beim freien Fall spielt sich entlang der $z$-Achse ab, \gDh $x(t) \equiv 0$ und $y(t) \equiv 0$. Setzt man die Definitionen für den freien Fall in \cref{eq: formel_r_gleichf_beschl_bewegung} ein, erhält man die Bewegungsgleichung
\begin{equation}\label{eq: bewegungsgleichung_z_freier_fall}
    z(t) = -\frac{g}{2}\cdot t^2 + h 
\end{equation}
sowie die Momentangeschwindigkeit 
\begin{equation}
    v(t) = -g\cdot t \mDot
\end{equation}
Die Geschwindigkeit ist negativ, sie zeigt also entlang der negativen $z$-Achse und steigt linear mit der Zeit. 
\begin{figure}[htb]
    \centering
    \includegraphics[width=0.6\linewidth]{Bilder/Kapitel_Mechanik/Kapitel_Kinematik/freierFall_edit.png}
    \caption{Die Höhe $z(t)$ beim freien Fall mit Anfangshöhe $z(0) = h$ als Funktion der Zeit $t$. Zum Zeitpunkt $t_\mathrm{E}$ erreicht den Massenpunkt den Boden [$z(t_\mathrm{E} = 0$)].}\label{fig: freier_Fall}
\end{figure}
Die Bewegungsgleichung $z(t)$ ist in \cref{fig: freier_Fall} dargestellt. Der Massenpunkt beschreibt eine Parabel in der $(t,z)$-Ebene mit Scheitelpunkt $z=h$. Die Falldauer $t_\mathrm{E}$ ist jener Zeitpunkt, zu dem die $z$-Koordinate verschwindet, \gDh $z(t_\mathrm{E}) = 0$. Die Falldauer lässt sich aus \cref{eq: bewegungsgleichung_z_freier_fall} bestimmen:
\begin{align}
  0 \eqexcl z(t_\mathrm{E}) &= -\frac{g}{2}t_E^2 + h & &| -h \notag \\
  0-h &= -\frac{g}{2}t_\mathrm{E}^2 & &| \cdot \left(-\frac{2}{g}\right) \notag \\
  \frac{2h}{g} &= t_\mathrm{E}^2 & &| \sqrt{\text{\,}} \notag \\
  t_\mathrm{E} &= \sqrt{\frac{2h}{g}} \label{eq: fallzeit_freier_fall}
\end{align}

\subsection{Der schräge Wurf}\label{subsec: schräge_wurf}
Der schräge Wurf ist ein weiteres Beispiel für eine gleichförmig beschleunigte Bewegung. Beim schrägen Wurf wird ein Massenpunkt von einer gewissen Höhe $z(0) = h$ mit einer Anfangsgeschwindigkeit $\ivecS{v}{0} \neq 0$ unter dem Einfluss der Gravitationsbeschleunigung [$\ivec{a} = \ivec{g} = \inlrowThree{0}{0}{\qty{-9.81}}\,\si{\meter\per\second\squared}$] abgeschossen. Ohne Beschränkung der Allgemeinheit fixieren wir die Bewegungsebene als die $(x,z)$-Ebene, womit $y \equiv 0$ und $v_y \equiv 0$ für alle Zeiten $t$. Die wirkende Beschleunigung ist wiederum die Gravitationsbeschleunigung. Die Anfangsbedingungen sind demnach 
\begin{equation}\label{eq: anfangsbed_schräger_wurf}
    \ivecS{r}{0} = \icolThree{x(0)}{y(0)}{z(0)} = \icolThree{0}{0}{h} \mComma \quad \ivecS{v}{0} = \icolThree{v_{0,x}}{0}{v_{0,z}} = \icolThree{v_0 \cos(\varphi)}{0}{v_0 \sin(\varphi)} \mComma
\end{equation}
wobei hier die Geschwindigkeit über die Koordinaten $(v_x, 0, v_z)$ [kartesische Koordinaten] oder den Abschusswinkel $\varphi$ [$\tan(\varphi) = v_{0,z}/v_{0,x}$] und den Betrag $v_0 = |\ivecS{v}{0}|$ [Polarkoordinaten] parametriert werden kann (siehe \cref{fig: schraeger_Wurf_wurfparabel}).
\begin{figure}[htb]
    \centering
    \includegraphics[width=0.5\linewidth]{Bilder/Kapitel_Mechanik/schrägerWurf.png}
    \caption{Schräger Wurf mit Anfangshöhe $z(0) = h$ und Anfangsgeschwindigkeit $\protect\ivecS{v}{0} = \inlrowThree{v_{0,x}}{0}{v_{0,z}}$ dargestellt in der $(x,z)$-Ebene.}\label{fig: schraeger_Wurf_wurfparabel}
\end{figure}
Damit ergeben sich folgende Bewegungsgleichungen
\begin{align}
    x(t) &= v_{0,x} \cdot t \mComma \label{eq: x_t_koord_schraeger_wurf}\\
    y(t) &= 0 \mComma\\
    z(t) &= -\frac{1}{2} g \cdot t^2 + v_{0,z}\cdot t + h \mDot \label{eq: z_t_koord_schraeger_wurf}
\end{align}
Die Bewegung in $x$- und $z$-Richtung sind unabhängig voneinander, da jede Koordinate nur von den Anfangsbedingungen und der Zeit $t$ abhängt. In $x$-Richtung bewegt sich das Teilchen linear gleichförmig ohne Beschleunigung. In $z$-Richtung führt das Teilchen einen senkrechten Wurf aus. \\
Aus \cref{eq: x_t_koord_schraeger_wurf} lässt sich die Funktion $t(x)$ durch Umformen auf $t$ extrahieren. Substituiert man $t \rightarrow t(x)$ in \cref{eq: z_t_koord_schraeger_wurf} erhält man die Funktion $z(x)$ (Wurfparabel). Zunächst stellt man also $x(t)$ nach $t$ um, 
\begin{equation}
    x(t) = v_{0,x}\cdot t \implies t(x) = \frac{x}{v_{0,x}}
\end{equation}
und setzen diesen Ausdruck dann in $z(t)$ ein. Damit erhalten wir
\begin{equation}\label{eq: wurfparabel_z_x_schräger_wurf}
    z(t(x)) = z(x) = - \frac{1}{2} \frac{g}{v_{0,x}^2}\cdot x^2 + \frac{v_{0,z}}{v_{0,x}}\cdot x + h \mDot
\end{equation}
Dies ist die Gleichung einer Parabel $z(x) = a x^2 + b x + c$ in der $(x,z)$-Ebene -- die sogenannte \textit{Wurfparabel}. Die maximale Höhe des Massenpunktes wird an der Position des Scheitels $x_s$ der Parabel erreicht. Der Scheitel kann über die Ableitung $\dd z/\dd x = 0$ gefunden werden,
\begin{equation}\begin{gathered}
    \frac{\dd z}{\dd x} = - \frac{g}{v_{0,x}^2} x_s + \frac{v_{0,z}}{v_{0,x}} = 0 \\
\Rightarrow x_s = \frac{v_{0,x} \cdot v_{0,z}}{g} = \frac{v_0^2 \sin(\varphi)\cos(\varphi)}{g} = \frac{v_0^2 \sin(2\varphi)}{2g}\mDot
\end{gathered}\end{equation}
Hier haben wir aus \cref{eq: anfangsbed_schräger_wurf} verwendet, dass $v_{0,x} = v_0 \cos(\varphi)$ und $v_{0,z} = v_0 \sin(\varphi)$. Außerdem wurde ein trigonometrisches Theorem angewendet: $2\sin(\varphi)\cos(\varphi) = \sin(2\varphi)$. Die Wurfweite $x_\mathrm{W}$ berechnet sich über die Tatsache, dass bei der maximalen Wurfweite die Höhe $0$ wird: $z(x_\mathrm{W}) \eqexcl 0$. Damit ergibt sich aus \cref{eq: wurfparabel_z_x_schräger_wurf}
\begin{equation}
    - \frac{1}{2} \frac{g}{v_{0,x}^2} \cdot x_\mathrm{W}^2 + \frac{v_{0,z}}{v_{0,x}} \cdot x_\mathrm{W} + h \eqexcl 0 \mDot
\end{equation}
Die quadratische Gleichung in der Variablen $x_\mathrm{W}$ hat die Lösungen (große Lösungsformel)
\begin{equation}\label{eq: wurfweite_kartesisch_schräger_wurf}
    x_\mathrm{W} = \frac{v_{0,x}\cdot v_{0,z}}{g} \pm \sqrt{\left[ {\left( \frac{v_{0,x} v_{0,z}}{g}\right)}^2 + \frac{2 v_{0,x}^2}{g}h \right]} \mDot
\end{equation}
Die Formel für die Wurfweite kann in Polarkoordinaten angegeben werden, \gDh $x_\mathrm{W}(\varphi, v_0)$, da $v_{x,0}\cdot v_{z,0} = \frac{1}{2}v_0^2 \sin(2\varphi)$. Damit wird \cref{eq: wurfweite_kartesisch_schräger_wurf} nun 
\begin{equation}\label{eq: wurfweite_polar_schräger_wurf}
    x_\mathrm{W} = \frac{v_{0}\sin(2\varphi)}{2g} \left[ v_0 + \sqrt{v_0^2 + \frac{2gh}{\sin^2(\varphi)}} \right] \mDot
\end{equation}
Diese Gleichung kann dann dazu verwendet werden, den optimalen Winkel für die maximale Wurfweite zu berechnen. Ohne Beweis wird hier festgehalten, dass der optimale Wurfwinkel, für den die maximale Wurfweite erzielt wird, $\varphi_{\mathrm{opt}} = \pi/4 = \SI{45}{\degree}$ ist.

\section{Nicht-gleichförmig beschleunigte Bewegung}
Bisher haben wir gleichförmige Bewegungen ($\ivec{a} = \ivec{0}$) und gleichförmig beschleunigte Bewegungen ($\ivec{a} = \const$) betrachtet. Nun analysieren wir Bewegungen, bei denen sich auch die Beschleunigung mit der Zeit ändert, \gDh $\ivec{a} = \ivec{a}(t)$. Dabei muss sich nicht unbedingt der Betrag der Beschleunigung ändern. Auch eine zeitliche Änderung der Richtung der Beschleunigung entspricht einer nicht-gleichförmig beschleunigten Bewegung. Zu dieser letzten Gruppe gehört die gleichförmige Kreisbewegung. 
\subsection{Gleichförmige Kreisbewegung}
Eine Bewegung entlang einer Kreisbahn mit konstanter Geschwindigkeit $|\ivec{v}(t)| = \const$ wird gleichförmige Kreisbewegung genannt. Zwar ändert sich der Betrag der Geschwindigkeit nicht, wohl aber die Richtung und daher muss eine Beschleunigung wirken
\begin{equation}
    \dot{\ivec{v}}(t) = \ivec{a}(t) \neq 0 \mDot
\end{equation}
In diesem Abschnitt möchten wir nun berechnen, wie groß diese Beschleunigung ist und wohin sie zeigt. Zunächst benötigen wir aber noch ein paar Grundbegriffe.

\subsubsection{Bogenlänge}\label{subsubsec: bogenlänge}
Die Bogenlänge $s$ entspricht dem zurückgelegten Weg auf einer Kreisbahn -- dargestellt in rot in \cref{fig: bogenlaenge}. Bei konstantem Winkel ($\varphi$) nimmt die Bogenlänge mit dem Radius zu, und bei konstantem Radius nimmt die Bogenlänge mit dem Winkel zu, daher muss 
\begin{equation}\label{eq: def_bogenlänge}
    s = R\cdot \varphi
\end{equation}
sein. Der Radius ist bei einer Kreisbewegung konstant, somit ist die differentielle Bogenlänge $\dd s = R \cdot \dd \varphi$. 
% \begin{figure}[htb]
%     \centering
%     \includegraphics[width=0.35\linewidth]{Bilder/Kapitel_Mechanik/Kapitel_Kinematik/bogenlänge_def.png}
%     \caption{Die Bogenlänge $s$ eines Kreissektors mit Radius $r$ und Öffnungswinkel $\varphi$.} 
%     \label{fig: bogenlaenge}
% \end{figure}
\begin{figure}[htb]
    \centering
    \resizebox{0.35\linewidth}{!}{
    \begin{tikzpicture}
        \def\radius{2.5}
        \def\angle{35}
        % Kreis
        \draw[black, line width=1.0pt] (0,0) circle (\radius);
        \fill (0,0) circle (1.0pt);
        % Radien
        \draw (0,0) -- (-\angle:\radius) node[pos=0.42, below=7.0pt] {\Large $R$};
        \draw (0,0) -- (\angle:\radius);
        % Winkel phi
        \draw (-\angle:0.99) arc (-\angle:\angle:0.99);
        \node at (0:0.6) {\Large $\varphi$};
        % Bogen s
        \draw[red, line width=2.0pt] (-\angle:\radius) arc (-\angle:\angle:\radius);
        \node[red] at (0:\radius+0.45) {\Large $s$};
    \end{tikzpicture}
    }
    \caption{Die Bogenlänge $s$ (rote Linie) eines Kreissektors mit Radius $R$ und Öffnungswinkel $\varphi$.}\label{fig: bogenlaenge}
\end{figure}
\subsubsection{Winkelgeschwindigkeit $\omega$}
Der (zeitlich veränderliche) Betrag der Geschwindigkeit ist weiterhin definiert als das Verhältnis von zurückgelegtem Weg und verstrichener Zeit. Damit ergibt sich für die Kreisbewegung, bei der der zurückgelegte Weg gleich der Bogenlänge $s$ ist, dass 
\begin{equation}\label{eq: betrag_geschwindigkeit_kreisbewegung}
    |\ivec{v}(t)| = v(t) = \frac{\dd s}{\dd t} = \underbrace{\frac{\dd s}{\dd \varphi}}_{= R} \cdot \underbrace{\frac{\dd \varphi}{\dd t}}_{\eqdef \omega} = R\cdot \omega \mDot
\end{equation}
Da die Abhängigkeit der Bogenlänge $s$ von der Zeit $t$ zunächst nicht direkt bekannt ist, haben wir die Ableitung $\dd s/\dd t$ mittels Kettenregel in zwei Anteile zerlegt. Der erste Anteil, $\dd s/\dd \varphi$ lässt sich einfach aus der Definition der Bogenlänge bestimmen, denn für $s = R\cdot \varphi \implies \dd s/\dd \varphi = R$. Der zweite Anteil ($\dd \varphi/\dd t$) -- die Änderung des Winkels mit der Zeit -- ist die sogenannte Winkelgeschwindigkeit $\omega$. 
\begin{importantbox}[]{Definition Winkelgeschwindigkeit}
Die Winkelgeschwindigkeit $\omega$ ist definiert als 
\begin{equation}\label{eq: def_winkelgeschwindigkeit_omega}
    \omega \defeq \frac{\dd \varphi}{\dd t}\mDot
\end{equation}
Sie gibt das Verhältnis aus zurückgelegtem Winkel $\dd \varphi$ und verstrichener Zeit $\dd t$ an. Die Maßeinheit der Winkelgeschwindigkeit ist $[\omega] = \SI{1}{\radian\per\second}$. 
\end{importantbox}
Die Winkelgeschwindigkeit ist demnach unabhängig vom Radius und somit eine praktikable Größe bei der Beschreibung von Kreisbewegungen. Die Definition in \cref{eq: def_winkelgeschwindigkeit_omega} kann auch umgestellt werden zu $\dd \varphi = \omega \cdot \dd t $, was für $\omega = \const$ elementar integriert wird zu 
\begin{equation}\begin{gathered}
    \int \dd \varphi = \int \omega\cdot  \dd t \\
    \varphi(t) = \omega \cdot t + \varphi_0 \mDot
\end{gathered}\end{equation}
Die Integrationskonstante ist hier der Startwinkel $\varphi_0 = \varphi(t=0)$. Äquivalent zum Zusammenhang zwischen Ort, Zeit und Geschwindigkeit, $s = s_0 +  v\cdot t$, hängen bei der Kreisbewegung die Größen Winkel, Zeit und Winkelgeschwindigkeit zusammen, $\varphi = \varphi_0 + \omega \cdot t$. 


\subsubsection{Geschwindigkeit}\label{subsubsec: geschwindigkeit_ebeneKreisbewegung}
Der Betrag des Geschwindigkeitsvektors ist laut \cref{eq: betrag_geschwindigkeit_kreisbewegung} $|\ivec{v}(t)| = v(t) = R \cdot \omega$. Die Richtung des Geschwindigkeitsvektors $\ivec{e}_v$ kann in Polarkoordinaten deutlich einfacher ermittelt werden. Ohne Beschränkung der Allgemeinheit betrachten wir einen Massenpunkt auf einer Kreisbahn (gegen den Uhrzeigersinn) mit Radius $R$ um den Ursprung. Für die zeitabhängigen Koordinaten gilt nun geometrisch:
\begin{equation}\label{eq: koordinaten_xy_kreisbewegung}\begin{aligned}
x(t) &= R \cdot \cos(\varphi(t)) \mComma \\
y(t) &= R \cdot \sin(\varphi(t)) \mDot
\end{aligned}\end{equation}
\begin{figure}[tbh]
    \centering
    \resizebox{0.5\linewidth}{!}{
    \begin{tikzpicture}[
        scale=2,
        axis/.style={->, >=Latex, line width=1.5pt}
    ]
    \definecolor{darkerRed}{RGB}{222, 40, 73}
    \def\axislength{3.0cm}    % The radius of the circle
    \def\radius{2.5cm}    % The radius of the circle
    \def\angleEnd{55}    % The angle phi in degrees
    % Draw the x-axis
    \draw[axis] (-\axislength,0) -- (\axislength,0) node[anchor=north west] {\LARGE $x$};
    \draw[axis] (0,-\axislength) -- (0,\axislength) node[anchor=south east] {\LARGE $y$};
    % --- Draw the circular path and sector ---
    \draw[line width=0.8pt] (0,0) circle (\radius);
    \fill[red!15, draw=black] (0,0) -- (\radius,0) arc (0:\angleEnd:\radius) -- cycle;
    % --- Add labels and annotations ---
    \draw[-, >=Latex] (0.8,0) arc (0:\angleEnd:0.8cm);
    \node at (\angleEnd/2:0.54cm) {\Large $\varphi (t)$};
    \node at (-\radius/2, 0.25) {\LARGE $R$};

    \coordinate (p) at (\angleEnd:\radius); % Define the point on the circle
    \coordinate (px) at (1.434,0); % px = r * cos(phi) = 2.5*cos(55°) = 1.434
    \draw[red, line width=1.8pt, ->, >=Latex] (\radius,0) arc (0:\angleEnd/2:\radius);
    \draw[red, line width=2.0pt] (\radius,0) arc (0:\angleEnd:\radius);
    % red vectors
    \draw[->, >={Latex[length=6mm]}, darkerRed, line width=2.0pt] (0,0) -- (p) node[left=2pt, pos=0.7] {\LARGE $\ivec{r}(t)$};
    \draw[line width=1.6pt, dashed, darkerRed] (px) -- (p) node[midway, right=3pt] {\Large $y(t)$};
    \draw[line width=1.6pt, dashed, darkerRed] (0,0.01) -- (1.434,0.01) node[midway, below=3pt] {\Large $x(t)$};
    \node[circle, fill=black, inner sep=1.9pt] at (p) {};
    \end{tikzpicture}
    }
    \caption{Bei der ebenen Kreisbewegung zeigt der Ortsvektor $\protect\ivec{r}(t)$ vom Ursprung zur Position am Kreis. Die $x$- und $y$-Koordinaten können durch den Kosinus und Sinus des Winkels $\varphi(t)$ angegeben werden.}\label{fig: kreisbewegung_v_tangential}
\end{figure}
Setzen wir $\varphi(t) = \omega \cdot t + \varphi_0$ in \cref{eq: koordinaten_xy_kreisbewegung} ein, erhalten wir die Bewegungsgleichungen der ebenen Kreisbewegung. 
\begin{importantbox}{Bewegungsgleichungen der ebenen Kreisbewegung}
    Die Bewegungsgleichungen eines Massenpunktes bei der ebenen, gleichförmigen Kreisbewegung mit konstanter Winkelgeschwindigkeit $\omega$ ($\omega = \const$) um den Ursprung $M=\ipTwo{0}{0}$ sind: 
    \begin{equation}\label{eq: kreisbewegung_xy_polarkoord}\begin{aligned}
        x(t) &= R \cdot \cos(\omega t + \varphi_0)  \\
        y(t) &= R \cdot \sin(\omega t + \varphi_0)
    \end{aligned}\end{equation}
    Der Ortsvektor des Massenpunktes ergibt sich damit allgemein zu: 
    \begin{equation}\label{eq: ortsvektor_gleichf_kreisbewegung_polar}
        \ivec{r}(t) = R \icolTwo{\cos(\omega t + \varphi_0)}{\sin(\omega t + \varphi_0)} = R \cdot \ivecS{e}{r}(t)\mDot
    \end{equation} 
\end{importantbox}
Der radiale Einheitsvektor $\ivecS{e}{r}(t)$ zeigt immer vom Ursprung des Kreises zur Position des Massenpunktes am Kreis. Ist der Mittelpunkt der Kreisbewegung nicht der Ursprung, dann muss zu \cref{eq: ortsvektor_gleichf_kreisbewegung_polar} noch der Ortsvektor des Ursprungs ($\ivecS{r}{M} = \inlrowTwo{x_M}{y_M}^\T$) addiert werden: 
\begin{equation}
    \ivec{r}(t) = \ivecS{r}{M} + R\cdot \ivecS{e}{r}(t) = \icolTwo{x_M}{y_M} + R \icolTwo{\cos(\omega t + \varphi_0)}{\sin(\omega t + \varphi_0)} \mDot
\end{equation}
Nun berechnen wir den zeitabhängigen Geschwindigkeitsvektor über die Ableitung des Ortsvektors aus \cref{eq: ortsvektor_gleichf_kreisbewegung_polar}, wobei wir \oBdA $\varphi_0 = 0$ und $\ivecS{r}{M} = \ivec{0}$ setzen: 
\begin{equation}\label{eq: herleitung_geschwvektor_gleichf_kreisbew}
     \ivec{v}(t) = \frac{\dd \ivec{r}(t)}{\dd t} = \frac{\dd}{\dd t} \bigg[ R \underbrace{\icolTwo{\cos(\omega t)}{\sin(\omega t) }}_{\ivecS{e}{r}} \bigg] = R \cdot \omega \underbrace{\icolTwo{-\sin(\omega t)}{\cos(\omega t)}}_{\ivecS{e}{v}} \mDot
\end{equation}
\begin{figure}[tb]
    \centering
    \resizebox{0.50\linewidth}{!}{
    \begin{tikzpicture}[
            scale=2,
            axis/.style={->, >=Latex, line width=1.8pt},
        ]
        \def\radius{2cm}    % The radius of the circle
        \def\vlength{1.2cm}
        \def\angleEnd{40}    % The angle phi in degrees
        \definecolor{darkerRed}{RGB}{222, 40, 73}
        % Draw the x,y-axis
        \draw[axis] (-2.5,0) -- (2.5,0) node[anchor=north west] {\LARGE $x$};
        \draw[axis] (0,-2.5) -- (0,2.5) node[anchor=south east] {\LARGE $y$};
        % Draw the main circle outline
        \draw[line width=0.8pt] (0,0) circle (\radius);        
        % --- Add labels and annotations ---
        \draw[->, >=Latex] (0.7,0) arc (0:\angleEnd:0.7cm);
        \node at (\angleEnd/2:0.44cm) {\LARGE $\varphi$};
        \coordinate (p) at (\angleEnd:\radius); % Define the point on the circle | (1.53209, 1.28558)
        \coordinate (px) at (1.532,0); % Define the point on the circle
        % blue vector
        \draw[->, >={Latex[length=5mm]}, blue, line width=2.0pt] (p) -- ++(\angleEnd+90:\vlength) node[above=8.0pt, pos=0.2] {\Large $\ivec{v}(t)$};
        \draw[line width=1.9pt, dashed, blue] (p) -- (0.760744, 1.28558) node[pos=0.7, below=3pt] {\Large $v_x$};
        \draw[line width=1.9pt, dashed, blue] (0.760744, 1.28558) -- (0.760744, 2.20483) node[pos=0.4, left=3pt] {\Large $v_y$};
        % red vectors
        \draw[->, >={Latex[length=5mm]}, darkerRed, line width=2.0pt] (0,0) -- (p) node[left=6pt, pos=0.6] {\Large $\ivec{r}(t)$};
        \draw[line width=1.9pt, dashed, darkerRed] (px) -- (p) node[midway, right=3pt] {\Large $y$};
        \draw[line width=1.9pt, dashed, darkerRed] (0,0.012) -- (1.532,0.012) node[midway, below=3pt] {\Large $x$};
        \node[circle, fill=blue, inner sep=1.5pt] at (p) {};
        \end{tikzpicture}
        }
    \caption{Ein Geschwindigkeitsvektor $\protect\ivec{v}(t)$ (blauer Pfeil) steht immer senkrecht auf den Ortsvektor $\protect\ivec{r}(t)$ (roter Pfeil).}\label{fig: geschw_senkrecht_ortsvektor_kreisbew}
\end{figure}
\begin{importantbox}{Geschwindigkeitsvektor $\ivec{v}(t)$ der gleichförmigen Kreisbewegung}
    Der Geschwindigkeitsvektor der gleichförmigen ebenen Kreisbewegung lautet ganz allgemein
    \begin{equation}\label{eq: geschwindigkeit_gleichf_kreisbew}
        \ivec{v}(t) = v \cdot \ivecS{e}{v} = (R \cdot \omega) \icolTwo{-\sin(\omega t + \varphi_0)}{\cos(\omega t + \varphi_0)}
    \end{equation}
    und steht immer senkrecht auf den Ortsvektor $\ivec{r}(t)$. Die Geschwindigkeit zeigt damit in Richtung der Tangente an die Kreisbahn (siehe \cref{fig: geschw_senkrecht_ortsvektor_kreisbew}). Der Betrag der Geschwindigkeit  
    \begin{equation}\label{eq: betrag_geschwindigkeit_gleichf_kreisbew}
        v = R \cdot \omega
    \end{equation}
    ist unabhängig von der Zeit. Die Einheit der Geschwindigkeit ist weiterhin $[v] = [R \cdot \omega] = \si{\meter\per\second}$. 
\end{importantbox}
Wir können mithilfe des Skalarproduktes zeigen, dass der Einheitsvektor der Geschwindigkeit $\ivecS{e}{v}$ senkrecht auf den Einheitsvektor des Ortes $\ivecS{e}{r}$ steht:
\begin{gather*}
    \ivecS{e}{v} \cdot \ivecS{e}{r} = \icolTwo{-\sin(\omega t)}{\cos(\omega t)} \cdot \icolTwo{\cos(\omega t)}{\sin(\omega t)} = -\sin(\omega t)\cos(\omega t) + \cos(\omega t)\sin(\omega t) = 0 \\
    \implies  \ivecS{e}{v} \perp \ivecS{e}{r} \mDot
\end{gather*}
Das kann auch geometrisch anhand von \cref{fig: geschw_senkrecht_ortsvektor_kreisbew} leicht überprüft werden. Für den Ortsvektor gilt, dass $\frac{y}{x} = \tan(\varphi)$ ist, während für den Geschwindigkeitsvektor $\frac{v_y}{v_x} = -\tan^{-1}(\varphi) = \tan(\varphi + \pi/2)$. Die letzte Gleichheit ist ein bekanntes Theorem für den Tangens. Sei die Richtung des Ortsvektors durch $\varphi$ gegeben, so ist die Richtung des Geschwindigkeitsvektors demnach $\varphi + \pi/2$ (bei mathematisch positiver Umlaufrichtung).  


\subsubsection{Beschleunigung}\label{subsubsec: beschleunigung_gleichf_kreisbew}
Die Beschleunigung der gleichförmigen Kreisbewegung kann ebenso geometrisch und analytisch berechnet werden. Zunächst betrachten wir die geometrische Herleitung des Betrags der Beschleunigung 
\begin{equation}
    a(t) = |\ivec{a}(t)| =  \lim_{\Delta t \to 0} \left|\frac{\Delta \ivec{v}(t)}{\Delta t} \right| = \left| \frac{\dd \ivec{v}(t)}{\dd t} \right| \mComma
\end{equation}
wobei hier $\Delta \ivec{v}(t) = \ivec{v}(t+\Delta t) - \ivec{v}(t)$.
\paragraph{Geometrische Betrachtung}
Man betrachte eine gleichförmige Kreisbewegung mit Radius $R$ und Mittelpunkt $C$ zu zwei verschiedenen Zeitpunkten mit Positionen $A$ und $B$, dargestellt in \cref{fig: beschleunigung_geometrisch_aehnliche_dreiecke}. Das Dreieck $\triangle ABC$ ist ein gleichschenkeliges Dreieck mit Seitenlänge $R$. Der spitze Winkel $\Delta \varphi$ hat eine gegenüberliegende Seitenlänge $\Delta \ivec{r} = \ivecS{r}{B}-\ivecS{r}{A}$ (blaue, strichlierte Linie). Da die Geschwindigkeitsvektoren $\ivecS{v}{A}$ und $\ivecS{v}{B}$ senkrecht auf ihre Ortsvektoren stehen und ihre Beträge gleich sind ($|\ivecS{v}{A}| = |\ivecS{v}{B}| = v$), ist das Dreieck, das von $\ivecS{v}{A}$, $\ivecS{v}{B}$ und $\Delta\ivec{v}$ gebildet wird, ein ähnliches Dreieck zum Dreieck $\triangle ABC$.
\begin{figure}[tb]
    \centering
    \resizebox{0.65\linewidth}{!}{
    \begin{tikzpicture}[
        vec/.style={-{Stealth}, ultra thick, green!50!black},
        point/.style={fill, circle, inner sep=1.pt}
        ]
        \def\myradius{4.5cm} % Radius des Kreises
        \def\angleA{110}     % Winkel für Punkt A (in Grad)
        \def\angleB{140}     % Winkel für Punkt B (in Grad)
        \def\vecLen{3cm}      % Länge der Geschwindigkeitsvektoren
        \def\fontSize{\large}
        % --- 2. Koordinaten definieren ---
        \coordinate (C) at (0,0);
        \coordinate (A) at (\angleA:\myradius);
        \coordinate (B) at (\angleB:\myradius);
        \coordinate (M_chord) at ($(A)!0.35!(B)$);
        \coordinate (M_arc) at ({(5*\angleA+5*\angleB)/(10)}:\myradius);
        % --- 3. Kreisbahn zeichnen ---
        \draw[thick] (\angleA-20:\myradius) arc (\angleA-20:\angleB+25:\myradius);    
        \draw[line width=2.2pt,red] (\angleA:\myradius) arc (\angleA:\angleB:\myradius);    
        % --- 4. Radien und Punkte zeichnen ---
        \draw (C) -- (A) node[pos=0.5, right=2pt] {\fontSize $R$};
        \draw (C) -- (B) node[pos=0.5, below left=1pt] {\fontSize $R$};
        % --- 5. Winkel und Bogenlänge beschriften ---
        \draw[-{Stealth}] (\angleA:1.8cm) arc (\angleA:\angleB:1.8cm);
        \node at ({(\angleA+\angleB)/2}:1.3cm) {\fontSize $\Delta \varphi$};
        % --- 6. Geschwindigkeitsvektoren ---
        \draw[vec] (A) -- ++(\angleA+90:\vecLen) node[pos=0.5, above=3.0pt] {\fontSize $\ivecS{v}{A}$};
        \draw[vec] (B) -- ++(\angleB+90:\vecLen) node[pos=0.5, above left=0.2pt] {\fontSize $\ivecS{v}{B}$};
        \draw[dashed, blue, line width=1.1pt] (A) -- (B);
        % Für Δr: Linie vom Mittelpunkt der Sehne (A)--(B)
        \draw[-, thick, blue] (M_chord) -- (-1.7,2.6) node[below] {\fontSize $\Delta \ivec{r}$};
        % Für Δs: Linie vom Mittelpunkt des Bogens zwischen A und B
        \draw[-, thick,red] (M_arc) -- (-2.3,3.1) node[below=0.0pt] {\fontSize $\Delta s$};
        \node[point, label={[label distance=3pt]below:C}] at (C) {};
        \node[point, label={[label distance=0pt]above:A}] at (A) {};
        \node[point, label={[label distance=2pt]below:B}] at (B) {};
    
        % zweiter Teil: 
        \coordinate (F) at (-1.8*\myradius,0.7*\myradius);
        \coordinate (TipV1) at ($(F) + (\angleA+90:\vecLen)$);
        \coordinate (TipV2) at ($(F) + (\angleB+90:\vecLen)$);
        \draw[vec] (F) -- (TipV1) node[pos=0.5, above=3.0pt] {\fontSize $\ivecS{v}{A}$};
        \draw[vec] (F) -- (TipV2) node[pos=0.5, below right=0.2pt] {\fontSize $\ivecS{v}{B}$};
        \draw[vec] (TipV1) -- (TipV2) node[pos=0.5, left=4.5pt] {\fontSize $\Delta \ivec{v}$};
        \draw[-{Stealth}] (F)++(\angleA+90:1.1) arc (\angleA+90:\angleB+90:1.1cm);
        \draw[-, thick]  ($(F)-(0.6,0.4)$) -- +(0.2,0.6) node[above=0.0pt] {\fontSize $\Delta \varphi$};
    \end{tikzpicture}
    }
    \caption{Geometrische Herleitung der Beschleunigung. Die Dreiecke $(ABC)$ und ($\protect\ivecS{v}{1}, \protect\ivecS{v}{2}, \Delta\protect\ivec{v}$) sind ähnlich, da beide gleichschenklige Dreiecke mit dem selben spitzen Winkel sind.}\label{fig: beschleunigung_geometrisch_aehnliche_dreiecke}
\end{figure}
Daraus erhalten wir die Beziehung:
\begin{equation}\label{eq: verhältnis_dv_dr_gleichf_kreisbew}
    \frac{|\Delta \ivec{r}|}{R} = \frac{|\Delta \ivec{v}|}{v} \implies |\Delta \ivec{v}| = \frac{v}{R} |\Delta \ivec{r}| \mDot
\end{equation}
Nun dividieren wir \cref{eq: verhältnis_dv_dr_gleichf_kreisbew} durch $|\Delta t|$ und bilden den Limes $\Delta t\to 0$, bei dem die endlichen Differenzen in Differentiale übergehen, und erhalten somit einen Ausdruck für den Betrag der Beschleunigung
\begin{equation}\label{eq: betrag_beschl_gleichf_kreisbew_noch_mit_limes}
    \lim_{\Delta t\to 0}\left|\frac{\Delta \ivec{v}}{\Delta t}\right| = \left| \frac{\dd \ivec{v}}{\dd t} \right| = |\ivec{a}(t)| = \lim_{\Delta t\to 0} \frac{v}{R} \left|\frac{\Delta \ivec{r}}{\Delta t}\right| \mDot
\end{equation}
\Cref{eq: betrag_beschl_gleichf_kreisbew_noch_mit_limes} enthält auf der rechten Seite allerdings noch den Ausdruck $|\Delta \ivec{r}/\Delta t|$, der die Sehnenlänge $|\Delta \ivec{r}|$ enthält. Für endliche Winkeländerungen $\Delta \varphi$ entspricht die Sehnenlänge nicht dem zurückgelegten Weg des Massenpunktes (siehe \cref{fig: beschleunigung_geometrisch_aehnliche_dreiecke}), weshalb $|\Delta \ivec{r}/\Delta t| \neq |\ivec{v}(t)|$ ist. In \cref{subsec: herleitung_bogenlaenge_sehnenlaenge} haben wir jedoch gezeigt, dass
\begin{equation}
    \lim_{\Delta t \to 0} |\Delta \ivec{r}| = \Delta s
\end{equation} 
[siehe \cref{eq: sehnenlänge_gleich_bogenlänge}] -- \gDh die Länge der Sehne $\overline{AB}$ konvergiert im Grenzübergang $\Delta t \to 0$ gegen die Bogenlänge. \\

Schließlich können wir im Limes die Sehnenlänge durch die Bogenlänge ersetzen und somit wird $\lim_{\Delta t \to 0} |\Delta \ivec{r}/\Delta t| = v$. In \cref{eq: betrag_beschl_gleichf_kreisbew_noch_mit_limes} wird der Betrag der Beschleunigung dann
\begin{equation}
    a(t) = |\ivec{a}(t)| = \lim_{\Delta t \to 0} \frac{v}{R} \left| \frac{\Delta \ivec{r}}{\Delta t} \right|  = \frac{v}{R} \cdot v = \frac{v^2}{R}
\end{equation}
Um die Richtung des Beschleunigungsvektors zu bestimmen, verwenden wir zunächst die Tatsache, dass der Geschwindigkeitsvektor $\ivec{v}(t)$ normal auf den Ortsvektor $\ivec{r}(t)$ steht und parallel zum differentiellen Verschiebungsvektor\footnote{Die geschlossene Formel für den differentiellen Verschiebungsvektor lautet \begin{equation}
    \dd \ivec{r} =  \left[\lim_{\Delta t \to 0} \left(\frac{\ivecS{r}{B} - \ivecS{r}{A}}{\Delta t}\right)\right] \dd t = \ivec{v} \dd t \mDot
\end{equation}} $\dd \ivec{r}$ ist: 
\begin{equation}\label{eq: er_perp_ev_parallel_dr}
    \ivecS{e}{r} \perp \ivecS{e}{v} \quad\mAND\quad \ivecS{e}{v} \parallel \dd \ivec{r} \mDot
\end{equation}
Dann muss aber der differentielle Geschwindigkeitsvektor\footnote{Die geschlossene Formel für den differentiellen Geschwindigkeitsvektor lautet \begin{equation}
    \dd \ivec{v} =  \left[\lim_{\Delta t \to 0} \left(\frac{\ivecS{v}{B} - \ivecS{v}{A}}{\Delta t}\right)\right] \dd t = \ivec{a} \dd t \mDot
\end{equation}}, $\dd \ivec{v}$,
\begin{equation}
    \dd \ivec{v} \parallel \ivecS{e}{a} \quad\mAND\quad \ivecS{e}{a} \perp \ivecS{e}{v}
\end{equation}
sein. Dass $\dd \ivec{v} \parallel \ivecS{e}{a}$, erschließt sich aus der Definition der Beschleunigung, $\ivec{a} = \dd \ivec{v}/\dd t$. Die Orthogonalität von $\ivecS{e}{a}$ mit $\ivecS{e}{v}$ ist bedingt durch $\ivec{r} \perp \ivec{v}$ und daher $\ivec{v} \perp \ivec{a}$. Es gilt daher, dass 
\begin{equation}
    \ivecS{e}{a} = -\ivecS{e}{r} 
\end{equation}
ist und somit zeigt der Beschleunigungsvektor $\ivec{a}$ zum Mittelpunkt des Kreises (antiparallel zum Ortsvektor) 
\begin{equation}
    \ivec{a}(t) = \frac{v}{R^2} \ivecS{e}{a} = -\frac{v}{R^2} \ivecS{e}{r}\mDot
\end{equation}



\paragraph{Analytische Betrachtung}
Die analytische Betrachtung geht von der Definition der Beschleunigung aus und führt die Differentiation mithilfe der Produktregel (siehe \cref{sec: Ableitungsregeln}) aus:
\begin{equation}
    \ivec{a}(t) = \frac{\dd \ivec{v}(t)}{\dd t} = \frac{\dd}{\dd t}(v \cdot \ivecS{e}{v}) = \underbrace{\frac{\dd v}{\dd t}}_{=0} \cdot \ivecS{e}{v} + v \cdot \frac{\dd \ivecS{e}{v}}{\dd t} = v \cdot \frac{\dd \ivecS{e}{v}}{\dd t} \mDot
\end{equation}
Die Änderung des Betrages der Geschwindigkeit verschwindet ($\dd v/\dd t = 0$), weil bei der gleichförmigen Kreisbewegung $v = \const$. 
Die Ableitung des Einheitsvektors $\ivecS{e}{v}$ der Geschwindigkeit (siehe \cref{eq: geschwindigkeit_gleichf_kreisbew}) ist
\begin{equation}
    \frac{\dd \ivecS{e}{v}}{\dd t} = \frac{\dd}{\dd t} \icolTwo{-\sin(\omega t)}{\cos(\omega t)} = -\omega \underbrace{\icolTwo{\cos(\omega t)}{\sin(\omega t)}}_{= \ivecS{e}{r}}  = -\omega \cdot \ivecS{e}{r} \mComma
\end{equation}
wobei wir wieder \oBdA $\varphi_0 = 0$ gewählt haben. Damit erhält man für den Beschleunigungsvektor
\begin{equation}
    \ivec{a}(t) = v \cdot \frac{\dd \ivecS{e}{v}}{\dd t} = v \cdot (-\omega \cdot \ivecS{e}{r}) = -(v \cdot \omega) \cdot \ivecS{e}{r} = - \frac{v^2}{R} \ivecS{e}{r}\mDot
\end{equation}
\begin{importantbox}{Zentripetalbeschleunigung $\ivec{a}(t)$}
    Der Beschleunigungsvektor der ebenen Kreisbewegung lautet
    $$ \ivec{a}(t) = a \cdot \ivecS{e}{a} = -\frac{v^2}{r} \cdot \ivecS{e}{r} = -\omega^2 \cdot r \cdot \ivecS{e}{r} $$
    und zeigt zum Kreismittelpunkt. Deshalb nennt man sie Zentripetalbeschleunigung (\gDQ{mittelpunktsuchende Beschleunigung}). Ihre Einheit beträgt $[\frac{v^2}{r}] = \si{\meter\per\second\squared}$.
\end{importantbox}

\section{Die allgemeine krummlinige Bewegung}
Bei der gleichförmigen Kreisbewegung nimmt die Beschleunigung eine sehr einfache Form an, da der Betrag der Geschwindigkeit konstant ist, $|\ivec{v}| = \const$. Im allgemeinen Fall wird sich $\ivec{v}$ aber nach Betrag und Richtung ändern und zwar weil sich die Beschleunigung sowohl nach Betrag als auch nach Richtung ändert. \\

Die Geschwindigkeit $\ivec{v}$ steht jedoch -- unabhängig von der Bewegungsform -- immer tangential zur Bahnkurve. Es ergibt sich nämlich aus der Definition $\ivec{v} = \frac{\dd \ivec{r}}{\dd t}$, dass die Geschwindigkeit parallel zur differentiellen Verschiebung ist. Die Beschleunigung $\ivec{a}$ kann eine beliebige Richtung aufweisen und lässt sich in zwei Komponenten zerlegen: 

\begin{importantbox}{Beschleunigung bei der krummlinigen Bewegung}
Die Beschleunigung $\ivec{a}(t)$ kann im Allgemeinen eine beliebige Richtung aufweisen
    \begin{equation}
        \ivec{a}(t) = \frac{\dd \ivec{v}(t)}{\dd t} = \frac{\dd}{\dd t}(v(t) \cdot \ivecS{e}{v}(t)) = \underbrace{\frac{\dd v(t)}{\dd t} \cdot \ivecS{e}{v}(t)}_{\ivecS{a}{t}(t)} + \underbrace{v(t) \cdot \frac{\dd \ivecS{e}{v}(t)}{\dd t}}_{\ivecS{a}{n}(t)} = \ivecS{a}{t}(t) + \ivecS{a}{n}(t)
    \end{equation}
und in die Tangential- und Normalbeschleunigung zerlegt werden.
\end{importantbox}
\begin{figure}
    \centering
    \includegraphics[width=0.65\linewidth]{Bilder/Kapitel_Mechanik/Kapitel_Kinematik/krummlinigeBewBahnkurve.png}
    \caption{Die Geschwindigkeit $\protect\ivec{v}(t)$ steht immer tangential auf die Bahnkurve. Die Beschleunigung hat jedoch zwei Anteile, die normal aufeinander stehen: die Tangentialbeschleunigung $\protect\ivecS{a}{t}$ und die Normalbeschleunigung $\protect\ivecS{a}{n}$.}\label{fig: allg_krummlinige_bewegung}
\end{figure}
\begin{rememberbox}{Tangential- und Normalbeschleunigung}
    \begin{itemize}
        \item \textbf{Die Tangentialbeschleunigung $\ivecS{a}{t} = \frac{\dd v}{\dd t} \cdot \ivecS{e}{v}$} ist ein Vektor, der in Tangentialrichtung der Bahnkurve zeigt, also parallel zu $\ivec{v}$. Sein Betrag $|\ivecS{a}{t}| = \frac{\dd v}{\dd t}$ beschreibt die Änderungsrate des Betrages der Geschwindigkeit.
        \item \textbf{Die Normalbeschleunigung $\ivecS{a}{n} = v \cdot \frac{\dd\ivecS{e}{v}}{\dd t}$} ist ein Vektor, der senkrecht auf die Geschwindigkeit und damit senkrecht auf die Tangente steht. Dieser Anteil beschreibt die Änderung der Richtung der Geschwindigkeit.
    \end{itemize}
\end{rememberbox}
Während die Tangentialbeschleunigung also die umgangssprachliche Erhöhung (bzw. Verringerung) der Geschwindigkeit bewirkt, veranlasst die Normalbeschleunigung einen Richtungswechsel und lenkt den Massenpunkt sozusagen. 

% Chapter end - always start new page after chapter
\newpage
%\chapter{Dynamik}\label{chap: Dynamik}
Nachdem wir die Grundlagen der unterschiedlichen Bewegungsformen in der Kinematik behandelt haben, wenden wir uns nun der Dynamik zu, also der Frage, warum ein Körper gerade die beobachtete Bewegung ausführt. 
\begin{figure}[h!]
    \centering
    \begin{tikzpicture}[
    main_node/.style={
        rectangle,
        fill=blue!60!black,
        text=white,
        font=\sffamily\bfseries\Large,
        minimum width=3.8cm,
        minimum height=1.5cm,
        text centered,
    },
    % Definiert den Stil für die Beschreibungstexte
    desc_node/.style={
        rectangle,
        draw=black, % Rand in der Farbe der Unterboxen
        font=\sffamily,
        align=center,
        text width=4cm,
        inner ysep=4pt,
        inner xsep=0pt,
    }
    ]
    \node[main_node] (mechanik) {Mechanik};
    \node[rectangle,draw=cyan!80!blue,line width=1pt, text=black, minimum width=4.5cm,minimum height=1.2cm,text centered,font=\sffamily\bfseries\Large,below left=1.3cm and -0.3cm of mechanik] (kinematik) {Kinematik};
    \node[rectangle,fill=cyan!40!blue, text=white,line width=1pt,minimum width=4.5cm,minimum height=1.2cm,text centered,font=\sffamily\bfseries\Large,below right=1.3cm and -0.3cm of mechanik] (dynamik) {Dynamik};
    % Beschreibungstexte über die Unterboxen legen
    \node[desc_node, minimum height=1.5cm,below=0.1cm of kinematik] {Gesetze der Bewegung \\ (ohne Kräfte)};
    \node[desc_node, minimum height=1.5cm,below=0.1cm of dynamik] {Wirkung von Kräften};
    % Verbindungslinien zeichnen
    \coordinate[below=0.75cm of mechanik] (midpoint);
    \draw[black, thick] (mechanik.south) -- (midpoint);
    \draw[black, thick] (kinematik.north) -- (midpoint) -- (dynamik.north);
    \end{tikzpicture}
\end{figure}
\vspace{0.3cm}


\section{Einleitung und die vier Grundkräfte}\label{sec: dynamik_einleitung}
Isaac Newton erkannte, dass die Ursache für den Bewegungszustand eines Körpers und dessen Änderungen ein Resultat von Wechselwirkungen dieses Körpers mit seiner Umgebung sein muss. Diese Erkenntnisse gipfelten in den drei Newtonschen Axiomen, die die Basis für die klassische Mechanik darstellen. Die Newtonschen Axiome stellen den Zusammenhang zwischen der Bewegungsänderung eines Körpers und den auf ihn einwirkenden Kräften her. In der modernen Physik kennen wir vier Grundkräfte (fundamentale Wechselwirkungen):
\begin{table}[htbp]
    \centering
    \begin{tabular}{l l l}
        \toprule
        \textbf{Wechselwirkung} & \textbf{Relative Stärke} & \textbf{Reichweite} \\
        \midrule
        Elektromagnetismus & \num{1} & unendlich \\
        Gravitation & $\approx \SI{e-36}{}$ & unendlich \\
        Starke Wechselwirkung & \num{100} & $\approx \SI{e-15}{\meter}$ \\
        Schwache Wechselwirkung & $\approx \SI{e-11}{} $ & $\approx \SI{e-18}{\meter}$ \\
        \bottomrule
    \end{tabular}
    %\caption{Die vier fundamentalen Wechselwirkungen der Physik.}
    \label{tab: vier_grundkraefte_der_physik}
\end{table}

\begin{figure}[htbp]
    \centering
    % (1) Gravitation
    \begin{subfigure}{0.22\textwidth} % NOTE: New environment syntax
        \begin{tikzpicture}
            \node[anchor=south west, inner sep=0] (image) at (0,0) {\includegraphics[width=\textwidth]{Bilder/Kapitel_Mechanik/Kapitel_Dynamik/tabelle_WW_1_Gravitation.jpg}};
            \node[anchor=north west, fill=white, fill opacity=0.8, text opacity=1, inner sep=2pt, rounded corners=2pt] at (image.north west) {\textbf{(1)}};
        \end{tikzpicture}
        \caption*{} % Optional: for spacing and labeling
        \label{fig:grundkraft_gravitation}
    \end{subfigure}
    \hfill % This still works to space them out
    % (2) Elektromagnetismus
    \begin{subfigure}{0.22\textwidth}
        \begin{tikzpicture}
            \node[anchor=south west, inner sep=0] (image) at (0,0) {\includegraphics[width=\textwidth]{Bilder/Kapitel_Mechanik/Kapitel_Dynamik/tabelle_WW_2_Elektromagnetismus.jpg}};
            \node[anchor=north west, fill=white, fill opacity=0.8, text opacity=1, inner sep=2pt, rounded corners=2pt] at (image.north west) {\textbf{(2)}};
        \end{tikzpicture}
        \caption*{}
        \label{fig:grundkraft_elektro}
    \end{subfigure}
    \hfill
    % (3) Starke Wechselwirkung
    \begin{subfigure}{0.22\textwidth}
        \begin{tikzpicture}
            \node[anchor=south west, inner sep=0] (image) at (0,0) {\includegraphics[width=\textwidth]{Bilder/Kapitel_Mechanik/Kapitel_Dynamik/tabelle_WW_3_starkeWechselwirkung.jpg}};
            \node[anchor=north west, fill=white, fill opacity=0.8, text opacity=1, inner sep=2pt, rounded corners=2pt] at (image.north west) {\textbf{(3)}};
        \end{tikzpicture}
        \caption*{}
        \label{fig:grundkraft_stark}
    \end{subfigure}
    \hfill
    % (4) Schwache Wechselwirkung
    \begin{subfigure}{0.22\textwidth}
        \begin{tikzpicture}
            \node[anchor=south west, inner sep=0] (image) at (0,0) {\includegraphics[width=\textwidth]{Bilder/Kapitel_Mechanik/Kapitel_Dynamik/tabelle_WW_4_schwacheWechselwirkung.jpg}};
            \node[anchor=north west, fill=white, fill opacity=0.8, text opacity=1, inner sep=2pt, rounded corners=2pt] at (image.north west) {\textbf{(4)}};
        \end{tikzpicture}
        \caption*{}
        \label{fig:grundkraft_schwach}
    \end{subfigure}

    \caption{Die vier fundamentalen Wechselwirkungen der Physik: (1) Gravitation, (2) Elektromagnetismus, (3) Starke Wechselwirkung und (4) Schwache Wechselwirkung.}
    \label{fig: grundkraefte_bilder}
\end{figure}

\section{Die Newtonschen Axiome}\label{sec: newtonsche_axiome}
Die drei Newtonschen Axiome sind die Grundpfeiler der klassischen Mechanik. Sie wurden von Isaac Newton formuliert, um die Ursache von Bewegungsänderungen zu erklären und damit die Dynamik zu begründen. 

Als Axiome sind sie fundamentale Lehrsätze einer Theorie, die nicht bewiesen, sondern als wahr angenommen werden, da ihre Folgerungen mit den Beobachtungen und Experimenten in der Natur übereinstimmen. Sie schaffen eine präzise Verbindung zwischen dem Begriff der Kraft und der daraus resultierenden Bewegung eines Körpers. Die Axiome sind logisch aufeinander aufgebaut und decken die grundlegenden Szenarien der Krafteinwirkung ab:

Das \textbf{erste Newtonsche Axiom} (Trägheitsgesetz) beschreibt den Zustand, in dem keine resultierende Kraft auf einen Körper wirkt. \\
Das \textbf{zweite Newtonsche Axiom} (Aktionsprinzip) erklärt, was geschieht, wenn eine resultierende Kraft auf einen Körper wirkt. \\
Das \textbf{dritte Newtonsche Axiom} (Wechselwirkungsprinzip) charakterisiert Kräfte als Ergebnis einer Wechselwirkung zwischen zwei Körpern. Es besagt, dass Kräfte immer paarweise auftreten (\gDQ{actio = reactio}).

\subsection{Erstes Newtonsches Axiom (Trägheitsgesetz)}\label{subsec: erstes_newtonsches_axiom}
\begin{importantbox}{1. Newtonsches Axiom}
    Ein Körper verharrt im Zustand der Ruhe oder der gleichförmigen geradlinigen Bewegung, solange keine äußeren Kräfte auf ihn wirken.
    \begin{equation}\label{eq: erstes_newtonsches_axiom_trägheit}
        \sum_{i}\ivec{F}_{i} = 0
    \end{equation}
\end{importantbox}

Einen Körper, auf den überhaupt keine Kraft wirkt oder für den die Vektorsumme aller Kräfte null ergibt, nennen wir \textbf{frei}. Die Eigenschaft eines Körpers, seinen Bewegungszustand beizubehalten, solange er nicht durch äußere Kräfte zu einer Änderung gezwungen wird, bezeichnen wir als \textbf{Trägheit}. Das 1. Newtonsche Axiom wird daher auch \textbf{Trägheitsgesetz} genannt.

Das Axiom gilt nur in sogenannten \textbf{Inertialsystemen}, also in nicht beschleunigten (und nicht rotierenden) Bezugssystemen. In einem beschleunigten Bezugssystem kann ein Körper auch ohne äußere Kraft seinen Bewegungszustand ändern. 

\begin{rememberbox}[]{Inertialsysteme}
    Ein Inertialsystem wird sogar meist über das 1. Newtonsche Axiom definiert: Jedes Bezugssystem, in dem sich ein kräftefreier Körper in Ruhe verharrt oder geradlinig gleichförmig bewegt, ist ein Inertialsystem. Alle Systeme, die sich mit konstanter Geschwindigkeit relativ zu einem Inertialsystem bewegen, sind ebenfalls Inertialsysteme.
\end{rememberbox}

\subsection{Grundbegriffe: Kraft, Masse und Impuls}\label{subsec: grundbegriffe_dynamik}

\begin{rememberbox}[]{Definition: Kraft}
    Eine \textbf{Kraft} ist eine äußere Einwirkung auf einen Körper, infolgedessen sich die Geschwindigkeit (und damit der Impuls) des Körpers ändert. Dies führt zu einer Beschleunigung des Körpers relativ zum Inertialsystem. Eine Kraft ist eine vektorielle Größe; sie kann sowohl Betrag als auch Richtung der Geschwindigkeit ändern. Die Einheit der Kraft ist das Newton: $[F] = \SI{1}{\newton} = \SI{1}{\kilogram\meter\per\second\squared}$.
\end{rememberbox}

\begin{rememberbox}[]{Definition: Masse}
    Körper besitzen einen inneren Widerstand gegen jegliche Art von Beschleunigung. Diese Eigenschaft wird \textbf{Masse} genannt. Sie ist ein Maß für die Trägheit eines Körpers. Bei gleicher einwirkender Kraft steigt der Widerstand gegenüber einer Bewegungsänderung des Körpers mit seiner Masse. Die Masse ist ein Skalar mit der Einheit $[m] = \SI{1}{\kilogram}$.
\end{rememberbox}


\begin{rememberbox}{Definition: Impuls}
    Der \textbf{Impuls} ist ein Maß für den Bewegungszustand eines Körpers. Er ist definiert als das Produkt aus Masse und Geschwindigkeit:
    \begin{equation}\label{eq:impuls}
        \ivec{p} \defeq m \cdot \ivec{v} 
    \end{equation}
    Der Impuls ist ein Vektor, der parallel zur Geschwindigkeit zeigt, und hat die Maßeinheit $[p] = \SI{1}{\kilogram\meter\per\second}$. Eine Impulsänderung ist immer auf eine Wechselwirkung (Kraft) zurückzuführen.
\end{rememberbox}


\subsection{Zweites Newtonsches Axiom (Aktionsprinzip)} \label{subsec:zweites_axiom}
\begin{importantbox}[]{2. Newtonsches Axiom}
    Die Ursache einer Impulsänderung eines Körpers liegt in der auf den Körper wirkenden Kraft. Die Kraft ist die zeitliche Ableitung des Impulses:
    \begin{equation}\label{eq: zweites_newtonsches_axiom_kraft}
        \ivec{F} = \frac{\dd\ivec{p}}{\dd t}
    \end{equation}
    Eine resultierende Kraft ungleich Null ($\sum \ivec{F}_i \neq 0$) führt zu einer Impulsänderung.
\end{importantbox}

Da $\ivec{p} = m\cdot\ivec{v}$, erhalten wir im allgemeinen Fall, in dem sich auch die Masse ändern kann (\zB bei einer Rakete, die Treibstoff ausstößt):
\begin{equation}\label{eq: newton2_rakete}
    \ivec{F} = \frac{\dd(m\ivec{v})}{\dd t} = m \cdot \frac{\dd \ivec{v}}{\dd t} + \frac{\dd m}{\dd t} \cdot \ivec{v} = m\ivec{a} + \frac{\dd m}{\dd t}\ivec{v} \mDot
\end{equation}

\begin{rememberbox}{Sonderfall: Konstante Masse}
    Im häufigen Fall, dass die Masse konstant ist ($m = \const$), vereinfacht sich die Formel zur bekannten Form:
    \begin{equation}\label{eq: zweites_newtonsche_axiom_F_equals_ma}
        \ivec{F} = m\cdot\frac{\dd\ivec{v}}{\dd t} = m\cdot\ivec{a} 
    \end{equation}
    Historisch wird das 2. Newtonsche Axiom oft für diesen Fall formuliert: Die Beschleunigung eines Körpers ist direkt proportional zur auf ihn wirkenden Gesamtkraft und umgekehrt proportional zu seiner Masse
    \begin{equation}
        \ivec{a} = \frac{\sum_{i}\ivec{F}_{i}}{m} \mDot
    \end{equation}
\end{rememberbox}


\subsection{Exkurs: Relativistische Masse} \label{subsec: relativistische_masse}
In der klassischen Mechanik ist die Masse eines Körpers ($m_0$) eine konstante Eigenschaft, unabhängig von seinem Bewegungszustand.
In der speziellen Relativitätstheorie hingegen ist der Impuls eines Teilchens um den Lorentz-Faktor $\gamma(v)$ größer:
\begin{equation}\label{eq: relativistischer_impuls_p_von_v}
    \ivec{p}(v) = \gamma(v)\cdot m_0 \cdot \ivec{v} \mComma
\end{equation}
wobei hier $v = |\ivec{v}|$. Der Lorentz-Faktor (auch Gamma-Faktor genannt) ergibt sich aus der speziellen Relativitätstheorie:
\begin{equation}\label{eq: gamma_faktor_relativität}
    \gamma(v)=\frac{1}{\sqrt{1-\frac{v^{2}}{c^{2}}}} \mComma
\end{equation}
mit der Lichtgeschwindigkeit $c \approx \SI{299792458}{\meter\per\second}$. 
\begin{figure}[htbp]
    \centering
    \includegraphics[width=0.5\textwidth]{Bilder/Kapitel_Mechanik/Kapitel_Dynamik/relativistischerGamma-Faktor.png}
    \caption{Der relativistische Gamma-Faktor $\gamma(v)$. Für klassische Geschwindigkeiten $v \ll c$ ist $\gamma\approx 1$. Für Geschwindigkeiten nahe der Lichtgeschwindigkeit $c$ divergiert der Faktor.}
    \label{fig: gamma_faktor_gamma_von_v}
\end{figure}
Vereint man den Gamma-Faktor mit der Ruhemasse, kann man eine \textbf{relativistische Masse} $m^{*}$ über $m^{*}:=\gamma \cdot m_{0}$ definieren, die dann über $\gamma(v)$ von der Geschwindigkeit abhängt. Für klassische Geschwindigkeiten kann die relativistische Massenzunahme meist ignoriert werden. Zum Beispiel gilt für $v = \SI{1000}{\meter\per\second}$, dass $\gamma(v) \approx 1 + \num{6e-12}$.


\subsection{Drittes Newtonsches Axiom (Wechselwirkungsprinzip)}\label{subsec: drittes_newtonsche_axiom}
\begin{figure}[htbp]
    \centering
    \includegraphics[width=0.45\textwidth]{Bilder/Kapitel_Mechanik/Kapitel_Dynamik/aktions-reaktions-paar.png}
    \caption{Die Gravitationskraft zwischen zwei Massen $m_1$ und $m_2$ ist ein Beispiel für ein Aktions-Reaktions-Paar ($\protect\ivecS{F}{1}=-\protect\ivecS{F}{2}$).}
    \label{fig: actio_reactio_reaktionspaar}
\end{figure}
\begin{importantbox}[]{3. Newtonsches Axiom (Actio = Reactio)}
    Wenn zwei Körper wechselwirken, so ist die Kraft, die Körper 1 auf Körper 2 ausübt ($\ivecS{F}{1\rightarrow 2}$), entgegengesetzt gleich der Kraft, die Körper 2 auf Körper 1 ausübt ($\ivecS{F}{2\rightarrow 1}$):
    \begin{equation}\label{eq: drittes_axiom_actio_reactio}
        \ivecS{F}{1\rightarrow 2} = -\ivecS{F}{2\rightarrow 1} 
    \end{equation}
\end{importantbox}
Keine der beiden Kräfte tritt zuerst auf und bewirkt damit die andere: Beide Kräfte treten immer gleichzeitig auf und bilden ein untrennbares „Aktions-Reaktions-Paar“, deren Betrag gleich ist $|\ivecS{F}{1\rightarrow 2}| = |\ivecS{F}{2\rightarrow 1}|$, aber deren Richtung entgegengesetzt ist. \\

Ein System von Massenpunkten (oder Teilchen), das von der Umgebung isoliert ist -- mit der Umgebung also überhaupt nicht wechselwirkt -- nennt man abgeschlossenes System. Auf ein \textbf{abgeschlossenes System} wirken keine äußeren Kräfte ($\ivecS{F}{\mathrm{ext}} = 0$). Wir wollen zeigen, dass daher der Gesamtimpuls des Systems konstant bleiben muss (Impulserhaltung). 

Der Gesamtimpuls $\ivec{P}$ eines Systems aus $N$ Teilchen ist durch die Vektorsumme der Einzelimpulse gegeben
\begin{equation}
    \ivec{P} = \ivecS{p}{1} + \ivecS{p}{2} + \dots + \ivecS{p}{N} = \sum_i \ivecS{p}{i} \mDot
\end{equation}
Eine äußere Kraft $\ivecS{F}{\mathrm{ext}}$ würde den Gesamtimpuls via $\dd \ivec{P}/\dd t = \ivecS{F}{\mathrm{ext}}$ ändern. Für ein abgeschlossenes System bedeutet das aber
\begin{equation}
    \ivecS{F}{\mathrm{ext}} \eqexcl 0 = \frac{\dd \ivec{P}}{\dd t} \implies \ivec{P} = \const \mDot
\end{equation}
\begin{rememberbox}{Impulserhaltung}
    Auf ein abgeschlossenes System wirken keine äußeren Kräfte und daher bleibt der Gesamtimpuls des Systems konstant (Impulserhaltung)
    \begin{equation}
        \ivec{P} = \sum_{i} \ivecS{p}{i} = \const \mDot
    \end{equation}
\end{rememberbox}
Für ein abgeschlossenes System aus zwei Teilchen gilt demnach 
\begin{equation*}
    \ivec{P} = \ivecS{p}{1} + \ivecS{p}{2} = \const \mDot
\end{equation*}
Leitet man diesen Ausdruck nach der Zeit ab, folgt 
\begin{equation*}
    \frac{\dd \ivecS{p}{1}}{\dd t} + \frac{\dd \ivecS{p}{2}}{\dd t} = 0 \implies \ivecS{F}{1} = -\ivecS{F}{2} \mDot
\end{equation*}
Das 3. Newtonsche Axiom ist also eine direkte Folge der Impulserhaltung.

\section{Anwendung der Axiome}\label{sec: anwendung_axiome}
\subsection{Superpositionsprinzip und Vektoraddition von Kräften}\label{subsec: superposition_von_kräften}
Da Beschleunigungen Vektoren sind, müssen auch Kräfte laut \cref{eq: zweites_newtonsche_axiom_F_equals_ma} Vektoren sein. Greifen an einem Punkt mehrere Kräfte an, so ist die resultierende Gesamtkraft die \textbf{Vektorsumme der Einzelkräfte}. Dies folgt aus dem sogenannten Superpositionsprinzip. 
\begin{rememberbox}{Superpositionsprinzip}
     Das \textbf{Superpositionsprinzip} besagt, dass bei allen linearen Systemen die verursachte Nettoreaktion, die durch zwei oder mehr Stimuli hervorgerufen wird, die Summe der Reaktionen ist, die durch jeden einzelnen Stimulus verursacht worden wäre.
\end{rememberbox}
Mit anderen Worten lässt sich aus einer Fülle an Einzelkräften durch vektorielle Addition der Einzelkräfte immer eine resultierende Gesamtkraft ermitteln. Dieses Superpositionsprinzip  gilt für jede Raumrichtung separat:
\begin{equation}
    \ivecS{F}{\text{ges}} = \sum_{i} \ivecS{F}{i} \quad \implies \quad F_{\text{ges},x} = \sum_{i} F_{i,x}\mComma \quad F_{\text{ges},y} = \sum_{i} F_{i,y}\mComma \quad F_{\text{ges},z} = \sum_{i} F_{i,z} \mDot
\end{equation}
Wenn ein Körper in Ruhe ist (statisches Gleichgewicht), müssen sich die angreifenden Kräfte in allen Raumrichtungen zu null aufheben, sodass $\ivecS{F}{\text{ges}} = \ivec{0}$. 
\begin{figure}[htbp]
    \centering
    \includegraphics[width=0.38\textwidth]{Bilder/Kapitel_Mechanik/Kapitel_Dynamik/vektoraddition_kräfte.png}
    \caption{Vektoraddition von Kräften, die alle im selben Punkt angreifen. Die resultierende Kraft $\protect\ivecS{F}{4}$ ist die Vektorsumme von $\protect\ivecS{F}{1}$, $\protect\ivecS{F}{2}$ und $\protect\ivecS{F}{3}$.}
    \label{fig: vektoraddition_von_kräften_dynamik}
\end{figure}


\subsection{Kraftfelder und Zentralkräfte}
\label{subsec:kraftfelder}

In vielen Fällen hängt die auf einen Körper wirkende Kraft vom Ort ab. Kann man jedem Raumpunkt eine Kraft zuordnen, so spricht man von einem \textbf{Kraftfeld} 
\begin{equation}
    \ivec{F} = \ivec{F}(x,y,z)\quad \text{oder} \quad \ivec{F} = \ivec{F}(r,\theta,\varphi)\mDot
\end{equation}. 
Hängt die Kraft in jedem Punkt nur vom Abstand $r$ zu einem Zentrum (\zB Nullpunkt) ab, nennt man es ein \textbf{Zentralkraftfeld}. Für solche Zentralkraftfelder gilt:
\begin{equation}
    \ivec{F} = f(r) \cdot \ivecS{e}{r}
\end{equation}
wobei $\ivecS{e}{r}$ der Einheitsvektor in radialer Richtung ist. Für Kräfte, die zum Zentrum zeigen, gilt $f(r) < 0$ und für Kräfte, die vom Zentrum weg weisen, gilt $f(r) > 0$.

\begin{rememberbox}{Gravitationsfeld der Erde als Zentralkraftfeld}
    Das Gravitationsfeld der Erde ist für Abstände $r$ größer als der Erdradius $R$ ein Zentralkraftfeld. Die Kraft ist gegeben durch das Newtonsche Gravitationsgesetz:
    \begin{equation}\label{eq: gravitationskraft_allgemein}
        \ivec{F}(r) = \underbrace{-G\frac{m\cdot M}{r^{2}}}_{f(r)}\cdot \ivecS{e}{r}
    \end{equation}
    Hierbei ist $G$ die Gravitationskonstante, $M$ die Masse der Erde und $m$ die Masse des Körpers. Das negative Vorzeichen zeigt, dass die Kraft anziehend ist (zum Zentrum gerichtet).
\end{rememberbox}



\section{Klassische Kräfte}\label{sec: klassische_Kräfte}

\subsection{Die Gewichtskraft} \label{subsec: gewichtskraft}
Die sogenannte \gDQ{schwere Masse} ist eine Eigenschaft eines Körpers, die dem Körper erlaubt, über die gravitative Wechselwirkung eine Anziehungskraft auf andere Körper auszuüben und selbst von anderen Körpern angezogen zu werden. Im Wesentlichen ist sie die Quelle der Gravitationskraft. Die schwere Masse ist nicht dasselbe wie die träge Masse, die beschreibt, wie stark sich ein Körper einer Beschleunigung widersetzt. Allerdings sind in der klassischen Physik und in der allgemeinen Relativitätstheorie die schwere und träge Masse äquivalent. 
\begin{rememberbox}[]{Gewichtskraft}
    Infolge der Gravitation der Erde wirkt auf jeden Körper mit Masse $m$ die Gewichtskraft $\ivecS{F}{\text{G}}$:
    \begin{equation}\label{eq: gewichtskraft_f_eq_mg}
        \ivecS{F}{\text{G}} = m \cdot \ivec{g}
    \end{equation}
    Dabei ist $\ivec{g}$ die Erdbeschleunigung, die zum Erdmittelpunkt zeigt und senkrecht auf die Erdoberfläche steht. Ihr Betrag beträgt im Mittel $|\ivec{g}| \approx \SI{9.81}{\meter\per\second\squared}$.
\end{rememberbox}
Der Wert von $\ivec{g}$ hängt vom Ort ab und ist am Äquator mit $\approx \SI{9.78}{\meter\per\second\squared}$ kleiner als an den Polen mit $\approx \SI{9.83}{\meter\per\second\squared}$. \\

Im Gegensatz zur Masse ist das Gewicht keine innere Eigenschaft eines Körpers, sondern hängt vom Ort ab. Eine Kugel mit $m = \SI{10}{kg}$ hat auf der Erde ein Gewicht von $F_{\text{G},\text{Erde}} \approx \SI{98.1}{N}$, auf dem Mond jedoch nur $F_{\text{G},\text{Mond}} \approx \SI{16.2}{N}$, da die Gravitationsbeschleunigung auf der Mondoberfläche nur $\approx \SI{1.62}{\meter\per\second\squared}$ ist. Um die Kugel jedoch horizontal zu beschleunigen (ihre Trägheit zu überwinden), ist auf Erde und Mond dieselbe Kraft erforderlich, da die Masse $m$ dieselbe ist.


\subsection{Das Newtonsche Gravitationsgesetz}
Die Gewichtskraft ist jedoch nur ein Spezialfall eines fundamentaleren Prinzips. Das Phänomen der Gravitation ist nämlich universell und wirkt nicht nur in unmittelbarer Nähe eines Himmelskörpers sondern zwischen \textit{allen} massebehafteten Körpern. Isaac Newton formulierte hierfür ein allgemeingültiges Gesetz, das die Wechselwirkung zwischen zwei beliebigen Massen im Universum beschreibt. Aus diesem allgemeinen Gesetz lässt sich die bekannte Formel für die Gewichtskraft und der Wert für die Erdbeschleunigung $\ivec{g}$ herleiten.

Wie in der Abbildung \cref{fig: gravitation_zwischen_m1_m2} dargestellt, besagt das Newtonsche Gravitationsgesetz, dass sich zwei beliebige Massen $m_1$ und $m_2$ gegenseitig anziehen. Die Kraft, die dabei auf jede Masse wirkt, ist entlang der geraden Verbindungslinie ihrer Massenmittelpunkte gerichtet. Gemäß dem 3. Newtonschen Axiom (actio et reactio) sind die Kräfte, die die beiden Körper aufeinander ausüben, entgegengesetzt gerichtet, aber vom Betrag her gleich groß: $|\ivecS{F}{2\rightarrow 1}| = |\ivecS{F}{1\rightarrow 2}|$.
\begin{figure}[htbp]
    \centering
    \resizebox{0.55\linewidth}{!}{
    \begin{tikzpicture}
        \coordinate (mA_center) at (0,0);
        \coordinate (mB_center) at (6,0);
        \def\mAradius{0.7};
        \def\mBradius{0.5};
        % --- Masse 1 (m1) ---
        \shade[shading=ball, ball color=red!30] (mA_center) circle (\mAradius);
        \node[below=7mm of mA_center] {$m_1$};
        % --- Masse 2 (m2) ---
        \shade[shading=ball, ball color=green!30] (mB_center) circle (\mBradius);
        \node[below=6mm of mB_center] {$m_2$};
        % --- Abstandslinie für r ---
        \draw[|<->|, >={Latex[length=3mm, width=1.2mm]}] (mA_center |- 0,1.5) -- (mB_center |- 0,1.5) node[midway, above] {$r$};
        % --- Kraftvektoren ---
        \draw[-{Stealth[length=3mm, width=3mm]}, red!70!black, ultra thick] 
              (mA_center) -- ++(2.5,0) node[above, pos=0.6, yshift=1mm] {$\vec{F}_{2\rightarrow 1}$};
        \draw[-{Stealth[length=3mm, width=3mm]}, green!60!black, ultra thick] 
              (mB_center) -- ++(-2.5,0) node[above, pos=0.6, yshift=1mm] {$\vec{F}_{1\rightarrow 2}$};
        % --- Einheitsvektoren --- 
        \draw[-{Stealth[length=2mm, width=1.3mm]}, black, thick] 
              (mA_center)++(1,-1) -- ++(1,0) node[above, pos=0.4, yshift=0.7mm] {$\vec{e}_{1\rightarrow 2}$};
        \draw[-{Stealth[length=2mm, width=1.3mm]}, black, thick] 
              (mB_center)++(-1,-1) -- ++(-1,0) node[above, pos=0.4, yshift=0.7mm] {$\vec{e}_{2\rightarrow 1}$};
    \end{tikzpicture}
    }
    \caption{Zwei Massen $m_1$ und $m_2$ im Abstand $r$ ziehen sich gegenseitig mit den Kräften $\protect\ivec{F}{1\rightarrow 2}$ und $\protect\ivec{F}{2\rightarrow 1}$ an.}
    \label{fig: gravitation_zwischen_m1_m2}
\end{figure}

Die Stärke dieser Anziehungskraft hängt von den beiden Massen und ihrem Abstand $r$ ab. Je größer die Massen, desto stärker die Kraft; je größer der Abstand, desto schwächer wird sie.
\begin{importantbox}[]{Newtonsches Gravitationsgesetz}
    Zwei Massen $m_1$ und $m_2$ im Abstand $r$ üben gegenseitig eine Gravitationskraft aufeinander aus. Die Gravitationskraft, die auf $m_1$ wirkt und von $m_2$ ausgelöst wird, lautet
    \begin{equation}\label{eq: gravitationsgesetz_allg}
        \ivecS{F}{\text{G}, 2\rightarrow 1} = -G \frac{m_1 \cdot m_2}{r^2} \cdot \ivecS{e}{2\rightarrow 1}
    \end{equation}
    Dabei ist $G \approx \SI{6.67e-11}{\newton\meter\squared\per\kilogram\squared}$ die universelle Gravitationskonstante und $\ivecS{e}{2\rightarrow 1}$ der Einheitsvektor, der von $m_2$ nach $m_1$ zeigt. 
\end{importantbox}
Die Gravitationskonstante $G$ ist eine fundamentale Naturkonstante und gilt überall im Universum. Das Minus in \cref{eq: gravitationsgesetz_allg} drückt aus, dass die Kraft auf die Masse $m_1$ (ausgeübt von $m_2$) in die entgegengesetzte Richtung des Einheitsvektors $\ivecS{e}{2\rightarrow 1}$ zeigt, der von $m_2$ nach $m_1$ weist – es handelt sich also um eine Anziehungskraft.

\subsection{Normalkräfte}
Man betrachte die Statue in \cref{fig: statue_normalkraft_reaktionskraft}. Auf die Statue im linken Bild wirkt die Gewichtskraft $\ivecS{F}{\text{G}}$. Würde diese alleine wirken, müsste sich die Statue bewegen. Die Vektorsumme der angreifenden Kräfte ist demnach null, da die Statue in Ruhe ist. Die der Gewichtskraft entgegenstehende Kraft ist die Normalkraft auf die Kontaktfläche, $\ivecS{F}{\text{N}}$. 

\textbf{Achtung:} Die Normalkraft und die Gewichtskraft ist kein Aktions-Reaktions-Kräftepaar. Die Reaktionskraft auf die Gewichtskraft wirkt auf die Erde. Das Aktions-Reaktionspaar der Normalkraft ist im rechten Bild dargestellt: Das Pendant zur Normalkraft auf die Statue $\ivecS{F}{\text{N}}$ ist die Normalkraft der Statue auf den Tisch $\ivecS{F'}{\text{N}}$.\footnote{Der Apostroph markiert hier nicht die Ableitung nach dem Ort. Der Apostroph wird oftmals dazu verwendet, gewisse gleichartige Größen zu nummerieren, zu unterscheiden oder zu gruppieren.} 

\begin{figure}[htbp]
    \centering
    \includegraphics[width=0.35\linewidth]{Bilder/Kapitel_Mechanik/Kapitel_Dynamik/statue_gewichtskraft_normalkraft.png}
    \caption{Da sich die Statue nicht bewegt, muss die Gesamtkraft null sein. Die Gewichtskraft $\protect\ivecS{F}{\text{G}}$ wird durch eine Normalkraft $\protect\ivecS{F}{\text{N}}$ kompensiert. Die Normalkraft hat eine Reaktionskraft $\protect\ivecS{F'}{\text{N}}$.}
    \label{fig: statue_normalkraft_reaktionskraft}
\end{figure}


\subsection{Rückstellkräfte (Federkraft)}\label{subsec: Rückstellkräfte}
Wird eine Masse mit einer Feder verbunden, so existiert eine Position, in der die Feder entspannt ist und die Masse in Ruhe verharrt, siehe \cref{fig: feder_rückstellkraft_a}. Diese Position wird als \textbf{Ruhelage} $x_0$ bezeichnet. Es ist zweckmäßig, den Ursprung des Koordinatensystems in diesen Punkt zu legen, sodass $x_0 = 0$.
\begin{figure}[htbp]
    \centering
    \labelphantom{fig: feder_rückstellkraft_a}
    \labelphantom{fig: feder_rückstellkraft_b}
    \labelphantom{fig: feder_rückstellkraft_c}
    \includegraphics[width=0.6\textwidth]{Bilder/Kapitel_Mechanik/Kapitel_Dynamik/rückstellkraft_feder_abc.png}
    \caption{Eine Masse, die an einer Feder befestigt ist. a) In der Ruhelage $x=x_0=0$. b) Auslenkung in die positive $x$-Richtung, $\Delta x > 0$. c) Kompression in die negative $x$-Richtung, $\Delta x < 0$.}
    \label{fig: feder_rückstellkraft}
\end{figure}

Lenkt man die Masse aus der Ruhelage aus, so erzeugt die Feder eine \textbf{Rückstellkraft} $\ivecS{F}{x}$, die stets bestrebt ist, die Masse in die Ruhelage zurückzuführen.
\begin{itemize}
    \item Bei einer \textbf{Auslenkung} in die positive $x$-Richtung (Streckung der Feder), dargestellt in \cref{fig: feder_rückstellkraft_b}, wirkt die Rückstellkraft in die negative $x$-Richtung.
    \item Bei einer \textbf{Kompression} in die negative $x$-Richtung, dargestellt in \cref{fig: feder_rückstellkraft_c}, wirkt die Rückstellkraft in die positive $x$-Richtung.
\end{itemize}

Für kleine Auslenkungen aus der Ruhelage ($\Delta x = x - x_0 \ll 1$) verhält sich die Feder linear. Das bedeutet, die Rückstellkraft ist direkt proportional zur Auslenkung. Aus dem obigen Verhalten ergibt sich das sogenannte Hooke'sche Gesetz, das im linearen Bereich einer Feder (kleine Auslenkungen) gültig ist. 

\begin{rememberbox}[]{Hooke'sches Gesetz}
Die Rückstellkraft einer idealen Feder ist gegeben durch:
\begin{equation}\label{eq: kraft_hooksches_gesetz}
    F_x = -k_F \cdot \left( x - x_0 \right) = -k_F \cdot \Delta x \mDot
\end{equation}
Hierbei ist $k_F$ die \textbf{Federkonstante}, eine Materialeigenschaft der Feder, die ihre Steifigkeit beschreibt. Das negative Vorzeichen verdeutlicht, dass die Rückstellkraft stets der Auslenkung entgegengesetzt ist.
\end{rememberbox}
In mehreren Dimensionen wird das Hooke'sche Gesetz zu 
\begin{equation}\label{eq: kraft_hooksches_gesetz_3D}
    \ivecS{F}{\text{Feder}} = -k_F \cdot \left( \ivec{r} - \ivecS{r}{0} \right) = -k_F \cdot \Delta \ivec{r} \mComma
\end{equation}
wobei $\ivecS{r}{0}$ dann die mehrdimensionale Ruhelage ist, beispielsweise $\ivecS{r}{0} = \inlrowThree{x_0}{y_0}{z_0}$.

\subsection{Reibungskräfte}\label{subsec: Reibungskräfte}
Reibungskräfte treten an den Kontaktflächen zwischen zwei Körpern auf. Sie entstehen durch mikroskopische Unebenheiten der Oberflächen und intermolekulare Anziehungskräfte. Die Stärke der Reibung hängt von der Art der Bewegung ab, weshalb man zwischen den folgenden Arten unterscheidet:
\begin{itemize}
    \item \textbf{Haftreibung}: Wirkt auf ruhende Körper.
    \item \textbf{Gleitreibung}: Wirkt auf sich bewegende Körper.
    \item \textbf{Rollreibung}: Wirkt auf rollende Körper.
\end{itemize}

\begin{figure}[htb]
    \centering
    \includegraphics[width=0.4\textwidth]{Bilder/Kapitel_Mechanik/Kapitel_Dynamik/kontaktkraft_unebenheit.png}
    \caption{Ein Körper bewegt sich mit der Geschwindigkeit $\protect\ivec{v}$ parallel zur Oberfläche der Unterlage. Die Gewichtskraft $\protect\ivecS{F}{\text{G}}$ presst den Körper normal auf die Unterlage. Die Unterlage kompensiert kompensiert hier die Gewichtskraft $\protect\ivecS{F}{\text{G}}$ durch die Normalkraft $\protect\ivecS{F}{\text{N}}$, die senkrecht auf die Kontaktfläche steht. Die vergrößerte Ansicht zeigt die mikroskopischen Unebenheiten der Oberflächen, die zur Reibung führen.}
    \label{fig: reibung_mikro_ursache}
\end{figure}

Die Oberflächen zweier Körper berühren sich nur an den Spitzen dieser Unebenheiten. Erhöht man die \textbf{Normalkraft} $\ivecS{F}{\text{N}}$ – die Kraft, die senkrecht auf die Kontaktfläche drückt – vergrößert sich die effektive Kontaktfläche. Die Reibungskraft $\ivecS{F}{\text{R}}$ ist proportional zu dieser mikroskopischen Kontaktfläche und somit auch proportional zur Normalkraft: $|\ivecS{F}{\text{R}}| \propto \ivecS{F}{\text{N}}$.

\begin{importantbox}[]{Allgemeine Reibungsformel}
Die Beziehung zwischen dem Betrag der Reibungskraft $|\ivecS{F}{\text{R}}|$ und dem Betrag der Normalkraft $|\ivecS{F}{\text{N}}|$ lässt sich durch die Einführung eines \textbf{Reibungskoeffizienten} $\mu$ beschreiben:
\begin{equation}\label{eq: reibungskraft_FR_equal_mu_FN}
    |\ivecS{F}{\text{R}}| = \mu \cdot |\ivecS{F}{\text{N}}|
\end{equation}
Diese Gleichung ist eine skalare Beziehung, da die Reibungskraft und die Normalkraft senkrecht aufeinander stehen ($\ivecS{F}{\text{R}} \perp \ivecS{F}{\text{N}}$).
\end{importantbox}

\subsection{Haftreibung}\label{subsec: Haftreibung}
Um einen ruhenden Körper in Bewegung zu versetzen, muss zunächst die Haftreibung überwunden werden. Auf einen ruhenden Körper auf einer horizontalen Fläche wirkt die Gewichtskraft $\ivecS{F}{\text{G}} = m\ivec{g}$, die von der Normalkraft $\ivecS{F}{\text{N}}$ des Untergrunds kompensiert wird, siehe \cref{fig: haftreibung_kraftdiagramm_kiste}. 

Zieht man nun mit einer horizontalen Kraft $\ivecS{F}{\text{A}}$ am Körper, baut sich eine entgegengesetzt gleiche Haftreibungskraft $\ivecS{F}{\text{R,h}} = -\ivecS{F}{\text{A}}$ auf, die die Bewegung verhindert. Der Körper bleibt so lange in Ruhe, bis die Zugkraft einen bestimmten Maximalwert überschreitet und der Körper zu gleiten beginnt. Die Haftreibungskraft wirkt immer dann, wenn zwei sich berührende Oberflächen nicht aneinander gleiten. 

% \begin{figure}[h!]
%     \centering
%     \includegraphics[width=0.4\textwidth]{Bilder/Kapitel_Mechanik/Kapitel_Dynamik/kiste_reibung_beschleunigung_grav_normalkraft.png}
%     \caption{Links: Kräftediagramm für einen Körper, an dem mit der Kraft $\protect\ivec{F}_A$ gezogen wird. Die Haftreibungskraft $\protect\ivec{F}_{R}$ wirkt entgegen. Rechts: Diagramm der Reibungskraft $F_R$ in Abhängigkeit von der Zugkraft $F_A$.}
%     \label{fig: haftreibung}
% \end{figure}
\begin{figure}[h!]
    \centering
    \begin{subfigure}{0.44\textwidth}
        \centering
        \includegraphics[width=\textwidth]{Bilder/Kapitel_Mechanik/Kapitel_Dynamik/kiste_reibung_beschleunigung_grav_normalkraft.png}
        \caption{}
        \label{fig: haftreibung_kraftdiagramm_kiste}
    \end{subfigure}
    \hfill    
    \begin{subfigure}{0.44\textwidth}
        \centering
        \includegraphics[width=\textwidth]{Bilder/Kapitel_Mechanik/Kapitel_Dynamik/reibungskraft_vs_beschleunigendeKraft.png}
        \caption{}
        \label{fig: haftreibungskraft_zugkraft_diagramm}
    \end{subfigure}
    \caption{Links: Kräftediagramm für einen Körper, an dem mit der Kraft $\protect\ivecS{F}{\text{A}}$ gezogen wird. Die Haftreibungskraft $\protect\ivecS{F}{\text{R}}$ wirkt der Bewegung entgegen. Die Gewichtskraft $m\cdot \protect\ivec{g}$ wird durch die Normalkraft $\protect\ivecS{F}{\text{N}}$ kompensiert. Rechts: Diagramm der Reibungskraft $|\protect\ivecS{F}{\text{R}}|$ in Abhängigkeit von der Zugkraft $|\protect\ivec{F}| = |\protect\ivecS{F}{\text{A}}|$.}
    \label{fig: haftreibung_figure}
\end{figure}

\begin{rememberbox}[]{Maximale Haftreibungskraft}
Die maximale Haftreibungskraft ist proportional zum Betrag der Normalkraft:
\begin{equation}\label{eq: max_haftreibungskraft}
    |\ivecS{F}{\text{R,h,max}}| = \mu_{\text{R,h}} \cdot |\ivecS{F}{\text{N}}| \mDot
\end{equation}
Der Proportionalitätsfaktor $\mu_{\text{R,h}}$ ist der \textbf{Haftreibungskoeffizient}. Sein Wert hängt von den Materialien, der Oberflächenbeschaffenheit und der Temperatur ab. Solange die angreifende Kraft kleiner als dieser Maximalwert ist ($|\ivecS{F}{\text{A}}| < |\ivecS{F}{\text{R,h,max}}|$), ist die Haftreibungskraft betragsmäßig gleich der angreifenden Kraft ($|\ivecS{F}{\text{R,h}}| = |\ivecS{F}{\text{A}}|$).
\end{rememberbox}

\subsection{Gleitreibung}\label{subsec: Gleitreibung}
Sobald die angreifende Kraft die maximale Haftreibungskraft übersteigt ($|\ivecS{F}{\text{A}}| > |\ivecS{F}{\text{R,h,max}}|$), beginnt der Körper zu gleiten. Im Moment des Übergangs von Haften zu Gleiten sinkt die Reibungskraft abrupt ab, da der Gleitreibungskoeffizient $\mu_{\text{R,g}} < \mu_{\text{R,h}}$. 

\begin{importantbox}[]{Gleitreibungskraft}
Die Gleitreibungskraft $\ivecS{F}{\text{R,g}}$ ist ebenfalls proportional zur Normalkraft, jedoch mit dem \textbf{Gleitreibungskoeffizienten} $\mu_{\text{R,g}}$:
\begin{equation}\label{eq: gleitreibungskraft}
    |\ivecS{F}{\text{R,g}}| = \mu_{\text{R,g}} \cdot |\ivecS{F}{\text{N}}|\mDot
\end{equation}
Im Gegensatz zur Haftreibung ist die Gleitreibungskraft (in guter Näherung) unabhängig von der angreifenden Kraft und der Geschwindigkeit beim Gleiten. Für fast alle Materialien gilt:
\begin{equation}
    \mu_{\text{R,g}} < \mu_{\text{R,h}}
\end{equation}
Dies erklärt, warum es mehr Kraft erfordert, einen Gegenstand in Bewegung zu setzen, als ihn in Bewegung zu halten.
\end{importantbox}
Da die Gleitreibungskraft nur proportional zur Normalkraft und nicht zur Zugkraft $\ivecS{F}{\text{A}}$ ist, bleibt die Gleitreibungskraft konstant für zunehmende Zugkraft. Daher ist $|\ivecS{F}{\text{R}}|$ in \cref{fig: haftreibungskraft_zugkraft_diagramm} konstant im Bereich der Gleitreibung ($|\ivecS{F}{\text{R}}| > |\ivecS{F}{\text{R,h,max}}|$).

\subsection{Rollreibung}
Ein ideal starres Rad, das mit konstanter Geschwindigkeit auf einer ideal starren horizontalen Straße rollt, würde keine Rollreibungskraft erfahren. Die Rollreibung entsteht durch die Verformung des Rades und der Fahrbahn. In der Realität führen diese Deformationen zu einer Kraft, die der Rollbewegung entgegenwirkt.

\begin{rememberbox}[]{Rollreibungskraft}
Analog zu den anderen Reibungsarten ist die Rollreibungskraft $\ivecS{F}{\text{R,r}}$ proportional zur Normalkraft:
\begin{equation}\label{eq: rollreibungskraft}
    |\ivecS{F}{\text{R,r}}| = \mu_{\text{R,r}} \cdot |\ivecS{F}{\text{N}}| \mDot
\end{equation}
Der \textbf{Rollreibungskoeffizient} $\mu_{\text{R,r}}$ hängt nicht nur von den Materialien, sondern auch vom Aufbau des Reifens und der Beschaffenheit der Straße ab. Typischerweise ist die Rollreibung deutlich kleiner als die Gleit- und Haftreibung:
\begin{equation}
    \mu_{\text{R,r}} \ll \mu_{\text{R,g}} < \mu_{\text{R,h}}\mDot
\end{equation}
\end{rememberbox}
Der Gleitreibungskoeffizient ist für die meisten Materialien über einen weiten Bereich von Geschwindigkeiten konstant. 
 
\section{Anwendungen auf der schiefen Ebene}\label{sec: schiefe_ebene_beispiele}
\subsection{Reibungsfreies Gleiten auf der schiefen Ebene}\label{subsec: reibungsfreies_gleiten}
Betrachten wir einen Körper der Masse $m$, der eine reibungsfreie, um den Winkel $\alpha$ geneigte Ebene hinabgleitet. Die auf den Körper wirkende Gewichtskraft $\ivecS{F}{\text{G}}$ kann in zwei Komponenten zerlegt werden (siehe \cref{fig: schiefe_ebene_reibungsfreies_gleiten}:
\begin{itemize}[itemsep=1.5pt]
    \item Eine \textbf{Normalkomponente} $\ivecS{F}{\text{N}}$, die senkrecht zur Ebene steht: \newline
    $|\ivecS{F}{\text{N}}| = |\ivecS{F}{\text{G}}| \cos(\alpha) = mg\cos(\alpha)$.
    \item Eine \textbf{Hangabtriebskraft} $\ivecS{F}{\text{A}}$, die parallel zur Ebene wirkt: \newline
    $|\ivecS{F}{\text{A}}| = |\ivecS{F}{\text{G}}| \sin(\alpha) = mg\sin(\alpha)$.
\end{itemize}

\begin{figure}[htb]
    \centering
    \includegraphics[width=0.5\textwidth]{Bilder/Kapitel_Mechanik/Kapitel_Dynamik/schiefe_ebene_reibungsfreies_gleiten.png}
    \caption{Kräftezerlegung der Gewichtskraft $|\protect\ivecS{F}{\text{G}}|$ für einen Körper auf einer reibungsfreien schiefen Ebene.}
    \label{fig: schiefe_ebene_reibungsfreies_gleiten}
\end{figure}
Nur die Hangabtriebskraft trägt zur Beschleunigung des Körpers bei. Die Normalkraft $\ivecS{F}{\text{N}}$ wird durch eine gleichgroße entgegengesetze Kraft $\ivecS{F'}{\text{N}}$ kompensiert. Nach dem zweiten Newtonschen Gesetz ($F = m\cdot a$) gilt:
\begin{equation}
    |\ivecS{F}{\text{A}}| = m a \implies mg\sin(\alpha) = m a \mDot
\end{equation}

\begin{rememberbox}[]{Beschleunigung auf der reibungsfreien schiefen Ebene}
Die Beschleunigung eines Körpers auf einer reibungsfreien schiefen Ebene ist unabhängig von seiner Masse $m$ und beträgt
\begin{equation}\label{eq: beschleunigung_reibungsfreies_gleiten}
    a = g \sin(\alpha) \mDot
\end{equation}
Für $\alpha = \SI{0}{\degree}$ ist $a=0$ (keine Beschleunigung auf einer horizontalen Ebene) und für $\alpha = \SI{90}{\degree}$ ist $a = g$ (freier Fall).
\end{rememberbox}

\subsection{Haftreibung auf der schiefen Ebene}\label{subsec: haftreibung_schiefe_ebene}
Wenn Reibung vorhanden ist, kann ein Körper auf einer um den Winkel $\alpha$ geneigten schiefen Ebene ruhen. Sofern die maximale Haftreibungskraft nicht überschritten wird ($|\ivecS{F}{\text{A}}| < |\ivecS{F}{\text{R,h,max}}|$), wird die Hangabtriebskraft $\ivecS{F}{\text{A}}$ durch die Haftreibungskraft $\ivecS{F}{\text{R,h}}$ vollständig kompensiert: 
\begin{equation}
    \ivecS{F}{\text{A}} + \ivecS{F}{\text{R,h}} = 0 \implies \ivecS{F}{\text{R,h}} = -\ivecS{F}{\text{A}} \mDot
\end{equation}
Der Körper beginnt zu gleiten, wenn die Hangabtriebskraft die maximale Haftreibungskraft erreicht oder übersteigt, \gDh wenn $|\ivecS{F}{\text{A}}| \geq |\ivecS{F}{\text{R,h,max}}|$. Die maximale Haftreibungskraft lautet hier 
\begin{equation}
    |\ivecS{F}{\text{R,h,max}}| = \mu_{\text{R,h}} \cdot |\ivecS{F}{\text{N}}| = \mu_{\text{R,h}}\cdot m\cdot g\cdot \cos(\alpha) \mDot
\end{equation}
Da die Hangabtriebskraft vom Neigungswinkel der schiefen Ebene abhängt, stellt sich die Frage, unter welchem Winkel die Kiste zu gleiten beginnt. Das passiert gerade dann, wenn die Hangabtriebskraft die maximale Haftreibungskraft erreicht
\begin{equation}\begin{aligned}
     |\ivecS{F}{\text{A}}| &= |\ivecS{F}{\text{R,h,max}}| \\
    mg\cdot \sin(\alpha_{\text{max}}) &= \mu_{\text{R,h}} mg \cdot\cos(\alpha_{\text{max}}) \\
    \tan(\alpha_{\text{max}}) &= \mu_{\text{R,h}} \\
    \implies \alpha_{\text{max}} &= \arctan(\mu_{\text{R,h}})\mDot
\end{aligned}\end{equation}
\begin{figure}[htb]
    \centering
    \includegraphics[width=0.5\textwidth]{Bilder/Kapitel_Mechanik/Kapitel_Dynamik/schiefe_ebene_haftreibung.png}
    \caption{Ein Körper der Masse $m$ wird auf der schiefen Ebene von der Haftreibungskraft gehalten, sofern die maximale Haftreibungskraft nicht überschritten wird: $|\protect\ivecS{F}{\text{A}}| \leq |\protect\ivecS{F}{\text{R,h,max}}|$.}
    \label{fig: schiefe_ebene_haftreibung}
\end{figure}
\begin{rememberbox}[]{Maximaler Haftreibungswinkel}
Der maximale Winkel $\alpha_{\text{R,h,max}}$, bei dem ein Körper gerade noch nicht zu rutschen beginnt, wird als Haftreibungswinkel bezeichnet. Er ergibt sich aus der Gleichheit der Kräfte:
\begin{equation}
    \tan(\alpha_{\text{max}}) = \mu_{\text{R,h}} \quad \Leftrightarrow \quad \alpha_{\text{max}} = \arctan(\mu_{\text{R,h}})
\end{equation}
\end{rememberbox}
Der Haftreibungskoeffizient $\mu_{\text{R,h}}$ ist also lediglich durch den den maximalen Haftreibungswinkel $\alpha_{\text{max}}$ gegeben. 

\subsection{Gleitreibung auf der schiefen Ebene}\label{subsec: gleiten_auf_der_schiefen_ebene}
Wird der maximale Haftreibungswinkel und damit die maximale Haftreibungskraft überschritten, beginnt der Körper zu gleiten. Auf ihn wirkt nun die Gleitreibungskraft $\ivecS{F}{\text{R,g}}$, die der Bewegung entgegenwirkt. Die resultierende Kraft, die den Körper beschleunigt, ist die Differenz aus Hangabtriebskraft und Gleitreibungskraft.
\begin{figure}[htb]
    \centering
    \includegraphics[width=0.5\textwidth]{Bilder/Kapitel_Mechanik/Kapitel_Dynamik/schiefe_ebene_gleitreibung.png}
    \caption{Ein Körper der Masse $m$ gleitet unter dem Einfluss der Gleitreibungskraft eine schiefe Ebene hinab. Die Gleitreibungskraft $\protect\ivecS{F}{\text{R,g}}$ wirkt dabei der Hangabtriebskraft }
    \label{fig: schiefe_ebene_gleitreibung}
\end{figure}
Das Kräftegleichgewicht lautet
\begin{equation}\label{eq: kraeftegleichgew_gleitreibung}
    m \ivec{a} = |\ivecS{F}{\text{A}}| - |\ivecS{F}{\text{R,g}}| \mDot
\end{equation}
Da alle Kräfte in \cref{eq: kraeftegleichgew_gleitreibung} parallel zur schiefen Ebene wirken, können wir auch nur mit den Beträgen weiterrechnen. Wir setzen die Hangabtriebskraft und die Gleitreibungskraft ein und erhalten 
\begin{equation}\begin{aligned}
    m a &= F_\text{G} \sin(\alpha) - \mu_{\text{R,g}} F_\text{N}  \\
    m a &= mg \sin(\alpha) - \mu_{\text{R,g}} mg \cos(\alpha) \\
    a &= g \cdot \left( \sin(\alpha) - \mu_{\text{R,g}} \cos(\alpha)\right) \mDot
\end{aligned}\end{equation}
\begin{importantbox}[]{Beschleunigung auf der rauen schiefen Ebene}
Die Beschleunigung eines Körpers, der eine schiefe Ebene hinabgleitet, ist:
\begin{equation}\label{eq: beschleunigung_gleiten_mit_reibung}
    a = g \left[\sin(\alpha) - \mu_{\text{R,g}} \cos(\alpha)\right]\mDot
\end{equation}
Beim Gleiten mit Reibung ist die reibungsfreie Beschleunigung \cref{eq: beschleunigung_reibungsfreies_gleiten} [$a(\mu_{\text{R,g}}=0) = g \sin(\alpha)$] um den Anteil $-\mu_{\text{R,g}} g \cos(\alpha)$ reduziert. 
\end{importantbox}

\begin{examplebox}{Beispiel Gleitreibung}
Ein Körper mit $m = \SI{3}{\kilo\gram}$ liegt auf einer schiefen Ebene mit Haftreibungskoeffizienten $\mu_{\text{R,h}} = 0.4$ und Gleitreibungskoeffizienten $\mu_{\text{R,g}} = 0.08$. Die Erdbeschleunigung beträgt $g=\SI{9.81}{\metre\per\second\squared}$. \\

Der maximale Haftreibungswinkel $\alpha_{\text{max}}$ errechnet sich aus 
\begin{equation}
   \alpha_{\text{max}} = \arctan(0.4) \approx \SI{0.3805}{\radian} = \SI{21.8}{\degree}\mDot
\end{equation}

Bei einem Winkel von $\alpha = \SI{30}{\degree}$ gleitet der Körper, da $\SI{30}{\degree} > \SI{21.8}{\degree}$. Die auftretende Beschleunigung beträgt 
\begin{equation}\begin{gathered}
    a = g\left[ \sin(\alpha) - \mu_{\text{R,g}} \cos(\alpha)\right] = \\
    = \SI{9.81}{\metre\per\second\squared} \cdot \left[\sin\left(\frac{\pi}{6}\si{\radian}\right) - 0.08 \cdot \cos\left(\frac{\pi}{6}\si{\radian}\right)\right] \mDot\\
    a \approx \num{9.81} \cdot (\num{0.5} - \num{0.08} \cdot \num{0.866}) \approx \SI{4.23}{\metre\per\second\squared}
\end{gathered}\end{equation}
\end{examplebox}


\subsection{Gleichgewicht mit Seil und Rolle}\label{subsec: gleichgewicht_seilrolle_schiefe_ebene}
\begin{examplebox}[lower separated=true, breakable]{Beispiel: Gleichgewicht auf der schiefen Ebene}
Eine Kiste mit Masse $m_1 = \SI{10.0}{\kilo\gram}$ ruht auf einer schiefen Ebene mit einem Neigungswinkel von $\theta = \SI{37}{\degree}$. Sie ist über ein masseloses Seil und eine reibungsfreie Rolle mit einer zweiten, hängenden Masse $m_2$ verbunden. Der Haftreibungskoeffizient zwischen Kiste und Ebene beträgt $\mu_\text{R,h} = 0.4$.\\
\begin{center}
    \includegraphics[width=0.45\textwidth]{Bilder/Kapitel_Mechanik/Kapitel_Dynamik/schiefe_ebene_gleichgewicht_seilrolle.png}
    \captionof{figure}{Die Kiste mit der Masse $m_1$ wird durch die Haftreibung und durch die Zugkraft, die von der Kiste $m_2$ über die Umlenkung vermittelt wird, in Ruhe gehalten.}
    \label{fig: gleichgewicht_schiefe_ebene_seilrolle}
\end{center}
\textbf{Frage:} In welchem Wertebereich für die Masse $m_2$ bleibt das System in Ruhe?
\tcblower
\textbf{Lösung:}
\begin{center}
    \includegraphics[width=0.65\textwidth]{Bilder/Kapitel_Mechanik/Kapitel_Dynamik/schiefe_ebene_gleichgewicht_seilrolle_kraefteglgew.png}
    \captionof{figure}{Für kleine Zugkräfte $\protect\ivecS{F}{Z}$ [Fall(i)] zeigt die Reibungskraft in die $-x$-Richtung und für sehr große Zugkräfte zeigt sie in die $+x$-Richtung [Fall (ii)].}
    \label{fig: gleichgewicht_schiefe_ebene_seilrolle_fallunterscheidung}
\end{center}
Die Kiste $m_1$ bleibt in Ruhe, solange die resultierende Kraft in $x$-Richtung (parallel zur Ebene) durch die Haftreibungskraft kompensiert werden kann. Die Richtung der Haftreibungskraft hängt davon ab, ob die Kiste tendenziell nach oben oder nach unten rutschen würde. Bei kleinen Massen $m_2$ und damit einer kleinen Zugkraft $\ivecS{F}{Z}$ zeigt die Haftreibungskraft auf $m_1$ die schiefe Ebene hinauf, da $m_1$ dazu tendiert hinab zu rutschen [Fall (i) in \cref{fig: gleichgewicht_schiefe_ebene_seilrolle_fallunterscheidung}]. Ist die Masse $m_2$ hingegen groß, wird die Zugkraft $\ivecS{F}{Z}$ groß die Haftreibungskraft zeigt die schiefe Ebene hinab, um zu verhindern, dass die Masse $m_1$ die schiefe Ebene hochgezogen wird.\newline
In der $y$-Richtung heben sich der Anteil der Gewichtskraft und die Normalkraft gerade auf. Da sich die Bewegung wiederum nur entlang der $x$-Achse abspielt, verzichten wir auf eine Vektorschreibweise.

Die Hangabtriebskraft auf $m_1$ ist 
\begin{equation}
    F_{\text{A}} = m_1 g \sin(\theta) \mDot
\end{equation}
Die Zugkraft auf $m_1$ vermittelt durch das Seil ist 
\begin{equation}
    F_{\text{Z}} = m_2 g \mDot
\end{equation}
Die maximale Haftreibungskraft lautet 
\begin{equation}
    F_{\text{R,h,max}} = \mu_{\text{R,h}} F_{\text{N}} = \mu_{\text{R,h}} m_1 g \cos(\theta)\mDot
\end{equation}

\textbf{Fall (i): Untere Grenze für $m_2$ (kleine Zugkraft)}\newline
Die Masse $m_2$ ist so klein, dass die Hangabtriebskraft $F_{\text{A}}$ überwiegt. Die Haftreibungskraft $F_{\text{R,h}}$ wirkt bergauf, um das Hinabrutschen zu verhindern. Im Grenzfall gilt:
\begin{equation}\begin{aligned}
    F_{\text{A}} - F_{\text{Z}} &= F_{\text{R,h,max}} \\
    m_1 g \sin(\theta) - m_2 g &= \mu_{\text{R,h}} m_1 g \cos(\theta) \mDot
\end{aligned}\end{equation}
Kürzen von $g$ und auflösen nach $m_2$ liefert die minimale Masse
\begin{equation}
    m_{2,\text{min}} = m_1 (\sin(\theta) - \mu_{\text{R,h}} \cos(\theta))
\end{equation}
Für die gegebenen Werte finden wir eine minimale Masse von $m_{2,\text{min}} = \SI{10.0}{kg} \cdot (\sin(\SI{37}{\degree}) - 0.4 \cdot \cos(\SI{37}{\degree})) \approx \SI{10.0}{kg} \cdot (0.6018 - 0.4 \cdot 0.7986) \approx \SI{2.82}{kg}$.\newline
Für eine Masse $m_2$ kleiner als $\SI{2.82}{\kilo\gram}$ ist die maximale Haftreibung überschritten und die Masse $m_1$ rutscht die schiefe Ebene hinab. \\

\textbf{Fall (ii): Obere Grenze für $m_2$ (große Zugkraft)}\newline
Die Masse $m_2$ ist so groß, dass die Zugkraft $F_{\text{Z}}$ überwiegt. Die Haftreibungskraft $F_{\text{R,h}}$ wirkt nun bergab. Im Grenzfall gilt:
\begin{equation}\begin{aligned}
    F_{\text{Z}} - F_{\text{A}} &= F_{\text{R,h,max}} \\
    m_2 g - m_1 g \sin(\theta) &= \mu_{\text{R,h}} m_1 g \cos(\theta)\mDot
\end{aligned}\end{equation}
Auflösen nach $m_2$ liefert die maximale Masse
\begin{equation}
    m_{2,\text{max}} = m_1 (\sin(\theta) + \mu_\text{R,h} \cos(\theta)) \mDot
\end{equation}
Für die gegebenen Werte finden wir, dass die maximale Masse $m_{2,\text{max}} = \SI{10.0}{kg} \cdot (\sin(\SI{37}{\degree}) + 0.4 \cdot \cos(\SI{37}{\degree})) \approx \SI{10.0}{kg} \cdot (0.6018 + 0.4 \cdot 0.7986) \approx \SI{9.21}{kg}$ beträgt.

\textbf{Zusammenfassung:}\newline
Das System bleibt in Ruhe, solange sich die Masse $m_2$ im Intervall
$m_2 \in [\num{2.82}; \num{9.21}]\,\si{\kilo\gram}$ befindet.
\end{examplebox}




\section{Drehimpuls und Drehmoment} \label{sec: Drehimpuls_Drehmoment}
Betrachten wir einen Massepunkt mit Masse $m$ und Impuls $\ivec{p} = m \cdot \ivec{v}$, der sich entlang einer Bahnkurve $\ivec{r}(t)$ bewegt.
\begin{importantbox}{Definition: Drehimpuls}
Der \textbf{Drehimpuls} $\ivec{L}$ des Teilchens bezüglich des Koordinatenursprungs O ist definiert als das Vektorprodukt aus dem Ortsvektor $\ivec{r}$ und dem Impuls $\ivec{p}$:
\begin{equation}\label{eq: drehimpuls_def}
\ivec{L} := \ivec{r} \times \ivec{p} = m \cdot (\ivec{r} \times \ivec{v})\mDot
\end{equation}
Der Drehimpulsvektor $\ivec{L}$ steht senkrecht auf die Ebene, die von den Vektoren $\ivec{r}$ und $\ivec{v}$ aufgespannt wird. Bei einer Bewegung in einer Ebene zeigt $\ivec{L}$ daher in Richtung der Normalen auf diese Ebene.
\end{importantbox}

\begin{figure}[htbp]
    \centering
    \includegraphics[width=0.6\textwidth]{Bilder/Kapitel_Mechanik/Kapitel_Dynamik/drehimpuls_ebene.png} 
    \caption{Der Drehimpuls $\protect\ivec{L}(t)$ eines Massenpunktes $m$ bei einer ebenen Bewegung. Die Geschwindigkeit $\protect\ivec{v}$ wird in eine radiale Komponente ($\protect\ivecS{v}{r}$) und eine tangentiale Komponente ($\protect\ivecS{v}{\varphi}$) zerlegt. Nur die tangentiale Komponente erzeugt einen Drehimpuls.}
    \label{fig: drehimpuls_ebene_bewegung}
\end{figure}
Um den Drehimpuls besser zu verstehen, zerlegen wir den Geschwindigkeitsvektor $\ivec{v}$ in eine radiale Komponente $\ivecS{v}{r}$, die parallel zum Ortsvektor $\ivec{r}$ ist, und eine tangentiale (oder azimutale) Komponente $\ivecS{v}{\varphi}$, die senkrecht zu $\ivec{r}$ steht. Es gilt also $\ivec{v} = \ivecS{v}{r} + \ivecS{v}{\varphi}$.

Setzen wir dies in die Definition des Drehimpulses ein:
\begin{equation}
\label{eq: drehimpuls_komponenten}
\ivec{L} = m \cdot [\ivec{r} \times (\ivecS{v}{r} + \ivecS{v}{\varphi})] = m \cdot (\ivec{r} \times \ivecS{v}{r}) + m \cdot (\ivec{r} \times \ivecS{v}{\varphi}) \mComma
\end{equation}
Da $\ivecS{v}{r} \parallel \ivec{r}$ ist, verschwindet ihr Vektorprodukt: $\ivec{r} \times \ivecS{v}{r} = \ivec{0}$. Damit vereinfacht sich der Ausdruck für den Drehimpuls zu
\begin{equation}
\label{eq:drehimpuls_tangential}
\ivec{L} = m \cdot (\ivec{r} \times \ivecS{v}{\varphi}) \mDot
\end{equation}
Daraus folgt, dass nur die zum Ortsvektor senkrechte Geschwindigkeitskomponente zum Drehimpuls beiträgt. Wenn sich ein Körper nur radial vom Ursprung weg oder auf ihn zu bewegt ($\ivec{v} = \ivecS{v}{r}$), ist sein Drehimpuls Null. Eine gekrümmte Bahnkurve ist daher stets mit einem von Null verschiedenen Drehimpuls verbunden.

Leiten wir nun den Drehimpuls nach der Zeit ab, um die Dynamik zu betrachten:
\begin{align}
\frac{\dd\ivec{L}}{\dd t} &= \frac{\dd}{\dd t} (\ivec{r} \times \ivec{p}) = \left(\frac{\dd\ivec{r}}{\dd t} \times \ivec{p}\right) + \left(\ivec{r} \times \frac{\dd\ivec{p}}{\dd t}\right) \\
&= (\underbrace{\ivec{v} \times (m\ivec{v})}_{=\ivec{0}}) + \left(\ivec{r} \times \ivec{F}\right) = \ivec{r} \times \ivec{F} \mDot
\end{align}
Dabei haben wir das zweite Newtonsche Gesetz in der Form $\ivec{F} = \frac{\dd\ivec{p}}{\dd t}$ verwendet und die Tatsache, dass das Vektorprodukt paralleler Vektoren ($\ivec{v}$ und $\ivec{p}$) null ist.

\begin{importantbox}{Definition: Drehmoment}
Die zeitliche Änderung des Drehimpulses wird als \textbf{Drehmoment} $\ivec{D}$ bezeichnet. Es ist das Vektorprodukt aus dem Ortsvektor $\ivec{r}$ und der am Teilchen angreifenden Kraft $\ivec{F}$
\begin{equation}\label{eq: drehmoment_def}
\ivec{D} := \frac{\dd\ivec{L}}{\dd t} = \ivec{r} \times \ivec{F} \mDot
\end{equation}
Sowohl der Drehimpuls als auch das Drehmoment sind immer in Bezug auf einen festen Punkt (hier den Ursprung) definiert.
\end{importantbox}

Aus \cref{eq: drehmoment_def} folgt direkt der \textbf{Drehimpulserhaltungssatz}: Wenn das gesamte äußere Drehmoment auf ein System verschwindet ($\ivec{D} = \ivec{0}$), dann bleibt der Drehimpuls $\ivec{L}$ zeitlich konstant.

Es besteht eine fundamentale Analogie zwischen der Translations- und der Rotationsbewegung:
$$ \vec{F} = \frac{\dd\vec{p}}{\dd t} \quad \longleftrightarrow \quad \vec{D} = \frac{\dd\vec{L}}{\dd t} $$
Die Kraft $\ivec{F}$ ist die Ursache für die Änderung des Impulses $\ivec{p}$, genauso wie das Drehmoment $\ivec{D}$ die Ursache für die Änderung des Drehimpulses $\ivec{L}$ ist.

\section{Analogie zwischen Translation und Rotation}
\label{sec:analogie_translation_rotation}

Die physikalischen Größen der Translation haben direkte Entsprechungen in der Rotation. Diese Analogie hilft, die Konzepte der Rotationsdynamik zu verstehen.

\begin{rememberbox}{Vektorielle Winkelgeschwindigkeit}
Die Winkelgeschwindigkeit $\vec{\omega}$ lässt sich vektoriell über die Formel
\begin{equation}\label{eq: omega_r_x_v_allgemein}
    \vec{\omega} = \frac{\vec{r} \times \vec{v}}{|\vec{r}|^2}
\end{equation}
berechnen. Wenn man nur am Betrag interessiert ist und die Rotationsachse ihre Richtung nicht ändert (wie bei einer ebenen Kreisbewegung), kann die skalare Beziehung $\omega = \frac{\dd\varphi}{\dd t}$ verwendet werden.
\end{rememberbox}

Die folgende Tabelle stellt die wichtigsten analogen Größen gegenüber.

\begin{table}[h!]
\centering
\caption{Gegenüberstellung von physikalischen Größen der Translation und Rotation.}
\label{tab:analogie_trans_rot}
\renewcommand{\arraystretch}{1.5}
\begin{tabular}{ll|ll}
\toprule
\multicolumn{2}{c|}{\textbf{Translation}} & \multicolumn{2}{c}{\textbf{Rotation}} \\
\midrule
Zeit & $t$ & Zeit & $t$ \\
Ort, Weg & $\vec{r}$, $s$ & Winkel & $\varphi$ \\
Geschwindigkeit & $\vec{v} = \frac{\dd\vec{r}}{\dd t}$ & Winkelgeschwindigkeit & $\vec{\omega} = \frac{1}{|\vec{r}|^2}(\vec{r}\times\vec{v})$ \\
Beschleunigung & $\vec{a} = \frac{\dd\vec{v}}{\dd t}$ & Winkelbeschleunigung & $\vec{\alpha} = \frac{\dd\vec{\omega}}{\dd t}$ \\
Masse (Trägheit) & $m$ & Trägheitsmoment & $I$ \\
Impuls & $\vec{p} = m \cdot \vec{v}$ & Drehimpuls & $\vec{L} = \vec{r} \times \vec{p}$ \\
Kraft & $\vec{F} = \frac{\dd\vec{p}}{\dd t}$ & Drehmoment & $\vec{D} = \vec{r} \times \vec{F} = \frac{\dd\vec{L}}{\dd t}$ \\
Kinetische Energie & $E_{\text{kin}} = \frac{1}{2} m v^2$ & Rotationsenergie & $E_{\text{rot}} = \frac{1}{2} I \omega^2$ \\
\bottomrule
\end{tabular}
\end{table}

\section{Beschleunigte Bezugssysteme und Scheinkräfte}
\label{sec:scheinkraefte}

In \textbf{Inertialsystemen} (unbeschleunigten Bezugssystemen) gelten die Newtonschen Gesetze in ihrer gewohnten Form. In \textbf{beschleunigten Bezugssystemen} müssen jedoch zusätzliche Kräfte, sogenannte \textbf{Scheinkräfte} oder \textbf{Trägheitskräfte}, eingeführt werden, um die Bewegung korrekt zu beschreiben.

\subsection{Das Fahrstuhlexperiment}\label{subsec: scheinkraft_fahrstuhlexperiment}
Stellen wir uns eine Masse $m$ vor, die in einem Fahrstuhl an einer Federwaage hängt. Auf die Masse wirkt die Gravitationskraft $\ivecS{F}{\text{G}} = -mg\ivecS{e}{z}$ nach unten.
\begin{figure}[htb]
    \centering
    \includegraphics[width=0.25\textwidth]{Bilder/Kapitel_Mechanik/Kapitel_Dynamik/scheinkraft_fahrstuhl.png}
    \caption{Kräftediagramm für eine Masse $m$, die in einem mit $\protect\ivec{a}$ nach unten beschleunigten Fahrstuhl an einer Federwaage hängt.}
    \label{fig:fahrstuhl}
\end{figure}
Betrachten wir den Fall, dass der Fahrstuhl mit einer Beschleunigung $\ivec{a} = -a\ivecS{e}{z}$ nach unten beschleunigt wird.

\paragraph{Sicht eines externen Beobachters (Inertialsystem):}
Für einen Beobachter, der außerhalb des Fahrstuhls im Ruhesystem steht, bewegt sich die Masse mit der Beschleunigung $\ivec{a}$ nach unten. Die Gesamtkraft $\ivecS{F}{\text{ges}}$ ist die Summe der Gewichtskraft $\ivecS{F}{\text{G}}$ und der Rückstellkraft der Feder $\ivecS{F}{\text{Feder}}$. Nach Newtons zweitem Gesetz gilt
\begin{equation}
    \ivecS{F}{\text{ges}} = m\ivec{a} = \ivecS{F}{\text{G}} + \ivecS{F}{\text{Feder}}
\end{equation}
Also ist die Federkraft
\begin{equation}
    \ivecS{F}{\text{Feder}} = m\ivec{a} - \ivecS{F}{\text{G}} = m(-a\ivecS{e}{z}) - (-mg\ivecS{e}{z}) = (mg - ma)\ivecS{e}{z}
\end{equation}
Die Federwaage zeigt also ein geringeres Gewicht an als im Ruhezustand. \\

\paragraph{Sicht eines internen Beobachters (beschleunigtes Bezugssystem):}
Ein Beobachter im Fahrstuhl stellt fest, dass die Masse relativ zu ihm in Ruhe ist. Nach seiner Auffassung müsste die Summe aller Kräfte null sein. Er spürt die Gewichtskraft $\ivecS{F}{\text{G}}$ und die Federkraft $\ivecS{F}{\text{Feder}}$. Die Summe dieser \textit{wirklichen} Kräfte ist:
\begin{equation}
    \ivecS{F}{\text{G}} + \ivecS{F}{\text{Feder}} = -mg\ivecS{e}{z} + (mg - ma)\ivecS{e}{z} = -ma\ivecS{e}{z} \neq \ivec{0} \mDot
\end{equation} 
Um diesen Widerspruch aufzulösen und das Ruhegleichgewicht zu erklären, muss der Beobachter im Fahrstuhl eine zusätzliche Kraft, die \textbf{Scheinkraft} $\ivecS{F}{\text{S}}$, einführen:
\begin{gather}
    \ivecS{F}{G} + \ivecS{F}{\text{Feder}} + \ivecS{F}{\text{S}} = \ivec{0} \\
    -ma\ivecS{e}{z} + \ivecS{F}{\text{S}} = \ivec{0} \\
    \llap{$\implies$\;} \ivecS{F}{\text{S}} = +ma\ivecS{e}{z} = -m\ivec{a} 
\end{gather}

\begin{rememberbox}{Schein- oder Trägheitskraft}
Die Scheinkraft $\ivecS{F}{S} = -m\ivec{a}$ ist eine Kraft, die nur in einem mit $\ivec{a}$ beschleunigten Bezugssystem benötigt wird, um die Bewegungsgesetze anwenden zu können. Sie basiert nicht auf einer physikalischen Wechselwirkung, sondern auf der Trägheit der Masse, die sich der Beschleunigung des Bezugssystems widersetzt. Deshalb wird sie auch als \textbf{Trägheitskraft} bezeichnet.
\end{rememberbox}

\subsection{Zentrifugalkraft und Corioliskraft}\label{subsec: zentrifugalkraft_corioliskraft}
Betrachten wir nun ein Bezugssystem $S'(x',y',z')$, das mit einer konstanten Winkelgeschwindigkeit $\ivec{\omega}$ gegenüber einem Inertialsystem $S(x,y,z)$ rotiert, dargestellt in \cref{fig: rotierendes_system_zwei_KS}. Beide Systeme haben einen gemeinsamen Ursprung ($O = O'$). Ein solches rotierendes System $S'$ ist ein klassisches Beispiel für ein beschleunigtes Bezugssystem und ist somit kein Inertialsystem. Der Vektor der Winkelgeschwindigkeit $\ivec{\omega}$ steht parallel zur Rotationsachse von $S'$.

\begin{figure}[htb]
    \centering
    \includegraphics[width=0.45\textwidth]{Bilder/Kapitel_Mechanik/Kapitel_Dynamik/scheinkraft_rotierendes_bezugssystem.png}
    \caption{Ein Inertialsystem $S$ ($x,y$-Ebene) und ein rotierendes Bezugssystem $S'$ ($x',y'$-Ebene), das mit konstanter Winkelgeschwindigkeit $\protect\ivec{\omega}$ um eine Achse rotiert. Der Ortsvektor $\protect\ivec{r} = \protect\ivec{r}'$ zeigt auf einen Punkt $A$.}
    \label{fig: rotierendes_system_zwei_KS}
\end{figure}

Ein Beobachter $B$ im ruhenden System $S$ beschreibt die Geschwindigkeit eines Punktes $A$ als $\ivec{v} = \frac{\dd\ivec{r}}{\dd t}$. Wir nehmen an, dass ein Beobachter $B'$ im rotierenden System $S'$ nicht weißt, dass sein Koordinatensystem rotiert und so misst er die Geschwindigkeit von $A$ relativ zu seinem System als $\ivec{v}'$. 

\paragraph{Wie hängen diese Geschwindigkeiten zusammen?}
Zur Zeit $t$ habe der Punkt $A$ im System $S$ den Ortsvektor 
\begin{equation}
    \ivec{r}(t) = x(t)\cdot \ivecS{e}{x} + y(t)\cdot \ivecS{e}{y} + z(t)\cdot \ivecS{e}{z} \mDot
\end{equation}
und die Geschwindigkeit 
\begin{equation}\label{eq: geschwindigkeit_v_in_S}
    \ivec{v}(t) = \frac{\dd x}{\dd t} \cdot \ivecS{e}{x} + \frac{\dd y}{\dd t}\cdot \ivecS{e}{y} + \frac{\dd z}{\dd t}\cdot \ivecS{e}{z} \mDot
\end{equation}
Die Einheitsvektoren im System $S$ hängen nicht von der Zeit ab und verändern sich daher nicht. \\
Im System $S'$ drückt der Beobachter $B'$ denselben Punkt $A$ zur Zeit $t$ durch den Ortsvektor 
\begin{equation}
    \ivec{r}'(t) = x(t)\cdot \ivecS{e}{x}' + y(t)\cdot \ivecS{e}{y}' + z(t)\cdot \ivecS{e}{z}'
\end{equation}
aus, wobei $\ivec{r}(t) = \ivec{r}'(t)$ gilt, da die Ursprünge der beiden Koordinatensysteme zusammenfallen (siehe \cref{fig: rotierendes_system_zwei_KS}). Wenn der Beobachter, der mit $S'$ rotiert, nicht berücksichtigt, dass sein System rotiert, wird er die Geschwindigkeit des Massenpunktes $A$ als 
\begin{equation}\label{eq: geschw_vD_herleitung_coriolis}
    \ivec{v}' = \frac{\dd \ivec{r}'}{\dd t} = \frac{\dd x'}{\dd t}\cdot \ivecS{e}{x}' + \frac{\dd y'}{\dd t}\cdot \ivecS{e}{y}' + \frac{\dd z'}{\dd t}\cdot \ivecS{e}{z}' 
\end{equation}
definieren. Der Fehler besteht hierbei darin, dass der Beobachter $B'$ in $S'$ die Einheitsvektoren als konstant (nicht-rotierend) annimmt, da er sich ja mit ihnen mit dreht und sie sich somit für ihn nicht ändern. \\

Wenn nun der Beobachter $B$ die Geschwindigkeit des Punktes $A$ in den Koordinaten des rotierenden Systems ausdrückt, so weiß er, dass die Achsen von $S'$ rotieren und die Einheitsvektoren $\ivecS{e}{x}',\ivecS{e}{y}',\ivecS{e}{z}'$ zeitlich nicht konstant sind. Er schreibt daher korrekterweise
\begin{multline}\label{eq: transformation_v_vS_u}
    \ivec{v}(x',y',z') = \\
    \left( \frac{\dd x'}{\dd t} \cdot \ivecS{e}{x}' + \frac{\dd y'}{\dd t} \cdot \ivecS{e}{y}' + \frac{\dd z'}{\dd t} \cdot \ivecS{e}{z}'\right)  +  \left( x'\cdot \frac{\dd \ivecS{e}{x}'}{\dd t} + y'\cdot \frac{\dd \ivecS{e}{y}'}{\dd t} + z'\cdot \frac{\dd \ivecS{e}{z}'}{\dd t} \right) \\
    = \ivec{v}' + \ivec{u}
\end{multline}
Der erste Anteil $\ivec{v}'$ ist die Geschwindigkeit, die $B'$ angibt, und der zweite Anteil $\ivec{u}$ kommt von der Änderung der Einheitsvektoren, \gDh von der Rotation des Koordinatensystems $S'$. \\

Die Endpunkte der Einheitsvektoren $\ivecS{e}{x}', \ivecS{e}{y}', \ivecS{e}{z}'$ führen eine Kreisbewegung mit der Winkelgeschwindigkeit $|\omega|$ aus. Damit können wir laut \cref{eq: omega_r_x_v_allgemein} deren Änderung auch als
\begin{equation}\label{eq: ableitung_einheitsvektor_dash}
    \frac{\dd \ivecS{e}{x}'}{\dd t} = \ivec{\omega} \times \ivecS{e}{x}'\mComma \dots
\end{equation}
schreiben. Laut der „Rechte-Hand-Regel“ zeigt $\left(\dd \ivecS{e}{x}'/\dd t\right) \times \ivecS{e}{x}$ entlang der Rotationsrichtung von $\ivec{\omega}$. Damit können wir den zweiten Term in \cref{eq: transformation_v_vS_u} umschreiben zu 
\begin{equation}\begin{aligned}
    \ivec{u} &= x' \cdot \left( \ivec{\omega} \times \ivecS{e}{x}'\right) + y' \cdot \left( \ivec{\omega} \times \ivecS{e}{y}'\right) + z' \cdot \left( \ivec{\omega} \times \ivecS{e}{z}'\right) \\
    &= \left( \ivec{\omega} \times x'\ivecS{e}{x}'\right) + \left( \ivec{\omega} \times y'\ivecS{e}{y}'\right) + \cdot \left( \ivec{\omega} \times z'\ivecS{e}{z}'\right) \\
    &= \ivec{\omega} \times \ivec{r}' = \ivec{\omega} \times \ivec{r} \mComma
\end{aligned}\end{equation}
weil ja $\ivec{r} = \ivec{r}'$. Damit erhalten wir die korrekte Transformation der Geschwindigkeit vom rotierenden ins ruhende System. 

\begin{rememberbox}[]{Transformation der Geschwindigkeit}
    \begin{equation}\label{eq: geschw_trafo_zwischen_S_Sd}
        \ivec{v} = \ivec{v}' + (\ivec{\omega} \times \ivec{r})
    \end{equation}
    Die Geschwindigkeit $\ivec{v}'$ ist die Geschwindigkeit, die $B'$ in seinem System angibt, wenn er die Rotation der Koordinatenachsen nicht berücksichtigt. Die Geschwindigkeit $v$ kann $B$ durch \cref{eq: geschwindigkeit_v_in_S} oder durch \cref{eq: geschw_trafo_zwischen_S_Sd} angegeben werden.
\end{rememberbox}

\paragraph{Transformation der Beschleunigungen}
Um die Transformation für die Beschleunigung zu finden, leiten wir \cref{eq: geschw_trafo_zwischen_S_Sd} nach der Zeit ab:=0
\begin{equation}\label{eq: beschleunigung_ableitung_von_v_coriolis}\begin{aligned}
    \ivec{a} &= \frac{\dd\ivec{v}}{\dd t} = \frac{\dd}{\dd t} \left(\ivec{v}' + \ivec{\omega} \times \ivec{r} \right) = \frac{\dd\ivec{v}'}{\dd t} + \underbrace{\frac{\dd\ivec{\omega}}{\dd t}}_{=0} \times \ivec{r} + \ivec{\omega} \times \frac{\dd\ivec{r}}{\dd t} \\
    &= \frac{\dd\ivec{v}'}{\dd t} + \ivec{\omega} \times \underbrace{\frac{\dd\ivec{r}}{\dd t}}_{\ivec{v}} = \frac{\dd\ivec{v}'}{\dd t} + \ivec{\omega} \times \ivec{v} \mComma
\end{aligned}\end{equation} 
weil $\ivec{\omega} = \const$. Der Schlüssel liegt hier in der Ableitung von $\ivec{v}'$. Auch wenn $\ivec{v}'$ jene Geschwindigkeit ist, die $B'$ in seinem System $S'$ angibt, wenn er die Rotation der Achsen vernachlässigt, heißt das nicht, dass die Achsen deshalb konstant sind. Wir finden daher für die Ableitung von $\ivec{v}' = v_x'\ivecS{e}{x}' + v_y'\ivecS{e}{y}' + v_z'\ivecS{e}{z}'$ durch die Produktregel
\begin{multline}\label{eq: dvD_dt_erster_teil}
    \frac{\dd\ivec{v}'}{\dd t} = \\
    \left( \frac{\dd v_x'}{\dd t} \ivecS{e}{x}' + \frac{\dd v_y'}{\dd t} \ivecS{e}{y}' + \frac{\dd v_z'}{\dd t} \ivecS{e}{z}'\right) + \left( \frac{\dd \ivecS{e}{x}'}{\dd t} v_x' + \frac{\dd \ivecS{e}{y}'}{\dd t} v_y' + \frac{\dd \ivecS{e}{z}'}{\dd t} v_z' \right)  = \\
    = \ivec{a}' + \left( \frac{\dd \ivecS{e}{x}'}{\dd t} v_x' + \frac{\dd \ivecS{e}{y}'}{\dd t} v_y' + \frac{\dd \ivecS{e}{z}'}{\dd t} v_z' \right) \mDot
\end{multline}
Den zweiten Teil in \cref{eq: dvD_dt_erster_teil} können wir wiederum mit \cref{eq: ableitung_einheitsvektor_dash} vereinfachen zu 
\begin{equation}\label{eq: zweiter_anteil_dvD_dt}\begin{aligned}
    \left( \frac{\dd \ivecS{e}{x}'}{\dd t} v_x' + \frac{\dd \ivecS{e}{y}'}{\dd t} v_y' + \frac{\dd \ivecS{e}{z}'}{\dd t} v_z' \right) &= \left( \ivec{\omega} \times \ivecS{e}{x}' \right) v_x' + \left( \ivec{\omega} \times \ivecS{e}{y}' \right) v_y' + \left( \ivec{\omega} \times \ivecS{e}{z}' \right) v_z' \\
    &= \left(\ivec{\omega} \times v_x' \ivecS{e}{x}'\right) + \left(\ivec{\omega} \times v_y' \ivecS{e}{y}'\right) + \left(\ivec{\omega} \times v_z' \ivecS{e}{z}'\right) \\
    &= \ivec{\omega} \times \ivec{v}' \mDot
\end{aligned}\end{equation}
Damit lässt sich die Ableitung $\dd \ivec{v}'/\dd t$ in \cref{eq: dvD_dt_erster_teil} final schreiben als 
\begin{equation}
    \frac{\dd \ivec{v}'}{\dd t} = \ivec{a}' + \ivec{\omega} \times \ivec{v}' \mComma
\end{equation}
Setzt man den Ausdruck in \cref{eq: beschleunigung_ableitung_von_v_coriolis} ein, erhält man schließlich für die Beschleunigung $\ivec{a}$, dass 
\begin{equation}\label{eq: beschleunigung_a_fast_geschafft}
    \ivec{a} =  \frac{\dd\ivec{v}'}{\dd t} + \ivec{\omega} \times \ivec{v} = \ivec{a}' + \ivec{\omega} \times \ivec{v}' + \ivec{\omega} \times \ivec{v}\mDot
\end{equation}
Nun drücken wir alles in gestrichenen Koordinaten aus und ersetzen $\ivec{v}$ durch das Transformationsgesetz in \cref{eq: geschw_trafo_zwischen_S_Sd}. Damit erhalten wir für den dritten Term in \cref{eq: beschleunigung_a_fast_geschafft} 
\begin{equation}\label{eq: omega_times_v_letzter_teil}\begin{aligned}
    \ivec{\omega} \times \ivec{v} &= \ivec{\omega} \times \left( \ivec{v'} + \ivec{\omega} \times \ivec{r} \right) \\
    &= \ivec{\omega} \times \ivec{v'} + \ivec{\omega} \times (\ivec{\omega} \times \ivec{r} ) \mDot 
\end{aligned}\end{equation}
Damit können wir nun endlich das Transformationsgesetz für die Beschleunigung aufstellen, indem wir \cref{eq: omega_times_v_letzter_teil} in \cref{eq: beschleunigung_a_fast_geschafft} einsetzen.
\begin{rememberbox}[]{Transformation der Beschleunigung}
    Die Beziehung zwischen der Beschleunigung $\ivec{a}$ im Inertialsystem und der im rotierenden System wahrgenommenen Beschleunigung $\ivec{a}'$ lautet
    \begin{equation}\label{eq: a_transformationsgesetz_beschleunigung_coriolis_zentrifugal}
        \ivec{a} = \ivec{a}' + 2(\ivec{\omega} \times \ivec{v}') + \ivec{\omega} \times (\ivec{\omega} \times \ivec{r}) \mDot
    \end{equation}\label{eq: aD_transformationsgesetz_beschleunigung_coriolis_zentrifugal}
    Diese Beziehung kann auch nach $\ivec{a}'$ aufgelöst werden 
    \begin{equation}\begin{aligned}
        \ivec{a}' &= \ivec{a} &&+ 2(\ivec{v}' \times \ivec{\omega}) &&+ \ivec{\omega} \times (\ivec{r} \times \ivec{\omega}) \\
        &= \ivec{a} &&+ \ivecS{a}{\text{Coriolis}} &&+ \ivecS{a}{\text{Zentrifugal}} \mDot 
    \end{aligned}\end{equation}
\end{rememberbox}
Während der Beobachter $B$ in seinem ruhenden System $S$ die Beschleunigung $\ivec{a} = \dd \ivec{v}/\dd t$ misst, muss der Beobachter $B'$ in seinem System $S'$ zusätzliche Beschleunigungen (Kräfte) einführen, um dieselbe Bewegung des Massenpunktes zu erhalten. \\

Multipliziert man diese Gleichung mit der Masse $m$ und stellt sie nach der Kraft im rotierenden System $m\vec{a}' = \vec{F}'$ um, ergibt sich das \textbf{Grundgesetz der Dynamik im rotierenden Bezugssystem}:
\begin{equation}
\label{eq:dyn_rot_system}
\ivec{F}' = \ivec{F} +\underbrace{2m(\ivec{v}' \times \ivec{\omega})}_{\text{Corioliskraft}} + \underbrace{m\ivec{\omega} \times (\ivec{r} \times \ivec{\omega})}_{\text{Zentrifugalkraft}}
\end{equation}
wobei $\ivec{F} = m\ivec{a}$ die \gDQ{wahre} Kraft ist, die im Inertialsystem wirkt. Um die Bewegung in $S'$ zu beschreiben, muss ein Beobachter zusätzlich zur wahren Kraft $\ivec{F}$ zwei Scheinkräfte berücksichtigen.
\begin{figure}[htb]
    \centering
    \includegraphics[width=0.55\textwidth]{Bilder/Kapitel_Mechanik/Kapitel_Dynamik/scheinkraft_coriolis-und-zentrifugalkraft.png}
    \caption{Die Zentrifugalbeschleunigung $\protect\ivec{a}_{\text{Zf}}$ und die Coriolisbeschleunigung $\protect\ivec{a}_{\text{C}}$ für einen Punkt $A$, der sich mit der Geschwindigkeit $\protect\ivec{v}'$ in einem rotierenden Bezugssystem bewegt.}
    \label{fig: coriolis_zentrifugalbeschl}
\end{figure}
\begin{importantbox}[]{Zentrifugalkraft und Corioliskraft}
Die beiden Scheinkräfte im rotierenden Bezugssystem sind (siehe \cref{fig: coriolis_zentrifugalbeschl}):
\begin{itemize}
    \item Die \textbf{Zentrifugalkraft}: $\ivec{F}_{\text{Zf}} = m\ivec{\omega} \times (\ivec{r} \times \ivec{\omega})$. \newline 
    Sie ist stets von der Rotationsachse nach außen gerichtet und wirkt auf jeden Körper im rotierenden System, egal ob er sich bewegt oder ruht.
    \item Die \textbf{Corioliskraft}: $\ivec{F}_{\text{C}} = 2m(\ivec{v}' \times \ivec{\omega})$.\newline
    Sie wirkt nur auf Körper, die sich relativ zum rotierenden System bewegen (für $\ivec{v}' \neq \ivec{0}$), und steht stets senkrecht zur Bewegungsrichtung $\ivec{v}'$ und zur Winkelgeschwindigkeit $\ivec{\omega}$.
\end{itemize}
Beide Kräfte sind Scheinkräfte, da sie nicht auf einer Wechselwirkung beruhen, sondern aus der Beschleunigung des Bezugssystems resultieren.
\end{importantbox}


Die Zentrifugalkraft ist für die \gDQ{Fliehkraft} verantwortlich, die wir in einem Karussell spüren. Die Corioliskraft ist von entscheidender Bedeutung für großräumige Phänomene in der Meteorologie und Ozeanographie, wie z.B. die Drehrichtung von Hoch- und Tiefdruckgebieten auf der Erde.


% \chapter{Gravitation}\label{chap: gravitation}

Die Erforschung des Himmels und der Bewegung der Gestirne zählt zu den ältesten wissenschaftlichen Bestrebungen der Menschheit. Über Jahrhunderte dominierte das geozentrische Weltbild von Ptolemäus, welches die Erde im Zentrum des Universums sah. Erst im 16. Jahrhundert revolutionierte Nikolaus Kopernikus mit seinem heliozentrischen Modell, das die Sonne in den Mittelpunkt rückte, dieses Dogma. Obwohl das kopernikanische Modell in seiner ursprünglichen Form noch von perfekten Kreisbahnen ausging, legte es den Grundstein für eine präzisere Beschreibung des Kosmos.

Auf diesem Fundament baute Johannes Kepler auf. Anhand der außerordentlich genauen und umfangreichen astronomischen Daten, die ein Kollege, Tycho Brahe, über Jahrzehnte gesammelt hatte, gelang es Kepler, die tatsächliche Form der Planetenbahnen zu entschlüsseln. Er erkannte, dass die Planeten sich nicht auf Kreisen, sondern auf Ellipsen bewegen und dass ihre Geschwindigkeit entlang dieser Bahn variiert. Diese revolutionären Erkenntnisse fasste er zwischen 1609 und 1619 in drei fundamentalen Gesetzen zusammen, die nicht nur die Planetenbewegung exakt beschrieben, sondern auch den Weg für Isaac Newtons universelles Gravitationsgesetz ebneten.

\section{Die Keplerschen Gesetze}\label{sec: keplerschen_gesetze}

\subsection{Erstes Keplersches Gesetz}\label{subsec: kepler1}
Das erste Gesetz bricht mit der antiken Vorstellung perfekter Kreisbahnen.

\begin{rememberbox}[]{1. Keplersches Gesetz (Planetengesetz)}
Die Planeten bewegen sich auf Ellipsen, in deren einem Brennpunkt die Sonne steht. 
\end{rememberbox}

Eine Ellipse ist geometrisch definiert als die Menge aller Punkte, für die die Summe der Abstände zu zwei festen Punkten, den Brennpunkten, konstant ist. Im Kontext der Planetenbahnen ist diese Ellipse dargestellt in \cref{fig: kepler1_ellipse} befindet sich die Sonne in einem dieser Brennpunkte. Der sonnennächste Punkt der Bahn wird als \textbf{Perihel} und der sonnenfernste als \textbf{Aphel} bezeichnet. Die Länge der großen Halbachse wird mit $a$ und die der kleinen Halbachse mit $b$ beschrieben. Die numerische Exzentrizität $\varepsilon$ beziffert die Abweichung der Ellipse von einer Kreisform -- für $\varepsilon = 0$ wird die Ellipse zu einem Kreis.

\begin{figure}[htb]
    \centering
    % \includegraphics[width=0.7\textwidth]{Bilder/Kapitel_Mechanik/Kapitel_KeplerGesetze/planetenbahn_aphel_perihel.png} 
    \resizebox{0.55\linewidth}{!}{
    \begin{tikzpicture}
        [dot/.style={circle, fill=black}]
        % --- Definitionen ---
        % Diese Werte können Sie anpassen, um die Form der Ellipse zu ändern
        \def\a{3}      % Große Halbachse
        \def\e{0.50}    % Numerische Exzentrizität (epsilon)
        \pgfmathsetmacro{\b}{\a*sqrt(1-\e^2)} % Kleine Halbachse
        \pgfmathsetmacro{\f}{\a*\e}           % Abstand des Brennpunkts vom Zentrum
        \def\planetangle{40}                  % Winkel für die Position des Planeten
        % --- Achsen zeichnen ---
        \draw (-\a - 1.0, 0) -- (\a + 1.0, 0); % Horizontale Achse
        \draw (0, -\b - 0.7) -- (0, \b + 0.7); % Vertikale Achse
        % --- Elliptische Umlaufbahn ---
        \draw[green!30!black, line width=1mm, smooth] (0,0) ellipse (\a cm and \b cm);
        % --- Sonne (S) im Brennpunkt ---
        % Ein Knoten für den Punkt und das Label "S" darunter
        \node[dot,inner sep=2.5pt, label={[label distance=1.mm]below:Sonne}] (sun) at (\f,0) {};
        % --- Planet ---
        % Positioniert den Planeten auf der Ellipse
        \coordinate (planet_pos) at (\planetangle:\a cm and \b cm);
        \node[dot,inner sep=2.0pt,label=above right:Planet] at (planet_pos) {};
        % --- Vektor r(t) ---Stealth, length=2mm
        \draw[-{Stealth[length=3mm, width=2mm]}, line width=0.5mm] (sun) -- (planet_pos) 
              node[pos=0.65, left=2pt] {\large $\ivec{r}(t)$};
        % --- Beschriftungen ---
        \node[anchor=south east, inner sep=3pt] at (-\a,0) {Aphel};
        \node[anchor=south west, inner sep=3pt] at (\a,0) {Perihel};
        \node[anchor=north, inner sep=3pt] at (-\a/2, 0) {\large $a$};
        \node[anchor=west, inner sep=3pt] at (0, \b/2) {\large $b$};
        % Abstand 'a * epsilon'
        % Striche im Zentrum und am Brennpunkt zur Markierung der Distanz
        \draw[line width=0.5mm] (0, 10pt) -- (0, 4pt);
        \draw[line width=0.5mm] (\f, 10pt) -- (\f, 4pt);
        \node[anchor=south] at (\f/2, 0.2) {\large $a \cdot \varepsilon$};
        \draw[|{Latex}-{Latex}|] (0,0.24) -- ({\a*\e},0.24);
    \end{tikzpicture}
    }
    \caption{Eine elliptische Planetenbahn um die Sonne, die in einem der beiden Brennpunkte steht. Dargestellt sind die große Halbachse $a$, die kleine Halbachse $b$ sowie Perihel, Aphel und der Ortsvektor $\protect\ivec{r}$ (Fahrstrahl) eines Planeten.}\label{fig: kepler1_ellipse}
\end{figure}

\subsection{Zweites Keplersches Gesetz}\label{subsec: kepler2}
Das zweite Gesetz beschreibt die Geschwindigkeit eines Planeten auf seiner Umlaufbahn.
\begin{figure}[htb]
    \centering
    \includegraphics[width=0.7\textwidth]{Bilder/Kapitel_Mechanik/Kapitel_KeplerGesetze/zweites_keplersches_gesetz.png} 
    \caption{Veranschaulichung des zweiten Keplerschen Gesetzes. In gleichen Zeitintervallen $\Delta t$ überstreicht der Fahrstrahl $\protect\ivec{r}(t)$ gleiche Flächen ($A_1 = A_2$). (Quelle:~\cite[S.~63]{Demtroeder2018})}\label{fig: kepler2_flaechensatz}
\end{figure}
\begin{rememberbox}[]{2. Keplersches Gesetz (Flächensatz)}
Der Fahrstrahl (der Verbindungsvektor von der Sonne zum Planeten) überstreicht in gleichen Zeiten gleiche Flächen. 
\end{rememberbox}

Das bedeutet, dass ein Planet sich schneller bewegt, wenn er sich näher an der Sonne befindet (im Perihel), und langsamer, wenn er weiter entfernt ist (im Aphel). Wenn die in der \cref{fig: kepler2_flaechensatz} dargestellten Flächen $A_1$ (nahe dem Aphel) und $A_2$ (nahe dem Perihel) in derselben Zeitspanne $\Delta t$ überstrichen werden, so sind diese Flächen gleich groß: $A_1 = A_2$. 

\paragraph{Herleitung aus der Drehimpulserhaltung}
Das zweite Keplersche Gesetz ist eine direkte Konsequenz der \textbf{Drehimpulserhaltung} für eine Zentralkraft. Betrachten wir ein infinitesimales Zeitintervall $\dd t$, in dem der Planet eine Strecke $\dd\ivec{s} = |\ivec{v}| \dd t$ zurücklegt, dargestellt in \cref{fig: kepler2_herleitung}. Die dabei überstrichene Fläche $\dd A$ kann durch die Fläche eines Dreiecks mit den Eckpunkten ($S, P_1, P_2$) angenähert werden. Die Seitenlängen betragen $\ivec{r}(t)$, $\ivec{r}(t+\dd t)$ und $\dd\ivec{r}$, wobei $\dd \ivec{r} = \ivec{r}(t+\dd t) - \ivec{r}(t)$. Wir wissen ja bereits aus \cref{subsec: herleitung_bogenlaenge_sehnenlaenge}, dass $\lim_{\Delta \varphi \to 0} |\Delta \ivec{r}| = \Delta s$ (\cref{eq: sehnenlänge_gleich_bogenlänge}), weshalb diese Näherung für $\dd t \to 0$ gegen die Fläche $\dd A$ konvergiert.
\begin{figure}[htb]
    \centering
     %\includegraphics[width=0.5\textwidth]{Bilder/Kapitel_Mechanik/Kapitel_KeplerGesetze/zweites_keplersches_gesetz_Zeitintervall.png} 
    \resizebox{0.55\linewidth}{!}{
    \begin{tikzpicture}[
        vec/.style={-{Stealth}, ultra thick, red!50!black},
        point/.style={fill, circle, inner sep=0.2pt}, 
        arrow_2/.style={-{Stealth[length=4mm, width=2mm]}, line width=1.6pt, black}
        ]
        \def\myradius{6.5cm} % Radius des Kreises
        \def\angleA{0}     % Winkel für Punkt A (in Grad)
        \def\angleB{35}     % Winkel für Punkt B (in Grad)
        \def\vecLen{2.1cm}      % Länge der Geschwindigkeitsvektoren
        \def\fontSize{\large}
        % --- 2. Koordinaten definieren ---
        \coordinate (C) at (0,0);
        \coordinate (A) at (\angleA:\myradius);
        \coordinate (B) at (\angleB:\myradius);    
        % --- 3. Kreisbahn zeichnen ---
        \draw[thick] (\angleA-10:\myradius) arc (\angleA-10:\angleB+10:\myradius);
        \path[fill=red!10, draw=black] (C) -- (A) arc (\angleA:\angleB:\myradius) -- cycle;    
        % --- 4. Radien und Punkte zeichnen ---
        \draw[arrow_2] (C) -- (A) node[pos=0.5, below=2pt,sloped] {\large $\ivec{r}(t)$};
        \draw[arrow_2] (C) -- (B) node[pos=0.5, above=2pt,sloped] {\large $\ivec{r}(t+\dd t)$};
        % --- 5. Winkel und Bogenlänge beschriften ---
        \draw[-{Stealth}] (\angleA:1.8cm) arc (\angleA:\angleB:1.8cm);
        \node at ({(\angleA+\angleB)/2}:1.3cm) {\fontSize $\dd \varphi$};
        % --- 6. Geschwindigkeitsvektoren ---
        \draw[vec] (A) -- ++(\angleA+90:\vecLen) node[pos=0.5, right=3.0pt] {\fontSize $\ivec{v}(t)$};
        \draw[dashed, black, line width=1.1pt] (A) -- (B);
        % Für Δr: Linie vom Mittelpunkt der Sehne (A)--(B)
        \coordinate (M_chord) at ($(A)!0.35!(B)$);
        \draw[-, thick, black] (M_chord) -- (5.5,0.7) node[below] {\fontSize $\dd \ivec{r}$};
        
        % Für Δs: Linie vom Mittelpunkt des Bogens zwischen A und B
        \coordinate (M_arc) at ({(3*\angleA+7*\angleB)/(10)}:\myradius);
        \node[font=\large, right=3pt] at (M_arc) {\fontSize $\dd s = |\ivec{v}|\, \dd t$};
    
        \node[point, label={[label distance=3pt]below:$S$}] at (C) {};
        \node[point, label={[label distance=0pt]below right:$P_1$}] at (A) {};
        \node[point, label={[label distance=1pt]above right:$P_2$}] at (B) {};
    
        \node[font=\Large\bfseries] at (barycentric cs:C=1,A=1.2,B=1.2) {$\dd A$};
    \end{tikzpicture}
    }
    \caption{Das infinitesimale Flächenelement $\dd A$, das vom Fahrstrahl $\protect\ivec{r}(t)$ in der Zeit $\dd t$ überstrichen wird.}\label{fig: kepler2_herleitung}
\end{figure}

Die Fläche dieses Dreiecks ist die Hälfte des Betrags des Kreuzprodukts\footnote{Der Betrag des Kreuzprodukts ergibt die Fläche des aufgespannten Parallelogramms.} der beiden aufspannenden Vektoren:
\begin{equation}\label{eq: dA_element}
    \dd A = \frac{1}{2} |\ivec{r} \times \dd\ivec{r}|  \mDot
\end{equation}
Die differentielle Verschiebung kann im Limes durch $\dd \ivec{r} \approx \dd \ivec{s} = \ivec{v} \dd t$ ersetzt werden und man erhält
\begin{equation}
    \dd A = \frac{1}{2} |\ivec{r} \times \ivec{v} \dd t|
\end{equation}
Die \textbf{Flächengeschwindigkeit} ist somit die zeitliche Änderung der Fläche
\begin{equation}\label{eq: flaechengeschwindigkeit}
    \frac{\dd A}{\dd t} = \frac{1}{2} |\ivec{r}(t) \times \ivec{v}(t)| \mDot
\end{equation}
Wenn wir den Impuls $\ivec{p} = m\ivec{v}$ einsetzen, indem wir $\ivec{v}$ durch $\ivec{p}$ ersetzen und dafür die rechte Seite durch $m$ dividieren, erkennen wir den Zusammenhang mit dem Drehimpuls $\ivec{L} = \ivec{r} \times \ivec{p}$:
\begin{equation}\label{eq: flaechengeschwindigkeit_drehimpuls}
    \frac{\dd A}{\dd t} = \frac{1}{2m} |\ivec{r}(t) \times \ivec{p}(t)| = \frac{|\ivec{L}(t)|}{2m}
\end{equation}
Da das zweite Keplersche Gesetz besagt, dass die Flächengeschwindigkeit konstant ist ($\dd A/\dd t = \const$), folgt daraus direkt, dass auch der Betrag des Drehimpulses $|\ivec{L}|$ konstant sein muss. Dies ist immer der Fall, wenn die wirkende Kraft (hier die Gravitationskraft) eine Zentralkraft ist, \gDh immer auf das Zentrum (die Sonne) gerichtet ist.




\subsection{Drittes Keplersches Gesetz}\label{subsec: kepler3}
Das dritte Gesetz stellt einen Zusammenhang zwischen den Umlaufzeiten und den Bahngrößen der Planeten her.

\begin{rememberbox}{3. Keplersches Gesetz (Periodengesetz)}
Die Quadrate der Umlaufzeiten ($T_1, T_2$) zweier Planeten verhalten sich wie die Kuben (dritten Potenzen) der großen Halbachsen ($a_1, a_2$) ihrer Bahnen. 
\begin{equation}\label{eq: kepler3}
    \frac{T_1^2}{T_2^2} = \frac{a_1^3}{a_2^3} \mDot
\end{equation}
\end{rememberbox}
Dies impliziert, dass der Quotient $T^2/a^3 = \const$ für alle Planeten, die um dasselbe Zentralgestirn (die Sonne) kreisen. Planeten auf größeren Bahnen benötigen also überproportional länger für einen Umlauf.

\section{Newtonsches Gravitationsgesetz}\label{sec: newton_gravitation}
Während Keplers Gesetze die Planetenbewegungen phänomenologisch beschrieben, war es Isaac Newton, der die dahinterliegende physikalische Ursache aufdeckte. Er postulierte, dass dieselbe Kraft, die einen Apfel zu Boden fallen lässt, auch die Planeten auf ihren Bahnen um die Sonne hält: die universelle \textbf{Massenanziehung} oder \textbf{Gravitation}. In diesem Abschnitt möchten wir zeigen, dass die Form der Gravitationskraft direkt aus den Keplerschen Gesetzen folgt.\\ 

Nach Newtons drittem Axiom (\gDQ{Actio gleich Reactio}) muss die Anziehungskraft zwischen zwei Körpern proportional zu beiden Massen, $m_1$ und $m_2$, sein: 
\begin{equation}
    F_{\text{G}} \propto m_1 \cdot m_2 \mDot
\end{equation}
Nun setzen wir einen Proportionalitätsfaktor $G$ ein. Außerdem wissen wir aus dem zweiten Keplerschen Gesetz, dass die Kraft vom Abstand $r$ der beiden Massen abhängen muss: 
\begin{equation}
    F_{\text{G}} = G \cdot m_1 \cdot m_2 \cdot f(r) \mDot
\end{equation}
Die genaue funktionale Abstandsabhängigkeit $f(r)$ dieser Kraft leitete Newton aus dem dritten Keplerschen Gesetz her, indem er den Spezialfall einer kreisförmigen Planetenbahn betrachtete ($\varepsilon=0$, also $a=r$) -- siehe \cref{fig: planet_kreisbahn}. 
\begin{figure}[tb]
    \centering
    \resizebox{0.45\linewidth}{!}{
    % Planetenbahn um die Sonne als Ellipse (Aphel, Perihel, Fahrstrahl)
    \begin{tikzpicture}[dot/.style={circle, fill=black}]
        % Diese Werte können Sie anpassen, um die Form der Ellipse zu ändern
        \def\a{3}      % Große Halbachse
        \def\e{0.0}    % Numerische Exzentrizität (epsilon)
        \pgfmathsetmacro{\b}{\a*sqrt(1-\e^2)} % Kleine Halbachse
        \pgfmathsetmacro{\f}{\a*\e}           % Abstand des Brennpunkts vom Zentrum
        \def\planetangle{40}                  % Winkel für die Position des Planeten
        % --- Achsen zeichnen ---
        \draw (-\a - 1.0, 0) -- (\a + 1.0, 0); % Horizontale Achse
        \draw (0, -\b - 0.7) -- (0, \b + 0.7); % Vertikale Achse
        % --- Elliptische Umlaufbahn ---
        \draw[green!50!blue, line width=1mm, smooth] (0,0) ellipse (\a cm and \b cm);
        % --- Sonne (S) im Brennpunkt ---
        \node[dot,inner sep=2.5pt, label={[label distance=0.2mm]below right:Sonne}] (sun) at (\f,0) {};
        % --- Planet ---
        % Positioniert den Planeten auf der Ellipse
        \coordinate (planet_pos) at (\planetangle:\a cm and \b cm);
        \node[dot,inner sep=2.0pt,label=above right:Planet] at (planet_pos) {};
        % --- Vektor r(t) ---Stealth, length=2mm
        \draw[-{Stealth[length=3mm, width=2mm]}, line width=0.5mm] (sun) -- (planet_pos) 
              node[pos=0.65, left=5pt] {\large $\ivec{r}(t)$};
        % --- Beschriftungen ---
        \node[anchor=north, inner sep=3pt] at (-\a/2, 0) {\large $a = r$};
        % Striche im Zentrum und am Brennpunkt zur Markierung der Distanz
    \end{tikzpicture}   
    }
    \caption{Ein Planet, der eine Kreisbahn um die Sonne ausführt.}\label{fig: planet_kreisbahn}
\end{figure}
Für eine gleichförmige Kreisbewegung mit Radius $r$ und Winkelgeschwindigkeit $\omega$ muss die Gravitationskraft $F_{\text{G}}$ die benötigte Zentripetalkraft $F_{\text{Zp}}$ aufbringen, um einen Planeten der Masse $m$ auf seiner Bahn zu halten: 
\begin{equation}\label{eq: grav_vs_zentri}
    \begin{gathered}
    F_\text{G} = F_\text{Zp} \\
    \implies G \cdot M \cdot m \cdot f(r) = m \omega^2 r
    \end{gathered}
\end{equation}
wobei $m$ die Planetenmasse, $M$ die Sonnenmasse und $G$ die Gravitationskonstante ist.

Das dritte Keplerschen Gesetz, $T^2/a^3 = \const$, wird für eine Kreisbahn ($a=r$) zu $T^2/r^3 = c = \const$. Mit der allgemeinen Beziehung für Umlaufzeiten $T = 2\pi/\omega$ folgt:
\begin{equation}\label{eq: omega_from_kepler3}
    \frac{T^2}{r^3} = \frac{{(2\pi/\omega)}^2}{r^3} = c \quad \implies \quad \omega^2 = \frac{4\pi^2}{c} \frac{1}{r^3} \mDot
\end{equation}
Setzt man diesen Ausdruck für $\omega^2$ in Gleichung \cref{eq: grav_vs_zentri} ein, erhält man:
\begin{equation}
    G \cdot M \cdot m \cdot f(r) = m \left( \frac{4\pi^2}{c} \frac{1}{r^3} \right) r = \left( \frac{4\pi^2}{c} \right) \cdot m \cdot \frac{1}{r^2}
\end{equation}
Durch Vergleich der Terme erkennt man, dass $4\pi^2 /c = G\cdot M$ und dass die Abstandsabhängigkeit der Kraft $f(r) = 1/r^2$ sein muss. Durch diese Betrachtung findet man demnach wiederum das Newtonsche Gravitationsgesetz. 

\begin{rememberbox}[]{Newtonsches Gravitationsgesetz}
Zwei beliebige Massenpunkte $m_1$ und $m_2$ im Abstand $r$ ziehen sich mit einer Kraft $\ivecS{F}{\text{G}}$ an, deren Betrag proportional zum Produkt der Massen und umgekehrt proportional zum Quadrat ihres Abstandes ist. 
\begin{equation}\label{eq: newton_gravitation}
    \ivecS{F}{\text{G}, 2 \to 1} = -G \frac{m_1 m_2}{r^2} \ivecS{e}{2\rightarrow 1}
\end{equation}
Die \textbf{Gravitationskonstante} hat den Wert $G \approx \SI{6.67e-11}{\newton\metre\squared\per\kilogram\squared}$. 
\end{rememberbox}

\section{Erste kosmische Geschwindigkeit (Umlaufgeschwindigkeit)}\label{sec: erste_kosmische_geschwindigkeit}
Die erste kosmische Geschwindigkeit $v_I$ ist die Mindestgeschwindigkeit, die ein Körper tangential zur Erdoberfläche haben muss, um eine stabile, niedrige Kreisbahn um die Erde einzunehmen. 
\begin{figure}[htb]
    \centering
    \includegraphics[width=0.3\textwidth]{Bilder/Kapitel_Mechanik/erste_kosmische_geschwindigkeit.png}
    \caption{Ein Objekt wird von der Oberfläche eines Himmelskörpers (Masse $M$, Radius $R$) mit der Anfangsgeschwindigkeit $\protect\ivecS{v}{0}$ tangential zur Oberfläche abgefeuert.}\label{fig: erste_kosmische_geschwindigkeit}
\end{figure}
Für eine stabile Kreisbahn muss die anziehende Gravitationskraft $\ivecS{F}{G}$ genau die erforderliche Zentripetalkraft $\ivecS{F}{\text{Zp}}$ aufbringen. Wir setzen die Beträge der beiden Kräfte gleich:
\begin{equation}
    F_{\text{Zp}} = F_G
\end{equation}
Die Zentripetalkraft ist $F_{\text{Zp}} = mv_I^2 /R$, und die Gravitationskraft auf der Erdoberfläche ist $F_G = G (M\cdot m)/R^2$. Einsetzen ergibt:
\begin{equation}
    \frac{m v_I^2}{R} = G \frac{Mm}{R^2} \mDot
\end{equation}
Wir können die Masse des Körpers $m$ und einen Faktor $R$ kürzen und erhalten:
\begin{equation}
    v_I^2 = \frac{GM}{R}
\end{equation}
Daraus folgt für die erste kosmische Geschwindigkeit:
\begin{equation}\label{eq: v_kosmisch_1}
    v_I = \sqrt{\frac{GM}{R}} \mDot
\end{equation}
Einsetzen der gegebenen Werte für die Erde ergibt
\begin{equation}
    v_I = \sqrt{\frac{(\SI{6.674e-11}{\newton\meter\squared\per\kilo\gram\squared}) \cdot (\SI{5.972e24}{\kilo\gram})}{\SI{6.371e6}{\meter}}} \approx \SI{7909}{\meter\per\second} \approx \SI{7.91}{\kilo\meter\per\second} \mDot
\end{equation}
Wenn ein Körper mit $v_I \approx \SI{7.91}{\kilo\meter\per\second}$ horizontal von der Erdoberfläche gestartet wird, umkreist der Körper die Erde in einer stabilen Umlaufbahn, ohne jemals auf die Oberfläche zu treffen.

\section{Zweite kosmische Geschwindigkeit (Fluchtgeschwindigkeit)}\label{sec: zweite_kosmische_geschwindigkeit}
Mit dem Gravitationsgesetz können wir berechnen, welche Geschwindigkeit ein Objekt benötigt, um das Schwerefeld eines Himmelskörpers, wie zum Beispiel der Erde, dauerhaft zu verlassen. Diese Geschwindigkeit wird als zweite kosmische Geschwindigkeit oder \textbf{Fluchtgeschwindigkeit} bezeichnet.

\begin{figure}[htb]
    \centering
    % \includegraphics[width=0.3\textwidth]{Bilder/Kapitel_Mechanik/Kapitel_KeplerGesetze/zweite_kosmische_geschwindigkeit.png} 
    \begin{tikzpicture}[scale=0.8,
        >=Latex, font=\large,
        dimline/.style={|-|, thick}
        ]

        % --- Definitionen ---
        \def\R{2.5}      % Radius des Planeten
        \def\rDist{6.0}  % Gesamtabstand r (vom Zentrum bis oben)
        \def\Angle{-20}  % Winkel für den Radius R

        % --- 1. Kugel (Planet) ---
        \draw[thick] (0,0) circle (\R);
        \node[below=5pt] at (0,0) {M};

        % Radius R einzeichnen
        \draw[thick] (0,0) -- (\Angle:\R) node[midway, below] {R};

        % --- 2. Vertikale Achse ---
        % Dünne Hilfslinie vom Zentrum nach oben
        \draw[thin] (0,0) -- (0, \rDist);

        % --- 3. Vektoren (Rot) ---
        
        % Einheitsvektor r_hat (vom Zentrum aus)
        \draw[->, red, line width=1.5pt] (0,0) -- (0, 1.2) 
            node[midway, right, text=black] {$\hat{r}$};

        % Geschwindigkeitsvektor v0 (ab der Oberfläche)
        \draw[->, red, line width=1.5pt] (0, \R) -- (0, \R + 1.8) 
            node[midway, left, text=black] {$v_0$};

        % --- 4. Bemaßung (Die "Klammern" als Pfeile) ---
        
        % Maßlinie für r (links)
        % Vom Zentrum (0,0) bis zur Spitze (\rDist)
        % Wir schieben sie etwas nach links (x = -0.8)
        \draw[dimline] (-0.8, 0) -- (-0.8, \rDist) 
            node[midway, fill=white] {$r$};

        % Hilfslinie gestrichelt von der Spitze zur Maßlinie r
        \draw[dashed, gray] (0, \rDist) -- (-0.8, \rDist);
        % Hilfslinie gestrichelt vom Zentrum zur Maßlinie r
        \draw[dashed, gray] (0, 0) -- (-0.8, 0);


        % Maßlinie für h = r - R (rechts)
        % Von der Oberfläche (\R) bis zur Spitze (\rDist)
        % Wir schieben sie etwas nach rechts (x = 0.8), damit sie v0 nicht stört
        \draw[dimline] (0.8, \R) -- (0.8, \rDist) 
            node[midway, right, xshift=5pt] {$h = r - R$};

        % Hilfslinie gestrichelt von der Oberfläche zur Maßlinie h
        \draw[dashed, gray] (0, \R) -- (0.8, \R);
        % Hilfslinie gestrichelt von der Spitze zur Maßlinie h
        \draw[dashed, gray] (0, \rDist) -- (0.8, \rDist);

    \end{tikzpicture}
    \caption{Ein Objekt wird von der Oberfläche eines Himmelskörpers (Masse $M$, Radius $R$) mit der Anfangsgeschwindigkeit $\protect\ivecS{v}{0}$ senkrecht nach oben abgefeuert.}\label{fig: zweite_kosmische_geschwindigkeit}
\end{figure}
Im Gegensatz zur konstanten Erdbeschleunigung $g$ in Bodennähe ist die Gravitationsbeschleunigung allgemein ortsabhängig: $a(r) = -GM/r^2$. Wir betrachten ein Objekt der Masse $m$, das von der Erdoberfläche (Radius $R$) senkrecht nach oben geschossen wird. Da die Beschleunigung vom Ort $r$ und nicht von der Zeit $t$ abhängt, formen wir den Beschleunigungsterm um, wobei wir nur an den Beträgen der Größen interessiert sind und daher nicht vektoriell rechnen:
\begin{equation}\label{eq: a_v_dv_dr}
    a = \frac{\dd v}{\dd t} = \frac{\dd v}{\dd r} \frac{\dd r}{\dd t} = v \frac{\dd v}{\dd r} \mDot  
\end{equation}
Setzen wir dies in die Bewegungsgleichung ein, erhalten wir eine Differentialgleichung, die durch Trennung der Variablen lösbar ist:
\begin{equation}\label{eq: dgl_flucht}
    a= v \frac{\dd v}{\dd r} = -\frac{GM}{r^2} \quad \implies \quad v \, \dd v = -\frac{GM}{r^2} \, \dd r
\end{equation}
Wir integrieren nun von den Anfangsbedingungen -- Start an der Erdoberfläche $r_0 = R$ mit Anfangsgeschwindigkeit $v_0$ -- bis zu einem variablen Endpunkt (Höhe $r_\text{f}$ mit Endgeschwindigkeit $v_\text{f}$): 
\begin{align}
    \int_{v_0}^{v_{\text{f}}} v \, \dd v &= \int_{R}^{r_{\text{f}}} -\frac{GM}{r^2} \, \dd r \\
    \left. \frac{1}{2} v^2 \right|_{v_0}^{v_{\text{f}}} &= \left. \frac{GM}{r} \right|_{R}^{r_{\text{f}}} \\
    \frac{1}{2}(v_{\text{f}}^2 - v_0^2) &= GM \left( \frac{1}{r_{\text{f}}} - \frac{1}{R} \right) \mDot
\end{align}
Auflösen nach der Endgeschwindigkeit $v_{\text{f}}^2$ ergibt:
\begin{equation}\label{eq: vf_general}
    v_{\text{f}}^2 = v_0^2 + 2GM \left( \frac{1}{r_{\text{f}}} - \frac{1}{R} \right)\mDot
\end{equation}
Um die maximale Höhe $h_{\maxText}$ zu finden, die bei einer Anfangsgeschwindigkeit $v_0$ erreicht wird, ersetzen wir $r_{\text{f}}$ durch $r_{\text{f}} = R + h_{\maxText}$. Für den Umkehrpunkt gilt außerdem $v_{\text{f}} = 0$: 
\begin{equation}
    0 = v_0^2 - 2GM \left( \frac{1}{R} - \frac{1}{R+h_{\maxText}} \right) = v_0^2 - 2GM \frac{h_{\maxText}}{R(R+h_{\maxText})} \mDot
\end{equation}
Mit der Erdbeschleunigung an der Oberfläche, $g = GM/R^2$, lässt sich dies umformen zu
\begin{equation}\label{eq: hmax}
    v_0^2 = \frac{2gR h_{\maxText}}{R+h_{\maxText}} \quad \implies \quad h_{\maxText} = \frac{v_0^2 R}{2gR - v_0^2} \mDot
\end{equation}
Aus dieser Gleichung wird ersichtlich, dass die maximale Höhe $h_{\maxText}$ unendlich wird, wenn der Nenner gegen Null geht. Dies geschieht, wenn die Anfangsgeschwindigkeit $v_0$ einen kritischen Wert erreicht.

\begin{importantbox}{Zweite kosmische Geschwindigkeit $v_{II}$}
Die Fluchtgeschwindigkeit ist die minimale Anfangsgeschwindigkeit, die ein Objekt benötigt, um das Gravitationsfeld eines Himmelskörpers ohne weiteren Antrieb zu verlassen ($h_{\maxText} \to \infty$). Man erhält sie, indem man den Nenner in Gleichung \cref{eq: hmax} gleich null setzt:
\begin{equation}\label{eq: v_escape}
    v_0^2 = 2gR \quad \implies \quad v_0 = v_{II} = \sqrt{2gR}
\end{equation}
Für die Erde beträgt die Fluchtgeschwindigkeit etwa $\SI{11.2}{\kilo\metre\per\second}$. 
\end{importantbox}

% Chapter end - always start new page after chapter
\newpage
%\chapter{Arbeit und Energie}\label{chap: Arbeit_und_Energie}
In der Mechanik sind die Konzepte von Arbeit und Energie von zentraler Bedeutung, da sie es ermöglichen, komplexe Bewegungsvorgänge auf eine oft einfachere Weise zu analysieren, als es allein mit den Newtonschen Axiomen möglich wäre. Sie bilden die Grundlage für einen der fundamentalsten Sätze der Physik: den Energieerhaltungssatz.

\section{Mechanische Arbeit}\label{sec: mechanische_arbeit}
In der Alltagssprache wird der Begriff \gDQ{Arbeit} vielfältig verwendet, doch in der Physik besitzt er eine präzise Definition.

\begin{rememberbox}[]{Definition: Mechanische Arbeit}
    Mechanische Arbeit $W$ (\textit{Engl.:} \gDQ{work}) ist die Übertragung von Energie von einem System auf ein anderes durch das Wirken einer Kraft entlang eines Weges. Arbeit ist eine skalare Größe, die positiv, negativ oder null sein kann. 
    \begin{itemize}[itemsep=1pt]
        \item Die von einem Körper A an einem Körper B verrichtete Arbeit $W$ ist \textbf{positiv}, wenn Energie von A auf B übertragen wird. A leistet eine Arbeit, $W > 0$.
        \item Sie ist \textbf{negativ}, wenn Energie von B auf A übertragen wird. An A wird eine Arbeit verrichtet $W < 0$. 
        \item Wird keine Energie übertragen, ist die Arbeit \textbf{null}. 
    \end{itemize}
\end{rememberbox}

Legt ein Massepunkt, der sich im Einfluss eines Kraftfeldes $\ivec{F}(\ivec{r})$ befindet, ein kleines, geradliniges Wegstück $\Delta\ivec{r}$ zurück, so verrichtet die Kraft an dem Massenpunkt die Arbeit $\Delta W$. Diese wird durch das Skalarprodukt von Kraft und Weg definiert: 
\begin{equation}\label{eq: arbeit_infinitesimal}
    \Delta W = \ivec{F}(\ivec{r}) \cdot \Delta\ivec{r} \mDot
\end{equation}
Beachte, dass es sich bei der Multiplikation in \cref{eq: arbeit_infinitesimal} um ein inneres Produkt (Skalarprodukt) aus $\ivec{F}$ und $\Delta \ivec{r}$ handelt und die Arbeit damit ein Skalar ist. Nur die Komponente der Kraft, die parallel zum Wegstück $\Delta\ivec{r}$ liegt, trägt zur Arbeit bei. Eine Kraft, die senkrecht zum Weg wirkt, verrichtet keine Arbeit, da das Skalarprodukt in diesem Fall null ist: 
\begin{equation}
    \ivec{F} \cdot \Delta \ivec{r} = 0 \Longleftrightarrow W = 0 \mDot
\end{equation}

\begin{figure}[h!]
    \centering
    \includegraphics[width=0.5\textwidth]{Bilder/Kapitel_ArbeitEnergie/bahnkurve_def_arbeit.png}
    \caption{Ein Massenpunkt bewegt sich unter dem Einfluss der Kraft $\protect\ivec{F}$ entlang seiner Bahnkurve das Wegelement $\Delta \protect\ivec{r}$ zurück und verrichtet dabei die Arbeit $\Delta W = \protect\ivec{F} \cdot \Delta \protect\ivec{r}$.}\label{fig: bahnkurve_arbeit_delta_r}
\end{figure}

Bewegt sich der Massenpunkt entlang einer beliebigen Bahnkurve von einem Punkt $P_1$ nach $P_2$, dargestellt in \cref{fig: bahnkurve_arbeit_delta_r} so setzt sich die Gesamtarbeit aus den Beiträgen aller kleinen Wegstücke $\Delta\ivecS{r}{i}$ zusammen:
\begin{equation}
    W = \sum_{i} \Delta W_{i} = \sum_{i} \ivec{F}(\ivecS{r}{i}) \cdot \Delta\ivecS{r}{i} \mDot
\end{equation}
Im Grenzwert unendlich kleiner Wegstücke ($\Delta\ivec{r} \to \dd\ivec{r}$) geht diese Summe in ein Wegintegral (auch Linienintegral genannt) über. Dies ist die allgemeine Formel für die von einer Kraft $\ivec{F}$ entlang eines Weges $\mathcal{C}$ von $P_1$ nach $P_2$ verrichtete Arbeit. 

\begin{importantbox}[]{Definition: Arbeit als Wegintegral}
    Die von einer Kraft $\ivec{F}(\ivec{r})$ entlang einer Bahnkurve $\mathcal{C}$ von einem Punkt $P_1$ zu einem Punkt $P_2$ verrichtete Arbeit $W$ ist durch das Wegintegral
    \begin{equation}\label{eq: arbeit_wegintegral}
        W = \int_{\mathcal{C}: P_1}^{P_2} \ivec{F}(\ivec{r}) \cdot \dd\ivec{r}
    \end{equation}
    gegeben.
\end{importantbox}

\begin{rememberbox}[]{Einheit der Arbeit}
    Die SI-Einheit der Arbeit ist das \textbf{Joule (J)}. Es gilt:
    \begin{equation}
        [W] = \SI{1}{\newton\meter} = \SI{1}{\joule} \mDot
    \end{equation}
    Die Einheit der Arbeit ist identisch mit der der Energie. 
\end{rememberbox}

Zur praktischen Berechnung kann das Linienintegral in kartesischen Koordinaten in eine Summe von drei gewöhnlichen Integralen zerlegt werden. Das Skalarprodukt lautet ausgeschrieben:
\begin{equation}
    \ivec{F}(\ivec{r}) \cdot \dd\ivec{r} = F_x \dd x + F_y \dd y + F_z \dd z \mDot
\end{equation}
Damit berechnet sich die Arbeit über:
\begin{equation}\label{eq: arbeit_integral_komponenten}\begin{aligned}
    W &= \int_{P_1}^{P_2} \ivec{F}(\ivec{r}) \cdot \dd \ivec{r} \\
    &= \int_{x_1}^{x_2} F_x(x,y,z) \dd x + \int_{y_1}^{y_2} F_y(x,y,z) \dd y + \int_{z_1}^{z_2} F_z(x,y,z) \dd z \mComma
\end{aligned}\end{equation}
wobei $P_1 = (x_1, y_1, z_1)$ und $P_2 = (x_2, y_2, z_2)$. 

\section{Kraftfelder}\label{sec: kraftfelder}
Ein Kraftfeld ist ein Raumbereich, in dem auf einen Probekörper an jedem Punkt eine definierte Kraft (nach Betrag und Richtung) wirkt: 
\begin{equation}
    \ivec{F} = \ivec{F}(\ivec{r}) = \ivec{F}(x,y,z)
\end{equation}Man unterscheidet zwischen konservativen und nicht-konservativen Kraftfeldern, je nachdem, ob die verrichtete Arbeit vom gewählten Weg abhängt.
\begin{examplebox}[breakable]{Kraftfeld: Schwerefeld der Erde}
Das Schwerefeld der Erde ist ein klassisches Beispiel für ein Kraftfeld. In guter Näherung kann es als Zentralkraftfeld beschrieben werden, bei dem die Kraft auf eine Probemasse $m$ stets zum Erdmittelpunkt gerichtet ist. Solche Zentralkraftfelder sind immer konservativ, wie wir sehen werden.

Die Kraft, die auf die Masse $m$ im Abstand $r$ vom Erdmittelpunkt wirkt, ist durch das Newtonsche Gravitationsgesetz gegeben. Das Kraftfeld der Erdanziehung, wobei der Ursprung des Koordinatensystems im Erdmittelpunkt liegt, lautet:
\begin{equation}\label{eq: gravitationsgesetz_erde}
    \ivecS{F}{\text{G}}(x,y,z) = \ivecS{F}{\text{G}}(r) = -G \frac{M_\text{E} \cdot m}{r^2} \ivecS{e}{r} \mDot
\end{equation}
mit $r = \sqrt{x^2 + y^2 + z^2}$. Hierbei ist die Gravitationskonstante $G \approx \SI{6.674e-11}{\newton\meter\squared\per\kilogram\squared}$, die Erdmasse $M_\text{E} \approx \SI{5.972e24}{\kilogram}$ und der radial vom Erdmittelpunkt nach außen zeigende Einheitsvektor $\ivecS{e}{r}$.\\

Wir berechnen die Anziehungskraft auf eine Probemasse von $m = \SI{1}{\kilo\gram}$ an zwei verschiedenen Orten. Der mittlere Erdradius beträgt $R_\text{E} \approx \SI{6371}{\kilo\meter}$.

\begin{itemize}[itemsep=1.5pt]
    \item \textbf{Auf Seehöhe} ($r = R_\text{E} = \SI{6371e3}{\meter}$):
    \begin{equation}
        |\ivecS{F}{\text{G}}(r)| = G \frac{M_\text{E} \cdot m}{r^2} \approx \num{6.674e-11} \cdot \frac{\num{5.972e24} \cdot \num{1}}{(\num{6.371e6})^2} \approx \SI{9.82}{\newton} \mDot
    \end{equation}
    Dies entspricht ungefähr der bekannten Erdbeschleunigung von $g \approx \SI{9.82}{\meter\per\second\squared}$.

    \item \textbf{Auf dem Mount Everest} ($h_{\text{MtE}} \approx \SI{8849}{\meter}$):
    Der Abstand zum Erdmittelpunkt ist nun $r = R_\text{E} + h_{\text{MtE}} = \SI{6371000}{\meter} + \SI{8849}{\meter} = \SI{6379849}{\meter}$.
    \begin{equation}
        |\ivecS{F}{\text{G}}(r)| = G \frac{M_\text{E} \cdot m}{r^2} \approx \num{6.674e-11} \cdot \frac{\num{5.972e24} \cdot \num{1}}{(\num{6.379849e6})^2} \approx \SI{9.79}{\newton} \mDot
    \end{equation}
\end{itemize}
Wie erwartet, ist die Anziehungskraft auf dem Gipfel des Mount Everest geringfügig schwächer als auf Seehöhe, da der Abstand zum Erdmittelpunkt größer ist.
\end{examplebox}

\subsection{Nicht-konservative Kraftfelder}\label{subsec: nicht-konservative_kraftfelder}
Bei einem nicht-konservativen Kraftfeld hängt die Arbeit, die beim Bewegen eines Massenpunktes von $P_1$ nach $P_2$ verrichtet wird, vom gewählten Weg ab. Dies ist exemplarisch in \cref{fig: nichtkonservatives_feld_wegabhaengig} dargestellt. Hier unterscheiden sich die Kräfte entlang des Weges $\ivecS{r}{O}$ und $\ivecS{r}{U}$, sodass für die Arbeit entlang der zwei verschiedenen Wege $\mathcal{C}_O$ (oberer Weg) und $\mathcal{C}_U$ (unterer Weg) gilt: 
\begin{equation}\label{eq: arbeit_nicht_konservativ}
    W_O = \int_{\mathcal{C}_O: P_1}^{P_2} \ivec{F}(\ivec{r}) \cdot \dd\ivec{r} \neq \int_{\mathcal{C}_U: P_1}^{P_2} \ivec{F}(\ivec{r}) \cdot \dd\ivec{r} = W_U \mDot
\end{equation}
Folglich ist die Arbeit, die auf einem geschlossenen Pfad (\zB von $P_1$ nach $P_2$ über $\mathcal{C}_O$ und zurück von $P_2$ nach $P_1$ über $\mathcal{C}_U$) verrichtet wird, ungleich null
\begin{equation}\label{eq: arbeit_nicht_konservativ_geschlossen}
    \oint \ivec{F}(\ivec{r}) \cdot \dd\ivec{r} \neq 0 \mComma
\end{equation}
weshalb solche Kraftfelder \textbf{nicht konservativ} genannt werden. Typische Beispiele für nicht-konservative Kräfte sind Reibungskräfte oder zeitlich veränderliche Kräfte.
\begin{figure}[h!]
    \centering
    \resizebox{0.55\linewidth}{!}{
    \begin{tikzpicture}[
            font=\large,
            point/.style={circle, fill=orange, draw=black, thick, inner sep=3.1pt},
            label_box/.style={fill=white, draw=gray, fill opacity=0.95, text opacity=1, rounded corners=2pt, inner sep=4pt}
        ]
        \coordinate (P1) at (-3, -2.5);
        \coordinate (P2) at (3.5, 1.5);
        % 1. Hintergrundbild platzieren
        \node at (0,0) {\includegraphics[width=10cm]{Bilder/Kapitel_ArbeitEnergie/force_field.pdf}};
        % 4. Kurvenpfad von P1 nach P2 zeichnen
        \draw[line width=1.0mm, black] (P1) .. controls (-3.5, 0) and (-2, 2.5) .. (0, 2.2)
            .. controls (2, 1.9) and (3, 2.5) .. (P2);
        \draw[line width=1.0mm, black] (P1) .. controls (-2.5, -4.1) and (-0.5, -3.5) .. (1, -3) 
             .. controls (2.5, -2) and (2.5, 0) .. (P2);
        % 5. Punkte P1 und P2 setzen und beschriften
        \node[point, label={[label_box, label distance=0.0cm]225:$P_1$}] at (P1) {};
        \node[point, label={[label_box, label distance=0.0cm]45:$P_2$}] at (P2) {};
        
        % 6. Beschriftungen für die Pfadsegmente hinzufügen
        \node[label_box, font=\Large] at (-2.2, 2.5) {$\ivecS{r}{O}(x,y)$};
        \node[label_box, font=\Large] at (2.4, -3.2) {$\ivecS{r}{U}(x,y)$};
    \end{tikzpicture}
    }
    \caption{Die blauen Pfeile stellen die ortsabhängige Kraft an jedem Punkt $(x,y)$ dar. In einem nicht-konservativen Kraftfeld hängt die verrichtete Arbeit, um einen Körper von $P_1$ nach $P_2$ zu bewegen, vom Weg ab ($W_O \neq W_U$). Ein typisches Beispiel sind Reibungskräfte.}\label{fig: nichtkonservatives_feld_wegabhaengig}
\end{figure}

\subsection{Konservative Kraftfelder}\label{subsec: konservative_kraftfelder}
Für konservative Kräfte ist die verrichtete Arbeit zwischen zwei Punkten $P_1$ und $P_2$ \textbf{unabhängig} vom zurückgelegten Weg, 
\begin{equation}\label{eq: arbeit_konservativ}
    W = \int_{\mathcal{C}_O: P_1}^{P_2} \ivec{F}(\ivec{r}) \cdot \dd\ivec{r} = \int_{\mathcal{C}_U: P_1}^{P_2} \ivec{F}(\ivec{r}) \cdot \dd\ivec{r} \mDot
\end{equation}
Daraus folgt direkt, dass die Arbeit entlang eines beliebigen geschlossenen Weges in einem konservativen Kraftfeld immer null ist: 
\begin{equation}\label{eq: arbeit_konservativ_geschlossen}
    \oint \ivec{F}(\ivec{r}) \cdot \dd\ivec{r} = 0 \mDot
\end{equation}
Das kommt daher, weil eine Richtungsumkehr das Vorzeichen des Integrals ändert
\begin{equation}
    \int_{\mathcal{C}_O: P_1}^{P_2} \ivec{F}(\ivec{r}) \cdot \dd\ivec{r} = - \int_{\mathcal{C}_O: P_2}^{P_1} \ivec{F}(\ivec{r}) \cdot \dd\ivec{r}
\end{equation}
und somit jeder geschlossene Weg zurück zum Ausgangspunkt ein verschwindendes Integral ergibt. Beispiele für konservative Kräfte sind die Gravitationskraft und die elektrostatische Coulomb-Kraft. Im Allgemeinen sind alle Zentralkräfte, deren Betrag nur vom Abstand abhängt ($\ivec{F} \sim F(r)\ivecS{e}{r}$), konservativ. 

\begin{importantbox}{Eigenschaft konservativer Kräfte}
    Bei konservativen Kräften hängt die verrichtete Arbeit nur vom Start- und Endpunkt der Bewegung ab, nicht aber vom Weg dazwischen. 
\end{importantbox}

\begin{figure}[h!]
    \centering
    \resizebox{0.7\linewidth}{!}{
    \begin{tikzpicture}[
        font=\large,
        point/.style={circle, fill=orange, draw=black, thick, inner sep=2.8pt},
        label_box/.style={fill=white, draw=gray, fill opacity=0.8, text opacity=1, rounded corners=2pt, inner sep=3pt}]
    
        \def\innerradius{2.75} % Radius des Kreises
        \def\outerradius{4.75}
        \def\angleA{30}     % Winkel für Punkt A (in Grad)
        \def\angleB{110}     % Winkel für Punkt B (in Grad)
        \coordinate (O) at (0.5, 0.42); 
        \coordinate (P1) at ($(O) + (\angleA:\innerradius)$); 
        \coordinate (P2) at ($(O) + (\angleB:\outerradius)$); 
    
        % 1. Hintergrundbild aus der PDF-Datei platzieren
        \node at (0,0) {\includegraphics[width=12cm]{Bilder/Kapitel_ArbeitEnergie/grav_field_example.pdf}};
    
        \draw[line width=2.6pt, dashed, black] (0.5,0.42) ++({\innerradius*cos(\angleA)}, {\innerradius*sin(\angleA)}) 
            arc (\angleA:110:\innerradius);
        \draw[line width=2.6pt, dashed, black] (0.5,0.42) ++({\outerradius*cos(\angleA)}, {\outerradius*sin(\angleA)}) 
            arc (\angleA:110:\outerradius);
        \draw[line width=2.6pt, black] (0.5,0.42) ++({\innerradius*cos(\angleA)}, {\innerradius*sin(\angleA)} ) -- 
            ++({2*cos(\angleA)},{2*sin(\angleA)});
        \draw[line width=2.6pt, black] (0.5,0.42) ++({\innerradius*cos(\angleB)}, {\innerradius*sin(\angleB)} ) -- 
        ++({2*cos(\angleB)},{2*sin(\angleB)});
    
         % 6. Beschriftungen für die Pfadsegmente hinzufügen
        \node[label_box, font=\Large] at (4.3, 4.8) {$\ivecS{r}{O}(x,y)$};
        \node[label_box, font=\Large] at (-1.8, 3.0) {$\ivecS{r}{U}(x,y)$};
        
        % --- Beschriftungen und Punkte ---
        \node[point, label={[label_box,label distance=0.1cm]below:$\ivec{P}_1$}] at (P1) {};
        \node[point, label={[label_box,label distance=0.1cm]left:$\ivec{P}_2$}] at (P2) {};
    \end{tikzpicture}
    }
    \caption{Ein konservatives Zentralkraftfeld mit $\protect\ivec{F} = \protect\ivec{F}(r)$. Die Arbeit von $P_1$ nach $P_2$ ist für alle Wege gleich. Entlang der gestrichelten Kreisbahnen wird keine Arbeit verrichtet, da die Kraft senkrecht zum Weg steht. Die Arbeit entlang der durchgezogenen Linien hebt sich jeweils gegenseitig auf.}\label{fig: konservatives_kraftfeld}
\end{figure}

\section{Leistung}\label{sec: leistung}
Die Definition der Arbeit beinhaltet keine Information über die Zeitspanne, in der die Arbeit verrichtet wird. Um dies zu quantifizieren, führt man die physikalische Größe der Leistung ein.

\begin{rememberbox}[]{Definition: Leistung}
    Die Leistung $P$ ist die Rate, mit der Arbeit verrichtet wird. Sie gibt an, wie viel Energie pro Zeiteinheit übertragen wird,
    \begin{equation}\label{eq: leistung_def}
        P = \frac{\dd W}{\dd t} \mDot
    \end{equation}
    Die SI-Einheit der Leistung ist das \textbf{Watt (W)}. Es gilt:
    \begin{equation}
        [P] = \SI{1}{\joule\per\second} = \SI{1}{\watt} \mDot
    \end{equation}
\end{rememberbox}

Ein Massenpunkt, der sich mit der Geschwindigkeit $\ivec{v}$ bewegt, erfahre im Zeitintervall $\dd t$ eine infinitesimale Verschiebung von $\dd\ivec{r} = \ivec{v}\dd t$. Die in dieser Zeit von einer Kraft $\ivec{F}$ verrichtete Arbeit ist 
\begin{equation}
    \dd W = \ivec{F} \cdot \dd\ivec{r} = \ivec{F} \cdot \ivec{v}\dd t \mDot
\end{equation} 
Setzt man dies in die Definition der Leistung ein, erhält man eine sehr nützliche Formel: 
\begin{equation}\label{eq: leistung_vektoriell}
    P = \frac{\dd W}{\dd t} = \frac{\ivec{F} \cdot \ivec{v}\dd t}{\dd t} = \ivec{F} \cdot \ivec{v} \mDot
\end{equation}

\begin{examplebox}[breakable,sidebyside, sidebyside align=top seam, lower separated=false, righthand width=3.5cm]{Leistung eines Motors}
    Ein Motor soll über einen reibungsfreien Seilzug eine Last von Ziegelsteinen mit einer Gewichtskraft von $F_\text{G} = \SI{800}{\newton}$ um eine Höhe von $h = \SI{10}{\meter}$ in einer Zeit von $t = \SI{20}{\second}$ heben. Welche durchschnittliche Leistung muss der Motor aufbringen? \\

    \textbf{Lösung:}\\
    Um die Last mit konstanter Geschwindigkeit zu heben, muss der Motor eine Kraft aufbringen, die der Gewichtskraft entgegengesetzt und betragsmäßig gleich ist: 
    $\ivecS{F}{\text{Motor}} = -\ivecS{F}{\text{G}}$, also $|\ivecS{F}{\text{Motor}}| = \SI{800}{\newton}$. 
    Die konstante Geschwindigkeit des Anhebens beträgt:
    \begin{equation}
        v = \frac{\Delta y}{\Delta t} = \frac{\SI{10}{\meter}}{\SI{20}{\second}} = \SI{0.5}{\meter\per\second} \mDot
    \end{equation}
    Da die Kraft des Motors und die Geschwindigkeit in die gleiche Richtung zeigen, ist der Winkel zwischen ihnen $\theta = \SI{0}{\degree}$. Die Leistung errechnet sich somit zu:
    \begin{equation}\begin{aligned}
        P &= \ivecS{F}{\text{Motor}} \cdot \ivec{v} = |\ivecS{F}{\text{Motor}}| \cdot |\ivec{v}| \cdot \cos(\SI{0}{\degree}) \\
        &= \SI{800}{\newton} \cdot \SI{0.5}{\meter\per\second} = \SI{400}{\watt} \mDot
    \end{aligned}\end{equation}
    \tcblower
     \begin{center}
        \includegraphics[width=0.9\linewidth]{Bilder/Kapitel_ArbeitEnergie/leistung_motor_beispiel.png}
    \end{center}
\end{examplebox}

\section{Energie}\label{sec: energie}
Das Konzept der Energie ist eines der wichtigsten vereinheitlichenden Prinzipien in allen Naturwissenschaften. 
\begin{rememberbox}{Definition: Energie}
    Die Energie eines Systems ist dessen Fähigkeit, Arbeit zu verrichten. Energie ist, wie die Arbeit, eine skalare Größe. 
\end{rememberbox}
Energie kann in verschiedenen Formen auftreten, wie \zB kinetische Energie (Bewegungsenergie), potenzielle Energie (Lageenergie) oder Wärmeenergie. Ein fundamentales Prinzip ist der Energieerhaltungssatz:
\begin{importantbox}{Energieerhaltungssatz}
    Energie kann nicht erzeugt oder vernichtet, sondern nur von einer Form in eine andere umgewandelt werden. Die Gesamtenergie eines abgeschlossenen Systems bleibt konstant. 
\end{importantbox}
Das gilt ganz grundsätzlich auch in jeder alltäglichen Situation, in der \gDQ{Verluste} aufzutreten scheinen. Energie geht niemals wirklich \gDQ{verloren}, sie wird lediglich in andere, oft weniger nützliche, Energieformen umgewandelt.

\begin{examplebox}[breakable]{Beispiel: Energieumwandlung beim Autofahren}
    Betrachten wir ein Auto, das mit konstanter Geschwindigkeit fährt. Seine Hauptenergieform ist die \textbf{kinetische Energie} (Bewegungsenergie). Um die Geschwindigkeit zu halten, muss das Auto konstant Energie nachführen, ansonsten würde das Auto langsamer werden. Man könnte meinen, dass also durchwegs Energie \gDQ{verloren} gehe, aber tatsächlich wurde sie nur umgewandelt:
    \begin{itemize}
        \item \textbf{Rollreibung:} Die Reibung zwischen den Reifen und der Straße erzeugt Wärme. Die kinetische Energie des Autos wird in Wärmeenergie in den Reifen und im Straßenbelag umgewandelt.
        \item \textbf{Luftreibung:} Das Auto muss die Luft vor sich verdrängen. Diese Wechselwirkung erwärmt die Luftmoleküle. Ein Teil der kinetischen Energie wird also in Wärmeenergie der umgebenden Luft umgewandelt.
        \item \textbf{Motor-Ineffizienz:} Weniger als die Hälfte der chemischen Energie des Brennstoffs kann in mechanische Energie an den Antriebsstrang übertragen werden. Auch das Getriebe und andere bewegliche Teile tragen durch Reibung zur Wärmeerzeugung bei.
    \end{itemize}
    Die ursprüngliche kinetische Energie des Fahrzeugs wird also nicht vernichtet, sondern vollständig in andere Energieformen -- hauptsächlich Wärme und zu einem kleinen Teil auch Schallenergie -- umgewandelt. Der Energieerhaltungssatz bleibt somit erfüllt, wenn man die Systemgrenze so zieht, dass alle Körper, die Energie aufnehmen oder abgegeben mitberücksichtigt. 
\end{examplebox}

\subsection{Kinetische Energie}\label{subsec: kinetische_energie}
Wenn eine resultierende äußere Kraft\footnote{Als \gDQ{äußere Kraft} bezeichnet man eine Kraft, die nicht zum betrachteten System gehört. Diese Kraft kann demnach die Energie des Systems ändern} an einem freien Körper Arbeit verrichtet, führt dies zu einer Änderung seiner Bewegungsenergie. Betrachten wir eine konstante Kraft $\ivec{F} = F_x \ivecS{e}{x}$, die auf eine Masse $m$ entlang der $x$-Achse wirkt. Die verrichtete Arbeit ist:
\begin{equation}\label{eq: arbeit_integration_entlang_x}
    W = \int_{x_\text{A}}^{x_\text{E}} F_x \cdot \dd x \mDot
\end{equation}
\begin{figure}[tb]
    \centering
    \resizebox{0.65\linewidth}{!}{
    \begin{tikzpicture}[
        particle/.style={
            draw=black,
            fill=yellow!60!orange!70, % Eine passende gelb-orange Füllfarbe
            minimum size=0.7cm
        },
        force_arrow/.style={
            -Stealth, % Pfeilspitze vom Typ "Stealth"
            blue!80!cyan,
            very thick, opacity=0.15
        }
        ]    
        % 5. Das externe Kraftfeld (blaue Pfeile)
        \begin{scope}[yshift=-0cm]
            \foreach \y in {-1.5, -0.9, -0.3, 0.3, 0.9, 1.5, 2.1, 2.7} {
                \foreach \x in {-3, -2, -1, 0, 1, 2, 3, 4, 5, 6, 7} {
                    \draw[force_arrow] (\x, \y) -- ++(0.7, 0);
                }
            }
        \end{scope}
        % 1. Die Systemgrenze (gestricheltes Rechteck mit abgerundeten Ecken)
        \draw[gray, dashed, dash pattern=on 5pt off 4pt, rounded corners=12pt, line width=1.1pt, fill=white, fill opacity=0.4] (-2, -1) rectangle (6.8,2.25) node[anchor=north east, pos=0.94, gray, fill=white, fill opacity=0.8, text opacity=1] {\large Systemgrenze};
    
        \draw[-{Stealth}, line width=1.2pt] (-1.4, 0) -- (5.95, 0) node[right] {\large $x$};
        \draw[-{Stealth}, line width=1.2pt] (0, -1.5) -- (0, 2.6) node[above=3pt] {\large $y$};
        \foreach \x in {-1, 0, 1, 2, 3,4,5} {
            \draw[line width=1.2pt] (\x, -0.15) -- (\x, 0.15);
        };
        
        \node[particle, circle] (left_particle) at (1, 0.5) {};
        \node[particle, circle,dashed,  draw=gray, fill opacity=0.86] at (4, 0.5) {};    
        \draw[-{Stealth[length=3.3mm]}, red, line width=2pt] (left_particle.north) ++(-0.5, 0.3) -- ++(1.4, 0)
              node[midway, above=2pt] {\Large $\ivec{v}$};
        \node[] at (1,-0.5) {\large $x_\text{A}$};
        \node[] at (4,-0.5) {\large $x_\text{E}$};
        % 6. Die Formel für die externe Kraft
        \node[below=0cm, blue!70!white] at (-1,-1.6) {\large $F_x^{\text{ext}}(x) = \text{const}$};
    \end{tikzpicture}
    }
    \caption{Eine konstante äußere Kraft $F_x^{\text{ext}}$ wirkt auf ein Teilchen, das sich von $x_\text{A}$ nach $x_\text{E}$ bewegt. Dabei wird an dem Teilchen eine Arbeit verrichtet, die zur Erhöhung der Geschwindigkeit $v_x$ und damit zu einer Erhöhung der kinetischen Energie führt.}\label{fig: placeholder}
\end{figure}

Nach dem zweiten Newtonschen Axiom ist 
\begin{equation}
    F_x = m a_x = m \frac{\dd v_x}{\dd t} \mDot
\end{equation}
Die Formel für die Kraft setzen wir in \cref{eq: arbeit_integration_entlang_x} ein, schreiben das Integral mithilfe der Kettenregel um und erhalten
\begin{align*}
    W &= \int_{x_\text{A}}^{x_\text{E}} m \frac{\dd v_x}{\dd t} \dd x 
      = \int_{v_\text{A}}^{v_\text{E}} m \underbrace{\frac{\dd x}{\dd t}}_{v_x} \dd v_x 
      = \int_{v_\text{A}}^{v_\text{E}} m v_x \dd v_x \\
      &= \left. \frac{1}{2} m v_x^2 \right|_{v_\text{A}}^{v_\text{E}} 
      = \frac{1}{2} m v_\text{E}^2 - \frac{1}{2} m v_\text{A}^2 \mDot
\end{align*}
Das Verrichten einer Arbeit hat also zu einer Änderung der Geschwindigkeit geführt. Dieser Zusammenhang führt zur Definition der kinetischen Energie.

\begin{importantbox}[]{Definition: Kinetische Energie}
    Die kinetische Energie (Bewegungsenergie) eines Körpers der Masse $m$, der sich mit dem Tempo $v = |\ivec{v}|$ bewegt, ist
    \begin{equation}\label{eq: kinetische_energie}
        E_{\text{kin}} = \frac{1}{2}mv^2 \mDot
    \end{equation}
    Die kinetische Energie hängt nur vom Betrag der Geschwindigkeit ab, $v^2 = |\ivec{v}|^2$. 
\end{importantbox}
Das obige Ergebnis -- 
\begin{equation}
    W = \frac{1}{2} m v_\text{E}^2 - \frac{1}{2} m v_\text{A}^2 = E_{\text{kin,E}} - E_{\text{kin,A}} = \Delta E_{\text{kin}}    
\end{equation}
-- ist als Arbeit-Energie-Satz bekannt.

\begin{rememberbox}[]{Arbeit-Energie-Satz}
    Die von allen äußeren Kräften an einem freien Körper verrichtete Gesamtarbeit $W_{\text{ges}}$ entspricht der Änderung seiner kinetischen Energie $\Delta E_{\text{kin}}$
    \begin{equation}\label{eq: arbeit_energie_satz}
        W_{\text{ges}} = \Delta E_{\text{kin}}\mDot
    \end{equation}
\end{rememberbox}

\subsection{Potenzielle Energie}\label{subsec: potentielle_energie}
Während die Arbeit an einem freien Teilchen nur dessen kinetische Energie ändert, kann die \textbf{Arbeit}, die \textbf{an einem System} (\zB Erde-Teilchen-System) mit inneren konservativen Kräften verrichtet wird, als \textbf{potenzielle Energie (Lageenergie)} gespeichert werden. 
\begin{figure}[htb]
    \centering
    \resizebox{0.5\linewidth}{!}{    
    \begin{tikzpicture}[
        particle/.style={
            draw=black,
            fill=yellow!60!orange!70, % Eine passende gelb-orange Füllfarbe
            minimum size=0.7cm
        },
        force_arrow/.style={
            -Stealth, % Pfeilspitze vom Typ "Stealth"
            blue!80!cyan,
            very thick, opacity=0.15
        }
        ]    
        % 5. Das externe Kraftfeld (blaue Pfeile)
        \begin{scope}[yshift=-0cm]
            \foreach \y in {-2,0,2,4,6,8,10} {
                \foreach \x in {-3,-1,1,3,5,7} {
                    \draw[force_arrow] (\x, \y) -- ++(0, 0.8);
                }
            }
        \end{scope}
        % 1. Die Systemgrenze (gestricheltes Rechteck mit abgerundeten Ecken)
        \draw[gray, dashed, dash pattern=on 5pt off 4pt, rounded corners=12pt, line width=1.1pt, fill=white, fill opacity=0.4] (-2, -1) rectangle (6.5,9.5) node[anchor=north east, pos=0.96, gray, fill=white, fill opacity=0.8, text opacity=1] {\Large Systemgrenze};
    
        \draw[-{Stealth}, line width=1.2pt] (-2.4, 0) -- (5.95, 0) node[right] {\large $x$};
        \draw[-{Stealth}, line width=1.2pt] (-1.1, -1.5) -- (-1.1, 6) node[above=3pt] {\large $y$};
        
        \node[particle, circle] (lower_particle) at (2, 4.9) [label={[label distance=0.05cm]0:\large $h_\text{A}$}] {};
        \node[particle, circle,dashed,  draw=gray, fill opacity=0.86] (upper_particle) at (2, 7.8) [label={[label distance=0.05cm]0:\large $h_\text{E}$}] {};
        \draw[-{Stealth[length=3.3mm]}, red, line width=2pt] (lower_particle.south) ++(0, -0.1) -- ++(0, -1.2)
              node[midway, left=2pt] {\Large $\vec{F}_{\text{G}}(h_\text{A})$};
        \draw[-{Stealth[length=3.3mm]}, red!40!white, line width=2pt] (upper_particle.south) ++(0, -0.1) -- ++(0, -1.2) node[midway, left=2pt] {\Large $\vec{F}_{\text{G}}(h_\text{E})$};
        \node[circle, fill=purple!70!cyan, minimum size=6cm] at (2,0) {\Large Erde};
    
        % 6. Die Formel für die externe Kraft
        \node[below=0cm, blue!70!white] at (6.5,-2.2) {\large $F_y^{\text{ext}}(y) = \text{const}$};
    \end{tikzpicture}
    }
    \caption{Eine konstante äußere Kraft $F_y^{\text{ext}}$ wirkt auf ein Teilchen, das Teil eines Systems (Erde-Teilchen) ist. Das Teilchen wird dabei von $h_\text{A}$ nach $h_\text{E}$ gehoben. Dabei wird an dem Erde-Teilchen System eine Arbeit verrichtet, die zur Erhöhung der potenziellen Energie führt.}\label{fig: lageenergie_erde_teilchen}
\end{figure}

Betrachten wir das orangefarbene Teilchen der Masse $m$ in \cref{fig: lageenergie_erde_teilchen}, das von einer externen Kraft $F_{y}^{\text{ext}}(y) = \const$ (\zB unsere Hand) von der Höhe $h_{\text{A}}$ auf die Höhe $h_E$ langsam angehoben wird, wobei wir annehmen, dass $h_\text{A}$ und $h_\text{E}$ viel kleiner als der Erdradius sind, sodass die Gravitationskraft als konstant angesehen werden kann. Die externe Kraft muss daher genau der Gravitationskraft entgegenwirken, weshalb $F_y^{\text{ext}} = mg$. Die von der externen Kraft verrichtete Arbeit ist:
\begin{equation}
    W^{\text{ext}} = \int_{h_{\text{A}}}^{h_{\text{E}}} F_y^{\text{ext}} \dd y = \int_{h_{\text{A}}}^{h_{\text{E}}} mg\, \dd y = mg\cdot y \,\bigg|_{h_{\text{A}}}^{h_{\text{E}}} = mg \left( h_{\text{E}} - h_{\text{A}} \right) = mg \Delta h\mDot
\end{equation}
Da die Bewegung langsam erfolgt, ist die Geschwindigkeitsänderung und somit $\Delta E_{\text{kin}} \approx 0$. Die zugeführte Arbeit $W$ wurde also im System als Lageenergie gespeichert. Diese gespeicherte Energie nennen wir potenzielle Energie. Die Änderung der potenziellen Energie ist gleich der von der \textit{externen Kraft} verrichteten Arbeit:
\begin{equation}
    W^{\text{ext}} = \Delta E_{\text{pot}} \mDot
\end{equation}
Wenn wir die Situation aus der Sicht des konservativen Kraftfeldes (hier: Gravitationsfeld) betrachten, so wird gegen das Gravitationsfeld Arbeit verrichtet $W_{\text{G}} = -W^{\text{ext}} = -mg \Delta h$. Die externe Arbeit leistet gegen das Gravitationskraftfeld eine Arbeit und erhöht dabei die potenzielle Energie des Systems. Dies motiviert die folgende allgemeine Definition für konservative Kräfte:
{\text{ext}}
\begin{importantbox}[]{Definition: Potenzielle Energie}
    Die Änderung der potenziellen Energie $\Delta E_{\text{pot}}$ in einem konservativen Kraftfeld ist definiert als die negative Arbeit, die vom konservativen Kraftfeld $\ivec{F} = \ivec{F}_{\text{kons}}$ verrichtet wird: 
    \begin{equation}\label{eq: potenzielle_energie_def}
        W^{\text{int}} = \int_{P_1}^{P_2} \ivec{F} \cdot \dd\ivec{r} \defeq -\Delta E_{\text{pot}}\mComma
    \end{equation}
    wobei $\Delta E_{\text{pot}} = E_{\text{pot}}(P_2) - E_{\text{pot}}(P_1) $. Folglich gilt:
    \begin{itemize}[itemsep=1.5pt]
        \item Verrichtet das Feld Arbeit ($W^{\text{int}} > 0$ für $\ivec{F} \cdot \dd \ivec{r} > 0$), \textbf{sinkt} die potenzielle Energie. 
        \item Wird gegen das Feld Arbeit verrichtet ($W^{\text{int}} < 0$ für $\ivec{F} \cdot \dd \ivec{r} < 0$), \textbf{steigt} die potenzielle Energie. 
    \end{itemize}
    Der absolute Wert der potenziellen Energie ist nicht festgelegt; nur ihre Differenz ist physikalisch relevant. Man kann den Nullpunkt der potenziellen Energie beliebig wählen. 
\end{importantbox}

\begin{examplebox}[breakable]{Beispiel: Potenzielle Energie im Schwerefeld der Erde}
    Wir betrachten eine Masse von $m = \SI{15}{\kilo\gram}$, die in der Nähe der Erdoberfläche gehoben und gesenkt wird. Das Gravitationsfeld übt eine konstante Kraft $\ivecS{F}{\text{G}} = m\cdot\ivec{g}$ aus, wobei die Erdbeschleunigung $\ivec{g}$ in die $-z$-Richtung zeigt und $g \approx \SI{9.81}{\meter\per\second\squared}$. Wir legen den Nullpunkt der potenziellen Energie auf die Erdoberfläche bei $z=0$, wodurch $h = \Delta h$. 

    \begin{enumerate}
        \item \textbf{Anheben der Masse um \SI{3}{\meter}:}
        Um die Masse um die Höhe $h = \SI{3}{\meter}$ anzuheben, muss eine externe Kraft entgegen der Schwerkraft wirken, die mindestens so groß ist wie die Gewichtskraft $\ivec{F}^{\text{ext}} = -\ivecS{F}{\text{G}} = +m g \ivecS{e}{z}$. Die Arbeit $W^{\text{ext}}$, die von dieser externen Kraft verrichtet wird, beträgt:
        \begin{equation}
            W^{\text{ext}} = \int_0^{h} \ivec{F}^{\text{ext}} \cdot \dd \ivec{r}  = m \cdot g \cdot h = \SI{15}{\kilo\gram} \cdot \SI{9.81}{\meter\per\second\squared} \cdot \SI{3}{\meter} = \SI{441.45}{\joule} \mDot
        \end{equation}
        Diese verrichtete Arbeit wird vollständig in eine Erhöhung der potenziellen Energie der Masse umgewandelt. Die Änderung der potenziellen Energie ist also positiv:
        \begin{equation}
            \Delta E_{\text{pot}} = E_{\text{pot}}(h) - E_{\text{pot}}(0) = W^{\text{ext}} = \SI{+441.45}{\joule} \mDot
        \end{equation}
        Gleichzeitig verrichtet das konservative Gravitationsfeld eine negative Arbeit $W_\text{G} = W^{\text{int}} = -W^{\text{ext}}$, da die Gravitationskraft der Bewegungsrichtung entgegengesetzt ist ($\ivec{g} \propto -\dd \ivec{r}$). Gemäß der Definition ist 
        \begin{equation}
            \Delta E_{\text{pot}} = -W^{\text{int}} = -(-\SI{441.45}{\joule}) = \SI{+441.45}{\joule} \mDot
        \end{equation}

        \item \textbf{Absenken der Masse um \SI{3}{\meter}:}
        Wird die Masse nun wieder auf ihre ursprüngliche Höhe ($z=0$) abgesenkt, verrichtet das Gravitationsfeld positive Arbeit, da die Kraft des Feldes nun in die gleiche Richtung wie die Bewegung zeigt:
        \begin{equation}
             W_\text{G} = W^{\text{int}} = \int_h^{0} \ivecS{F}{\text{G}} \cdot \dd \ivec{r} = m \cdot g \cdot h = \SI{441.45}{\joule} \mDot
        \end{equation}
        Die potenzielle Energie der Masse nimmt dabei ab, da die Änderung der potenziellen Energie gleich der negativen Arbeit ist, die vom konservativen Feld verrichtet wird:
         \begin{equation}
            \Delta E_{\text{pot}} = E_{\text{pot}}(0) - E_{\text{pot}}(h) = - W_\text{G} = \SI{-441.45}{\joule} \mDot
        \end{equation}
        Die potenzielle Energie, die beim Anheben gespeichert wurde, wird beim Absenken wieder freigesetzt.
    \end{enumerate}
\end{examplebox}

\section{Erhaltung der mechanischen Energie}\label{sec: erhaltung_mechanische_energie}
Wir kombinieren nun die bisherigen Erkenntnisse für ein System, in dem nur konservative Kräfte wirken. Aus dem Arbeit-Energie-Satz wissen wir, dass die vom Feld verrichtete Arbeit die kinetische Energie ändert:
\begin{equation}
    W = \Delta E_{\text{kin}} = E_{\text{kin}}(P_2) - E_{\text{kin}}(P_1) \mDot
\end{equation}
Gleichzeitig gilt nach der Definition der potenziellen Energie:
\begin{equation}
    W = -\Delta E_{\text{pot}} = -(E_{\text{pot}}(P_2) - E_{\text{pot}}(P_1)) = E_{\text{pot}}(P_1) - E_{\text{pot}}(P_2) \mDot
\end{equation}
Durch Gleichsetzen der beiden Ausdrücke für $W$ erhalten wir: 
\begin{equation}
    E_{\text{kin}}(P_2) - E_{\text{kin}}(P_1) = E_{\text{pot}}(P_1) - E_{\text{pot}}(P_2) \mDot
\end{equation}
Umsortieren der Terme nach den Punkten $P_1$ und $P_2$ führt zum Erhaltungssatz der mechanischen Energie:
\begin{equation}
    E_{\text{kin}}(P_1) + E_{\text{pot}}(P_1) = E_{\text{kin}}(P_2) + E_{\text{pot}}(P_2) \mDot
\end{equation}

\begin{importantbox}{Erhaltungssatz der mechanischen Energie}
    In einem abgeschlossenen System, in dem nur konservative Kräfte wirken, ist die Summe aus kinetischer und potenzieller Energie – die mechanische Gesamtenergie $E$ – zu allen Zeiten konstant. 
    \begin{equation}\label{eq: mechanische_energieerhaltung}
        E_{\text{mech}} = E_{\text{kin}} + E_{\text{pot}} = \const \mDot
    \end{equation}
\end{importantbox}


\section{Das Federpendel}\label{sec: beispiel_federpendel}
Wir betrachten einen Block, der reibungsfrei auf einem Tisch gleiten kann und dabei, befestigt an einer Feder, schwingt. Dies ist ein Paradebeispiel für die Anwendung der Energieerhaltung.

\subsection{Arbeit einer Feder}\label{subsec: arbeit_feder}
Eine ideale Feder übt eine Kraft aus, die durch das Hookesche Gesetz beschrieben wird. Wenn die Ruhelage der Feder bei $x_0=0$ liegt, ist die Federkraft:
\begin{equation}
    F_x = -kx \mComma
\end{equation}
wobei $k$ die Federkonstante und $x$ die momentane Auslenkung aus der Ruhelage ist. Obwohl auf den Block auch noch die Gewichtskraft und die Normalkraft vom Tisch wirken, verrichten diese beiden Kräfte keine Arbeit, da sie normal auf die Bewegungsrichtung stehen. \\

Die Arbeit, die von der Feder verrichtet wird, wenn der Block von einer Anfangsposition $x_\text{A}$ zu einer Endposition $x_\text{E}$ bewegt wird, ist:
\begin{equation}\label{eq: arbeit_feder}
    W_{\text{Feder}} = \int_{x_\text{A}}^{x_\text{E}} F_x \cdot \dd x = \int_{x_\text{A}}^{x_\text{E}} (-kx) \dd x = \left. -k \frac{x^2}{2}\right|_{x_\text{A}}^{x_\text{E}} = -\left(\frac{1}{2}k x_\text{E}^2 - \frac{1}{2}k x_\text{A}^2 \right) \mDot
\end{equation}
Da die Arbeit nur von den Endpunkten abhängt, ist die Federkraft eine konservative Kraft. Das Resultat zeigt, dass für 
\begin{itemize}
    \item $x_\text{E} > x_\text{A} \Rightarrow W_\text{Feder} < 0$ -- die Feder verrichtet Arbeit am Block, 
    \item $x_\text{A} > x_\text{E} \Rightarrow W_\text{Feder} > 0$ -- der Block verrichtet Arbeit an der Feder.
\end{itemize}

\subsection{Potenzielle Energie einer Feder}\label{subsec: potentielle_energie_feder}
Wir können die potenzielle Energie der Feder mit \cref{eq: potenzielle_energie_def} und \cref{eq: arbeit_feder} bestimmen. Wir definieren den Nullpunkt der potenziellen Energie in der Ruhelage der Feder, \gDh $E_{\text{pot}}(x=0) \defeq 0$. Die potenzielle Energie bei einer Auslenkung von $0$ nach $x$ ist dann:
\begin{align*}
    \Delta E_{\text{pot}} = E_{\text{pot}}(x) - \underbrace{E_{\text{pot}}(0)}_{=0} &= -W_{\text{Feder}} = -\left[-\left( \frac{1}{2}k x^2 - \frac{1}{2}k \cdot 0^2 \right)\right] \\
    \Rightarrow E_{\text{pot}}(x) &= \frac{1}{2}kx^2 \mDot
\end{align*}

\begin{rememberbox}{Potenzielle Energie einer Feder}
    Die in einer um $x$ aus ihrer Ruhelage ausgelenkten Feder gespeicherte potenzielle Energie beträgt:
    \begin{equation}\label{eq: potenzielle_energie_feder}
        E_{\text{pot}}^{\text{Feder}} = \frac{1}{2}kx^2 \mDot
    \end{equation}
\end{rememberbox}

\subsection{Energieerhaltung beim Federpendel}\label{subsec: energieerhaltung_federpendel}
Für das System aus Masse und (masseloser) Feder lautet die mechanische Gesamtenergie:
\begin{equation}
    E_{\text{mech}} = E_{\text{kin}} + E_{\text{pot}} = \frac{1}{2}mv^2 + \frac{1}{2}kx^2 \mDot
\end{equation}
Da keine nicht-konservativen Kräfte (wie Reibung) wirken, bleibt diese Gesamtenergie erhalten. Die Energie wird nur periodisch zwischen potenzieller und kinetischer Energie umgewandelt, während ihre Summe konstant bleibt.
\begin{figure}[htb]
    \centering
    \includegraphics[width=0.7\textwidth]{Bilder/Kapitel_ArbeitEnergie/energieerhaltung_federpendel.png}
    \caption{Die, auf die Gesamtenergie normierte, potenzielle (blau), kinetische (orangen) und konstante Gesamtenergie (grün) eines idealen Federpendels als Funktion der normierten Auslenkung $x/x_\text{max}$.}\label{fig: federpendel_energie_diagramm}
\end{figure}

\begin{examplebox}[breakable]{Schwingung eines Federpendels}
    Ein Block mit Masse $m$ wird an einer Feder mit Federkonstante $k$ befestigt. Das System wird um eine Strecke $x_\text{S}$ ausgelenkt und aus der Ruhe losgelassen. 

    \textbf{1. Anfangszustand (maximale Auslenkung $x=x_\text{S}$):}\\
    
    Der Block ist anfangs in Ruhe ($v=0$), also ist die kinetische Energie null. Die potenzielle Energie ist demnach maximal.
    \begin{equation}\label{eq: e_total_feder_max_auslenkung}
        E_{\text{total}} = E_{\text{kin}} + E_{\text{pot}} = 0 + \frac{1}{2}kx_\text{S}^2 = \frac{1}{2}kx_\text{S}^2 \mDot
    \end{equation}

    \textbf{2. Zustand in der Ruhelage ($x=0$):}
    Wenn der Block die Ruhelage durchläuft, ist die potenzielle Energie der Feder null (per Definition). Die gesamte Energie liegt als kinetische Energie vor. Die Geschwindigkeit ist hier maximal ($v=v_{\text{max}}$).
    \begin{equation}\label{eq: e_total_feder_ruhelage}
        E_{\text{total}} = E_{\text{kin}} + E_{\text{pot}} = \frac{1}{2}mv_{\text{max}}^2 + 0 \mDot
    \end{equation}
    Durch Gleichsetzen von \cref{eq: e_total_feder_max_auslenkung,eq: e_total_feder_ruhelage} findet man die maximale Geschwindigkeit
    \begin{equation}
        \frac{1}{2}kx_\text{S}^2 = \frac{1}{2}mv_{\text{max}}^2 \implies v_{\text{max}} = \sqrt{\frac{k}{m}} \cdot |x_\text{S}| \mDot
    \end{equation}
\end{examplebox}




\section{Allgemeiner Energieerhaltungssatz}\label{sec: allgemeiner_energieerhaltungssatz}
In der realen, makroskopischen Welt treten oft nicht-konservative Kräfte wie Reibung auf. Diese Kräfte sind \textbf{dissipativ}, das heißt, sie wandeln mechanische Energie in andere Energieformen um, typischerweise in Wärmeenergie (innere Energie). Dadurch nimmt die mechanische Gesamtenergie eines Systems ab.

Der Energieerhaltungssatz ist jedoch ein universelles Prinzip, wenn man \textit{alle} Energieformen berücksichtigt.
\begin{importantbox}{Allgemeiner Energieerhaltungssatz}
    Die Gesamtenergie eines \textbf{abgeschlossenen} (isolierten) Systems ist immer konstant. Energie kann ihre Form ändern (z.B. mechanisch, thermisch, chemisch, nuklear, elektromagnetisch), aber die Gesamtsumme bleibt erhalten. 
    \begin{equation}
        E_{\text{total}} = E_{\text{mech}} + E_{\text{Wärme}} + E_{\text{chem}} + E_{\text{Kern}} + \dots = \const
    \end{equation}
    Für ein \textbf{offenes} System kann sich die Gesamtenergie ändern, aber nur um den Betrag, der mit der Umgebung ausgetauscht wird. 
\end{importantbox}
Zwei Beispiele für nichtkonservative Systeme: 
\begin{itemize}
    \item Bei chemischen Reaktionen kann die Summe aus mechanischer Energie und Wärmeenergie unter Umständen nicht erhalten bleiben, wenn ein Anteil der Energie in gewissen Molekülstrukturen und deren Umwandlung gespeichert wird. 
    \item Bei der Kernspaltung, zerfällt ein Atom in kleinere Atome. Das größere Ausgangsatom hat eine größere Bindungsenergie als die Zerfallsprodukte und ein Teil der dabei freiwerdenden Energie, wird in Form von elektromagnetische Strahlung abgegeben [$\beta$- und $\gamma$-Strahlung]). 
\end{itemize}






% ---------------------------------------------------------
% ---------------------------------------------------------
% ---------------------------------------------------------



\chapter{Wärmelehre}\label{chap: Waermelehre}
Die Erkenntnis, dass Wärme eine Form von Energie ist -- genauer gesagt, die mikroskopische Bewegungsenergie der Atome und Moleküle eines Körpers -- und dass mechanische Arbeit direkt in Wärme umgewandelt werden kann, ist eine der fundamentalen Säulen der modernen Physik. Diese Einsicht, die uns heute fast selbstverständlich erscheint, wurde maßgeblich durch die Arbeiten von Pionieren wie Julius Robert Mayer geprägt, der 1842 das Prinzip der Energieerhaltung bei der Umwandlung von mechanischer in thermische Energie formulierte.

Die Entwicklung der \textbf{kinetischen Gastheorie} im 19. Jahrhundert lieferte schließlich die mikroskopische Deutung der Wärmeenergie als die Summe der kinetischen und potenziellen Energien der Moleküle eines Körpers.

\section{Temperatur und kinetische Gastheorie}\label{sec: Temperatur_kinetische_Gastheorie}
In der kinetischen Gastheorie wird die Temperatur eines Systems direkt mit der durchschnittlichen kinetischen Energie seiner Teilchen verknüpft. Die \textbf{absolute Temperatur} $T$, gemessen in Kelvin (\si{\kelvin}), ist proportional zur mittleren kinetischen Energie der Translation $\overline{E_\mathrm{kin}}$ der Moleküle:
\begin{equation}\label{eq:def_temperatur_kinetisch}
    \overline{E_\mathrm{kin}} = \frac{1}{2}m\overline{v^2} = \frac{3}{2} k_B T \mDot
\end{equation}
Hierbei ist $m$ die Masse eines Moleküls, $\overline{v^2}$ das mittlere Geschwindigkeitsquadrat und $k_B = \SI{1.380649e-23}{\joule\per\kelvin}$ die \textbf{Boltzmann-Konstante}. Diese Gleichung ist eine der zentralen Aussagen der kinetischen Gastheorie und verbindet die makroskopische Größe der Temperatur mit der mikroskopischen Welt der Teilchenbewegung.

\begin{rememberbox}{Das thermodynamische System}
    Als thermodynamisches System bezeichnen wir eine Ansammlung von Atomen oder Molekülen, die mit ihrer Umgebung Energie in Form von Wärme oder mechanischer Arbeit austauschen kann. Ein solches System wird durch makroskopische Zustandsgrößen wie Temperatur, Druck, Volumen und Teilchenzahl beschrieben.
\end{rememberbox}

\section{Temperaturmessung und Skalen}\label{sec: Temperaturmessung_Skalen}
Zur Messung der Temperatur können prinzipiell alle physikalischen Eigenschaften genutzt werden, die sich vorhersagbar mit der Temperatur ändern. Beispiele hierfür sind:
\begin{itemize}[itemsep=1.5pt]
    \item Die \textbf{thermische Ausdehnung} von Festkörpern, Flüssigkeiten oder Gasen.
    \item Der \textbf{elektrische Widerstand} von Materialien, der bei Metallen typischerweise mit der Temperatur ansteigt.
    \item Die \textbf{elektrische Kontaktspannung} zwischen zwei unterschiedlichen Metallen (Thermoelement).
    \item Die \textbf{Strahlungsleistung}, die von einem heißen Körper emittiert wird.
\end{itemize}
Geräte zur Temperaturmessung werden als \textbf{Thermometer} bezeichnet. Die Funktionsweise und Genauigkeit eines Thermometers hängen von der gewählten physikalischen Eigenschaft und der definierten Temperaturskala ab. Im alltäglichen Gebrauch findet man vor allem Thermometer auf Basis der Volumenänderung von Flüssigkeiten (\textbf{Flüssigkeitsthermometer}) oder Thermometer auf Basis der Änderung der Kontaktspannung zwischen zwei Metallen (\textbf{Thermoelemente}). 

\subsection{Die Celsius- und Fahrenheit-Skala}\label{subsec: Celsius_Fahrenheit}
Historisch wurden verschiedene Temperaturskalen entwickelt, die auf unterschiedlichen Fixpunkten basieren.

\paragraph{Celsius-Skala}
Der schwedische Astronom Anders Celsius schlug 1742 eine Skala vor, die auf zwei leicht reproduzierbaren Fixpunkten basiert:
\begin{itemize}
    \item \textbf{\SI{0}{\celsius}} entspricht dem Schmelzpunkt von Eis.
    \item \textbf{\SI{100}{\celsius}} entspricht dem Siedepunkt von Wasser bei Normaldruck (\SI{1013.25}{\hecto\pascal}).
\end{itemize}
Der Bereich zwischen diesen beiden Punkten wird in 100 gleiche Teile unterteilt, die Grade Celsius ($\si{\celsius}$) genannt werden. Jedes Skalenteil entspricht daher $\SI{1}{\celsius}$.

\paragraph{Fahrenheit-Skala}
Diese Skala, die heute vor allem in den USA gebräuchlich ist, wurde von Daniel Gabriel Fahrenheit definiert. Ihre Fixpunkte sind der Schmelzpunkt einer speziellen Eis-Salz-Mischung (\SI{0}{\fahrenheit} $\approx \SI{-17.8}{\celsius}$) und die angenommene menschliche Körpertemperatur (\SI{100}{\fahrenheit} $\approx \SI{37.8}{\celsius}$). Auf dieser Skala gefriert Wasser bei \SI{32}{\fahrenheit} und siedet bei \SI{212}{\fahrenheit}. Auch auf dieser Skala wird der Bereich zwischen \SI{0}{\fahrenheit} und \SI{100}{\fahrenheit} in 100 Skalenteile eingeteilt. 

\begin{figure}[htb]
    \centering
    \includegraphics[width=0.55\linewidth]{Bilder/Kapitel_Waermelehre/fixpunkte_celsius_fahrenheit.png}
    \caption{Vergleich der Celsius- und Fahrenheit-Skala mit ihren jeweiligen Fixpunkten (in rot).}\label{fig: temp_skalen_vergleich}
\end{figure}

Die Umrechnung von Fahrenheit ($T_F$) nach Celsius ($T_C$) erfolgt über folgende Formel:
\begin{equation}\label{eq: fahrenheit_celsius_umrechnung}
    T_C\, [\si{\celsius}] = \frac{5}{9} (T_F\, [\si{\fahrenheit}]- 32) \mDot
\end{equation}

\subsection{Die Kelvin-Skala}\label{subsec: Kelvin_Skala}
Die physikalisch fundamentale Skala ist die \textbf{absolute Temperaturskala} oder \textbf{Kelvin-Skala}, benannt nach Lord Kelvin. Sie ist die SI-Basiseinheit der Temperatur.
Im Gegensatz zur Celsius-Skala benötigt die Kelvin-Skala eigentlich nur einen einzigen Fixpunkt:
\begin{itemize}
    \item Der \textbf{absolute Nullpunkt} bei \SI{0}{\kelvin} ist so gesehen kein Fixpunkt. Dies ist die theoretisch tiefstmögliche Temperatur, bei der die Teilchen eines Systems ihre geringstmögliche Energie besitzen.
    \item Der zweite Referenzpunkt ist der \textbf{Tripelpunkt von Wasser}. Dies ist der Punkt, an dem die feste, flüssige und gasförmige Phase von Wasser im thermodynamischen Gleichgewicht existieren. Seine Temperatur ist per Definition exakt auf \SI{273.16}{\kelvin} festgelegt, was \SI{0.01}{\celsius} entspricht.
\end{itemize}

\begin{figure}[htb]
    \centering
    \includegraphics[width=0.5\linewidth]{Bilder/Kapitel_Waermelehre/phasendiagramm_wasser.png}
    \caption{Schematisches Phasendiagramm $(T, p)$ von Wasser, das den Tripelpunkt zeigt, an dem alle drei Phasen (fest, flüssig, gasförmig) koexistieren.}\label{fig: phasendiagramm_wasser}
\end{figure}

Die Celsius-Skala ist heute über die Kelvin-Skala definiert. Die Umrechnung lautet:
\begin{equation}\label{eq: kelvin_celsius_umrechnung}
    T_C\, [\si{\celsius}] = T\,[\si{\kelvin}] - \num{273.15} \mDot
\end{equation}

\begin{importantbox}[]{Temperaturdifferenzen}
    Die Einteilung der Kelvin- und der Celsius-Skala ist identisch. Das bedeutet, eine Temperaturdifferenz von \SI{1}{\kelvin} entspricht exakt einer Temperaturdifferenz von \SI{1}{\celsius}:
    \begin{equation}
        \Delta T = \SI{1}{\kelvin} \,\Longleftrightarrow \,\Delta T = \SI{1}{\celsius}
    \end{equation}
\end{importantbox}


\subsection{Flüssigkeitsthermometer}\label{subsec: fluessigkeitsthermometer}
Das wohl bekannteste Messinstrument ist das \textbf{Flüssigkeitsthermometer}. Es nutzt die thermische Ausdehnung von Flüssigkeiten, um die Temperatur anzuzeigen. Typischerweise wird dafür Quecksilber (Hg) oder gefärbter Alkohol in einem dünnen Glasröhrchen (Kapillare) verwendet.
\begin{figure}[htb]
    \centering
    \includegraphics[width=0.22\linewidth]{Bilder/Kapitel_Waermelehre/hg_alkohol_fluessigthermometer_vergleich.png}
    \caption{Vergleich eines Quecksilber- (Hg) und eines Alkoholthermometers. Das unterschiedliche Ausdehnungsverhalten der Flüssigkeiten führt zu einer nicht-linearen Skala, wenn eine hohe Genauigkeit erforderlich ist.}\label{fig: hg_alkohol_thermometer}
\end{figure}
Die angezeigte Temperaturskala hängt dabei nicht nur von der Art der Flüssigkeit ab, sondern auch von der Glasart des Röhrchens, da sich auch das Glas selbst ausdehnt. Ein wichtiger Punkt ist, dass sich die Flüssigkeiten im Allgemeinen nicht perfekt linear über den gesamten Temperaturbereich ausdehnen. Vergleicht man beispielsweise ein Alkohol- mit einem Quecksilberthermometer, die beide bei \SI{0}{\celsius} und \SI{100}{\celsius} kalibriert wurden, können bei Zwischentemperaturen leicht unterschiedliche Werte angezeigt werden. Für alltägliche Zwecke ist diese Abweichung oft vernachlässigbar, für präzise wissenschaftliche Messungen sind jedoch gleichmäßigere Skalen erforderlich, wie sie beispielsweise Gasthermometer bieten.




\section{Thermische Ausdehnung}\label{sec: thermische_Ausdehnung}
Die meisten Materialien dehnen sich bei Erwärmung aus und ziehen sich bei Abkühlung zusammen. Dieses Phänomen ist die Grundlage für viele Thermometer und muss in technischen Anwendungen, wie dem Brückenbau oder bei Eisenbahnschienen, berücksichtigt werden.

\subsection{Lineare Ausdehnung fester Körper}\label{subsec: lineare_ausdehnung}
Betrachten wir einen Stab der Länge $L_0$ bei einer Anfangstemperatur $T_0$, dargestellt in \cref{fig: lineare_ausdehnung_L0_L}. Erwärmt man den Stab auf eine Temperatur $T = T_0 + \Delta T$, so ändert sich seine Länge. Für einen begrenzten Temperaturbereich ist die Längenänderung $\Delta L = L-L_0$ in guter Näherung proportional zur Temperaturänderung $\Delta T = T - T_0$ und zur ursprünglichen Länge $L_0$:
\begin{equation}\label{eq: lineare_ausdehnung_delta_L}
    \Delta L = \alpha \cdot L_0 \cdot \Delta T \mDot
\end{equation}
Die Längenänderung $\Delta L$ fällt demnach größer aus, je länger der Stab $(L_0)$ bei der Ausgangstemperatur ist und je mehr man ihn erhitzt ($\Delta T$). Die neue Länge $L(T)$ ist somit:
\begin{equation}\label{eq: lineare_ausdehnung_L_T}
    L(\Delta T) =L_0 + \Delta L =  L_0 (1 + \alpha \cdot \Delta T) \mDot
\end{equation}
Der Proportionalitätsfaktor $\alpha$ wird als \textbf{linearer Längenausdehnungskoeffizient} bezeichnet. Er ist eine materialspezifische Eigenschaft und gibt die relative Längenänderung pro Kelvin an. Aus \cref{eq: lineare_ausdehnung_delta_L} erhalten wir
\begin{equation}\label{eq: def_L_von_DeltaT}
    \alpha = \frac{1}{L_0} \frac{\Delta L}{\Delta T} \mComma 
\end{equation}
wobei $[\alpha] = \si{\kelvin^{-1}}$.
\begin{figure}[htb]
    \centering
    \includegraphics[width=0.5\linewidth]{Bilder/Kapitel_Waermelehre/lineare_ausdehnung_L0_L.png}
    \caption{Ein Stab der Länge $L_0$ bei Temperatur $T_0$ dehnt sich bei einer Erwärmung auf $T = T_0 + \Delta T$ auf die Länge $L = L_0 + \Delta L$ aus.}\label{fig: lineare_ausdehnung_L0_L}
\end{figure}
Genauere Messungen zeigen, dass der Ausdehnungskoeffizient $\alpha$ schwach von der Temperatur abhängt. Für die meisten technischen Anwendungen ist die Annahme eines konstanten Wertes jedoch eine ausreichend gute Näherung. Die in \Cref{tab: lineare_ausdehnungskoeffizienten} zusammengefassten linearen Ausdehnungskoeffizienten verdeutlichen, wie stark sich die Längenausdehnung verschiedener Materialien unterscheiden kann – teilweise um zwei Größenordnungen oder mehr.
\begin{table}[htb]
    \centering
    \caption{Linearer Längenausdehnungskoeffizient $\alpha$ verschiedener fester Stoffe bei Raumtemperatur.}\label{tab: lineare_ausdehnungskoeffizienten}
    \vspace{4pt}
    \begin{tabular}{lc}
        \toprule
        \textbf{Fester Stoff} & \textbf{Linearer Ausdehnungskoeffizient $\alpha$ (\SI{e-6}{\kelvin^{-1}})} \\
        \midrule
        Aluminium     & 23,8 \\
        Eisen         & 12   \\
        V2A-Stahl     & 16   \\
        Kupfer        & 16,8 \\
        Natrium       & 71   \\
        Wolfram       & 4,3  \\
        Invar         & 1,5  \\
        Hartgummi     & 75--100 \\
        \bottomrule
    \end{tabular}
\end{table}


\subsection{Nichtlineare Ausdehnung}\label{subsec: nichtlineare_ausdehnung}
Für größere Temperaturänderungen oder bei Materialien, die ein ausgeprägt nichtlineares Verhalten zeigen, reicht die lineare Näherung aus \cref{eq: lineare_ausdehnung_L_T} nicht mehr aus. In solchen Fällen muss die Temperaturabhängigkeit des Ausdehnungskoeffizienten $\alpha$ selbst berücksichtigt werden. Eine genauere Beschreibung wird oft durch einen quadratischen Ansatz erreicht:
\begin{equation}\label{eq: alpha_von_DeltaT}
    \alpha(\Delta T) = \alpha_0 + \beta \cdot \Delta T \mDot
\end{equation}
Hierbei ist $\alpha_0$ der lineare Ausdehnungskoeffizient bei der Referenztemperatur $T_0$ und $\beta$ ist ein weiterer materialabhängiger Koeffizient, der den quadratischen Anteil der Ausdehnung beschreibt. Setzt man diesen temperaturabhängigen Koeffizienten in die Längengleichung ein, erhält man:
\begin{equation}\label{eq:nichtlineare_ausdehnung}
    L(\Delta T) = L_0 (1+ \alpha(\Delta T)\cdot \Delta T) = L_0 (1 + \alpha_0 \Delta T + \beta \Delta T^2) \mDot
\end{equation}
Der Term $\beta \Delta T^2$ repräsentiert die Abweichung von der linearen Ausdehnung.

\begin{examplebox}[]{Beispiel: Nichtlineare Ausdehnung von Aluminium}
    Für Aluminium bei \SI{0}{\celsius} betragen die Ausdehnungskoeffizienten etwa $\alpha_0 = \SI{23.8e-6}{\kelvin^{-1}}$ und $\beta = \SI{1.8e-8}{\kelvin^{-2}}$. Das Verhältnis der quadratischen Ausdehnung zur linearen Ausdehnung beträgt in diesem Fall
    \begin{equation}
        \frac{\beta \Delta T^2}{\alpha_0 \Delta T} = \frac{\beta \Delta T}{\alpha_0} = \frac{\num{1.8e-8}}{\num{23.8e-6}} \cdot \Delta T \approx \num{7.56e-4} \Delta T \mDot
    \end{equation}
    Bei einer Erhöhung der Temperatur um $\Delta T = \SI{100}{\celsius}$ ändert sich der lineare Ausdehnungskoeffizient von $\alpha_0$ auf $\alpha_0(1+\num{0.0756})$. Der nichtlineare Anteil macht in diesem Fall also bereits etwa \SI{7.56}{\percent} der gesamten Ausdehnung aus.
\end{examplebox}

\subsection{Bimetallthermometer}\label{subsec: Bimetallthermometer}
Das Prinzip der unterschiedlichen thermischen Ausdehnung wird im \textbf{Bimetallthermometer} genutzt. Hierbei werden zwei Streifen aus unterschiedlichen Metallen (z.B. Stahl und Kupfer) fest miteinander verbunden. Da die Metalle unterschiedliche Ausdehnungskoeffizienten haben, krümmt sich der Verbundstreifen bei einer Temperaturänderung. Im Ausgangszustand $T_0$ haben beide Materialien dieselbe Länge, siehe \cref{fig: bimetallstreifen_bimetallthermometer}. Das Material des schwarzen Streifens hat einen höheren Ausdehnungskoeffizienten $\alpha_\text{schwarz} > \alpha_\text{rot}$. Bei Erwärmung biegt sich der Bimetallstreifen so, dass das Material mit dem größeren Ausdehnungskoeffizienten länger wird. Je größer der Unterschied in den Ausdehnungskoeffizienten, desto größer ist die Biegung. \\

Diese Krümmung kann auf eine geeichte Skala übertragen werden, um die Temperatur anzuzeigen. Dazu rollt man den Bimetallstreifen auf und befestigt ihn derart, dass eine Erwärmung $\Delta T$ über einen Zeiger auf einer geeigneten Skala ablesbar wird. 
\begin{figure}[htb]
    \centering
    \includegraphics[width=0.5\linewidth]{Bilder/Kapitel_Waermelehre/bimetallthermometer_beschreibung.png}
    \caption{(Links) Funktionsweise eines Bimetallstreifens. Das Material mit dem größeren $\alpha$ (schwarz) dehnt sich stärker aus und zwingt den Streifen zur Krümmung. (Rechts) In einem Zeigerthermometer wird ein aufgerollter Bimetallstreifen verwendet, um eine Drehbewegung zu erzeugen.}\label{fig: bimetallstreifen_bimetallthermometer}
\end{figure}

\subsection{Volumenausdehnung}\label{subsec: volumenausdehnung}
Analog zur linearen Ausdehnung erfahren Körper auch eine Volumenausdehnung. Der \textbf{Volumenausdehnungskoeffizient} $\gamma$ (auch kubischer Ausdehnungskoeffizient) ist analog zu \cref{eq: def_L_von_DeltaT} definiert als:
\begin{equation}
    \gamma = \frac{1}{V_0} \frac{\Delta V}{\Delta T} \mDot
\end{equation}
Die Volumenänderung $\Delta V$ bei einer Temperaturänderung $\Delta T$ ist gegeben durch:
\begin{equation}\label{eq: volumenausdehnung}
    \Delta V = \gamma \cdot V_0 \cdot \Delta T \quad \implies \quad V(\Delta T) = V_0 (1 + \gamma \cdot \Delta T) \mDot
\end{equation}
Für homogene, isotrope Festkörper, die sich in alle Richtungen gleich ausdehnen, gilt der Zusammenhang:
\begin{equation}
    \gamma = 3\alpha \mComma
\end{equation}
wenn $\alpha$ der lineare Längenausdehnungskoeffizient ist. Diese Näherung ergibt sich aus 
\begin{equation}
    V(\Delta T) = L{(\Delta T)}^3 = {[L_0 \cdot (1+\alpha\Delta T)]}^3 = V_0\cdot{(1+\alpha\Delta T)}^3 \mDot
\end{equation} 
Der kubische Term ergibt 
\begin{equation}
    {(1+\alpha \Delta T)}^3 = 1 + 3\alpha \Delta T + 3\alpha^2 \Delta T^2 + \alpha^3 \Delta T^3 
\end{equation}
und für Festkörper gilt meist $\alpha\Delta T \ll 1$. Daher können die quadratischen und kubischen Terme in $\Delta T$ vernachlässigt werden. Somit bleibt 
\begin{equation}
    {(1+\alpha \cdot \Delta T)}^3 \approx 1 + 3\alpha \Delta T = 1 + \gamma \Delta T\mComma
\end{equation}
sofern $\alpha \Delta T \ll 1$. 
In \cref{tab: volumen_ausdehnungskoeffizienten} sind einige Volumenausdehnungskoeffizienten für verschiedene feste, flüssige und gasförmige Stoffe aufgezählt. 
\begin{table}[htb]
    \centering
    \caption{Volumenausdehnungskoeffizienten $\gamma$ für ausgewählte feste, flüssige und gasförmige Stoffe.}\label{tab: volumen_ausdehnungskoeffizienten}
    \vspace{4pt}
    \begin{tabular}{llc}
        \toprule
        \textbf{Stoff} & \textbf{Phase} & \textbf{Ausdehnungskoeffizient $\gamma$ (\SI{e-6}{\kelvin^{-1}})} \\
        \midrule
        Aluminium      & fest           & 75   \\
        Eisen          & fest           & 35   \\
        Kupfer         & fest           & 50   \\
        \midrule
        Benzin         & flüssig        & 950  \\
        Quecksilber    & flüssig        & 180  \\
        Ethanol        & flüssig        & 1100 \\
        Wasser         & flüssig        & 210  \\ % Korrigierter Wert, 2 aus PDF ist zu niedrig für den Bereich bis 100°C
        \midrule
        Ideales Gas    & gasförmig      & 3661 \\
        Helium (He)    & gasförmig      & 3660 \\
        Argon (Ar)     & gasförmig      & 3671 \\
        Sauerstoff ($\text{O}_2$) & gasförmig  & 3674 \\
        Kohlendioxid ($\text{CO}_2$) & gasförmig & 3726 \\
        \bottomrule
    \end{tabular}
\end{table}


\subsection{Ausdehnung von Gasen (Gesetze von Gay-Lussac)}\label{subsec: ausdehnung_gase}
Auch Gase dehnen sich bei Erwärmung aus, jedoch deutlich stärker als Festkörper oder Flüssigkeiten. Die Ausdehnungskoeffizienten von Gasen sind ungefähr \qtyrange{2}{3}{} Größenordnungen größer als die von festen oder flüssigen Körpern. Die Gesetze von Gay-Lussac beschreiben dieses Verhalten für ideale Gase:
\begin{enumerate}
    \item \textbf{Bei konstantem Druck (isobar):} Hält man den Druck eines Gases konstant, so ist das Volumen direkt proportional zur \textbf{absoluten Temperatur} $T$ (in Kelvin). Wir starten mit der Volumenausdehnung als Funktion der Celsius-Temperatur $\vartheta$:
    \begin{equation}\label{eq: gay_lussac_isobar}
        V(\vartheta) = V_0(1 + \gamma \vartheta), % \quad \text{oder} \quad \frac{V}{T} = \const
    \end{equation}
    wobei für ideale Gase $\gamma = \dfrac{1}{\num{273.15}}\,\si{\celsius^{-1}}$ eine Konstante ist und $\vartheta$ die Temperatur in $\si{\celsius}$ ist. Bezieht man die Volumenänderung auf eine Referenztemperatur von $\vartheta_0 = \SI{0}{\celsius}$ ($T_0 = \SI{273.15}{\kelvin}$), so lässt sich \cref{eq: gay_lussac_isobar} auch auf die absolute Temperatur $T = \qty{273.15} + \vartheta$ umformen  
    \begin{equation} \begin{gathered}
        V(\vartheta) = V_0(1+\gamma \vartheta) = V_0 \left(1 + \frac{\vartheta}{\num{273.15}} \right) = V_0\left( \frac{\num{273.15} + \vartheta}{\num{273.15}} \right) = \\
        \implies V(T) = V_0 \left( \frac{T}{T_0} \right) \mDot
    \end{gathered}\end{equation}
    Somit findet man, dass 
    \begin{equation}\label{eq: gay_lussac_V_ueber_V0}
        \frac{V}{V_0} = \frac{T}{T_0} \quad \text{oder} \quad \frac{V}{T} = \frac{V_0}{T_0} = \const \mDot
    \end{equation}
    \item \textbf{Bei konstantem Volumen (isochor):} Hält man das Volumen eines Gases konstant, so ist der Druck eines Gases direkt proportional zur absoluten Temperatur. Auch hier beginnt man mit der Druckänderung als Funktion der Celsius-Temperatur $\vartheta$
    \begin{equation}
        p(\vartheta) = p_0(1 + \gamma_p \vartheta)
    \end{equation}
    und formt dies auf die absolute Temperatur $T$ (in Kelvin) um. Somit erhält man 
    \begin{equation}\label{eq: gay_lussac_p_ueber_p0}
        \frac{p}{p_0} = \frac{T}{T_0} \quad \text{oder} \quad \frac{p}{T} = \frac{p_0}{T_0} = \const \mDot
    \end{equation}
\end{enumerate}
Diese beiden Gesetze werden in dem Gesetz von Gay-Lussac zusammengefasst, indem man die \cref{eq: gay_lussac_V_ueber_V0,eq: gay_lussac_p_ueber_p0} etwas umformt, wie im Folgenden für die isochore Druckänderung dargestellt: 
\begin{align}
    \frac{p}{T} &= \frac{p_0}{T_0} \nonumber \\
    p &= p_0 \cdot \frac{T}{T_0} \nonumber \\
    p - p_0 &=  \left( p_0 \cdot \frac{T}{T_0} \right) - p_0 = p_0 \left( \frac{T}{T_0} - 1 \right) = p_0 \left( \frac{T - T_0}{T_0} \right) = \nonumber \\
    &= \frac{p_0}{T_0} \left( T - T_0 \right) \mDot
\end{align}
Somit folgt für die Druckänderung 
\begin{equation}
    (p-p_0) \propto (T-T_0) \mDot
\end{equation}
Die Volumenänderung bei konstantem Druck lässt sich analog herleiten. Zusammengefasst ergibt sich das sogenannte \textit{Gesetz von Gay-Lussac}.
\begin{importantbox}[]{Gesetz von Gay-Lussac}
    Das Gay-Lussac Gesetz beschreibt die Druckänderung bei konstantem Volumen (isochor) und die Volumenänderung bei konstantem Druck (isobar)
    \begin{equation}\begin{aligned}
        (p - p_0) &\propto (T - T_0) \mComma \quad \text{für konstantes Volumen}\\
        (V - V_0) &\propto (T - T_0) \mComma \quad \text{für konstanten Druck.}
    \end{aligned}\end{equation}
\end{importantbox}

\subsection{Gasthermometer}\label{subsec: Gasthermometer}
Bei einem \textbf{Gasthermometer} mit konstantem Volumen dient die Druckänderung als Maß für die Temperatur. Da sich Gase sehr gleichmäßig ausdehnen und Flüssigkeiten näherungsweise inkompressibel sind, sind Gasthermometer besonders präzise und werden zur Kalibrierung anderer Thermometer verwendet. 

Die Funktionsweise eines Gasthermometers erklärt sich anhand der \cref{fig: gasthermometer}. Das System besteht aus einem mit Gas gefüllten Kolben ($B_1$), der über einen flexiblen Schlauch mit einem Quecksilber-Manometer verbunden ist. Das entscheidende Anwendungsmerkmal eines Gasthermometers ist das \textbf{konstante Volumen}:
Um eine Messung durchzuführen, wird der bewegliche Behälter $B_3$ vertikal so verschoben, dass der Quecksilberspiegel im Schenkel $B_2$ exakt auf der Referenzmarke (bei $h=0$) steht. Damit nimmt das Gas in $B_1$ immer dasselbe Volumen ein.\\

Der absolute Druck $p$ im Inneren des Gasgefäßes $B_1$ wird durch die Höhendifferenz $h$ zwischen den beiden Quecksilbersäulen bestimmt. Der Druck $p$ des Gases in $B_1$ muss genau den äußeren Luftdruck $p_{\atm}$ und den hydrostatischen Druck der Quecksilbersäule kompensieren:
\begin{equation}\label{eq: druck_gasthermometer}
    p = p_{\atm} + \rho_{\text{Hg}} \cdot g \cdot h \mDot
\end{equation}
Hierbei ist $\rho_{\text{Hg}}$ die Dichte des Quecksilbers und $g$ die Erdbeschleunigung.

\begin{figure}[htb]
    \centering
    \includegraphics[width=0.5\linewidth]{Bilder/Kapitel_Waermelehre/gasthermometer.png}
    \caption{Aufbau eines Gasthermometers mit konstantem Volumen. Der Behälter $B_3$ wird gehoben oder gesenkt, um den Meniskus in $B_2$ stets bei der Nullmarke zu halten.}\label{fig: gasthermometer}
\end{figure}

\subsubsection*{Kalibrierung}
Um dem gemessenen Druck eine Temperatur in Grad Celsius zuzuordnen, muss das Thermometer kalibriert werden. Dazu werden zwei Fixpunkte genutzt, üblicherweise der Gefrierpunkt und der Siedepunkt von Wasser.

\begin{enumerate}
    \item \textbf{Messung am Eispunkt ($\SI{0}{\celsius}$):}
    Das Gefäß $B_1$ wird in ein Eiswasserbad getaucht. Nachdem sich das thermische Gleichgewicht eingestellt hat, wird $B_3$ justiert, bis der Pegel in $B_2$ wieder bei Null steht. Man misst die Höhendifferenz $h_0$ und berechnet den Druck $p_0$:
    \begin{equation}
        p_0 = p_{\text{atm}} + \rho_{\text{Hg}} \cdot g \cdot h_0 \mDot
    \end{equation}
    
    \item \textbf{Messung am Siedepunkt ($\SI{100}{\celsius}$):}
    Anschließend wird das Gefäß in kochendes Wasser (oder Dampf) gebracht. Durch die Erwärmung steigt der Druck, wodurch das Quecksilber in $B_2$ nach unten gedrückt würde. Um das Volumen konstant zu halten, muss $B_3$ stark angehoben werden, bis der Pegel in $B_2$ wieder die Nullmarke erreicht. Die neue Höhendifferenz $h_{100}$ liefert den Druck $p_{100}$:
    \begin{equation}
        p_{100} = p_{\text{atm}} + \rho_{\text{Hg}} \cdot g \cdot h_{100} \mDot
    \end{equation}
\end{enumerate}

Da der Druck bei konstantem Volumen linear mit der Temperatur steigt (Gesetz von Amontons), kann nun für jeden beliebigen gemessenen Druck $p_T$ die zugehörige Temperatur $T$ interpoliert werden.

\begin{rememberbox}{Temperaturskala am Gasthermometer}
    Die Temperatur $T$ in Grad Celsius ergibt sich aus dem Verhältnis der Druckänderung zur Druckdifferenz der beiden Fixpunkte:
    \begin{equation}\label{eq: formel_gasthermometer_celsius}
        T[\si{\celsius}] = \frac{p_T - p_0}{p_{100} - p_0} \cdot \SI{100}{\celsius} \mDot
    \end{equation}
    Hierbei ist $p_T$ der Druck bei der zu messenden Temperatur und $p_0$ bzw. $p_{100}$ sind die Drücke am Gefrier- bzw. Siedepunkt von Wasser.
\end{rememberbox}

\subsubsection{Herleitung der Formel}
Die Herleitung von \cref{eq: formel_gasthermometer_celsius} basiert auf der Annahme, dass sich der Druck eines idealen Gases bei konstantem Volumen linear mit der Temperatur ändert (Gesetz von Amontons). Geometrisch entspricht dies einer Geraden im $(p, T)$-Diagramm, die durch zwei experimentell bestimmte Punkte verläuft: Den Eispunkt $(p_0, T_0)$ und den Siedepunkt $(p_{100}, T_{100})$ von Wasser. Wir verwenden die Zweipunktform der Geradengleichung mit dem unbekannten Punkt $(p_T, T)$:
\begin{equation}\begin{aligned}
    \frac{T - T_{0}}{p_T - p_{0}} &= \frac{T_{100} - T_{0}}{p_{100} - p_{0}} \mComma \\
    \frac{T - 0}{p_T - p_0} &= \frac{100 - 0}{p_{100} - p_0} \mDot
\end{aligned}\end{equation}
Durch Umformen nach $T$ folgt direkt die gesuchte Beziehung:
\begin{align}
    \frac{T}{p_T - p_0} &= \frac{100}{p_{100} - p_0} & | \cdot (p_T - p_0) \notag \\
    T &= \frac{100}{p_{100} - p_0} \cdot (p_T - p_0) \notag \\
    T &= \frac{p_T - p_0}{p_{100} - p_0} \cdot \SI{100}{\celsius} \mDot \label{eq: herleitung_ende}
\end{align}
Diese lineare Skala ist die \textbf{empirische Celsiusskala} des Gasthermometers.
\subsubsection*{Vereinfachung in der Praxis}
In der Praxis ist es oft nicht nötig, die absoluten Drücke zu berechnen. Setzen wir die Definition des hydrostatischen Drucks $p = p_{\text{atm}} + \rho_{\text{Hg}} g h$ in die Temperaturformel ein, so sehen wir, dass sich der Atmosphärendruck $p_{\text{atm}}$ durch die Differenzbildung eliminiert:
\begin{align}
    p_T - p_0 &= (p_{\text{atm}} + \rho_{\text{Hg}} g h_T) - (p_{\text{atm}} + \rho_{\text{Hg}} g h_0) \notag \\
    &= \rho_{\text{Hg}} g (h_T - h_0) \mDot
\end{align}
Dasselbe gilt für den Nenner ($p_{100} - p_0$). Setzt man dies in \cref{eq: formel_gasthermometer_celsius} ein, kürzen sich auch die Dichte $\rho_{\text{Hg}}$ und die Erdbeschleunigung $g$ heraus:

\begin{equation}
    T = \frac{\cancel{\rho_{\text{Hg}} g} (h_T - h_0)}{\cancel{\rho_{\text{Hg}} g} (h_{100} - h_0)} \cdot 100 
    = \frac{h_T - h_0}{h_{100} - h_0} \cdot \SI{100}{\celsius} \mDot
\end{equation}

\begin{importantbox}{Praktische Messung}
    Solange der Atmosphärendruck während der Kalibrierung und Messung konstant bleibt, genügt es, die \textbf{Höhendifferenzen} der Quecksilbersäulen zu messen. Die Temperatur ist proportional zum Verhältnis der Höhenänderungen:
    \begin{equation}
        T[\si{\celsius}] = \frac{h_T - h_0}{h_{100} - h_0} \cdot \SI{100}{\celsius} \mDot
    \end{equation}
\end{importantbox}





\section{Das ideale Gas und die Zustandsgleichung}\label{sec: ideales_gas}
Die Funktionsweise des Gasthermometers in \cref{subsec: Gasthermometer} basierte auf der Beobachtung, dass der Druck eines Gases linear mit der Temperatur steigt, sofern das Volumen konstant gehalten wird (Gesetz von Amontons). Dies ist jedoch nur einer von mehreren Aspekten, wie sich Gase verhalten.

Im Gegensatz zu Festkörpern und Flüssigkeiten, deren Teilchen dicht gepackt sind und starke Bindungskräfte aufweisen, füllen Gase jeden verfügbaren Raum aus und lassen sich leicht komprimieren. Um das Verhalten von Gasen vollständig zu beschreiben, benötigen wir einen Zusammenhang zwischen allen drei Zustandsgrößen: Druck $p$, Volumen $V$ und Temperatur $T$.

Wir betrachten hierzu das Modell des \textbf{idealen Gases}. In diesem idealisierten Modell vernachlässigen wir das Eigenvolumen der Gasteilchen sowie die Kräfte zwischen ihnen.

\subsection{Das Gesetz von Boyle-Mariotte (Isotherme Zustandsänderung)}\label{subsec: boyle_mariotte}
Während das Gasthermometer bei konstantem Volumen arbeitet, betrachten wir nun, was passiert, wenn wir das Volumen ändern, aber die Temperatur konstant halten ($T = \const$).
Ein solches Experiment lässt sich mit einem Kolben in einem Zylinder durchführen (siehe \cref{fig: kolben_zylinder_gas_pV}). Drückt man den Kolben langsam herunter, verringert sich das Volumen. Gleichzeitig beobachtet man, dass der Druck im Inneren steigt.

\begin{figure}[tb]
    \centering
    \resizebox{0.35\linewidth}{!}{
    \begin{tikzpicture}[>=Latex] 
        % Parameter für einfache Größenanpassung
        \def\cylW{3.2}   % Breite des Zylinders
        \def\cylH{3.8}   % Höhe der Wände
        \def\pistY{2.0}  % y-Position des Kolbenbodens (etwas tiefer gesetzt für Kompression)
        \def\pistH{0.5}  % Dicke des Kolbens

        % 1. Zylinder
        \draw[thick] (0, \cylH) -- (0, 0) -- (\cylW, 0) -- (\cylW, \cylH);
        % 2. Kolben
        \filldraw[fill=red!20, draw=black, thick] (0, \pistY) rectangle (\cylW, \pistY+\pistH);
        
        % 3. Gas (Punkte)
        \foreach \i in {1,...,50} {
            \fill[blue!70] (0.03 + rnd*\cylW*0.96, 0.03 + rnd*\pistY*0.96) circle (1.5pt);
        }
        
        % 4. Beschriftung 
        \node[draw , fill=white] at (0.5*\cylW, 0.5*\pistY) {\large $V, p$};
        \node[right=0.1] at (\cylW, 0.5*\pistY) {\large $T=\const$};
        
        % 5. Kraft Pfeil
        \draw[-{Stealth}, ultra thick] (0.5*\cylW, \cylH + 0.5) -- (0.5*\cylW, \pistY+\pistH);
        \node[right] at (0.5*\cylW, \cylH + 0.2) {Kompression};
    \end{tikzpicture}
    }
    \caption{Gesetz von Boyle-Mariotte: Wird das Volumen $V$ bei konstanter Temperatur verringert, steigt der Druck $p$, da die Teilchen pro Zeiteinheit öfter gegen die Wand stoßen.}\label{fig: kolben_zylinder_gas_pV}
\end{figure}

Robert Boyle und Edme Mariotte fanden unabhängig voneinander heraus, dass für eine abgeschlossene Gasmenge bei \textbf{konstanter Temperatur} der Druck umgekehrt proportional zum Volumen ist ($p \propto 1/V$). Das bedeutet, das Produkt aus Druck und Volumen ist konstant:
\begin{equation}\label{eq: boyle_mariotte_gesetz}
    p \cdot V = \const \qquad (\text{für } T = \const) \mDot
\end{equation}
Halbiert man beispielsweise das Volumen eines Gases bei konstanter Temperatur, verdoppelt sich sein Druck.

\subsection{Die ideale Gasgleichung}\label{subsec: ideale_gasgleichung}
Fassen wir nun die experimentellen Befunde zusammen:
\begin{enumerate}
    \item \textbf{Gesetz von Amontons (Isochor):} $p \propto T$ (bei $V=\const$)
    \item \textbf{Gesetz von Gay-Lussac (Isobar):} $V \propto T$ (bei $p=\const$)
    \item \textbf{Gesetz von Boyle-Mariotte (Isotherm):} $p \propto 1/V$ (bei $T=\const$)
\end{enumerate}
Alle drei Gesetze sind Spezialfälle einer einzigen, fundamentalen Beziehung. Kombiniert man die Proportionalitäten ($p \propto T$ und $p \propto 1/V$), erhält man:
\begin{equation}
    p \propto \frac{T}{V} \quad \Longleftrightarrow \quad p \cdot V \propto T \mDot
\end{equation}
Um aus der Proportionalität eine Gleichung zu machen, benötigen wir eine Proportionalitätskonstante. Diese hängt von der Menge des Gases ab.

\begin{importantbox}{Ideale Gasgleichung}
    Betrachten wir die Anzahl der Teilchen $N$ im Gas, so erhalten wir die \textbf{thermische Zustandsgleichung idealer Gase}:
    \begin{equation}\label{eq: ideale_gasgleichung}
        p \cdot V = N \cdot k_B \cdot T
    \end{equation}
    Hierbei ist $p$ der absolute Druck in \si{\pascal}, $V$ das Volumen in \si{\cubic\meter}, $N$ die absolute Anzahl der Gasteilchen, $T$ die absolute Temperatur in \si{\kelvin} und $k_B \approx \SI{1.38e-23}{\joule\per\kelvin}$ die Boltzmann-Konstante.
\end{importantbox}
Diese Gleichung ist das Herzstück der Wärmelehre für Gase. Sie erlaubt es, eine unbekannte Größe zu berechnen, wenn die anderen bekannt sind. Zudem impliziert sie alle vorherigen Gesetze: Ist zum Beispiel $T$ konstant, wird die rechte Seite zu einer Konstanten und wir erhalten wieder $p \cdot V = \text{const}$ (Boyle-Mariotte).


\section{Die barometrische Höhenformel}\label{sec: barometrische_hoehenformel}

\begin{figure}[tb]
    \centering
    \resizebox{0.45\linewidth}{!}{
    \begin{tikzpicture}[>=Latex]
        % Parameter
        \def\cylW{3.5}       % Breite des Zylinders
        \def\cylH{5.0}       % Höhe
        \def\hBase{1.2}      % Höhe h
        \def\dh{1.8}         % Abstand dh (relative Höhe zu h)
        \def\hTop{\hBase+\dh}% Höhe h + dh
        \def\hNull{0.7}

        % 1. Füllung (Linearer Farbverlauf orange -> weiß)
        % 'middle color' kann optional für feinere Steuerung genutzt werden, 
        % hier reicht bottom zu top.
        \shade[bottom color=orange!60, top color=white] 
            (0,\hBase) rectangle (\cylW, \cylH);

        % 2. Zylinderwände (nur links und rechts)
        \draw[thick] (0,0) -- (0,\cylH);
        \draw[thick] (\cylW,0) -- (\cylW,\cylH);

        % 3. Markierungslinien
        % Untere Linie bei h (dick orange + gestrichelt schwarz für den Effekt)
        \draw[line width=4pt, orange] (0, \hBase) -- (\cylW, \hBase);
        \draw[thick, dashed] (0, \hBase) -- (\cylW, \hBase);
        % Obere Linie bei h + dh (nur gestrichelt)
        \draw[thick, dashed] (0, \hTop) -- (\cylW, \hTop);

        % 4. Koordinatenachse z (links außen)
        \draw[->, ultra thick] (-1.1, 3*\cylH/4) -- (-1.1, \cylH) node[right, midway] {\Large $z$};

        % 5. Druck-Pfeil p(z) (von oben kommend)
        \draw[->, ultra thick] (\cylW/2, \cylH + 0.5) -- (\cylW/2, \hBase + 0.2) 
            node[right=0.1, pos=0.05] {\Large $p(z)$};

        % 6. Beschriftungen
        % Links (Höhen)
        \node[left=0.2] at (0, \hBase) {\large $h$};
        \node[left=0.2] at (0, \hTop) {\large $h + dh$};

        % Rechts (Dichte/Druck Werte)
        \node[right=0.2] at (\cylW, \hBase) {\large $\rho(h), p(h)$};
        % Multiline Node für den oberen Text
        \node[right=0.2, align=left] at (\cylW, \hTop) {\large $\rho(h + dh),$ \\ \large $p(h + dh)$};

        % Mitte (Fläche A) - in orange
        \node[orange!90!black] at (\cylW/2, \hBase - 0.6) {\Large $A$};
    \end{tikzpicture}
    }
    \caption{Mit zunehmender Höhe $h$ nimmt das Gewicht der darüberliegenden Luftsäule ab, wodurch Druck und Dichte sinken.}\label{fig: luftsaeule_hoehenformel_anschauung}
\end{figure}

Der Luftdruck, den wir auf der Erdoberfläche messen, entsteht durch die Gewichtskraft der darüber liegenden Luftsäule (siehe \cref{fig: luftsaeule_hoehenformel_anschauung}). Da die Dichte der Luft mit der Höhe abnimmt, ist auch der Druck nicht konstant. Wir wollen eine Formel für den Druck als Funktion der Höhe $h$ herleiten.

Bei einem infinitesimalen Höhenanstieg von $h$ auf $h + \dd h$ nimmt das Gewicht der Luftsäule um $\rho\cdot g \cdot \dd V = \rho\cdot g \cdot A \cdot \dd h$ ab und daher sinkt der Druck $p$ um
\begin{equation}
    dp = -\rho(h) \cdot g \cdot dh \mComma
\end{equation}
wobei $\rho(h)$ die Dichte in der Höhe $h$ ist. Das negative Vorzeichen zeigt, dass der Druck mit zunehmender Höhe abnimmt.

Im Gegensatz zu einer inkompressiblen Flüssigkeit ist die Dichte der Luft vom Druck abhängig. Aus dem Gesetz von Boyle-Mariotte folgt für eine isotherme Atmosphäre ($T = \const$), dass $\rho/p = \const = \rho_0/p_0$. Wir können also $\rho$ ersetzen:
\begin{equation}
    dp = - \frac{\rho_0}{p_0} p \cdot g \cdot dh \implies \frac{dp}{p} = -\frac{\rho_0 g}{p_0} dh \mDot
\end{equation}
Integration dieser Differentialgleichung von Höhe $0$ (mit Druck $p_0$ und Dichte $\rho_0$) bis zur Höhe $h$ (mit Druck $p(h)$) liefert:
\begin{equation}
    \int_{p_0}^{p(h)} \frac{1}{p} dp = -\int_0^h \frac{\rho_0 g}{p_0} dh \implies \ln\left(\frac{p(h)}{p_0}\right) = -\frac{\rho_0 g}{p_0} h \mDot
\end{equation}
\begin{figure}[tb]
    \centering
    \includegraphics[width=0.4\textwidth]{Bilder/Kapitel_Waermelehre/hoehenformel.pdf}
    \caption{Der exponentiell abnehmende Luftdruck $p(z)$ über der Höhe $h$ in km.}\label{fig: barom_hoehenformel_grafik}\label{fig: barometrische_hoehenformel_exp_abnahme}
\end{figure}
\begin{importantbox}{Barometrische Höhenformel}
    Durch Auflösen nach $p(h)$ erhält man die \textbf{barometrische Höhenformel} für eine isotherme Atmosphäre:
    \begin{equation}\label{eq: barometrische_hoehenformel}
        p(h) = p_0 \cdot e^{-\frac{\rho_0 g}{p_0}h} = p_0 \cdot e^{-h/H} \mDot
    \end{equation}
    Der Druck in der Atmosphäre nimmt exponentiell mit der Höhe ab (siehe \cref{fig: barometrische_hoehenformel_exp_abnahme}). Die Größe $H = p_0/(\rho_0 g)$ wird als Skalenhöhe bezeichnet und beträgt für Luft bei \SI{0}{\celsius} etwa \SI{8}{\kilo\meter}. In einer Höhe von ca. \SI{5.8}{\kilo\meter} beträgt der Luftdruck nur noch die Hälfte des Bodenwertes.
\end{importantbox}
Laut dieser isothermen Näherung erhält man für den Luftdruck auf der Spitze des Mount Everest (\SI{8849}{\meter}) nur noch $\approx \SI{34.5}{\percent}$ des Luftdrucks auf Meereshöhe, was nur ungefähr $\SI{2}{\percent}$ vom realen Wert abweicht. 

Die \textbf{Druckabnahme} einer isothermen Lufthülle folgt also einem \textbf{Exponentialgesetz} im Gegensatz zur \textbf{linearen Druckabnahme in einer Flüssigkeit}. Dieser Unterschied ist in \cref{fig: gassaeule_fluessigkeit_gegenueberstellung} dargestellt. Die Kompressibilität der Luft führt zu einer exponentiellen (schneller als linear) Abnahme des Drucks. Die Lufthülle der Erde hat jedoch keine scharfe Grenze. In der realen Erdatmosphäre ändert sich die Temperatur allerdings mit der Höhe, weshalb die barometrische Höhenformel nur für die untere Atmosphäre eine gute Näherung darstellt. 

\begin{figure}[tb]
    \centering
    \resizebox{0.90\linewidth}{!}{
        \begin{tikzpicture}[>=Latex]
            % =========================
            % Gemeinsame Parameter
            % =========================
            \def\cylW{4.4}       % Breite des Zylinders
            \def\cylH{5.0}       % Gesamthöhe
            \def\hBase{1.2}      % Referenzhöhe h (etwas angehoben, damit man sie sieht)
            \def\dh{1.0}         % Abstand dh
            \def\offset{8.0}     % Abstand zwischen den Zylindern
            \def\AOffset{0.4}    % Vertikaler Versatz für die A-Beschriftung
            % ============================================================
            % LINKER ZYLINDER: GAS (Kompressibel)
            % ============================================================
            \begin{scope}
                \node[above, font=\bfseries] at (\cylW/2, \cylH+0.5) {Gas (kompressibel)};

                % 1. Füllung: Linearer Farbverlauf (orange -> weiß)
                % Visualisiert abnehmende Dichte mit der Höhe
                \shade[bottom color=orange!60, top color=white] (0,0) rectangle (\cylW, \cylH*0.99);

                % 2. Zylinderwände
                \draw[thick] (0,0) -- (0,\cylH);
                \draw[thick] (\cylW,0) -- (\cylW,\cylH);
                \draw[thick] (0,0) -- (\cylW,0); % Boden

                % 3. Markierungslinien für das Volumenelement
                % Untere Linie bei h
                \draw[line width=4pt, orange!80!black, opacity=0.7] (0, \hBase) -- (\cylW, \hBase);

                % 4. Koordinatenachse z
                \draw[->, ultra thick] (-1.4, \cylH*0.7) -- (-1.4, \cylH) node[right, above] {\Large $z$};
                
                % 5. Druck-Pfeil und Formel
                % Exponentielle Abnahme
                \draw[->, ultra thick, orange!90!black] (\cylW/2, \cylH-1.2) -- (\cylW/2, \hBase+0.1) 
                    node[pos=-0.2, align=center, fill=white, fill opacity=0.0, text opacity=1, rounded corners] 
                    {$p(z) = p_0 \cdot e^{-\frac{z}{H}}$\\[0.3em] \small (Exponentielle Abnahme)};

                % 6. Beschriftungen
                % Links (Höhen)
                \node[left=0.2] at (0, \hBase) {\large $p(h), \rho(h)$};
                \draw[line width=2pt] (-0.2, \hBase) -- (0.0, \hBase); % Tick mark
                % p_0
                \node[left=0.2] at (0, 0) {\large $p_0, \rho_0$};
                \draw[line width=2pt] (-0.2, 0.03) -- (0.0, 0.03); % Tick mark

                % Rechts (Dichte/Druck Werte)
                % Dichte variiert mit der Höhe

                % Mitte (Fläche A)
                \node[orange!90!black, font=\bfseries] at (\cylW/2, \hBase - \AOffset) {\Large $A$};
            \end{scope}

            % ============================================================
            % RECHTER ZYLINDER: FLÜSSIGKEIT (Inkompressibel)
            % ============================================================
            \begin{scope}[xshift=\offset cm]
                \node[above, font=\bfseries] at (\cylW/2, \cylH+0.5) {Flüssigkeit (inkompressibel)};

                % 1. Füllung: Solide Farbe (blau)
                % Visualisiert konstante Dichte
                \fill[blue!40] (0,0) rectangle (\cylW, \cylH);
                % Optional: Eine leichte Schattierung für 3D-Effekt, aber die Dichte wirkt konstant
                \shade[left color=blue!50, right color=blue!30, opacity=0.3] (0,0) rectangle (\cylW, \cylH);


                % 2. Zylinderwände
                \draw[thick] (0,0) -- (0,\cylH);
                \draw[thick] (\cylW,0) -- (\cylW,\cylH);
                \draw[thick] (0,0) -- (\cylW,0); % Boden
                % Wasseroberfläche andeuten
                \draw[thick, blue!80!black, decorate, decoration={snake, amplitude=1pt, segment length=5pt}] (0,\cylH) -- (\cylW,\cylH);

                % 3. Markierungslinien
                % Untere Linie bei h
                \draw[line width=4pt, blue!80!black, opacity=0.7] (0, \hBase) -- (\cylW, \hBase);

                % 5. Druck-Pfeil und Formel
                % Lineare Abnahme mit der Höhe (oder Zunahme mit der Tiefe)
                \draw[->, ultra thick, blue!90!black] (\cylW/2, \cylH-1.2) -- (\cylW/2, \hBase+0.1) 
                    node[pos=-0.2, align=center, fill=white, fill opacity=0.0, text opacity=1, rounded corners] 
                    {$p(z) = p_{0} - \rho g z$\\[0.3em] \small (Lineare Abnahme)};

                % 6. Beschriftungen
                % Links (Höhen)
                \node[left=0.2] at (0, \hBase) {\large $p(h), \rho_0$};
                \draw[line width=2pt] (-0.2, \hBase) -- (0.0, \hBase);

                \node[left=0.2] at (0, 0) {\large $p_0, \rho_0$};
                \draw[line width=2pt] (-0.2, 0.03) -- (0.0, 0.03);

                % Mitte (Fläche A)
                \node[blue!90!black, font=\bfseries] at (\cylW/2, \hBase - \AOffset) {\Large $A$};
            \end{scope}
        \end{tikzpicture}
    }
    \caption{Gegenüberstellung von Gas (kompressibel) und Flüssigkeit (inkompressibel) bezüglich Druck- und Dichteverteilung. Gase zeigen einen exponentiellen Druckabfall mit steigender Höhe, während Flüssigkeiten eine lineare Abnahme aufweisen.}\label{fig: gassaeule_fluessigkeit_gegenueberstellung}
\end{figure}

\subsubsection{Archimedisches Prinzip}
Gleich wie bei Flüssigkeiten gilt auch in Gasen das Archimedische Prinzip:
\begin{rememberbox}{Archimedisches Prinzip für Gase}
    Ein in ein Gas eingetauchter Körper erfährt eine Auftriebskraft $F_A$, die dem Gewicht des von ihm verdrängten Gases entspricht:
    \begin{equation}\label{eq: auftrieb_gase}
        F_{\text{Auf}} = \rho_{\text{Gas}} \cdot V_{\text{Körper}} \cdot g = m_{\text{verdr. Gas}} \cdot g = F_{\text{Gew}} \mDot
    \end{equation}
    Die Auftriebskraft wirkt jedoch der Gewichtskraft des Körpers entgegen, $\ivecS{F}{\text{Gew}} = -\ivecS{F}{\text{Auf}}$.
\end{rememberbox}
Für Gase ist die Dichte $\rho_{\text{Gas}}$ eigentlich nicht konstant über die Höhe des Körpers. In der Praxis ist die Variation jedoch meist vernachlässigbar klein, weshalb \cref{eq: auftrieb_gase} praktisch exakt ist.



\section{Kinetische Gastheorie}\label{sec: kinetische_gastheorie}
Die kinetische Gastheorie, entwickelt in der zweiten Hälfte des 19. Jahrhunderts unter anderem von James Clerk Maxwell, Ludwig Boltzmann und Rudolf Clausius, erklärt die makroskopischen Eigenschaften von Gasen (wie Druck und Temperatur) durch die Bewegung und die Wechselwirkung der Gasatome auf mikroskopischer Ebene.

Das einfachste Gasmodell ist das des idealen Gases, bei dem die Gasteilchen als punktförmige, nicht wechselwirkende Massen betrachtet werden, die sich mit statistisch verteilten Geschwindigkeiten bewegen und elastische Stöße ausführen.
\subsection{Das Modell des idealen Gases}\label{subsec: modell_ideales_gas}
Das Modell des idealen Gases basiert auf folgenden Annahmen:
\begin{itemize}
    \item Das Gas besteht aus einer großen Anzahl von Teilchen (Atomen/Molekülen), die als punktförmige Massen betrachtet werden können. Ihr Eigenvolumen ist vernachlässigbar gegenüber dem Gesamtvolumen.
    \item Die Teilchen bewegen sich zufällig (statistisch verteilt) in alle Richtungen.
    \item Zwischen den Teilchen gibt es keine anziehenden oder abstoßenden Kräfte, außer bei direkten Kollisionen.
    \item Alle Stöße der Teilchen untereinander und mit den Wänden des Behälters sind vollkommen elastisch, d.h., die kinetische Gesamtenergie bleibt erhalten.
\end{itemize}
\textit{Erklärendes Beispiel:} Um zu beurteilen, ob ein reales Gas als ideales Gas betrachtet werden kann, vergleichen wir den mittleren Abstand $\lrangle{r}$ der Gasteilchen mit ihrem effektiven Durchmesser $r_0$. Ist der mittlere Abstand deutlich größer als der Durchmesser ($r_0/\lrangle{r} \ll 1$), sind die Wechselwirkungen zwischen den Teilchen vernachlässigbar, und das Gas verhält sich ideal.
Bei einem Druck von \SI{1}{\barPr} und Raumtemperatur enthält \SI{1}{\centi\meter\cubed} eines Gases etwa $3\cdot 10^{19}$ Moleküle. Ihr mittlerer Abstand ist dann $\lrangle{r} \approx \SI{3}{\nano\meter}$. Für Heliumatome ist $r_0 \approx \SI{0.05}{\nano\meter}$, weshalb $r_0/\lrangle{r} \approx \qty{0.017} \ll 1$. Helium ist daher bei einem Druck von \SI{1}{\barPr} als ideales Gas zu betrachten.

\subsection{Herleitung des Gasdrucks}\label{subsec: herleitung_gasdruck}
\begin{figure}[htb]
    \centering
    \resizebox{0.5\linewidth}{!}{
    \begin{tikzpicture}[>=Latex, font=\large]
        % --- Definitionen für Konsistenz ---
        \definecolor{myblue}{RGB}{0, 102, 204} % Dunkelblau
        \def\wallHeight{3.0}
        \def\wallWidth{1.0}
        
        \def\xBall{2.2} % x-Position des Balls
        \def\yBall{2.0} % y-Position des Balls
        \def\factPos{0.75} % Faktor für die Position des ausgehenden Balls
        
        \def\vecLenX{1.3} % Länge der x-Vektorkomponente
        \def\vecLenY{\yBall*\vecLenX/\xBall} % Länge der y-Vektorkomponente

        % Koordinaten oben und unten mit gleichem Winkel
        \coordinate (O) at (0,0); % Aufprallpunkt
        \coordinate (BallIn) at (-\xBall, -\yBall); % Position Ball unten
        \coordinate (BallOut) at (-\factPos*\xBall, \factPos*\yBall); % Position Ball oben

        % 1. Wand (rechts)
        \fill[red!15] (0, -\wallHeight) rectangle (\wallWidth, \wallHeight);
        \draw[thick] (0, -\wallHeight) -- (0, \wallHeight);

        % 2. Hilfslinien für die Bahn (dünn)
        \draw[thin, gray] (BallIn) -- (O);
        \draw[thin, gray] (O) -- (BallOut);

        % 3. Impulsänderung Delta p (Pfeil nach rechts aus der Wand)
        \draw[->, thick] (O) -- (2.5, 0) node[right] {$\Delta p = 2mv_x$};

        % --- UNTEN: Eingehender Ball (Dunkelblau) ---
        \filldraw[fill=myblue, draw=black] (BallIn) circle (0.25);
        
        % Vektoren unten
        % vx (schwarz, horizontal)
        \draw[->, thick] (BallIn) -- ++(\vecLenX, 0) node[midway, below] {$v_x$};
        % vy (gestrichelt, vertikal)
        \draw[->, thick] (BallIn) ++(\vecLenX, 0) -- ++(0, \vecLenY) node[midway, right] {$v_y$};
        % v (rot, diagonal)
        \draw[->, red, thick] (BallIn) -- ++(\vecLenX, \vecLenY) node[midway, above left] {$\vec{v}$};

        % % --- OBEN: Ausgehender Ball (Hellblau, gestrichelt) ---
        \filldraw[fill=myblue!30, draw=black, dashed] (BallOut) circle (0.25);
        
        % Vektoren oben
        % vy (schwarz, vertikal nach oben)
        \draw[->, thick] (BallOut) -- ++(0, \vecLenY) node[midway, right] {$v_y$};
        % -vx (schwarz, horizontal nach links, startet an der Spitze von vy)
        \draw[->, thick] (BallOut) ++(0, \vecLenY) -- ++(-\vecLenX, 0) node[midway, above] {$-v_x$};
        % v (rot, diagonal nach links oben)
        \draw[->, red, thick] (BallOut) -- ++(-\vecLenX, \vecLenY) node[midway, below left] {$\vec{v}$};

        % 4. Textbeschriftung Mitte
        \node[below right, align=left] at (-3.25, 0.45) {$\Delta v_x = 2v_x$\\ $\Delta v_y = 0$};
        
        % 5. Koordinatenachsen x,y
                \draw[->, ultra thick] (2.5,1.5) -- (3.5, 1.5) node[right] {\Large $x$};
                \draw[->, ultra thick] (2.5,1.5) -- (2.5, 2.5) node[above] {\Large $y$};
    \end{tikzpicture}
    }
    \caption{Impulsübertragung beim elastischen Stoß auf eine Wand beträgt $\Delta p = 2mv_x$, wenn $v_x$ die Geschwindigkeitskomponente senkrecht zur Wand ist.}\label{fig: impulsuebertrag_wand}
\end{figure}

Der Druck, den ein Gas auf eine Wand ausübt, entsteht durch den Impulsübertrag der unzähligen Teilchen, die pro Zeiteinheit auf die Wand prallen.
In \cref{fig: impulsuebertrag_wand} betrachten wir ein Teilchen der Masse $m$, das sich mit der Geschwindigkeitskomponente $v_x$ auf eine Wand in der $(y,z)$-Ebene zubewegt. Beim elastischen Stoß kehrt sich die $x$-Komponente der Geschwindigkeit um ($v_x \rightarrow -v_x$), während die anderen Komponenten unverändert bleiben. Der Impulsübertrag auf die Wand entspricht der negativen Änderung des Impulses des Teilchens:
\begin{equation}
    \Delta p_{x, \text{Wand}} = -(p_{x,\text{nachher}} - p_{x,\text{vorher}}) = -(-mv_x - mv_x) = 2mv_x
\end{equation}

Der Impulsübertrag findet demnach immer durch die Normalkomponente der Geschwindigkeit zur Wand statt. Wir erinnern uns, dass der Impuls $\ivec{p} = m\cdot \ivec{v}$ ist und eine Kraft $\ivec{F}$ gleich der zeitlichen Änderung des Impulses ist:
\begin{equation}
    \ivec{F} = \frac{\dd \ivec{p}}{\dd t} \mDot
\end{equation}
 Um den Druck $p$ zu berechnen, betrachten wir ein kleines Flächenelement $A$ der Wand. Der Druck ist definiert als die Kraft pro Fläche:
\begin{equation}
    p = \frac{F}{A} = \frac{\dd \ivec{p}/\dd t}{A} = \frac{\dd}{\dd t} \left( \frac{\text{auf Fläche $A$ übertragener Impuls}}{\text{Fläche $A$}}\right)  \mDot 
\end{equation}
Die Schwierigkeit besteht darin, den gesamten Impuls zu bestimmen, der in der Zeit $\Delta t$ auf die Fläche $A$ übertragen wird, da viele Teilchen mit unterschiedlichen Geschwindigkeiten auf die Wand prallen. 
Wir betrachten daher zunächst das Volumen $dV$ in \cref{fig: teilchen_in_box}. Damit ein Teilchen innerhalb des Zeitintervalls $\dd t$ auf die rechte (graue) Wand treffen kann, muss es eine genügend große Geschwindigkeitskomponente $v_x$ haben -- die Komponenten $v_y$ und $v_z$ sind bezüglich des Abstands zur rechten Wand unerheblich. Fixieren wir die betrachtete Geschwindigkeit $v_x$, so erreichen nur jene Teilchen die Wand, die maximal einen Abstand $\dd s$ von der Wand haben:
\begin{equation}
    \dd s = v_x \cdot \dd t \mDot
\end{equation}
Dies definiert ein gedankliches Volumen $V_S$ (siehe \cref{fig: teilchen_in_box}), einen Quader mit der Grundfläche $A$ und der Länge $\dd s$:
\begin{equation}
    V_S = A \cdot \dd s = A \cdot v_x \cdot \dd t \mDot
\end{equation}
Alle Teilchen in diesem Volumen, die sich auf die Wand zubewegen, werden im Zeitintervall $\dd t$ mit ihr kollidieren. Sei $N$ die Gesamtzahl der Teilchen im Gesamtvolumen $V$ und $n = N/V$ die Teilchendichte. Die Anzahl der Teilchen im Volumen $V_S$ beträgt somit $n \cdot V_S$.
\begin{figure}[htb]
    \centering
    \includegraphics[width=0.6\textwidth]{Bilder/Kapitel_Waermelehre/particles_in_box.pdf}
    \caption{Im kleinen Volumenelement $V_S$ erreichen alle Teilchen mit Geschwindigkeit $v_x$ die Wand $A$ im Zeitintervall $\dd t$, weil die Seitenlänge des Quaders $\dd s=v_x \dd t$.}\label{fig: teilchen_in_box}
\end{figure}
Da sich die Teilchen jedoch statistisch ungeordnet bewegen (siehe \cref{sec: maxwell_boltzmann_verteilung}), bewegt sich im Durchschnitt nur die Hälfte der Teilchen in positive $x$-Richtung (auf die Wand zu), die andere Hälfte bewegt sich davon weg. Die Anzahl der stoßenden Teilchen $\dd N_{\text{Stoß}}$ ist daher:
\begin{equation}
    \dd N_{\text{Stoß}} = \frac{1}{2} \cdot n \cdot V_S = \frac{1}{2} n A v_x \dd t \mDot
\end{equation}
Der gesamte Impulsübertrag $\dd P$ auf die Wand in der Zeit $\dd t$ ergibt sich aus der Anzahl der Stöße multipliziert mit dem Impulsübertrag pro Stoß ($\Delta p = 2mv_x$):
\begin{equation}
    \dd P = \dd N_{\text{Stoß}} \cdot (2 m v_x) = \left( \frac{1}{2} n A v_x \dd t \right) \cdot 2 m v_x = n m v_x^2 A \dd t \mDot
\end{equation}
Die Kraft $F$, die auf die Wand wirkt, ist die zeitliche Änderung des Impulses $F = \dd P / \dd t$:
\begin{equation}
    F = n m v_x^2 A \mDot
\end{equation}
Daraus folgt, dass die Teilchen mit Geschwindigkeit $v_x$ den Druckanteil ($p = F/A$)
\begin{equation}
    p = n m v_x^2
\end{equation}
ausüben. In einem realen Gas haben nicht alle Teilchen dieselbe Geschwindigkeit $v_x$. Wir müssen daher das mittlere Geschwindigkeitsquadrat $\overline{v_x^2}$ betrachten. Zudem bewegen sich die Teilchen im dreidimensionalen Raum. Aufgrund der \textit{Isotropie des Raumes} (keine Raumrichtung ist ausgezeichnet) muss das mittlere Geschwindigkeitsquadrat in alle drei Richtungen gleich groß sein:
\begin{equation}
    \overline{v_x^2} = \overline{v_y^2} = \overline{v_z^2} \mDot
\end{equation}
Das Quadrat der Gesamtgeschwindigkeit $\ivec{v}$ setzt sich nach dem Satz des Pythagoras aus den Komponenten zusammen: $\overline{v^2} = \overline{v_x^2} + \overline{v_y^2} + \overline{v_z^2} = 3 \overline{v_x^2}$.
Daraus folgt für die $x$-Komponente:
\begin{equation}
    \overline{v_x^2} = \frac{1}{3} \overline{v^2} \mDot
\end{equation}
Setzen wir dies in den Ausdruck für den Druck ein, erhalten wir die Grundgleichung der kinetischen Gastheorie.

\begin{importantbox}{Grundgleichung der kinetischen Gastheorie}
    Der Druck eines idealen Gases ergibt sich aus der Teilchendichte $n=N/V$, der Masse $m$ eines Teilchens und dem mittleren Geschwindigkeitsquadrat $\overline{v^2}$:
    \begin{equation}\label{eq: grundglg_kinetische_gastheorie}
        p = \frac{1}{3} n m \overline{v^2} \mDot
    \end{equation}
\end{importantbox}

\subsubsection{Interpretation als Energiedichte}
Die Gleichung lässt sich physikalisch sehr anschaulich interpretieren, wenn man sie leicht umformt und die mittlere kinetische Energie eines Teilchens $\overline{\Ekin} = \frac{1}{2}m\overline{v^2}$ einführt:
\begin{equation}
    p = \frac{1}{3} n m \overline{v^2} = \frac{2}{3} n \cdot \left( \frac{1}{2} m \overline{v^2} \right) = \frac{2}{3} \cdot \frac{N}{V} \cdot \overline{\Ekin} \mDot
\end{equation}
Bringt man das Volumen auf die andere Seite, erhält man
\begin{equation}\label{eq: druck_energie_beziehung}
    p \cdot V = \frac{2}{3} N \cdot \overline{\Ekin} \mDot
\end{equation}
Vergleicht man dieses Ergebnis mit der idealen Gasgleichung aus der Thermodynamik, $p \cdot V = N \cdot \kB \cdot T$, so findet man den direkten mikroskopischen Zugang zum Begriff der Temperatur.

\begin{rememberbox}{Mikroskopische Deutung der Temperatur}
    Durch Gleichsetzen von \cref{eq: druck_energie_beziehung} und der idealen Gasgleichung folgt:
    \begin{equation}
        \frac{2}{3} N \overline{\Ekin} = N \kB T \quad \implies \quad \overline{\Ekin} = \frac{3}{2} \kB T \mDot
    \end{equation}
    Die absolute Temperatur $T$ ist also ein Maß für die mittlere kinetische Energie der Gasteilchen.
\end{rememberbox}


\section{Maxwell-Boltzmann-Verteilung}\label{sec: maxwell_boltzmann_verteilung}
Im vorangegangenen Abschnitt haben wir den Gasdruck mithilfe des mittleren Geschwindigkeitsquadrats $\overline{v^2}$ hergeleitet. Dies könnte den Eindruck erwecken, dass alle Gasteilchen mit derselben Geschwindigkeit unterwegs sind.
In der Realität herrscht im Mikrokosmos jedoch ein ständiges Chaos: Die Gasteilchen stoßen permanent miteinander und tauschen dabei kinetische Energie aus. Ein Teilchen kann durch einen zufälligen Stoß stark beschleunigt werden, während ein anderes fast bis zum Stillstand abgebremst wird.

Selbst im thermischen Gleichgewicht, wo makroskopische Größen wie $p$ und $T$ konstant sind, ändern sich die Geschwindigkeiten der einzelnen Teilchen fortwährend.
Betrachtet man jedoch eine sehr große Anzahl von Teilchen ($N_A \approx \SI{6.022e23}{\mole^{-1}}$), so stellt sich eine zeitlich konstante \textbf{Geschwindigkeitsverteilung} ein.

\subsection{Die Verteilungsfunktion}\label{subsec: verteilungsfunktion}
Die Wahrscheinlichkeit, ein Teilchen mit einer bestimmten Geschwindigkeit anzutreffen, wird durch die \textbf{Maxwell-Boltzmann-Verteilung} $f(v)$ beschrieben.
Genauer gesagt gibt die Größe $f(v)\,\dd v$ den Bruchteil der Teilchen an, deren Geschwindigkeitsbetrag im Intervall $[v, v+\dd v]$ liegt (siehe \cref{fig: f-v-dv_verteilung_anschauung}).

\begin{figure}[tb]
    \centering
    \resizebox{0.55\linewidth}{!}{
        \begin{tikzpicture}[
                >=Latex, 
                font=\large,
                declare function={
                    gauss(\x) = 3.5 * exp(-0.22 * \x^2);
                }
            ]
                \def\vpos{1.1}       
                \def\dv{0.6}         
                \def\vend{\vpos+\dv} 

                % 1. Fläche füllen
                \fill[red!15] (\vpos, 0) -- (\vpos, {gauss(\vpos)})
                    -- plot[domain=\vpos:\vend, samples=50] (\x, {gauss(\x)})
                    -- (\vend, 0) -- cycle;
                % 2. Achsen
                \draw[->, thick] (0,0) -- (6,0) node[below] {\Large $v$};
                \draw[->, thick] (0,0) -- (0,4) node[left] {\large $f(v)$};
                \node[below left] at (0,-0.1) {$0$};
                % 3. Graph
                \draw[red!85!black, ultra thick, samples=100, domain=0:5.5] 
                    plot (\x, {gauss(\x)});
                % 4. Begrenzungslinien
                \draw[thick] (\vpos, 0) -- (\vpos, {gauss(\vpos)});
                \draw[thick] (\vend, 0) -- (\vend, {gauss(\vend)});

                % 5. Beschriftung x-Achse
                \node[below=6pt] at (\vpos, 0) {$v$};
                \node[below right=4pt] at (\vend-0.25, 0) {$v + \dd v$};

                % 6. Pfeile für dv
                \draw[->] (\vpos - 0.8, 0.6) -- (\vpos, 0.6) node[midway, above] {$\dd v$};
                \draw[->] (\vend + 0.8, 0.6) -- (\vend, 0.6); 

                % 7. Beschriftung Fläche
                \draw[-, thick] (\vpos + 0.45*\dv, {gauss(\vpos)*0.6}) -- ++(0.5, 1.7) node[above right] {$f(v)\,\dd v$};
        \end{tikzpicture}
}
    \caption{Die schraffierte Fläche $f(v)\dd v$ unter der Kurve entspricht dem Anteil der Moleküle mit Geschwindigkeitsbeträgen im Intervall zwischen $v$ und $v+\dd v$.}\label{fig: f-v-dv_verteilung_anschauung}
\end{figure}

Die analytische Form dieser Verteilung für den Betrag der Geschwindigkeit $v$ lautet:
\begin{equation}\label{eq:maxwell_boltzmann}
    f(v) = \underbrace{4\pi \left(\frac{m}{2\pi k_B T}\right)^{3/2}}_{\text{Normierung}} \cdot \underbrace{v^2}_{\text{Geometrie}} \cdot \underbrace{e^{-\frac{mv^2}{2k_B T}}}_{\text{Boltzmann-Faktor}} \mDot
\end{equation}
Der Verlauf der Kurve in \cref{fig: f-v-dv_verteilung_anschauung} ergibt sich aus dem Produkt zweier konkurrierender Terme:
\begin{itemize}
    \item \textbf{Der geometrische Anteil ($v^2$):} Dieser Term nimmt quadratisch zu. Er berücksichtigt, dass es mit zunehmendem Geschwindigkeitsbetrag rein geometrisch mehr Möglichkeiten für die Geschwindigkeitsvektoren gibt (das Volumen einer Kugelschale im Geschwindigkeitsraum wächst mit $4\pi v^2$).
    \item \textbf{Der Boltzmann-Faktor ($e^{-E_{kin}/k_B T}$):} Dieser Term fällt exponentiell ab. Er drückt aus, dass Zustände mit sehr hoher Energie extrem unwahrscheinlich sind.
\end{itemize}
Das Produkt dieser Terme führt dazu, dass die Kurve bei $v=0$ beginnt, ein Maximum durchläuft (die wahrscheinlichste Geschwindigkeit) und für sehr große Geschwindigkeiten gegen Null geht.

\subsection{Einfluss der Temperatur}
Die Form der Verteilung hängt stark von der Temperatur $T$ und der Teilchenmasse $m$ ab.
Erhöht man die Temperatur, steht mehr thermische Energie zur Verfügung. Das Maximum der Kurve verschiebt sich zu höheren Geschwindigkeiten und die Verteilung wird breiter und flacher. Die Fläche unter der Kurve bleibt dabei stets 1 (da die Summe aller Wahrscheinlichkeiten 100\,\% ergeben muss).


\subsection{Charakteristische Geschwindigkeiten}
Da die Verteilung nicht symmetrisch ist, fallen der häufigste Wert und der Mittelwert nicht zusammen. Wir unterscheiden drei wichtige Kennzahlen:

\begin{enumerate}
    \item \textbf{Die wahrscheinlichste Geschwindigkeit $v_W$}:
    Sie entspricht dem Maximum der Kurve (dort, wo die Ableitung $f'(v)=0$ ist). Die meisten Teilchen besitzen eine Geschwindigkeit nahe diesem Wert.
    \begin{equation}
        v_W = \sqrt{\frac{2k_B T}{m}}
    \end{equation}
    
    \item \textbf{Die mittlere Geschwindigkeit $\overline{v}$}:
    Dies ist das arithmetische Mittel der Geschwindigkeitsbeträge.
    \begin{equation}
        \overline{v} = \sqrt{\frac{8k_B T}{\pi m}} \approx 1{,}13 \, v_W
    \end{equation}
    
    \item \textbf{Die mittlere quadratische Geschwindigkeit $v_{\text{rms}}$}:
    Die \textit{root-mean-square} velocity ist entscheidend für die kinetische Energie ($\overline{E_{\text{kin}}} \propto \overline{v^2}$). Sie ist etwas größer als die mittlere Geschwindigkeit, da schnelle Teilchen quadratisch stärker ins Gewicht fallen.
    \begin{equation}
        v_{\text{rms}} = \sqrt{\overline{v^2}} = \sqrt{\frac{3k_B T}{m}} \approx 1{,}22 \, v_W
    \end{equation}
\end{enumerate}

\begin{examplebox}{Beispiel: Stickstoff in der Luft}
    Für Stickstoffmoleküle ($N_2$) mit $m \approx \SI{4.67e-26}{\kilo\gram}$ bei Raumtemperatur ($T = \SI{300}{\kelvin}$) ergeben sich folgende Werte:
    \begin{itemize}
        \item $v_W \approx \SI{422}{\meter\per\second}$
        \item $\overline{v} \approx \SI{476}{\meter\per\second}$
        \item $\sqrt{\overline{v^2}} \approx \SI{517}{\meter\per\second}$
    \end{itemize}
    Diese Geschwindigkeiten sind bemerkenswert hoch und liegen im Bereich der Schallgeschwindigkeit.
\end{examplebox}


\section{OLD: Maxwell-Boltzmann-Verteilung}\label{sec: maxwell_boltzmann_verteilung}
\begin{figure}[htb]
    \centering
    \resizebox{0.5\linewidth}{!}{
        \begin{tikzpicture}[
                >=Latex, 
                font=\large,
                % Definition der Funktion
                declare function={
                    gauss(\x) = 3.5 * exp(-0.22 * \x^2);
                }
            ]
                % Parameter für den Streifen
                \def\vpos{1.1}       
                \def\dv{0.6}         
                \def\vend{\vpos+\dv} 

                % 1. Fläche füllen
                \fill[red!15] (\vpos, 0) -- (\vpos, {gauss(\vpos)})
                    -- plot[domain=\vpos:\vend, samples=50] (\x, {gauss(\x)})
                    -- (\vend, 0) -- cycle;

                % 2. Achsen
                \draw[->, thick] (0,0) -- (6,0) node[below] {\Large $v$};
                \draw[->, thick] (0,0) -- (0,4) node[left] {\large $f(v)$};
                \node[below left] at (0,-0.1) {$0$};

                % 3. Graph
                \draw[red!85!black, ultra thick, samples=100, domain=0:5.5] 
                    plot (\x, {gauss(\x)});

                % 4. Begrenzungslinien
                \draw[thick] (\vpos, 0) -- (\vpos, {gauss(\vpos)});
                \draw[thick] (\vend, 0) -- (\vend, {gauss(\vend)});

                % 5. Beschriftung x-Achse
                % Align the labels at the same vertical level using baseline and positioning
                \node[below=6pt] at (\vpos, 0) {$v$};
                \node[below right=4pt] at (\vend-0.25, 0) {$v + \dd v$};

                % 6. Pfeile für dv
                \draw[->] (\vpos - 0.8, 0.6) -- (\vpos, 0.6) node[midway, above] {$\dd v$};
                \draw[->] (\vend + 0.8, 0.6) -- (\vend, 0.6); 

                % 7. Beschriftung Fläche
                \draw[-, thick] (\vpos + 0.45*\dv, {gauss(\vpos)*0.6}) -- ++(0.5, 1.7) node[above right] {$f(v)\,\dd v$};
        \end{tikzpicture}
}
    \caption{Die Fläche $f(v)\dd v$ gibt die Anzahl der Moleküle mit Geschwindigkeitsbeträgen zwischen $v$ und $v+\dd v$ an.}\label{fig: f-v-dv_verteilung_anschauung}
\end{figure}
In einem Gas haben nicht alle Teilchen die gleiche Geschwindigkeit. Ihre Geschwindigkeiten sind statistisch verteilt. Die \textbf{Maxwell-Boltzmann-Verteilungsfunktion} $f(v)$ beschreibt, welcher Anteil der Moleküle eine Geschwindigkeit im Intervall zwischen $v$ und $v+dv$ besitzt.
Die Verteilungsfunktion lautet:
\begin{equation}\label{eq:maxwell_boltzmann}
    f(v) = 4\pi \left(\frac{m}{2\pi k_B T}\right)^{3/2} v^2 e^{-\frac{mv^2}{2k_B T}}
\end{equation}

% \begin{figure}[htb]
%     \centering
%     \includegraphics[width=0.7\linewidth]{Bilder/Kapitel_Waermelehre/maxwell_boltzmann_verteilung.png}
%     \caption{Maxwell-Boltzmann-Verteilung für Wasserstoff (H) und Helium (He) bei verschiedenen Temperaturen. Mit steigender Temperatur wird die Verteilung breiter und verschiebt sich zu höheren Geschwindigkeiten. Schwerere Teilchen (He) sind bei gleicher Temperatur im Mittel langsamer.}
%     \label{fig:maxwell_boltzmann_verteilung}
% \end{figure}

Aus dieser Verteilung lassen sich verschiedene charakteristische Geschwindigkeiten berechnen:
\begin{itemize}
    \item \textbf{Die wahrscheinlichste Geschwindigkeit $v_W$}: Die Geschwindigkeit am Maximum der Verteilung.
    \begin{equation}
        v_W = \sqrt{\frac{2k_B T}{m}}
    \end{equation}
    \item \textbf{Die mittlere Geschwindigkeit $\overline{v}$}: Der Durchschnittswert aller Geschwindigkeiten.
    \begin{equation}
        \overline{v} = \int_0^\infty v f(v) dv = \sqrt{\frac{8k_B T}{\pi m}} \approx 1.13 \, v_W
    \end{equation}
    \item \textbf{Die mittlere quadratische Geschwindigkeit $\sqrt{\overline{v^2}}$} (rms-Geschwindigkeit): Die Wurzel aus dem mittleren Geschwindigkeitsquadrat, die direkt mit der kinetischen Energie zusammenhängt.
    \begin{equation}
        \sqrt{\overline{v^2}} = \left(\int_0^\infty v^2 f(v) dv\right)^{1/2} = \sqrt{\frac{3k_B T}{m}} \approx 1.22 \, v_W
    \end{equation}
\end{itemize}

\begin{examplebox}{Beispiel: Stickstoff in der Luft}
    Für Stickstoffmoleküle ($N_2$) mit $m \approx \SI{4.67e-26}{\kilo\gram}$ bei Raumtemperatur ($T = \SI{300}{\kelvin}$) ergeben sich folgende Werte:
    \begin{itemize}
        \item $v_W \approx \SI{422}{\meter\per\second}$
        \item $\overline{v} \approx \SI{476}{\meter\per\second}$
        \item $\sqrt{\overline{v^2}} \approx \SI{517}{\meter\per\second}$
    \end{itemize}
    Diese Geschwindigkeiten sind bemerkenswert hoch und liegen im Bereich der Schallgeschwindigkeit.
\end{examplebox}

\section{Wärmemenge und Wärmekapazität}\label{sec: waermemenge_waermekapazitaet}
Führt man einem Körper Energie zu, beispielsweise durch mechanische Arbeit $\Delta W$, so steigt in der Regel seine Temperatur um einen Betrag $\Delta T$. Die Energie, die zu einer Temperaturänderung führt, wird als \textbf{Wärmemenge} $\Delta Q$ bezeichnet. Für diesen Prozess gilt der Energieerhaltungssatz:
\begin{equation}\label{eq:waermemenge_def}
    \Delta Q = c \cdot m \cdot \Delta T \mComma
\end{equation}
wobei $m$ die Masse des Körpers ist. Der Proportionalitätsfaktor $c$ ist eine Materialkonstante und wird \textbf{spezifische Wärmekapazität} genannt. Sie gibt an, wie viel Energie notwendig ist, um die Temperatur von \SI{1}{\kilo\gram} eines Stoffes um \SI{1}{\kelvin} (oder \SI{1}{\celsius}) zu erhöhen. Ihre Einheit ist $[\si{\joule\per\kilogram\per\kelvin}]$.

Das Produkt aus Masse und spezifischer Wärmekapazität wird als \textbf{Wärmekapazität} $C$ des gesamten Körpers bezeichnet:
\begin{equation}
    C = m \cdot c \mDot
\end{equation}

\begin{rememberbox}{Die Einheit Kalorie}
    Historisch wurde für die Wärmemenge oft die Einheit \textbf{Kalorie} (\si{\calorie}) verwendet. Eine Kalorie ist definiert als die Wärmemenge, die benötigt wird, um \SI{1}{\gram} Wasser von \SI{15}{\celsius} auf \SI{16}{\celsius} zu erwärmen. Heute ist die SI-Einheit für Energie, das Joule, gebräuchlich. Der Umrechnungsfaktor beträgt experimentell:
    \begin{equation}
        \SI{1}{\calorie} = \SI{4.1868}{\joule} \mDot
    \end{equation}
\end{rememberbox}

\subsection{Molare Größen und die Gasgleichung}\label{subsec: molare_groessen}
In der Thermodynamik, insbesondere bei Gasen, ist es oft praktischer, Stoffmengen in \textbf{Mol} anzugeben.
\begin{itemize}
    \item \textbf{Ein Mol} ist die Stoffmenge, die genauso viele Teilchen enthält, wie Atome in \SI{12}{\gram} des Kohlenstoff-Isotops ${}^{12}C$ enthalten sind.
    \item Diese Anzahl an Teilchen pro Mol ist eine universelle Konstante, die \textbf{Avogadro-Konstante} $N_A$:
    \begin{equation}
        N_A = \SI{6.022e23}{\per\mol}
    \end{equation}
    \item Das \textbf{Molvolumen} $V_M$ ist das Volumen, das ein Mol eines Stoffes einnimmt. Für ideale Gase unter Normbedingungen ($p=\SI{1}{bar}$, $T=\SI{0}{\celsius}$) beträgt es $V_M = \SI{22.7}{\deci\meter\cubed}$.
\end{itemize}

Die ideale Gasgleichung (\cref{eq: ideale_gasgleichung}) lässt sich auch in molarer Form schreiben. Für $v$ Mole eines Gases mit Gesamtvolumen $V = v \cdot V_M$ gilt:
\begin{equation}\label{eq:ideale_gasgleichung_molar}
    p \cdot V = v \cdot R \cdot T \mComma
\end{equation}
wobei $R$ die \textbf{allgemeine Gaskonstante} ist. Sie ist das Produkt aus der Boltzmann-Konstante und der Avogadro-Konstante:
\begin{equation}
    R = N_A \cdot k_B \approx \SI{8.314}{\joule\per\mol\per\kelvin} \mDot
\end{equation}

\subsection{Wärmekapazität von Gasen}\label{subsec: waermekapazitaet_gase}
Auch für Gase kann die zugeführte Wärmemenge über die Anzahl der Mole $v$ und die \textbf{molare Wärmekapazität} $C_m$ ausgedrückt werden:
\begin{equation}
    \Delta Q = v \cdot C_m \cdot \Delta T \mDot
\end{equation}
Die molare Wärmekapazität hängt davon ab, welche Zustandsgröße während der Wärmezufuhr konstant gehalten wird. Für ideale Gase unterscheidet man:
\begin{itemize}
    \item \textbf{Molare Wärmekapazität bei konstantem Volumen ($C_V$):} Die gesamte zugeführte Wärme erhöht nur die innere Energie.
    \begin{equation}
        C_V = \frac{f}{2}R
    \end{equation}
    \item \textbf{Molare Wärmekapazität bei konstantem Druck ($C_p$):} Ein Teil der Wärme wird für die Expansionsarbeit gegen den äußeren Druck aufgewendet.
    \begin{equation}
        C_p = C_V + R = \frac{f+2}{2}R
    \end{equation}
\end{itemize}
Hier ist $f$ die Anzahl der \textbf{Freiheitsgrade} des Gasmoleküls, welche sich aus Translation, Rotation und Vibration zusammensetzen kann ($f=f_\text{trans} + f_\text{rot} + f_\text{vib}$).

\begin{figure}[htb]
    \centering
    % \includegraphics[width=0.6\linewidth]{Bilder/Kapitel_Waermelehre/Cv_gase.png}
    \includegraphics[width=0.6\linewidth]{Bilder/Allgemein/placeholder.png}
    \caption{Die molare Wärmekapazität $C_V$ (normiert auf $R$) für verschiedene Gase als Funktion der Temperatur. Man erkennt, dass mit steigender Temperatur mehr Freiheitsgrade (Rotation, Vibration) angeregt werden, was zu einer Erhöhung von $C_V$ führt. Helium (He), als einatomiges Gas, hat konstant $f=3$ Freiheitsgrade.}
    \label{fig:cv_gase}
\end{figure}

\subsection{Innere Energie}\label{subsec: innere_energie}
Die \textbf{innere Energie} $U$ eines Systems ist die Summe aller Energien seiner Teilchen – also die Summe aus kinetischer Energie (Translation, Rotation, Vibration) und potenzieller Energie durch intermolekulare Kräfte. Für ein ideales Gas mit $N$ Teilchen und $f$ Freiheitsgraden, bei dem die potenzielle Energie vernachlässigt wird, ist die innere Energie nur von der Temperatur abhängig:
\begin{equation}
    U = N \cdot \overline{E_\mathrm{kin}} = N \cdot \frac{f}{2} k_B T = v \cdot \frac{f}{2} R T \mDot
\end{equation}
Im thermischen Gleichgewicht verteilt sich die Energie gleichmäßig auf alle Freiheitsgrade (Gleichverteilungssatz).

\subsection{Wärmekapazität von Festkörpern}\label{subsec: waermekapazitaet_festkoerper}
In Festkörpern sind die Atome in einem Gitter gebunden. Sie können nicht translatieren oder rotieren, sondern nur um ihre Gleichgewichtslage schwingen. Jedes Atom hat 3 Schwingungsfreiheitsgrade, und jede Schwingung trägt sowohl kinetische als auch potenzielle Energie bei, was zu insgesamt $f=6$ Freiheitsgraden pro Atom führt. Nach dem Gleichverteilungssatz wäre die molare Wärmekapazität somit konstant:
\begin{equation}\label{eq:dulong_petit}
    C_V = \frac{6}{2} R = 3R \approx \SI{25}{\joule\per\mol\per\kelvin} \mDot
\end{equation}
Dieses Ergebnis ist als das \textbf{Gesetz von Dulong-Petit} bekannt und gilt für viele Festkörper bei Raumtemperatur recht gut. Bei tiefen Temperaturen werden die Schwingungsmoden jedoch "eingefroren", und die Wärmekapazität sinkt gegen null, was nur quantenmechanisch erklärt werden kann.

\begin{figure}[htb]
    \centering
    % \includegraphics[width=0.5\linewidth]{Bilder/Kapitel_Waermelehre/Cv_festkoerper.png}
    \includegraphics[width=0.5\linewidth]{Bilder/Allgemein/placeholder.png}
    \caption{Temperaturabhängigkeit der molaren Wärmekapazität $C_V$ für die Festkörper Blei (Pb), Kupfer (Cu) und Kohlenstoff (C). Bei hohen Temperaturen nähern sich die Werte dem Grenzwert des Dulong-Petit-Gesetzes von $C_V/R = 3$ an.}
    \label{fig:cv_festkoerper}
\end{figure}

\section{Phasenübergänge}\label{sec: phasenuebergaenge}
Die Zustände fest, flüssig und gasförmig werden als \textbf{Aggregatzustände} oder \textbf{Phasen} eines Stoffes bezeichnet. Die Übergänge zwischen diesen Phasen haben spezifische Namen, wie Schmelzen, Gefrieren, Verdampfen, Kondensieren, Sublimieren und Resublimieren.

\begin{figure}[htb]
    \centering
    % \includegraphics[width=0.5\linewidth]{Bilder/Kapitel_Waermelehre/phasenuebergaenge.png}
    \includegraphics[width=0.5\linewidth]{Bilder/Allgemein/placeholder.png}
    \caption{Die drei Aggregatzustände und die Übergänge zwischen ihnen.}
    \label{fig:phasenuebergaenge}
\end{figure}

\subsection{Schmelz- und Verdampfungswärme}\label{subsec: schmelz_verdampfungswaerme}
Führt man einem Stoff bei einem Phasenübergang kontinuierlich Wärme zu, bleibt seine Temperatur konstant, bis die gesamte Substanz ihre Phase geändert hat. Die zugeführte Energie wird nicht zur Erhöhung der kinetischen Energie der Teilchen (Temperaturerhöhung), sondern zur Änderung ihrer potenziellen Energie (Aufbrechen von Bindungen) verwendet. Diese Energie wird als \textbf{latente Wärme} bezeichnet.

\begin{itemize}
    \item Die \textbf{spezifische Schmelzwärme} $\lambda_s$ ist die Energiemenge, die benötigt wird, um \SI{1}{\kilo\gram} eines festen Stoffes bei konstanter Schmelztemperatur zu verflüssigen.
    \item Die \textbf{spezifische Verdampfungswärme} $\lambda_v$ ist die Energiemenge, die benötigt wird, um \SI{1}{\kilo\gram} einer Flüssigkeit bei konstanter Siedetemperatur zu verdampfen.
\end{itemize}

\begin{figure}[htb]
    \centering
    % \includegraphics[width=0.7\linewidth]{Bilder/Kapitel_Waermelehre/erwaermungskurve_wasser.png}
    \includegraphics[width=0.7\linewidth]{Bilder/Allgemein/placeholder.png}
    \caption{Temperaturverlauf beim Erwärmen von Eis mit konstanter Wärmezufuhr. Während der Phasenübergänge (Schmelzen bei \SI{0}{\celsius}, Verdampfen bei \SI{100}{\celsius}) bleibt die Temperatur konstant, da die zugeführte Energie als latente Wärme zum Aufbrechen der Bindungen im Gitter bzw. in der Flüssigkeit benötigt wird.}
    \label{fig:erwaermungskurve_wasser}
\end{figure}

\section{Reale Gase und Phasendiagramme}\label{sec: reale_gase_phasendiagramme}
Das ideale Gasmodell vernachlässigt sowohl das Eigenvolumen der Teilchen als auch die Anziehungskräfte zwischen ihnen. Bei hohem Druck und niedriger Temperatur, wenn die Teilchen nahe beieinander sind, werden diese Effekte jedoch relevant und das Verhalten realer Gase weicht von der idealen Gasgleichung ab.

\subsection{Die Van-der-Waals-Gleichung}\label{subsec: van_der_waals}
Eine verbesserte Zustandsgleichung für reale Gase ist die \textbf{Van-der-Waals-Gleichung}:
\begin{equation}\label{eq:van_der_waals}
    \left( p + a \frac{v^2}{V^2} \right) (V - v b) = v R T
\end{equation}
Hierbei korrigiert der Term $a$ den Druckabfall durch die Anziehungskräfte zwischen den Teilchen (Kohäsionsdruck), und der Term $b$ berücksichtigt das von den Teilchen selbst eingenommene Volumen (Kovolumen).

\begin{figure}[htb]
    \centering
    % \includegraphics[width=0.6\linewidth]{Bilder/Kapitel_Waermelehre/vanderwaals_isothermen.png}
    \includegraphics[width=0.6\linewidth]{Bilder/Allgemein/placeholder.png}
    \caption{Van-der-Waals-Isothermen für $\text{CO}_2$. Unterhalb einer kritischen Temperatur zeigen die Isothermen einen Bereich, in dem Gas und Flüssigkeit koexistieren. Komprimiert man das Gas entlang der $T=\SI{273}{\kelvin}$-Isotherme, beginnt es bei Punkt A zu kondensieren. Entlang der Linie A-C bleibt der Druck konstant, während der Flüssigkeitsanteil zunimmt, bis bei C das gesamte Gas verflüssigt ist.}
    \label{fig:vanderwaals_isothermen}
\end{figure}

\subsection{Phasendiagramme}\label{subsec: phasendiagramme}
Ein \textbf{Phasendiagramm} stellt in einem Druck-Temperatur-Diagramm (p-T-Diagramm) die Existenzbereiche der verschiedenen Phasen eines Stoffes dar.
\begin{itemize}
    \item Die \textbf{Dampfdruckkurve} trennt die flüssige und die gasförmige Phase.
    \item Die \textbf{Schmelzkurve} trennt die feste und die flüssige Phase.
    \item Die \textbf{Sublimationskurve} trennt die feste und die gasförmige Phase.
\end{itemize}
Der Schnittpunkt der drei Kurven ist der \textbf{Tripelpunkt}, an dem alle drei Phasen im Gleichgewicht koexistieren können.

\begin{figure}[htb]
    \centering
    \begin{minipage}{0.48\linewidth}
        \centering
        % \includegraphics[width=\linewidth]{Bilder/Kapitel_Waermelehre/phasendiagramm_co2.png}
        \includegraphics[width=\linewidth]{Bilder/Allgemein/placeholder.png}
        \caption{Phasendiagramm von $\text{CO}_2$. Die Schmelzkurve hat eine positive Steigung, was bedeutet, dass der Schmelzpunkt mit steigendem Druck zunimmt.}
        \label{fig:phasendiagramm_co2}
    \end{minipage}\hfill
    \begin{minipage}{0.48\linewidth}
        \centering
        % \includegraphics[width=\linewidth]{Bilder/Kapitel_Waermelehre/phasendiagramm_h2o.png}
        \includegraphics[width=\linewidth]{Bilder/Allgemein/placeholder.png}
        \caption{Phasendiagramm von Wasser ($\text{H}_2\text{O}$). Die Schmelzkurve hat eine negative Steigung. Dies ist die \textbf{Anomalie des Wassers}: Eis hat eine geringere Dichte als flüssiges Wasser, weshalb es schwimmt und unter Druck schmelzen kann.}
        \label{fig:phasendiagramm_h2o}
    \end{minipage}
\end{figure}

\chapter{Die Hauptsätze der Thermodynamik}\label{chap: Hauptsaetze_Thermodynamik}
Die Hauptsätze der Thermodynamik sind fundamentale Prinzipien, die den Umgang mit Energie, insbesondere Wärme, und die Richtung thermodynamischer Prozesse beschreiben.

\section{Grundbegriffe und der Erste Hauptsatz}\label{sec: erster_hauptsatz}
Ein thermodynamisches System wird durch seine \textbf{Zustandsgrößen} wie Druck $p$, Volumen $V$, Temperatur $T$ und die innere Energie $U$ charakterisiert. Der \textbf{Erste Hauptsatz der Thermodynamik} ist eine Formulierung des Energieerhaltungssatzes. Er besagt, dass die Änderung der inneren Energie $\Delta U$ eines Systems gleich der Summe der ihm zugeführten Wärme $\Delta Q$ und der an ihm verrichteten Arbeit $\Delta W$ ist:
\begin{equation}\label{eq:erster_hauptsatz}
    \Delta U = \Delta Q + \Delta W \mDot
\end{equation}
In differentieller Form schreibt man:
\begin{equation}
    dU = dQ + dW \mDot
\end{equation}
Die Arbeit, die mit einer Volumenänderung verbunden ist, nennt man Volumenarbeit. Komprimiert man ein Gas (Volumenänderung $dV < 0$), wird am System Arbeit verrichtet ($dW > 0$). Expandiert es ($dV > 0$), verrichtet das System Arbeit ($dW < 0$). Es gilt:
\begin{equation}\label{eq:volumenarbeit}
    dW = -p_{ext} \cdot dV
\end{equation}
Damit lautet der erste Hauptsatz für Prozesse, die nur Volumenarbeit beinhalten:
\begin{equation}
    dU = dQ - p \cdot dV \mDot
\end{equation}

\subsection{Spezielle Zustandsänderungen}\label{subsec: spezielle_zustaende}
\begin{itemize}
    \item \textbf{Isochorer Prozess ($V = \const$):} Da $dV=0$, wird keine Volumenarbeit verrichtet ($dW=0$). Die gesamte zugeführte Wärme geht in die innere Energie: $dQ = dU$.
    \item \textbf{Isobarer Prozess ($p = \const$):} Die zugeführte Wärme erhöht sowohl die innere Energie als auch das Volumen: $dQ = dU + p \cdot dV$. Man führt die Zustandsgröße \textbf{Enthalpie} $H = U + pV$ ein, für die bei isobaren Prozessen gilt: $dH = dQ$.
    \item \textbf{Isothermer Prozess ($T = \const$):} Für ein ideales Gas hängt die innere Energie nur von der Temperatur ab, also ist $dU=0$. Der erste Hauptsatz wird zu $dQ = -dW = p \cdot dV$. Die gesamte zugeführte Wärme wird in Arbeit umgewandelt.
    \item \textbf{Adiabatischer Prozess ($Q = \const$):} Es wird keine Wärme mit der Umgebung ausgetauscht ($dQ=0$). Der erste Hauptsatz lautet $dU = dW = -p \cdot dV$. Jede Arbeit, die das System leistet, geht zulasten seiner inneren Energie, was zu einer Abkühlung führt.
\end{itemize}

\begin{figure}[htb]
    \centering
    % \includegraphics[width=0.7\linewidth]{Bilder/Kapitel_Hauptsaetze/pv_diagramm_prozesse.png}
    \includegraphics[width=0.7\linewidth]{Bilder/Allgemein/placeholder.png}
    \caption{Darstellung der vier speziellen Zustandsänderungen in einem p-V-Diagramm. Isothermen ($T=\const$) sind Hyperbeln. Adiabaten ($Q=\const$) verlaufen steiler als Isothermen, da sich das Gas bei der Expansion abkühlt.}
    \label{fig: pv_diagramm_prozesse}
\end{figure}

\section{Der Zweite Hauptsatz der Thermodynamik}\label{sec: zweiter_hauptsatz}
Während der erste Hauptsatz nur die Energiebilanz betrachtet, trifft der zweite Hauptsatz eine Aussage über die \textbf{Richtung} von Prozessen und die \textbf{Qualität} von Energie. Er erklärt, warum manche Prozesse in der Natur spontan ablaufen und andere nicht. Es gibt mehrere äquivalente Formulierungen:
\begin{itemize}
    \item \textbf(Formulierung nach Clausius): Wärme fließt von selbst immer nur von einem wärmeren zu einem kälteren Körper, niemals umgekehrt.
    \item \textbf(Formulierung nach Kelvin/Thomson): Es ist unmöglich, eine periodisch arbeitende Maschine zu bauen, die einem Wärmereservoir Wärme entzieht und diese vollständig in Arbeit umwandelt, ohne sonstige Veränderungen zu bewirken (Verbot des Perpetuum mobile zweiter Art).
\end{itemize}
Der zweite Hauptsatz impliziert, dass Prozesse in isolierten Systemen immer in Richtung eines Zustands größerer "Unordnung" oder Wahrscheinlichkeit ablaufen. Diese Prozesse sind \textbf{irreversibel}.

\section{Der Carnot'sche Kreisprozess und der Wirkungsgrad}\label{sec: carnot_prozess}
Der Carnot-Prozess ist ein idealisierter, reversibler Kreisprozess, der aus vier Schritten besteht: zwei isotherme und zwei adiabatische Zustandsänderungen. Er beschreibt die theoretisch maximale Effizienz, mit der eine Wärmekraftmaschine Wärme in Arbeit umwandeln kann.
Die Maschine nimmt dabei eine Wärmemenge $Q_1$ von einem heißen Reservoir der Temperatur $T_1$ auf, verrichtet die Arbeit $W$ und gibt die Abwärme $Q_2$ an ein kaltes Reservoir der Temperatur $T_2$ ab.

\begin{figure}[htb]
    \centering
    % \includegraphics[width=0.55\linewidth]{Bilder/Kapitel_Hauptsaetze/carnot_prozess.png}
    \includegraphics[width=0.55\linewidth]{Bilder/Allgemein/placeholder.png}
    \caption{Der Carnot-Prozess im p-V-Diagramm. Die von der Kurve umschlossene Fläche entspricht der pro Zyklus verrichteten Nettoarbeit $\Delta W$.}
    \label{fig: carnot_prozess}
\end{figure}

\begin{importantbox}{Der Carnot-Wirkungsgrad}
    Der \textbf{Wirkungsgrad} $\eta$ einer Wärmekraftmaschine ist das Verhältnis der gewonnenen Arbeit zur aufgewendeten Wärmeenergie, $\eta = |\Delta W / \Delta Q_1|$. Für den idealen Carnot-Prozess hängt der Wirkungsgrad nur von den Temperaturen der beiden Wärmereservoirs ab:
    \begin{equation}\label{eq: carnot_wirkungsgrad}
        \eta_{\text{Carnot}} = \frac{T_1 - T_2}{T_1} = 1 - \frac{T_2}{T_1}
    \end{equation}
    Der Wirkungsgrad ist immer kleiner als 1 (bzw. 100\%), da $T_2 > 0$ sein muss. Dies ist eine direkte Konsequenz des zweiten Hauptsatzes: Es ist unmöglich, die gesamte zugeführte Wärme in Arbeit umzuwandeln; ein Teil muss immer als Abwärme abgeführt werden. Keine reale Wärmekraftmaschine kann einen höheren Wirkungsgrad als die Carnot-Maschine haben, die zwischen denselben Temperaturen arbeitet.
\end{importantbox}

\section{Entropie}\label{sec: entropie}
Der zweite Hauptsatz führt zur Definition einer neuen Zustandsgröße, der \textbf{Entropie} $S$. Die Änderung der Entropie ist für einen reversiblen Prozess definiert als:
\begin{equation}\label{eq: entropie_def}
    dS = \frac{dQ_{\text{rev}}}{T}
\end{equation}
Die Entropie ist ein Maß für die "Unordnung" oder die Anzahl der mikroskopischen Realisierungsmöglichkeiten (Mikrozustände) eines makroskopischen Zustands. In einem abgeschlossenen System kann die Entropie bei irreversiblen Prozessen nur zunehmen oder (bei reversiblen Prozessen) konstant bleiben. Sie nimmt niemals ab. Dies ist eine weitere, fundamentale Formulierung des zweiten Hauptsatzes.

\section{Der Dritte Hauptsatz der Thermodynamik}\label{sec: dritter_hauptsatz}
Der dritte Hauptsatz, auch Nernst-Theorem genannt, macht eine Aussage über das Verhalten der Entropie am absoluten Nullpunkt:
\begin{importantbox}{Dritter Hauptsatz der Thermodynamik}
    Die Entropie eines perfekt kristallinen, reinen Stoffes nähert sich für $T \to 0$ dem Wert null.
    \begin{equation}
        \lim_{T \to 0} S(T) = 0
    \end{equation}
    Am absoluten Nullpunkt befindet sich das System in seinem energetisch tiefstmöglichen Zustand, dem Grundzustand. Dieser Zustand maximaler Ordnung hat nur eine einzige Realisierungsmöglichkeit, weshalb die Entropie null ist. Eine Konsequenz des dritten Hauptsatzes ist, dass der absolute Nullpunkt unerreichbar ist.
\end{importantbox}

















\appendix

\chapter{Herleitungen}
\section{Kapitel Kinematik}
\subsubsection{Bogenlänge und Sehnenlänge}\label{subsubsec: herleitung_bogenlänge_sehnenlänge}
Wir möchten zeigen, dass im Grenzübergang $\Delta \varphi \to 0$, die Sehnenlänge gegen die Bogenlänge konvergiert
\begin{equation*}\label{eq: question_sehnenlänge_gleich_bogenlänge}
    \lim_{\Delta \varphi \to 0} |\Delta \ivec{r}| \eqquestion \Delta s \mDot
\end{equation*}
In \cref{fig: bogenlänge_sehnenlänge} ist exemplarisch die Bogenlänge $\Delta s$ für einen Kreissektor mit dem Winkel $\Delta \varphi$ zusammen mit der Länge $|\Delta \ivec{r}|$ der Sehne $\overline{AB}$ gezeichnet. 

\begin{figure}[tb]
    \centering
    \resizebox{0.45\linewidth}{!}{
    \begin{tikzpicture}[
        vec/.style={-{Stealth}, ultra thick, red!50!black},
        point/.style={fill, circle, inner sep=1.pt}
        ]
        \def\myradius{4.5cm} % Radius des Kreises
        \def\angleA{110}     % Winkel für Punkt A (in Grad)
        \def\angleB{140}     % Winkel für Punkt B (in Grad)
        \def\vecLen{3cm}      % Länge der Geschwindigkeitsvektoren
        \def\fontSize{\large}
        % --- 2. Koordinaten definieren ---
        \coordinate (C) at (0,0);
        \coordinate (A) at (\angleA:\myradius);
        \coordinate (B) at (\angleB:\myradius);
        \coordinate (M_chord) at ($(A)!0.35!(B)$);
        \coordinate (M_arc) at ({(5*\angleA+5*\angleB)/(10)}:\myradius);
        % --- 3. Kreisbahn zeichnen ---
        % Zeichnet das Bogenstück, das etwas über A und B hinausgeht.
        \draw[thick] (\angleA-20:\myradius) arc (\angleA-20:\angleB+25:\myradius);
        \draw[line width=2.0pt, red] (\angleA:\myradius) arc (\angleA:\angleB:\myradius);

        % --- 4. Radien und Punkte zeichnen ---
        \draw (C) -- (A);
        \draw (C) -- (B) node[pos=0.5, below left=1pt] {\fontSize $R$};
        % --- 5. Winkel und Bogenlänge beschriften ---
        \draw[-{Stealth}] (\angleA:1.8cm) arc (\angleA:\angleB:1.8cm);
        \node at ({(\angleA+\angleB)/2}:1.3cm) {\fontSize $\Delta \varphi$}; 
        \draw[dashed, blue, line width=1.1pt] (A) -- (B);
        % Für Δr: Linie vom Mittelpunkt der Sehne (A)--(B)
        \draw[-, thick, blue] (M_chord) -- (-1.7,2.6) node[below] {\fontSize $|\Delta \ivec{r}|$};
        % Für Δs: Linie vom Mittelpunkt des Bogens zwischen A und B
        \draw[-, thick, red] (M_arc) -- (-2.3,3.1) node[below=0.0pt] {\fontSize $\Delta s$};
        \draw[vec] (C) -- (A) node[pos=0.3, right=1.5pt] {\fontSize $\ivecS{r}{A}$};
        \draw[vec] (C) -- (B) node[pos=0.2, left=3.0pt] {\fontSize $\ivecS{r}{B}$};
        \node[point, label={[label distance=3pt]below:C}] at (C) {};
        \node[point, label={[label distance=0pt]above:A}] at (A) {};
        \node[point, label={[label distance=2pt]below:B}] at (B) {};
    \end{tikzpicture}
    }
    \caption{Die Bogenlänge $\Delta s$ (rote Linie) ist über den simplen Zusammenhang $\Delta s = R \cdot \Delta \varphi$ gegeben. Die Länge der Sehne $\overline{AB}$ (blau strichlierte Linie) berechnet sich jedoch aus $|\Delta \protect\ivec{r}| = |\protect\ivecS{r}{B}-\protect\ivecS{r}{A}|$.}     
    \label{fig: bogenlänge_sehnenlänge}
\end{figure}

Die Bogenlänge ergibt sich aus der bekannten Formel $\Delta s = R \Delta \varphi$. Die Sehnenlänge erhalten wir über den Betrag der Differenz der beiden Ortsvektoren 
\begin{equation*}
    \ivecS{r}{A} = \icolTwo{\cos(\varphi)}{\sin(\varphi)} \mComma \quad \ivecS{r}{B} = \icolTwo{\cos(\varphi +\Delta \varphi)}{\sin(\varphi +\Delta \varphi)}\mDot
\end{equation*} 
Da die Sehnenlänge vom Winkel $\varphi$ unabhängig ist, setzen wir \oBdA $\varphi = 0$, wodurch 
\begin{equation}\label{eq: rA_rB_sehnenlänge}
    \ivecS{r}{A} = \icolTwo{1}{0} \mComma \quad \ivecS{r}{B} = \icolTwo{\cos(\Delta \varphi)}{\sin(\Delta \varphi)}\mComma
\end{equation} 
wird. Mithilfe der beiden Vektoren in \cref{eq: rA_rB_sehnenlänge} berechnet sich die Sehnenlänge zu 
\begin{multline}
    |\Delta \ivec{r}| = |\ivecS{r}{B}-\ivecS{r}{A}| = R\left|\icolTwo{\cos(\Delta \varphi)}{\sin(\Delta \varphi)} - \icolTwo{1}{0}\right| = \\
    R\sqrt{(\cos(\Delta \varphi)-1)^2 + (\sin(\Delta \varphi))^2} = \\
    R\sqrt{\cos^2(\Delta \varphi) - 2\cos(\Delta \varphi) + 1^2 + \sin^2(\Delta \varphi)} = \\
    R\sqrt{2-2 \cos(\Delta \varphi)} = R\sqrt{2}\sqrt{1-\cos(\Delta \varphi)} \mComma
\end{multline}
wobei wir $\cos^2(\Delta \varphi) + \sin^2(\Delta \varphi) = 1$ verwendet haben. Unter Zuhilfenahme der Halbwinkelidentität $1-\cos(\Delta \varphi) = 2\sin^2(\Delta \varphi/2)$ erhalten wir einen exakten Ausdruck für die Sehnenlänge
\begin{equation}\label{eq: sehnenlänge_exakt_sinus}
    |\Delta \ivec{r}| = R\sqrt{2}\sqrt{1-\cos(\Delta \varphi)} = R \sqrt{2}\sqrt{2\sin^2\left( \frac{\Delta\varphi}{2} \right)} = 2R\cdot \sin\left( \frac{\Delta \varphi}{2} \right)  \mDot
\end{equation}
Nun verwenden wir die Approximation des Sinus für kleine Winkel $\Delta \varphi$, die besagt, dass 
\begin{equation}
    \sin(\Delta \varphi) \approx \Delta \varphi + \bigO[\Delta \varphi^3] \mComma \quad \Delta \varphi \ll 1\mDot
\end{equation}
Damit nähern wir den exakten Wert in \cref{eq: sehnenlänge_exakt_sinus} an und erhalten 
\begin{equation} \label{eq: näherund_sehnenlänge_kleinwinkel}
    |\Delta \ivec{r}| = 2R\cdot \sin\left( \frac{\Delta \varphi}{2} \right) \approx 2R \cdot \frac{\Delta \varphi}{2} = R \cdot \Delta\varphi \mDot
\end{equation}
Die Näherung für die Sehnenlänge $|\Delta \ivec{r}|$ in \cref{eq: näherund_sehnenlänge_kleinwinkel} entspricht aber genau der Bogenlänge $\Delta s = R\cdot \Delta\varphi$. \\

Damit haben wir gezeigt, dass 
\begin{equation}\label{eq: sehnenlänge_gleich_bogenlänge}
    \lim_{\Delta \varphi \to 0} |\Delta \ivec{r}| = \Delta s \mComma
\end{equation}


\section{Kapitel Wärmelehre}
\subsection{Herleitung der Maxwell-Boltzmann-Verteilung}\label{subsec: herleitung_maxwell_boltzmann}
Die Maxwell-Boltzmann-Verteilung beschreibt die Verteilung der Geschwindigkeiten von Teilchen in einem idealen Gas im thermischen Gleichgewicht. Wir leiten diese Verteilung her, indem wir die Annahmen der kinetischen Gastheorie und die Prinzipien der statistischen Mechanik verwenden.


\newpage
%\chapter{Mathematische Einschübe}\label{chap: MathematischeEinschübe}
\section{Folgen und Reihen}\label{sec: Folgen_Reihen}
\subsubsection{Folgen}\label{subsec: Folgen}
Eine \textbf{Folge} ist eine geordnete Liste von Zahlen, die entweder komplett angegeben ist oder nach einer bestimmten Regel gebildet wird. Jedes Glied der Folge wird als Element bezeichnet. Folgen können endlich oder unendlich sein. \\ 

\noindent\textit{Beispiel:} Man möchte die Folge $\{1, \frac{1}{2},\frac{1}{4}, \frac{1}{8}, \frac{1}{16}, \cdots \}$ untersuchen. Die Beschreibung in der Klammer lässt aber keinen vollständigen Schluss zu, wie die Folge weitergeht, obwohl dies zunächst so scheint. Daher definiert man die Folge durch $(a_n)_{n \in \Real} = \frac{1}{n}$. Das bedeutet, dass das $n$-te Glied der Folge $\frac{1}{n}$ ist. Nun sind alle Glieder der Folge eindeutig bestimmt. 


\subsubsection{Reihen}\label{subsec: Reihen}
Eine \textbf{Reihe} ist die Summe der Glieder einer Folge. Wenn man die Glieder einer unendlichen Folge addiert, spricht man von einer unendlichen Reihe. Reihen können konvergieren oder divergieren. \\

\noindent \textit{Beispiel:} Die Summe der Glieder der Folge von oben schreiben wir als
\begin{equation*}
    \sum_{n=1}^{\infty} \frac{1}{n^2} = 1 + \frac{1}{4} + \frac{1}{9} + \cdot .
\end{equation*}
Wenn sich die Summe einem endlichen Wert annähert und im Limes diesen Wert annimmt, spricht man von einer \textbf{konvergenten Reihe}. Oszilliert die Summe um einen bestimmten Wert oder wächst unbeschränkt und nimmt daher im Limes keinen endlichen Wert an, nennt man sie \textbf{divergente Reihe}.
\newpage

\section{Polynome}\label{sec: Polynome}
Polynome sind mathematische Ausdrücke, die aus Variablen und Koeffizienten bestehen, verbunden durch Addition, Subtraktion und Multiplikation. Ein allgemeines Polynom $P(x)$ in einer Variablen $x$ hat die Form:
\begin{equation}
P(x) = a_n x^n + a_{n-1} x^{n-1} + \cdots + a_1 x + a_0 = \sum_{k=0}^{n} a_k x^k \mComma
\end{equation}
wobei $a_n, a_{n-1}, \ldots, a_0$ die Koeffizienten sind und $x$ ist die Variable. Der höchste vorkommende Exponent der Variablen im Polynom -- die höchste Potenz von $x$ ist hier $n$ -- wird der Grad $n$ des Polynoms genannt. \\

Polynome haben mehrere interessante Eigenschaften:
\begin{itemize}
    \item \textbf{Addition und Subtraktion:} Polynome können addiert und subtrahiert werden, indem man die Koeffizienten der gleichen Potenzen addiert oder subtrahiert.
    \item \textbf{Multiplikation:} Die Multiplikation von Polynomen erfolgt durch die Anwendung des Distributivgesetzes.
    \item \textbf{Nullstellen:} Die Nullstellen eines Polynoms sind die Werte von $x$, für die $P(x) = 0$ gilt.
\end{itemize}
Polynome sind in vielen Bereichen der Mathematik und der angewandten Wissenschaften von Bedeutung. Sie werden verwendet, um Kurven zu modellieren, in der numerischen Analyse und in der Algebra.

\section{Tupel}\label{sec: Tupel}
Ein Tupel ist eine geordnete Liste von Elementen. Im Kontext von mehrdimensionalen Räumen sind Punkte, eine Verallgemeinerung von Skalaren und werden oft als Tupel dargestellt. Stellen Sie sich ein Tupel als eine feste Reihenfolge von Werten vor. Im Gegensatz zu einer Menge kommt es bei einem Tupel auf die Reihenfolge der Elemente an, und Elemente können mehrfach vorkommen. Hier sind ein paar Beispiele, um das Konzept zu verdeutlichen:\\

\noindent\textbf{2D-Raum (Ebene):} Ein Punkt im zweidimensionalen Raum wird als Paar von Koordinaten $(x,y)$ dargestellt. Das ist ein $2$-Tupel. Zum Beispiel ist $(3,5)$ ein Punkt, wobei $3$ die $x$-Koordinate und $5$ die $y$-Koordinate ist. Der Punkt $(5,3)$ wäre ein anderer Punkt. \\

\noindent\textbf{3D-Raum:} Ein Punkt im dreidimensionalen Raum ist ein Tripel von Koordinaten $(x,y,z)$. Das ist ein $3$-Tupel. Ein Beispiel wäre $(1,2,7)$ oder $(-3,4,-3)$.\\

\noindent\textbf{n-dimensionaler Raum:} Allgemein wird ein Punkt in einem $n$-dimensionalen Raum durch ein $n$-Tupel von Koordinaten $(x_1, x_2, \,\dots, x_n)$ beschrieben. Jede dieser Koordinaten ist ein Skalar.\\

Ein Tupel von Koordinaten kann keine Richtung vermitteln. Das Tupel beschreibt lediglich einen Ort im $n$-dimensionalen Raum. Zusammenfassend lässt sich sagen, dass ein Tupel eine mathematische Struktur ist, die es ermöglicht, die Koordinaten eines Punktes in einem mehrdimensionalen Raum zu erfassen.

\newpage
\section{Vektorrechnung}\label{sec: Vektorrechnung}
\subsection{Unterschied zwischen Vektoren und Skalaren}\label{subsec: Vektor_vs_Skalar}
In mehrdimensionalen Räumen sind \textbf{Punkte} $P = (x,y,...)$ die Verallgemeinerung von \textbf{Skalaren}. \textbf{Punkte} vermitteln \textbf{keine Richtung} -- sie sind ein $n$-Tupel von Skalaren. Punkte sind „ortsfest“, aber ihre Koordinaten hängen vom gewählten Koordinatensystem ab.

% tikz picture
\begin{figure}[h!]
    \centering
    \begin{minipage}[b]{0.48\textwidth}
        \centering
        \resizebox{\linewidth}{!}{
            \begin{tikzpicture}
                % 1. Draw the horizontal axis
                \draw[->, line width=1.2pt] (-3, 0) -- (9, 0) node[right] {\large $x$};
                % 2. Add the tick marks and labels
                \foreach \x in {-2, 0, 2, 4, 6, 8} {
                    \draw[line width=1.2pt] (\x, -0.15) -- (\x, 0.15);
                    \node[below=5pt] at (\x, 0) {\x};
                }
                % 3. Draw the solid blue point at x=3
                \node[circle, fill=pointblue, inner sep=2.8pt] at (3, 0) {};
                % 4. Add the label above the point
                \node[above=7pt, color=pointblue] at (3, 0) {\LARGE $x=3$};
                \node[color=black] at (-2, 3.6) {\Huge $\Real^1$};
            \end{tikzpicture}
        }
        \caption{Ein Punkt auf dem Zahlenstrahl ($\Real^1$), der die Gleichung $x = 3$ darstellt.}
    \end{minipage}
    \hfill 
    \begin{minipage}[b]{0.48\textwidth}
        \centering
        \resizebox{\linewidth}{!}{
            \begin{tikzpicture}
                % 1. Draw the axes
                \draw[->, line width=1.2pt] (-3, 0) -- (7, 0) node[right] {\large $x$};
                \draw[->, line width=1.2pt] (0, -1.5) -- (0, 2.8) node[above] {\large $y$};
                % 2. Add the tick marks and labels
                \foreach \x in {-2, 0, 2, 4, 6} {
                    \draw[line width=1.2pt] (\x, -0.15) -- (\x, 0.15);
                    \ifnum\x=0
                        \node[below left=2pt] at (\x, 0) {0};
                    \else
                        \node[below=5pt] at (\x, 0) {\x};
                    \fi
                };
                \foreach \y in {-1, 1, 2} {
                    \draw[line width=1.2pt] (-0.15, \y) -- (0.15, \y);
                    \node[left=5pt] at (0, \y) {\y};
                };
                % 3. Draw the solid blue point
                \node[circle, fill=pointblue, inner sep=2.5pt] at (4,2) {};
                % 4. Add the label above the point
                \node[anchor=south west, color=pointblue] at (4,2) {\Large $P(4, 2)$};
                \node[color=black] at (-2, 3) {\huge $\Real^2$};
            \end{tikzpicture}
        }
        \caption{Ein Punkt in der kartesischen Ebene ($\Real^2$) an den Koordinaten $(4, 2)$.}
    \end{minipage}
\end{figure}

\begin{rememberbox}[]{Definition Vektor}
    \textbf{Vektoren} zeichnen sich im Gegensatz zu einem Skalar dadurch aus, dass sie eine \textbf{Länge} und eine \textbf{Richtung} haben. Vektoren haben keinen festen Start- und Endpunkt und sind damit nicht \gDQ{ortsfest}. Vektoren können über die Vorschrift 
    \begin{equation}
        \mathrm{Vektor} = \mathrm{Endpunkt} - \mathrm{Anfangspunkt}
    \end{equation}
    generiert werden. Ein Vektor vermittelt somit die \textbf{Verschiebung} des Anfangspunktes A zum Endpunkt E und wird auch als $\ivec{AE}$ notiert. 
\end{rememberbox}
Unterschiedliche Start- und Endpunkte können demnach dieselben Vektoren erzeugen. Im Beispielbild rechts sind der blaue Vektor $\ivecS{r}{1}$ und der rote Vektor $\ivecS{r}{2}$ ident, $\ivecS{r}{1}=\begin{psmallmatrix} 2 \\ 1 \end{psmallmatrix} = \ivecS{r}{2}$, obwohl sie von unterschiedlichen Start- und Endpunkten erzeugt wurden. Die Koordinaten eines Vektors hängen ebenso vom Koordinatensystem ab.
\begin{figure}[h!]
    \centering
    \resizebox{0.7\linewidth}{!}{
    \begin{tikzpicture}
    \definecolor{pointblue}{HTML}{004A8F}
    % 1. Draw the horizontal axis
    \draw[->, line width=1.2pt] (-3, 0) -- (9, 0) node[right] {\large $x$};
    \draw[->, line width=1.2pt] (0, -1.5) -- (0, 4.5) node[above] {\large $y$};
    % 2. Add the tick marks and labels
    % A 'foreach' loop is used to place them at regular intervals.
    \foreach \x in {-2, 0, 2, 4, 6, 8} {
        % Draw thick vertical lines for the ticks.
        \draw[line width=1.2pt] (\x, -0.15) -- (\x, 0.15);
        % Place the corresponding number below each tick.
        \ifnum\x=0
            \node[below left=2.1pt] at (\x -0.2, 0) {\x};
        \else
            \node[below=5pt] at (\x, 0) {\x}; 
    \fi
    };
    % small ticks
    \foreach \x in {-1, 1, 3, 5, 7} {
        % Draw small vertical lines for the ticks.
        \draw[line width=1.0pt] (\x, -0.1) -- (\x, 0.1);
    };
    \foreach \y in {-1, 1, 2, 3, 4} {
        % Draw thick lines for the ticks.
        \draw[line width=1.2pt] (-0.15, \y) -- (0.15, \y);
        % Place the corresponding number below each tick.
        \ifnum \y=0
        \else
            \node[left=5pt] at (0, \y) {\y}; 
        \fi
    };
    % small ticks
    \foreach \y in {-0.5, 0.5, 1.5, 2.5, 3.5} {
        % Draw small vertical lines for the ticks.
        \draw[line width=1.0pt] (-0.1, \y) -- (0.1, \y);
    };
    % 3. Draw the vector end points
    \node[circle, fill=black, inner sep=1.5pt] at (1,2) {};
    \node[circle, fill=black, inner sep=1.5pt] at (4,4) {};
    \node[circle, fill=black, inner sep=1.5pt] at (4,1) {};
    \node[circle, fill=black, inner sep=1.5pt] at (7,3) {};
    \draw[-Stealth, color=blue,line width=2.2pt] (1,2) -- (4,4);
    \draw[-Stealth, color=red,line width=2.2pt] (4,1) -- (7,3);
    % 4. Add the label above the point
    \node[anchor=south west, color=black] at (3,2.1) {\LARGE \color{blue}$\vec{r}_1 \color{black}= \color{red}\vec{r}_2$};
    \end{tikzpicture}
    }
    \caption{Die beiden Vektoren $\protect\ivecS{r}{1}$ (blau) und $\protect\ivecS{r}{2}$ (rot) sind ident. Sie haben dieselbe Richtung und dieselbe Länge.}\label{fig: zweiVektorenIdent}
\end{figure}

\subsection{Länge eines Vektors (Betrag)}\label{subsec: Laenge_Betrag_Vektor}
\subsubsection{Norm}
Eine Norm ist eine Abbildung, 
\begin{equation}
    ||\cdot||:\, V \to \Real_0^{+}, \quad \ivec{x} \mapsto ||\ivec{x}||, 
\end{equation}
für die drei definierende Axiome erfüllt sind:
\begin{itemize}[itemsep=3pt]
    \item \textbf{Definitheit:} $||\ivec{x}|| = 0 \Rightarrow \ivec{x} = \ivec{0}$
    \item \textbf{Absolute Homogenität:} $||\lambda \cdot \ivec{x}|| = |\lambda| \cdot ||\ivec{x}||$
    \item \textbf{Dreiecksungleichung:} $||\ivec{x}+\ivec{y}|| \le ||\ivec{x}||+||\ivec{y}||$
\end{itemize}

\subsubsection{Betrag eines Vektors}
\begin{rememberbox}[]{Betrag eines Vektors}
    Die \textbf{Länge} (auch Betrag) eines Vektors ergibt sich aus dem Satz des Pythagoras. Die verwendete Norm nennt man euklidische Norm oder $2$-Norm ($p = 2$). Die Länge des Vektors $\ivec{r} = (x, y, ...)$ ist dann
    \begin{equation}
        |\ivec{r}| = \sqrt{x^2 + y^2 + z^2}\mDot
    \end{equation}
Der Betrag eines Vektors ist ein (positiver) Skalar.
\end{rememberbox}
Der Verweis auf die Tatsache, dass der Betrag eines Vektors mathematisch gesehen eine Norm ist, sei nur der Vollständigkeit halber angegeben. Von Bedeutung ist diesbezüglich nur die positive Definitheit, \gDh der Betrag ist immer größer oder gleich $0$.
\begin{figure}[h!]
    \centering
    \resizebox{0.6\linewidth}{!}{
    \begin{tikzpicture}
    % 1. Draw the horizontal axis
    \draw[->, line width=1.2pt] (-1, 0) -- (9, 0) node[right] {\large $x$};
    \draw[->, line width=1.2pt] (0, -1) -- (0, 4.5) node[above] {\large $y$};
    % 2. Add the tick marks and labels
    % A 'foreach' loop is used to place them at regular intervals.
    \foreach \x in {0, 2, 4, 6, 8} {
        % Draw thick vertical lines for the ticks.
        \draw[line width=1.2pt] (\x, -0.15) -- (\x, 0.15);
        % Place the corresponding number below each tick.
        \ifnum\x=0
            \node[below=5pt] at (\x -0.2, 0) {\x};
        \else
            \node[below=5pt] at (\x, 0) {\x}; 
    \fi
    };
    % small ticks
    \foreach \x in {1, 3, 5, 7} {
        % Draw small vertical lines for the ticks.
        \draw[line width=1.0pt] (\x, -0.1) -- (\x, 0.1);
    };
    \foreach \y in {1, 2, 3, 4} {
        % Draw thick lines for the ticks.
        \draw[line width=1.2pt] (-0.15, \y) -- (0.15, \y);
        % Place the corresponding number below each tick.
        \ifnum \y=0
        \else
            \node[left=5pt] at (0, \y) {\y}; 
        \fi
    };
    % small ticks
    \foreach \y in {0.5, 1.5, 2.5, 3.5} {
        % Draw small vertical lines for the ticks.
        \draw[line width=1.0pt] (-0.1, \y) -- (0.1, \y);
    };
    % 3. Draw the vector end points
    \node[circle, fill=black, inner sep=1.5pt] at (2,1.5) {};
    \node[circle, fill=black, inner sep=1.5pt] at (6,4) {};
    \draw[-Stealth, color=orange,line width=2.2pt] (2,1.5) -- (6,4);
    \draw[-, color=black,line width=1.5pt] (2,1.5) -- (6,1.5);
    \draw[-, color=black,line width=1.5pt] (6,1.5) -- (6,4);
    % 4. Add the label above the point
    \node[anchor=south west, color=orange] at (3.3,3) {\LARGE $\vec{r}$};
    \node[anchor=south west, color=black] at (3.8,0.75) {\Large $x$};
    \node[anchor=south west, color=black] at (6.2,2.4) {\Large $y$};
    \end{tikzpicture}
    }
    \caption{Die Länge eines Vektors ergibt sich geometrisch aus dem Satz des Pythagoras: $|\protect\ivec{r}|^2 = x^2 + y^2$.}\label{fig: laengeEinesVektors}
\end{figure}


\subsection{Richtung eines Vektors}\label{subsec: Richtung_Vektor}
\begin{rememberbox}[]{Richtung eines Vektors}
    Die Richtung eines Vektors $\ivec{r}$ in $\Real^2$ kann über den Winkel $\varphi$ angegeben werden, den der Vektor mit der $x$-Achse einschließt. Dieser Winkel ergibt sich geometrisch zu
    \begin{equation}
        \tan(\varphi) = \frac{y}{x}
    \end{equation}
    mit $\ivec{r} = (x, y) \in \Real^2$. Der Winkel ergibt sich über die Umkehrfunktion $\varphi = \arctan(y/x)$.
\end{rememberbox}

\begin{figure}[h!]
    \centering
        \resizebox{0.6\linewidth}{!}{
        \begin{tikzpicture}
            % 1. Draw the horizontal axis
            \draw[->, line width=1.2pt] (-1, 0) -- (9, 0) node[right] {\large $x$};
            \draw[->, line width=1.2pt] (0, -1) -- (0, 4.5) node[above] {\large $y$};
            % 2. Add the tick marks and labels
            \foreach \x in {0, 2, 4, 6, 8} {
                \draw[line width=1.2pt] (\x, -0.15) -- (\x, 0.15);
                \ifnum\x=0
                    \node[below=5pt] at (\x -0.2, 0) {\x};
                \else
                    \node[below=5pt] at (\x, 0) {\x}; 
                \fi
            };
            \foreach \x in {1, 3, 5, 7} {
                \draw[line width=1.0pt] (\x, -0.1) -- (\x, 0.1);
            };
            \foreach \y in {1, 2, 3, 4} {
                \draw[line width=1.2pt] (-0.15, \y) -- (0.15, \y);
                \ifnum \y=0
                \else
                    \node[left=5pt] at (0, \y) {\y}; 
                \fi
            };
            \foreach \y in {0.5, 1.5, 2.5, 3.5} {
                \draw[line width=1.0pt] (-0.1, \y) -- (0.1, \y);
            };
            % 3. Draw the vector end points
            \draw[line width=1pt] (4,1.5) arc[start angle=0, end angle=40, radius=1.5];
            \node[circle, fill=black, inner sep=1.5pt] at (2,1.5) {};
            \node[circle, fill=black, inner sep=1.5pt] at (6,4) {};
            \draw[-Stealth, color=darkyellow,line width=2.2pt] (2,1.5) -- (6,4);
            \draw[-, color=black,line width=1.5pt] (2,1.5) -- (6,1.5);
            \draw[-, color=black,line width=1.5pt] (6,1.5) -- (6,4);
            % 4. Add the label above the point
            \node[anchor=south west, color=darkyellow] at (3.3,3) {\LARGE $\vec{r}$};
            \node[anchor=south west, color=black] at (3.8,0.75) {\Large $x$};
            \node[anchor=south west, color=black] at (6.2,2.4) {\Large $y$};
            \node[anchor=south west, color=black] at (3.2,1.6) {\Large $\varphi$};
        \end{tikzpicture}
        }
        \caption{Die Richtung eines Vektors wird über den Winkel $\varphi$ definiert, den der Vektor $\protect\ivec{r}$ mit der $x$-Achse einschließt.}\label{fig: richtungEinesVektorsPhi}
\end{figure}
Im $\Real^3$ benötigt man zwei Winkel ($\varphi, \theta$) zur Beschreibung der Richtung und in höherdimensionalen Räumen benötigt man $n-1$ Winkel (praktisch meist irrelevant).\\

\textbf{Achtung:} Da der Tangens Unstetigkeiten aufweist (siehe \cref{fig: abb_tangens_arkustanges}), muss die Berechnung korrigiert werden. Wenn der Punkt $(x, y)$ im 2. oder 3. Quadranten liegt, erhält man nur dann den korrekten Winkel, wenn man $\pi\, (=\ang{180})$ dazuaddiert:
\begin{rememberbox}[sidebyside, sidebyside align=center, lower separated=false]{}
    Im Quadranten $Q_1$ und $Q_4$:
    \begin{equation}
        \varphi = \arctan\left(\frac{y}{x}\right)
    \end{equation}
    \tcblower
    Im Quadranten $Q_2$ und $Q_3$: 
    \begin{equation}
        \varphi = \arctan\left(\frac{y}{x}\right) + \pi
    \end{equation}
\end{rememberbox}
\begin{figure}[h!]
    \centering
    \includegraphics[width=0.95\linewidth]{Bilder/Kapitel_MathEinschübe/tan_arctan.png}
    \caption{Graphen von Tangens und Arkustangens}\label{fig: abb_tangens_arkustanges}
\end{figure}
Die manuelle Korrektur des Arkustangens ist notwendig, da das Argument aus dem Bruch $(\frac{y}{x})$ besteht. Für einen Winkel im 1. Quadranten ist $x > 0$ und $y > 0$ und demnach ist der Bruch $(\frac{y}{x}) > 0$. Zwar gilt für einen Punkt im 3. Quadranten $x < 0$ und $y < 0$, aber da sich beide negativen Vorzeichen im Argument aufheben, ist der Bruch $(\frac{y}{x}) > 0$. Das bedeutet, dass man ohne Korrektur für Punkte im 3. Quadranten einen Winkel $\varphi \in [0, \pi/2]$ (1. Quadrant) bekommen würde. Für Punkte im 4. Quadranten würde man aus demselben Grund (beide Male gilt $(\frac{y}{x}) < 0$) Winkel im 2. Quadranten ($\varphi \in [\pi/2, \pi]$) erhalten. Indem man $\pi$ zu einem Winkel im 1. Quadranten (2. Quadranten) addiert, landet man beim äquivalenten Winkel im 3. Quadranten (4. Quadranten). Dieses Verhalten sieht man auch im Graph des Arkustangens in \cref{fig: abb_tangens_arkustanges}.  


\subsection{Einheitsvektor und Normierung}\label{subsec: Einheitsvektor_Normierung}
Neben der Darstellung eines Vektors $\ivec{r}$ mittels Koordinaten, $\ivec{r} = (x,y,\,\dots)$,  kann ein Vektor auch als Produkt aus Betrag und Einheitsvektors geschrieben werden. 
\begin{rememberbox}[]{}
    Ein Vektor $\ivec{r}$ kann immer mit Hilfe des \textbf{Betrags} $r = |\ivec{r}|$ und des \textbf{Einheitsvektors} $\ivec{e_r}$ als 
    \begin{equation}
        \ivec{r} = r \cdot \ivecS{e}{r} \mDot
    \end{equation}
    geschrieben werden. Der Einheitsvektor $\ivecS{e}{r}$ zeigt dabei in dieselbe Richtung wie $\ivec{r}$, hat aber die normierte Länge $1$.
\end{rememberbox}
 Die Länge von $\ivec{r}$ bleibt somit erhalten, da der Einheitsvektor nichts zur Länge beiträgt: $|\ivec{r}| = |r \cdot \ivecS{e}{r}| = \underbrace{|r|}_r \cdot \underbrace{|\ivecS{e}{r}|}_1 = r$. Umgekehrt skaliert der Betrag $r$ nur den Einheitsvektor $\ivecS{e}{r}$ und die Richtung von $\ivec{r}$ wird nur durch $\ivecS{e}{r}$ bestimmt. \newline
 \subsubsection{Normierung}
 Als \textbf{Normierung} bezeichnet man die Berechnung des Einheitsvektors
 \begin{equation}\label{eq: normierungVektor}
     \ivec{e_r} = \frac{1}{|\ivec{r}|} \cdot \ivec{r}\mDot
 \end{equation}
Aus \cref{eq: normierungVektor} sieht man, dass zur Normierung eines Vektors zunächst sein Betrag $|\ivec{r}|$ berechnet werden muss.
\begin{figure}[h!]
    \centering
        \resizebox{0.5\linewidth}{!}{
        \begin{tikzpicture}
        % 1. Draw the axis
        \draw[->, line width=1.2pt] (-0.5, 0) -- (6.5, 0) node[right] {\large $x$};
        \draw[->, line width=1.2pt] (0, -0.5) -- (0, 4) node[above] {\large $y$};
    
        % 2. Add the tick marks and labels
        \foreach \x in {0, 2, 4, 6} {
            % Draw thick vertical lines for the ticks.
            \draw[line width=1.2pt] (\x, -0.15) -- (\x, 0.15);
            % Place the corresponding number below each tick.
            \ifnum\x=0
                \node[below=5pt] at (\x -0.2, 0) {\x};
            \else
                \node[below=5pt] at (\x, 0) {\x}; 
        \fi
        };
        % small ticks
        \foreach \x in {1, 3, 5} {
            % Draw small vertical lines for the ticks.
            \draw[line width=1.0pt] (\x, -0.1) -- (\x, 0.1);
        };
        \foreach \y in {1, 2, 3} {
            % Draw thick lines for the ticks.
            \draw[line width=1.2pt] (-0.15, \y) -- (0.15, \y);
            % Place the corresponding number below each tick.
            \ifnum \y=0
            \else
                \node[left=5pt] at (0, \y) {\y}; 
            \fi
        };
        % small ticks
        \foreach \y in {0.5, 1.5, 2.5, 3.5} {
            % Draw small vertical lines for the ticks.
            \draw[line width=1.0pt] (-0.1, \y) -- (0.1, \y);
        };    
        % 3. Draw the vector end points
        \draw[-Stealth, color=blue,line width=2.2pt] (2,1) -- (5,3.5);
        \draw[-Stealth, color=orange,line width=2.2pt] (2,2) -- (2.77,2.64);
        % 4. Add the label above the point
        \node[anchor=south west, color=blue] at (4,1.5) {\LARGE $\vec{r}$};
        \node[anchor=south west, color=orange] at (1.8,2.5) {\LARGE $\vec{e}_r$};
        \end{tikzpicture}
        }
    \caption{Der Einheitsvektor $\protect\ivecS{e}{r}$ hat dieselbe Richtung wie $\protect\ivec{r}$, aber die Länge $1$.}\label{fig: vektorEinheitsvektor}
\end{figure}

\subsection{Addition und Subtraktion}\label{subsec: Vektoraddition-subtraktion}
Vektoren und Skalare sind unterschiedliche mathematische Objekte. Operationen, die für
Skalare definiert sind, sind zunächst für Vektoren nicht definiert. Zu diesen Operationen gehören zum Beispiel Addition, Subtraktion, Multiplikation und Division.
Um auszudrücken, ob es sich bei einer Operation, um die skalare Variante oder die vektorielle Variante handelt, kann man andere Symbole für die beiden Varianten verwenden $(\oplus, \ominus, \otimes, *)$\footnote{Diese eingekreisten Symbole werden häufig für Tensoroperationen verwendet. Hier werden sie ausschließlich für Vektoroperationen verwendet.}.
In der Literatur wird meist nur bei Zweideutigkeit ein anderes Symbol verwendet. \\

\subsubsection{Vektoraddition}
Die Vektoraddition $(\oplus)$ ist eine Verallgemeinerung der Addition von Skalaren, die die 4 definierenden Eigenschaften der skalaren Addition erhält:
\begin{itemize}[itemsep=1.5pt]
    \item Kommutativität,
    \item Assoziativität,
    \item Existenz des neutralen Elements,
    \item  Existenz des inversen Elements.
\end{itemize}
\begin{rememberbox}[]{Vektoraddition}
    Die \textbf{Vektoraddition} entspricht einer Addition der Koordinaten der beiden zu summierenden Vektoren
    \begin{equation}
        \ivec{r} = \ivec{r_1} \oplus \ivec{r_2} \defeq (x_1+x_2, y_1+y_2, ...), 
    \end{equation}
mit $\ivec{r_1}=(x_1, y_1, ...)$ und $\ivec{r_2}=(x_2, y_2, ...)$. Das Resultat einer Vektoraddition ist wieder ein Vektor.
\end{rememberbox}
Grafisch entspricht die Vektoraddition einer Aneinanderreihung der Vektoren, wie in \cref{fig: vektoraddition} verbildlicht. Eine Vektoraddition ist demnach eine Hintereinanderausführung von zwei Verschiebungen. Dementsprechend ist es wenig überraschend, dass die Vektoraddition kommutativ ist, \gDh $\ivecS{r}{1} \oplus \ivecS{r}{2} = \ivecS{r}{2} \oplus \ivecS{r}{1}$. 

\subsubsection{Vektorsubtraktion}
Die \textbf{Vektorsubtraktion} $\ivec{r} = \ivecS{r}{1} \ominus \ivecS{r}{2}$ ergibt eine Gesamtverschiebung (Vektor), der ebenso aus zwei einzelnen Verschiebungen zusammengesetzt werden kann. Die Gesamtverschiebung $\ivec{r}$ besteht aus der Verschiebung $\ivecS{r}{1}$ und der anschließenden Verschiebung um $-\ivecS{r}{2}$, dem sogenannten inversen Element zu $\ivecS{r}{2}$.
\begin{rememberbox}[]{Inverses Element}
    Das inverse Element $-\ivec{r}$ eines Vektors $\ivec{r}$ hat dieselbe Länge, aber die umgekehrte Richtung. Das inverse Element eines Vektors $\ivec{r} = (x,y,\,\dots)$ erhält man, indem man alle Koordinaten des Vektors spiegelt, 
    \begin{equation}
        -\ivec{r} \defeq (-x,-y,\,\dots)\mDot
    \end{equation} 
\end{rememberbox}
Die Vektorsubtraktion zweier Vektoren kann daher von einer Subtraktion in eine Addition mit dem inversen zweiten Argument umgeschrieben werden: 
\begin{equation}
    \ivecS{r}{1} \ominus \ivecS{r}{2} \equiv \ivecS{r}{1} \oplus (-\ivecS{r}{2})\mDot
\end{equation}
\begin{rememberbox}[]{Vektorsubtraktion}
    Die Vektorsubtraktion $(\ominus)$ entspricht einer Vektoraddition mit dem inversen Element. Die Vektorsubtraktion erfolgt ebenso koordinatenweise:
    \begin{equation}
        \ivec{r} = \ivecS{r}{1} \ominus \ivecS{r}{2} = \ivecS{r}{1} \oplus (-\ivecS{r}{2}) \defeq (x_1 - x_2, y_1 - y_2,\,\dots),
    \end{equation}
mit $\ivecS{r}{1} = (x_1, y_1,\,\dots)$ und $\ivecS{r}{2} = (x_2, y_2, \,\dots)$. Das Resultat der Vektorsubtraktion ist wieder ein Vektor.
\end{rememberbox}

\begin{figure}[h!]
    \centering
    \begin{minipage}[b]{0.48\textwidth}
    \centering
    % Platzhalter für das Additionsbild
    \resizebox{\linewidth}{!}{
        \begin{tikzpicture}
            % r1 = PQ , r2 = QR und r = PR
            \coordinate (P) at (1,1);
            \coordinate (Q) at (3,3);
            \coordinate (R) at (6,2.5);
            % 1. Draw the horizontal axis
            \draw[->, line width=1.2pt] (-0.5, 0) -- (6.5, 0) node[right] {\large $x$};
            \draw[->, line width=1.2pt] (0, -0.5) -- (0, 4) node[above] {\large $y$};
        
            % 2. Add the tick marks and labels
            % A 'foreach' loop is used to place them at regular intervals.
            \foreach \x in {0, 2, 4, 6} {
                % Draw thick vertical lines for the ticks.
                \draw[line width=1.2pt] (\x, -0.15) -- (\x, 0.15);
                % Place the corresponding number below each tick.
                \ifnum\x=0
                    \node[below=5pt] at (\x -0.2, 0) {\x};
                \else
                    \node[below=5pt] at (\x, 0) {\x}; 
            \fi
            };
            % small ticks
            \foreach \x in {1, 3, 5} {
                % Draw small vertical lines for the ticks.
                \draw[line width=1.0pt] (\x, -0.1) -- (\x, 0.1);
            };
            \foreach \y in {1, 2, 3} {
                % Draw thick lines for the ticks.
                \draw[line width=1.2pt] (-0.15, \y) -- (0.15, \y);
                % Place the corresponding number below each tick.
                \ifnum \y=0
                \else
                    \node[left=5pt] at (0, \y) {\y}; 
                \fi
            };
            % small ticks
            \foreach \y in {0.5, 1.5, 2.5, 3.5} {
                % Draw small vertical lines for the ticks.
                \draw[line width=1.0pt] (-0.1, \y) -- (0.1, \y);
            };
            % 3. Draw the vector end points
            \node[circle, fill=black, inner sep=1.5pt] at (P) {};
            \node[circle, fill=black, inner sep=1.5pt] at (Q) {};
            \node[circle, fill=black, inner sep=1.5pt] at (R) {};
            \draw[-Stealth, color=orange,line width=2.2pt] (P) -- (Q);
            \draw[-Stealth, color=purpleCol,line width=2.2pt] (Q) -- (R);
            \draw[-Stealth, color=blue,line width=2.2pt] (P) -- (R);
            % 4. Add the label above the point
            \node[anchor=south west, color=blue] at (3.4,1.0) {\LARGE $\vec{r}$};
            \node[anchor=south west, color=orange] at (1.4,2.1) {\LARGE $\vec{r}_1$};
            \node[anchor=south west, color=purpleCol] at (4,2.8) {\LARGE $\vec{r}_2$};
        \end{tikzpicture}
    }
    \caption{Grafische Darstellung der Vektoraddition $\protect\ivec{r} = \protect\ivecS{r}{1} \oplus \protect\ivecS{r}{2}$.}\label{fig: vektoraddition}
    \end{minipage}
    \hfill
    \begin{minipage}[b]{0.48\textwidth}
    \centering
    % Platzhalter für das Subtraktionsbild
    \resizebox{\linewidth}{!}{
        \begin{tikzpicture}
            % r1 = PQ , r2 = QR und r = PR
            \coordinate (P) at (1,1);
            \coordinate (Q) at (4,3);
            \coordinate (R) at (2,3.2);
            \coordinate (S) at (6,2.8);
            % 1. Draw the horizontal axis
            \draw[->, line width=1.2pt] (-0.5, 0) -- (6.5, 0) node[right] {\large $x$};
            \draw[->, line width=1.2pt] (0, -0.5) -- (0, 4) node[above] {\large $y$};
            % 2. Add the tick marks and labels
            \foreach \x in {0, 2, 4, 6} {
                % Draw thick vertical lines for the ticks.
                \draw[line width=1.2pt] (\x, -0.15) -- (\x, 0.15);
                % Place the corresponding number below each tick.
                \ifnum\x=0
                    \node[below=5pt] at (\x -0.2, 0) {\x};
                \else
                    \node[below=5pt] at (\x, 0) {\x}; 
            \fi
            };
            % small ticks
            \foreach \x in {1, 3, 5} {
                % Draw small vertical lines for the ticks.
                \draw[line width=1.0pt] (\x, -0.1) -- (\x, 0.1);
            };
            \foreach \y in {1, 2, 3} {
                % Draw thick lines for the ticks.
                \draw[line width=1.2pt] (-0.15, \y) -- (0.15, \y);
                % Place the corresponding number below each tick.
                \ifnum \y=0
                \else
                    \node[left=5pt] at (0, \y) {\y}; 
                \fi
            };
            % small ticks
            \foreach \y in {0.5, 1.5, 2.5, 3.5} {
                % Draw small vertical lines for the ticks.
                \draw[line width=1.0pt] (-0.1, \y) -- (0.1, \y);
            };
            % 3. Draw the vector end points
            \node[circle, fill=black, inner sep=1.5pt] at (P) {};
            \node[circle, fill=black, inner sep=1.5pt] at (Q) {};
            \node[circle, fill=black, inner sep=1.5pt] at (R) {};
            \draw[-Stealth, color=orange,line width=2.2pt] (P) -- (Q);
            \draw[-Stealth, color=purpleCol,line width=2.2pt] (Q) -- (R);
            \draw[-Stealth, color=purpleColFade,line width=2.2pt] (Q) -- (S);
            \draw[-Stealth, color=blue,line width=2.2pt] (P) -- (R);
            % 4. Add the label above the point
            \node[anchor=south west, color=blue] at (0.8,2.0) {\LARGE $\vec{r}$};
            \node[anchor=south west, color=orange] at (2.6,1.2) {\LARGE $\vec{r}_1$};
            \node[anchor=south west, color=purpleCol] at (2.3,3.2) {\LARGE $-\vec{r}_2$};
            \node[anchor=south west, color=purpleColFade] at (4.5,3.0) {\LARGE $\vec{r}_2$};
        \end{tikzpicture}
    }
    \caption{Grafische Darstellung der Vektorsubtraktion $\protect\ivec{r} = \protect\ivecS{r}{1} \ominus \protect\ivecS{r}{2}$.}\label{fig: vektorsubtraktion}
    \end{minipage}
\end{figure}


\subsection{Skalarmultiplikation}\label{subsec: Skalarmultiplikation}
Die Multiplikation von Skalaren entspricht einer wiederholten Addition: \[5 \cdot 3 = 5 + 5 + 5 = 15\mDot\]
Für Vektoren definieren wir die sogenannte Skalarmultiplikation auf dieselbe Weise.
\begin{rememberbox}[]{Skalarmultiplikation}
    Die Skalarmultiplikation eines Vektors $\ivec{r} = (x,y,\,\dots)$ mit einem Skalar $s$ entspricht einer $s$-fachen Vektoraddition mit sich selbst,
    \begin{equation}
        \ivec{u} = s \cdot \ivec{r} := s\cdot (x,y,\,\dots) = (s \cdot x, s \cdot y, \,\dots)\mDot
    \end{equation}
    Das Resultat der Skalarmultiplikation ist ein Vektor mit derselben Richtung $(\ivec{u} \parallel \ivec{r})$ und der skalierten Länge $|\ivec{u}| = s \cdot |\ivec{r}|$.
\end{rememberbox}

\begin{itemize}[itemsep=1.5pt]
    \item Wenn $|s| > 1$ ist, wird der Vektor gestreckt.
    \item Für $s = 1$ erhält man den Ausgangsvektor, $\ivec{u} = 1\cdot \ivec{r}$.
    \item Ist $0 < |s| < 1$, wird der Vektor gestaucht.
    \item Wenn $s < 0$, wird die Richtung des Vektors zusätzlich zur Streckung oder Stauchung umgekehrt.
    \item Für $s = 0$ erhält man den Nullvektor, $\ivec{u} = 0\cdot \ivec{r} = (0,0,\,\dots) = \ivec{0}$.
\end{itemize}
\textit{Beispiel:} Wenn $s = -0.5$ ist -- also negativ und $0 < |s| < 1$ -- so entspricht der Vektor $ -0.5 \cdot \ivec{r}$ einem Vektor, der in die entgegengesetzte Richtung von $\ivec{r}$ zeigt und nur halb so lang ist.
\begin{figure}[h!]
    \centering
    \resizebox{0.55\linewidth}{!}{
    \begin{tikzpicture}
        \definecolor{purpleCol}{HTML}{a834eb}
        \definecolor{purpleColFade}{HTML}{ccb2db}
        % r = (1, 0.5)
        \coordinate (P) at (1,0.5);
        \coordinate (Q) at (5,2.5);
        \coordinate (A) at (0.8,1.2);
        \coordinate (B) at (1.8,1.7);
        \coordinate (C) at (2.8,2.2);
        \coordinate (D) at (3.8,2.7);
        \coordinate (E) at (4.8,3.2);
        % 1. Draw the horizontal axis
        \draw[->, line width=1.2pt] (-0.5, 0) -- (6.5, 0) node[right] {\large $x$};
        \draw[->, line width=1.2pt] (0, -0.5) -- (0, 4) node[above] {\large $y$};
        % 2. Add the tick marks and labels
        % A 'foreach' loop is used to place them at regular intervals.
        \foreach \x in {0, 2, 4, 6} {
            % Draw thick vertical lines for the ticks.
            \draw[line width=1.2pt] (\x, -0.15) -- (\x, 0.15);
            % Place the corresponding number below each tick.
            \ifnum\x=0
                \node[below=5pt] at (\x -0.2, 0) {\x};
            \else
                \node[below=5pt] at (\x, 0) {\x}; 
        \fi
        };
        % small ticks
        \foreach \x in {1, 3, 5} {
            % Draw small vertical lines for the ticks.
            \draw[line width=1.0pt] (\x, -0.1) -- (\x, 0.1);
        };
        \foreach \y in {1, 2, 3} {
            % Draw thick lines for the ticks.
            \draw[line width=1.2pt] (-0.15, \y) -- (0.15, \y);
            % Place the corresponding number below each tick.
            \ifnum \y=0
            \else
                \node[left=5pt] at (0, \y) {\y}; 
            \fi
        };
        % small ticks
        \foreach \y in {0.5, 1.5, 2.5, 3.5} {
            % Draw small vertical lines for the ticks.
            \draw[line width=1.0pt] (-0.1, \y) -- (0.1, \y);
        };
        % 3. Draw the vector end points
        \node[circle, fill=black, inner sep=1.5pt] at (P) {};
        \node[circle, fill=black, inner sep=1.5pt] at (Q) {};
        \node[circle, fill=black, inner sep=1.5pt] at (A) {};
        \node[circle, fill=black, inner sep=1.5pt] at (B) {};
        \node[circle, fill=black, inner sep=1.5pt] at (C) {};
        \node[circle, fill=black, inner sep=1.5pt] at (D) {};
        \node[circle, fill=black, inner sep=1.5pt] at (E) {};
        \draw[-Stealth, color=blue,line width=2.2pt] (P) -- (Q);
        \draw[-Stealth, color=orange,line width=2.2pt] (A) -- (B);
        \draw[-Stealth, color=orange,line width=2.2pt] (B) -- (C);
        \draw[-Stealth, color=orange,line width=2.2pt] (C) -- (D);
        \draw[-Stealth, color=orange,line width=2.2pt] (D) -- (E);
        % 4. Add the label above the point
        \node[anchor=south west, color=orange] at (0.8,1.7) {\LARGE $\vec{r}$};
        \node[anchor=south west, color=orange] at (1.8,2.2) {\LARGE $\vec{r}$};
        \node[anchor=south west, color=orange] at (2.8,2.7) {\LARGE $\vec{r}$};
        \node[anchor=south west, color=orange] at (3.8,3.2) {\LARGE $\vec{r}$};
        \node[anchor=south west, color=blue] at (3.0,0.8) {\LARGE $\vec{u}$};
    \end{tikzpicture}
    }
    \caption{Die Skalarmultiplikation, $\protect\ivec{u} = s \cdot \protect\ivec{r}$, entspricht einer wiederholten Vektoraddition. Der Vektor $\protect\ivec{r}$ wird 4-mal addiert zu $\protect\ivec{u}$ aufsummiert, \gDh $s = 4$.}\label{fig: skalarmultiplikation}
\end{figure}


\subsection{Skalarprodukt (Inneres Produkt)}\label{subsec: Skalarprodukt_InneresProdukt}
Das Skalarprodukt -- nicht zu verwechseln mit der Skalarmultiplikation -- ist das Produkt der Multiplikation zweier Vektoren, dessen Resultat ein Skalar ist. 
\begin{rememberbox}[sidebyside, sidebyside align=center, lower separated=false,righthand width=3.5cm]{Skalarmultiplikation}
    Das Skalarprodukt $(\,\cdot\,)$\footnotemark{} ist eine positiv definite symmetrische
    Bilinearform $(\ivec{u}, \ivec{v}, \ivec{w} \in V, \lambda \in \Real)$
    \begin{equation}
        \ivec{u} \cdot \ivec{v} :\, V \times V \in \Real\mComma
    \end{equation}
    mit den Eigenschaften 
    \begin{itemize}[nosep]
        \item bilinear,
        \item symmetrisch, 
        \item positiv definit.
    \end{itemize}
    \vspace{0.2cm}
    Für zwei Vektoren $\ivec{u}, \ivec{v}$ aus $\Real^n$ berechnet sich das Skalarprodukt als
    \begin{equation}
        \ivec{u} \cdot \ivec{v} = u_1 \cdot v_1 + u_2 \cdot v_2 + \,\dots + u_n \cdot v_n
    \end{equation}
    
    \tcblower
    
    \resizebox{0.99\linewidth}{!}{
    \begin{tikzpicture}
        \definecolor{lightGreen}{HTML}{B3FFB3}
        \coordinate (P) at (1,0.5);
        \coordinate (Q) at (5,2.5);
        \coordinate (S) at (2, 3.3);
        \fill[lightGreen] (P) -- ++(26:1.5) arc[start angle=26, end angle=69.5, radius=1.5] -- cycle;
        % Draw the arc with a black outline
        \draw[black, line width=0.7pt] (P) ++(26:1.5) arc[start angle=26, end angle=69.5, radius=1.5];
        % 1. Draw the horizontal axis
        \draw[-Stealth, line width=1.2pt] (P) -- (Q);
        \draw[-Stealth, line width=1.2pt] (P) -- (S);
        \node[anchor=south west, color=black] at (3.0,0.8) {\Large $\vec{a}$};
        \node[anchor=south west, color=black] at (0.7,1.5) {\Large $\vec{b}$};
        \node[anchor=south west, color=black] at (1.4,1.0) {\Large $\varphi$};
    \end{tikzpicture}
    }
\end{rememberbox}
\footnotetext{Es sollte zu keiner Verwirrung führen, dass wir hier wieder das Symbol $\cdot$ für eine andere Operation (Skalarprodukt) verwenden. Bei dem
Skalarprodukt werden zwei Vektoren miteinander multipliziert, $\ivec{u}\cdot\ivec{v}$. Damit sollte es keine Verwechslung mit der skalaren Multiplikation
geben $a\cdot b$.}
Das Skalarprodukt gibt Rückschluss auf den Winkel $\varphi$ zwischen zwei Vektoren:
\begin{equation}\label{eq: skalarprodukt_cosPhi}
    \ivec{a} \cdot \ivec{b} = |\ivec{a}| \cdot |\ivec{b}| \cdot \cos(\varphi) \mDot
\end{equation}
Damit kann es verwendet werden, um zu bestimmen, ob \textbf{zwei Vektoren normal aufeinander stehen}: $\ivec{a} \cdot \ivec{b} = 0 \Leftrightarrow \cos(\varphi)=0 \Leftrightarrow \ivec{a} \perp \ivec{b}$. Aus \cref{eq: skalarprodukt_cosPhi} kann man eine Formel zur Berechnung des Winkels $\varphi$ ableiten,
\begin{equation}
    \varphi = \arccos\left( \frac{\ivec{a} \cdot \ivec{b}}{|\ivec{a}| \cdot |\ivec{b}|}\right)\mDot
\end{equation}
Beachte, dass das Produkt im Zähler ein inneres Produkt (Skalarprodukt) zwischen zwei Vektoren ist, während das Produkt im Nenner ein Produkt zweier Skalare ist. 

Der Vollständigkeit halber sei hier erwähnt, was die oben erwähnten Eigenschaften mathematisch bedeuten.\newline
\textbf{Bilinearität:}
\begin{align*}
    (\ivec{u} + \ivec{w}) \cdot \ivec{v} &= \ivec{u} \cdot \ivec{v} + \ivec{w} \cdot \ivec{v} \\
    \ivec{u} \cdot (\ivec{v} + \ivec{w}) &= \ivec{u} \cdot \ivec{v} + \ivec{u} \cdot \ivec{w} \\
    (\lambda \cdot \ivec{u}) \cdot \ivec{v} &= \lambda \cdot (\ivec{u} \cdot \ivec{v}) \\
    \ivec{u} \cdot (\lambda \cdot \ivec{v}) &= \lambda \cdot (\ivec{u} \cdot \ivec{v})
\end{align*}
\textbf{Symmetrie:}
\begin{align*}
    \ivec{u} \cdot \ivec{v} = \ivec{v} \cdot \ivec{u}
\end{align*}
\textbf{Positive Definitheit:}
\begin{align*}
    \ivec{u} \cdot \ivec{u} &\ge 0 \\
    \ivec{u} \cdot \ivec{u} = 0 &\Leftrightarrow \ivec{u} = \ivec{0}
\end{align*}

\subsection{Kreuzprodukt (Äußeres Produkt)}\label{subsec: Kreuzprodukt_ÄußeresProdukt}
Das Kreuzprodukt ist ein Produkt zweier Vektoren, dessen Ergebnis ein Vektor ist, der auf beide Faktoren (Vektoren) orthogonal steht. 
\begin{rememberbox}[]{Kreuzprodukt}
    Das Kreuzprodukt $(\,\times\,)$ ist eine Bilinearform $(\ivec{u}, \ivec{v}, \ivec{w} \in V,\; \lambda,\mu \in \Real)$
    \begin{equation}
        \ivec{u} \times \ivec{v} :\; V \times V \in V\mComma
    \end{equation}
    mit den Eigenschaften 
    \begin{itemize}[nosep]
        \item bilinear,
        \item antikommutativ, 
        \item Jacobi-Identität (nicht-assoziativ).
    \end{itemize}
    \vspace{0.2cm}
    Für zwei Vektoren $\ivec{u}, \ivec{v}$ aus $\mathbb{R}^3$ wird es wie folgt berechnet:
    \begin{equation}
        \ivec{w} = \ivec{u} \times \ivec{v} = \begin{pmatrix} u_1 \\ u_2 \\ u_3 \end{pmatrix} \times \begin{pmatrix} v_1 \\ v_2 \\ v_3 \end{pmatrix} = \begin{pmatrix} u_2v_3 - u_3v_2 \\ -(u_1v_3 - u_3v_1) \\ u_1v_2 - u_2v_1 \end{pmatrix}\mComma
    \end{equation}
    wobei $\ivec{u}, \ivec{v} \perp \ivec{w}$.
\end{rememberbox}
Der Betrag des Kreuzproduktes entspricht der Fläche des von den Vektoren aufgespannten Parallelogramms, siehe \cref{fig: kreuzprodukt_parallelogramm}. Es gilt:
\begin{equation} 
    \ivec{u} \times \ivec{v} = \underbrace{(|\ivec{u}| \cdot |\ivec{v}| \cdot \sin(\theta))}_{A_\mathrm{Parallelogramm}} \cdot\,\ivec{n} \mDot
\end{equation}
\begin{figure}
    \centering
    \resizebox{0.6\linewidth}{!}{
    \begin{tikzpicture}[vector/.style={line width=2pt, -Stealth}]
        % Koordinaten definieren
        \def\angleTheta{60}
        \def\lengthA{6}
        \def\lengthB{4}
        \coordinate (O) at (0,0);
        \coordinate (A) at (\lengthA,0);
        \coordinate (B) at (\angleTheta:\lengthB);
        \coordinate (C) at (8, 3.464);
        \coordinate (H) at (B |- O); % Projektion von B auf die x-Achse
        \fill[purple!10] (O) -- (A) -- (C) -- (B) -- cycle;
        \draw[vector, blue] (O) -- (A) node[midway, below=2pt] {\Large $\vec{a}$};
        \draw[vector, red] (O) -- (B) node[midway, left=7pt] {\Large $\vec{b}$};
        % Die anderen beiden Seiten des Parallelogramms und Höhe
        \draw (A) -- (C);
        \draw (B) -- (C);
        \draw (B) -- (H) node[midway, right=2pt] {$b \cdot \sin\theta$};
        % Winkel Theta markieren
        \draw[black, line width=0.7pt] (O) ++(0:0.8) arc[start angle=0, end angle=60, radius=0.8];
        \node[anchor=south west, color=black] at (0.22,0.05) {\large $\theta$};
        % Rechten Winkel markieren (Quadrat mit Punkt)
        \draw (H) -- ++(0.3,0) -- ++(0,0.3) -- ++(-0.3,0);
    \end{tikzpicture}
    }
    \caption{Zwei Vektoren $\protect\ivec{a}, \protect\ivec{b}$ spannen ein Parallelogramm auf, dessen Fläche dem Betrag des Kreuzprodukts entspricht, $A_\mathrm{Parallelogramm} = \big|\protect\ivec{a} \times \protect\ivec{b}\big|$.}\label{fig: kreuzprodukt_parallelogramm}
\end{figure}
Hier ist $\theta$ der Winkel zwischen $\ivec{u}$ und $\ivec{v}$ und $\ivec{n}$ ist der Einheitsvektor, der normal auf $\ivec{u}$ und $\ivec{v}$ steht und die \gDQ{Rechte-Hand-Regel} erfüllt.
Das Kreuzprodukt ist antikommutativ: $\ivec{u} \times \ivec{v} = - \ivec{v} \times \ivec{u}$.\newline
Wir wollen auch hier die mathematischen Eigenschaften des Kreuzproduktes erläutern.\newline
\textbf{Bilinearität:}
\begin{align*}
    \ivec{u} \times (\lambda \cdot \ivec{v} + \mu \cdot \ivec{w}) &= \lambda \cdot (\ivec{u} \times \ivec{v}) + \mu \cdot (\ivec{u} \times \ivec{w}) \\
    (\lambda \cdot \ivec{u} + \mu \cdot \ivec{v}) \times \ivec{w} &= \lambda \cdot (\ivec{u} \times \ivec{w}) + \mu \cdot (\ivec{v} \times \ivec{w}) \\
    \ivec{u} \times (\lambda \cdot \ivec{v}) &= \lambda \cdot (\ivec{u} \times \ivec{v}) = (\lambda \cdot \ivec{u}) \times \ivec{v}
\end{align*}
\textbf{Antikommutativ:}
\begin{align*}
    \ivec{u} \times \ivec{v} = -\ivec{v} \times \ivec{u}
\end{align*}
\textbf{Nicht-assoziativ:}
\begin{align*}
    \ivec{u} \times (\ivec{v} \times \ivec{w}) \neq (\ivec{u} \times \ivec{v}) \times \ivec{w}
\end{align*}
\textbf{Jacobi-Identität:}
\begin{align*}
    \ivec{u} \times (\ivec{v} \times \ivec{w}) + \ivec{v} \times (\ivec{w} \times \ivec{u}) + \ivec{w} \times (\ivec{u} \times \ivec{v}) = 0
\end{align*}

\subsection{Spalten- und Zeilenvektor}\label{subsec: Spalten_vs_Zeilenvektor}
Ein \textbf{Vektor} ist ein mathematisches Objekt, das eine geordnete Liste von Zahlen, den sogenannten \textbf{Komponenten} oder \textbf{Koordinaten}, darstellt. Die Anzahl der Komponenten bestimmt die \textbf{Dimension} des Vektors. Vektoren können auf zwei grundlegende Weisen geschrieben werden: als Spaltenvektor oder als Zeilenvektor.

\subsection*{Der Spaltenvektor}
Beim Spaltenvektor werden die Komponenten untereinander in einer Spalte angeordnet. Ein allgemeiner $n$-dimensionaler Spaltenvektor $\ivec{v}$ hat die Form:
\begin{equation}
    \ivec{v} = \begin{pmatrix}
    v_1 \\
    v_2 \\
    \vdots \\
    v_n
    \end{pmatrix}\mComma
\end{equation}
Dabei sind $v_1, v_2, \dots, v_n$ die Komponenten des Vektors.

In vielen Bereichen der linearen Algebra und Physik ist der Spaltenvektor die \textbf{Standarddarstellung} eines Vektors. Wenn nur von einem \gDQ{Vektor} die Rede ist, ist meist ein Spaltenvektor gemeint.

\subsection*{Der Zeilenvektor}
Beim Zeilenvektor werden die Komponenten nebeneinander in einer Zeile geschrieben. Ein allgemeiner $m$-dimensionaler Zeilenvektor $\ivec{u}$ hat die Form:
\begin{equation}
    \ivec{u} = \begin{pmatrix} u_1, & u_2, & \dots, & u_m \end{pmatrix}\mDot
\end{equation}
Auch hier sind $u_1, u_2, \dots, u_m$ die Komponenten.

\subsection*{Die Transposition}
Die \textbf{Transposition} ist eine Operation, die einen Spaltenvektor in einen Zeilenvektor umwandelt und umgekehrt. Sie wird durch den hochgestellten Buchstaben $T$ symbolisiert.
\begin{itemize}
    \item \textbf{Transposition eines Spaltenvektors:} Das Transponieren eines Spaltenvektors ergibt den entsprechenden Zeilenvektor.
    $$
    \text{Wenn } \ivec{v} = \begin{pmatrix} v_1 \\ v_2 \\ \vdots \\ v_n \end{pmatrix}, \text{ dann ist } \ivec{v}^T = \begin{pmatrix} v_1, & v_2, & \dots, & v_n \end{pmatrix}.
    $$
    \item \textbf{Transposition eines Zeilenvektors:} Das Transponieren eines Zeilenvektors ergibt den entsprechenden Spaltenvektor.
    $$
    \text{Wenn } \ivec{u} = \begin{pmatrix} u_1, & u_2, & \dots, & u_m \end{pmatrix}, \text{ dann ist } \ivec{u}^T = \begin{pmatrix} u_1 \\ u_2 \\ \vdots \\ u_m \end{pmatrix}.
    $$
\end{itemize}
Die Transposition ist nützlich, um Schreibweise zu sparen. Anstatt einen hohen Spaltenvektor im Fließtext darzustellen, kann man den transponierten Zeilenvektor schreiben, \zB $\ivec{v} = \begin{pmatrix} v_1, v_2, \dots, v_n \end{pmatrix}^T$.\\

Zusammenfassend lässt sich sagen, dass Spalten- und Zeilenvektoren zwei unterschiedliche Schreibweisen für dasselbe grundlegende Konzept sind -- eine geordnete Liste von Zahlen. Die Wahl der Darstellung hängt vom mathematischen Kontext ab, wobei dem Spaltenvektor oft eine prominentere Rolle zukommt. 


%% -----------------------------------------------------------
%% -----------  KAPITEL  DIFFERENTIALRECHNUNG   -----------------
%% -----------------------------------------------------------

\chapter{Differentialrechnung}\label{chap: Differentialrechnung}

\section{Ableitungen spezieller Funktionen}\label{sec: Ableitung_spezFunktionen}
\subsubsection{Ableitung einer Konstanten}
Für eine Funktion $f(x) = c$ mit $c \in \Real, \Complex$ gilt:
\begin{equation}
    f'(x) = 0\mDot
\end{equation}

\subsubsection{Ableitung einer Polynomfunktion}
Für eine Funktion $f(x) = x^n$ gilt:
\begin{equation}
    f'(x) = n \cdot x^{n-1} \mDot
\end{equation}

\subsubsection{Ableitung einer Exponentialfunktion}
Für eine Funktion $f(x) = a^x$ gilt:
\begin{equation}
    f'(x) = a^x \cdot \ln(a) \mDot
\end{equation}
Die Ableitung der $e$-Funktion $f(x) = e^x$  (auch \textit{natürliche Exponentialfunktion}) ergibt:
\begin{equation}
    f'(x) = e^x \mDot
\end{equation}
Die $e$-Funktion ist ein Sonderfall einer Exponentialfunktion mit $a = e$. Der natürliche Logarithmus ist die Umkehrfunktion der $e$-Funktion: $\ln(e) = 1$.

\subsubsection{Ableitung einer Logarithmusfunktion}
Für eine Funktion $f(x) = \log_a{x}$ ergibt die Ableitung:
\begin{equation}
    f'(x) = \frac{1}{\ln(a)}\frac{1}{x} \mDot
\end{equation}
Damit ergibt die Ableitung des natürlichen Logarithmus $f(x) = \log_e{x} = \ln{x}$
\begin{equation}
    f'(x) = \frac{1}{x} \mDot
\end{equation}

\subsubsection{Ableitung von Winkelfunktionen}
Für die Winkelfunktionen (Sinus, Kosinus und Tangens) gilt:
\begin{align*}
    f(x) &= \sin(x) \Rightarrow f'(x) = \cos(x)\mComma \\
    f(x) &= \cos(x) \Rightarrow f'(x) = -\sin(x)\mComma \\
    f(x) &= \tan(x) \Rightarrow f'(x) = \frac{1}{\cos^2(x)} = 1 + \tan^2(x)\mDot
\end{align*}
Die Umkehrfunktionen haben folgende Ableitungen:
\begin{align*}
    f(x) &= \arcsin(x) \Rightarrow f'(x) = \frac{1}{\sqrt{1-x^2}}\mComma \\
    f(x) &= \arccos(x) \Rightarrow f'(x) = -\frac{1}{\sqrt{1-x^2}}\mComma \\
    f(x) &= \arctan(x) \Rightarrow f'(x) = \frac{1}{1+x^2}\mDot
\end{align*}

\section{Ableitungsregeln}\label{sec: Ableitungsregeln}
Die folgenden Ableitungsregeln lassen sich sehr einfach aus der Grenzwertbildung herleiten. Im Folgenden seien die Funktionen $g(x), h(x)$ differenzierbare Funktionen.  
\begin{rememberbox}[]{Konstante Faktoren}
Sei $f(x) = c\cdot g(x)$, wobei $c \in \Real, \Complex$. Für die Ableitung von $f(x)$ gilt:
\begin{equation}
f'(x) = c \cdot g'(x) \mDot
\end{equation}
\end{rememberbox}
Diese Regel ist eine direkte Konsequenz aus der Produktregel (unten) und der Tatsache, dass die Ableitung von Konstanten verschwindet. Konstante Faktoren werden demnach nicht abgeleitet.\newline
\textit{Beispiel:}
\begin{itemize}
    \item Sei $f(x) = 8 x^2$. Mit $c = 8$ und $g(x) = x^2$ ergibt sich die Ableitung $g'(x) = 2x$. \newline
    Somit ist $f'(x) = 8\cdot (2x) = 16 x$.
    \item Sei $f(x) = k\cdot e^x$. Mit $c = k$ ($k$ hängt nicht von $x$ ab) und $g(x) = e^x$ ergibt sich die Ableitung $g'(x) = e^x$.\newline
    Somit ist $f'(x) = k\cdot e^x$.
\end{itemize}
%\subsubsection{Summenregel}
\begin{rememberbox}[]{Summenregel}
Für die Ableitung der Summenfunktion $f(x) = g(x) + h(x)$ gilt:
\begin{equation}
f'(x) = g'(x) + h'(x) \mDot
\end{equation}
\end{rememberbox}
\textit{Beispiel:}
\begin{itemize}
    \item Sei $f(x) = x^2 + \sin(x)$. Mit $g(x) = x^2$ und $h(x) = \sin(x)$ ergeben sich die Ableitungen $g'(x) = 2x$ und $h'(x) = \cos(x)$. \newline
    Somit ist $f'(x) = 2x + \cos(x)$.
    \item Sei $f(x) = e^x + 5$. Mit $g(x) = e^x$ und $h(x) = 5$ ergeben sich die Ableitungen $g'(x) = e^x$ und $h'(x) = 0$.\newline
    Somit ist $f'(x) = e^x + 0 = e^x$.
\end{itemize}

%\subsubsection{Produktregel}
\begin{rememberbox}{Produktregel}
Für die Ableitung einer Produktfunktion $f(x) = g(x) \cdot h(x)$ gilt:
\begin{equation}
    f'(x) = g'(x)h(x) + g(x)h'(x)\mDot
\end{equation}
\end{rememberbox}
\textit{Beispiel:}
\begin{itemize}
    \item Sei $f(x) = x^3 \cdot \cos(x)$. Mit $g(x) = x^3$ und $h(x) = \cos(x)$ ergeben sich die Ableitungen $g'(x) = 3x^2$ und $h'(x) = -\sin(x)$.\newline
    Somit ist $f'(x) = 3x^2 \cos(x) - x^3 \sin(x)$.
    \item Sei $f(x) = (x^2+1) \cdot e^x$. Mit $g(x) = x^2+1$ und $h(x) = e^x$ ergeben sich die Ableitungen $g'(x) = 2x$ und $h'(x) = e^x$.\newline
    Somit ist $f'(x) = 2x \cdot e^x + (x^2+1) \cdot e^x = e^x(x^2+2x+1) = e^x(x+1)^2$.
\end{itemize}

%\subsubsection{Quotientenregel}
\begin{rememberbox}{Quotientenregel}
Sei $f(x)$ eine Quotientenfunktion $f(x) = \frac{g(x)}{h(x)}$, wobei $h(x) \neq 0$. Dann gilt für die Ableitung von $f(x)$:
\begin{equation}
    f'(x) = \frac{g'(x)h(x) - g(x)h'(x)}{[h(x)]^2}\mDot
\end{equation}
\end{rememberbox}
\textit{Beispiel:}
\begin{itemize}
    \item Sei $f(x) = \frac{\sin(x)}{x}$. Mit $g(x) = \sin(x)$ und $h(x) = x$ ergeben sich die Ableitungen $g'(x) = \cos(x)$ und $h'(x) = 1$.\newline
    Somit ist $f'(x) = \frac{x\cos(x) - \sin(x)}{x^2}$.
    \item Sei $f(x) = \frac{x^2+3}{2x-1}$. Mit $g(x) = x^2+3$ und $h(x) = 2x-1$ ergeben sich die Ableitungen $g'(x) = 2x$ und $h'(x) = 2$.\newline
    Somit ist $f'(x) = \frac{2x(2x-1) - 2(x^2+3)}{(2x-1)^2} = \frac{4x^2-2x-2x^2-6}{(2x-1)^2} = \frac{2x^2-2x-6}{(2x-1)^2}$.
\end{itemize}

% \subsubsection{Kettenregel}
\begin{rememberbox}{Kettenregel}
Sei $f(x) = h(g(x))$ eine verschachtelte (verkettete) Funktion, bei der die äußere Funktion $h$ nach der inneren Funktion $g$ abgeleitet wird.
\begin{equation}
    f'(x) = h'(g(x)) \cdot g'(x) = \frac{\dd h}{\dd g}\frac{\dd g}{\dd x}\mDot
\end{equation}
\end{rememberbox}
\textit{Beispiel:}
\begin{itemize}
    \item Sei $f(x) = (3x+1)^4$. Mit der äußeren Funktion $h(u) = u^4$ und der inneren Funktion $g(x) = 3x+1$ ergeben sich die Ableitungen $h'(u) = 4u^3$ und $g'(x) = 3$.\newline
    Somit ist $f'(x) = 4(3x+1)^3 \cdot 3 = 12(3x+1)^3$.
    \item Sei $f(x) = \cos(x^2)$. Mit der äußeren Funktion $h(u) = \cos(u)$ und der inneren Funktion $g(x) = x^2$ ergeben sich die Ableitungen $h'(u) = -\sin(u)$ und $g'(x) = 2x$.\newline
    Somit ist $f'(x) = -\sin(x^2) \cdot 2x = -2x\sin(x^2)$.
\end{itemize}

\section{Partielle Ableitung}\label{sec: Partielle_Ableitung}
Die sogenannte partielle Ableitung ist die Ableitung einer Funktion mit mehreren Argumenten nach einem dieser Argumente. Die Werte der übrigen Argumente werden konstant gehalten.

Sei $f: (x,y) \to \Real$ eine reelle Funktion in zwei Variablen $x$ und $y$. Die partielle Ableitung von $f$ nach $x$ an der Stelle $(x,y) = (x_1,y_1)$ ist
\begin{equation}
    \frac{\partial f}{\partial x}(x_1, y_1) \defeq \lim_{h \rightarrow 0} \frac{f(x_1+ h,y_1)-f(x_1,y_1)}{h} \mDot
\end{equation}
Die Auswertestelle muss dabei nicht angegeben werden -- dann erhält man die allgemeine partielle Ableitung an jeder beliebigen Stelle. \\

\textit{Beispiel:}
Die Funktion
$$ f(x,y) = \cos(x) + \sin(x)$$
ist in \cref{fig: partielleAbleitung_sattel} dargestellt. Die partiellen Ableitungen nach $x$ und $y$ ergeben
$$ \frac{\partial f}{\partial x} = -\sin(x) + \underbrace{\frac{\partial (\sin(y))}{\partial x}}_{=0}$$ 
so ändert sich die Funktion entlang der $x$-Achse.
$$ \frac{\partial f}{\partial y} = \underbrace{\frac{\partial (\cos(x))}{\partial y}}_{=0} + \cos(y)$$ so ändert sich die Funktion entlang der $y$-Achse.
\begin{figure}[h!]
    \centering
    \includegraphics[width=0.5\linewidth]{Bilder/Kapitel_MathEinschübe/partielleAbleitungSattel.png}
    \caption{Die Funktion $f(x,y)$ und die beiden partiellen Ableitungen entlang der $x$- und $y$-Achse (roter und grüner Pfeil).}\label{fig: partielleAbleitung_sattel}
\end{figure}

\section{Totales Differential}\label{sec: Totales_Differential}
Das sogenannte totale Differential (auch vollständiges Differential) bezeichnet das Differential $\dd f$ einer Funktion, insbesondere bei Funktionen mehrerer Variablen.

Sei $f: (x,y) \to \Real$ eine reelle Funktion in zwei Variablen $x$ und $y$, dann ist das totale Differential
\begin{equation}
    \dd f = \frac{\partial f}{\partial x} \dd x + \frac{\partial f}{\partial y} \dd y.
\end{equation}
Das totale Differential beinhaltet demnach die Änderungen aller Variablen. \Cref{fig: totalesDifferentialVisualisierung} zeigt, wie sich die totale Änderung $\dd f$ aus den beiden Änderungen $\frac{\partial f}{\partial x}\dd x$ und $\frac{\partial f}{\partial y}\dd y$ entlang der $x$- und $y$-Achse zusammensetzt.

\begin{figure}[h!]
    \centering
    \includegraphics[width=0.7\linewidth]{Bilder/Kapitel_MathEinschübe/totalesDifferential.png}
    \caption{Visualisierung des totalen Differentials.}\label{fig: totalesDifferentialVisualisierung}
\end{figure}

In der Physik hat man es häufig mit einem Fall zu tun, bei dem die Ortskoordinaten $x$ und $y$ von der Zeit abhängen (implizite Zeitabhängigkeit) und die Funktion selbst direkt von der Zeit abhängt (explizite Zeitabhängigkeit):
\begin{equation}
    g: (x(t), y(t), t) \to \Real.    
\end{equation}
Ist man nun an der totalen Ableitung von $g$ nach der Zeit interessiert, muss man die Kettenregel beachten
\begin{equation}
    \frac{\dd g(x(t), y(t), t)}{\dd t} = \frac{\partial g}{\partial x}\frac{\dd x}{\dd t} + \frac{\partial g}{\partial y}\frac{\dd y}{\dd t} + \frac{\partial g}{\partial t}.
\end{equation}
Da die Koordinaten selbst wieder von der Zeit abhängen, $x(t), y(t)$, tritt hier noch die explizite Ableitung der Koordinaten nach der Zeit auf, $\frac{\dd x}{\dd t}$, $\frac{\dd y}{\dd t}$.


\newpage
\chapter{Integralrechnung}\label{chap: Integralrechnung}
\section{Bestimmtes Integral}\label{sec: Bestimmtes_Integral}
Die (bestimmte) Integralrechnung erlaubt es, Flächen (oder Volumina) zu berechnen, die durch gekrümmte Linien (oder Flächen) begrenzt sind. Es gibt zwei Arten von Integralen: das \textbf{bestimmte Integral} und das \textbf{unbestimmte Integral}.

Zur funktionalen Herleitung des Integrals betrachten wir zunächst die Fläche unter der Kurve $f(t)$ im Intervall $[t_{1},t_{2}]$ in \cref{fig: integral_ft_Rectangles}. Dazu unterteilen wir die Kurve in Teilschritte der Länge $\Delta t_{1},\dots,\Delta t_{n}$. Um den Flächeninhalt anzunähern, verwenden wir Rechtecke der Höhe $f_{i}$, wobei $f_{i}$ jeweils der Funktionswert in der Mitte des Intervalls $\Delta t_i$ ist.

        
\begin{figure}[h!]
    \centering
    \begin{minipage}[b]{0.6\textwidth}
        \centering
        \includegraphics[width=\linewidth]{Bilder/Kapitel_MathEinschübe/Integral_ft_Rectangles.png} 
        \caption{Annäherung der Fläche unter einer Kurve $f(t)$ durch eine Summe von Rechtecksflächen. Jedes Rechteck hat die Breite $\Delta t_i$ und die Höhe $f_i$.}\label{fig: integral_ft_Rectangles}
    \end{minipage}
    \hfill
    \begin{minipage}[b]{0.3\textwidth}
        \centering
        % The TikZ picture is resized to the full width of the subfigure
        \resizebox{\linewidth}{!}{
            \begin{tikzpicture}
            % --- Horizontale Achse und Beschriftungen ---
            \draw[thick, figGrayBorder] (-0.4,0) -- (4.5,0);
            \foreach \x in {0,1,2,3,4} {
                %\draw[line width=2pt, figGrayBorder] (\x, 0.1) -- (\x, -0.1);
                \draw[line width=0.5pt, black] (\x, -0.8) -- (\x, -0.3);
            }
            \node[below=3pt] at (0.5, -0.15) {$\cdot$};
            \node[below=3pt] at (1.5, -0.15) {$\cdot$};
            \node[below=3pt] at (2.5, -0.15) {$\Delta t_i$};
            \node[below=3pt] at (3.5, -0.15) {$\cdot$};
            % --- Hintergrund-Rechtecke (Säulendiagramm) ---
            \draw[fill=figGrayFill, draw=figGrayBorder] (0,0) rectangle (1,1.8);
            \draw[fill=figGrayFill, draw=figGrayBorder] (1,0) rectangle (2,2.4);
            \draw[fill=figOrange, opacity=0.2] (2,0) rectangle (3,3.4);
            \draw[draw=figGrayBorder, line width=1.4pt] (2,0) rectangle (3,3.4);
            \draw[fill=figGrayFill, draw=figGrayBorder] (3,0) rectangle (4,4.8);
            %\draw[line width=1.5pt, dashed] (2,0) rectangle (3,3.4);
            % --- Die Funktion (orangefarbene Kurve) ---
            \draw[domain=-0.3:4, figOrange, samples=30, smooth, line width=2.2pt] 
                plot (\x, {0.2*(\x+0.5)^2 + 1.6});
            \begin{scope}
                \clip (2,0) rectangle (3,3.4);
                \fill[figGreen, opacity=0.85] plot[domain=0:4, samples=30, smooth] (\x, {0.2*(\x+0.5)^2 + 1.6});
            \end{scope}
            \begin{scope}
                \clip (2,3.4) rectangle (3,5.4);
                \fill[figRed, opacity=0.85] plot[domain=2:3, samples=30, smooth] (\x, {0.2*(\x+0.5)^2 + 1.6}) -- (3,3.4) -- (2,3.4) -- cycle;
            \end{scope}
            \end{tikzpicture}
        }
        \caption{Der Fehler, der bei der Mittelauswertung von $f(t)$ im Intervall $\Delta t_i$ gemacht wird.}\label{fig: zoom_integral_fehler}
    \end{minipage}  
\end{figure}
Als Approximation für die Fläche unter der Kurve $f(t)$ schreiben wir dann:
\begin{equation}
    \text{Fläche} \approx \sum_{i=1}^{n}f_{i}\cdot\Delta t_{i} \mDot
\end{equation}
Der Fehler, den wir bei der Mittelauswertung von $f(t)$ im Intervall $\Delta t_i$ machen, ist in \cref{fig: zoom_integral_fehler} dargestellt. Er besteht darin, dass die grüne Fläche Teil der Summe ist, obwohl sie gar nicht unter der Kurve liegt. Umgekehrt ist die rote Fläche nicht Teil der berechneten Fläche, obwohl sie unter der Kurve liegt. Die Hypothese ist zunächst, dass der Fehler für immer kleiner werdende Intervallgrößen, $\Delta t_i \rightarrow 0$, gegen $0$ geht.

\section{Einschub: Obersumme und Untersumme}\label{sec: Obersumme_vs_Untersumme}
Anstatt die Funktion in der Mitte des Intervalls auszuwerten, kann man auch immer den kleinsten Wert im jeweiligen Intervall $\Delta t_{i}$ (Untersumme) oder den größten Wert im Intervall $\Delta t_{i}$ (Obersumme) wählen.
\begin{figure}[h!]
    \centering
    \begin{minipage}[b]{0.48\textwidth}
        \centering
        \includegraphics[width=\linewidth]{Bilder/Kapitel_MathEinschübe/obersumme.png}
        \caption{Wählt man den jeweils größten Wert im Teilintervall $[t_{i-1}, t_i]$, erhält man die Obersumme.}\label{fig: obersumme}
    \end{minipage}
    \hfill
    \begin{minipage}[b]{0.48\textwidth}
        \centering
        \includegraphics[width=\linewidth]{Bilder/Kapitel_MathEinschübe/untersumme.png}
        \caption{Wählt man den jeweils kleinsten Wert im Teilintervall $[t_{i-1}, t_i]$, erhält man die Untersumme.}\label{fig: untersumme}
    \end{minipage}
\end{figure}
\subsubsection{Obersumme}
Wählt man den jeweils größten Wert im Teilintervall $[t_{i-1},t_{i}]$, erhält man die Obersumme:
\begin{equation}
    f_{i} = \sup_{t\in[t_{i-1},t_{i}]}f(t) \text{}
\end{equation}
\Cref{fig: obersumme} zeigt exemplarisch, wie die Höhe der einzelnen Rechtecke der Obersumme jeweils dem größten Wert im Intervall entsprechen. Das Supremum (sup) ist die kleinste obere Schranke einer Menge, das heißt, es ist der kleinste Wert, der größer oder gleich jedem Element der Menge ist. 
\subsubsection{Untersumme}
Wählt man den jeweils kleinsten Wert im Teilintervall $[t_{i-1},t_{i}]$, erhält man die Untersumme.
\begin{equation}
    f_{i} = \inf_{t\in[t_{i-1},t_{i}]}f(t) \text{}
\end{equation}
\Cref{fig: untersumme} zeigt exemplarisch, wie die Höhe der einzelnen Rechtecke der Untersumme jeweils dem kleinsten Wert im Intervall entsprechen. Das Infimum (inf)ist die größte untere Schranke einer Menge, also der größte Wert, der kleiner oder gleich jedem Element der Menge ist.

\section{Riemann-Integral}\label{sec: Riemann-Integral}
\begin{rememberbox}[]{Riemann-Integral}
    Eine Funktion ist Riemann-integrierbar (eigentlich Darboux-integrierbar), wenn Obersumme und Untersumme für eine Verkleinerung der Intervalle ($\Delta t_{i}\rightarrow 0$) gegen einen gemeinsamen (endlichen) Wert $A$ konvergieren. Diesen Wert $A$ nennt man das \textbf{Riemann-Integral} von $f$ über dem Intervall $[a,b]$ und man schreibt dafür:
    \begin{equation}
        A = \int_{a}^{b}f(t)\,\dd t \mDot
    \end{equation}
    Der Wert $A$ entspricht der Fläche zwischen der Kurve $f(t)$ und der Abszisse ($x$-Achse).
\end{rememberbox}
Das Integral wird als \textbf{bestimmtes Integral} bezeichnet, wenn die Grenzen $a$ und $b$ definiert sind und die Fläche keine Funktion der Grenzen ist.

\begin{figure}[h!]
    \centering
    % Platzhalter für das Bild auf Seite 3
    \includegraphics[width=0.5\textwidth]{Bilder/Kapitel_MathEinschübe/positiveNegativeAreas_Integral.png}
    \caption{Flächen unterhalb der $x$-Achse ergeben einen negativen Beitrag zum Integral, da dort $f(t) < 0$ ist. Flächen oberhalb der $x$-Achse sind positiv.}\label{fig: pos_neg_flächen_integral}
\end{figure}

An der Definition des Riemann-Integrals erkannt man, dass Flächen, die unter der $x$-Achse liegen, negativ gezählt werden, da dort $f(t) < 0$ ist. In \cref{fig: pos_neg_flächen_integral} ist schematisch dargstellt, welches Vorzeichen die einzelnen Teilflächen unter der Kurve haben. Ist dieser Effekt unerwünscht -- wenn beispielsweise die gesamte Fläche unabhängig vom Vorzeichen berechnet werden soll -- teilt man das Integral in Abschnitte auf und addiert die Beträge der Flächen. Es gilt nämlich allgemein
\begin{equation}
    A = \int_{a}^{b}f(t)\,\dd t = \int_{a}^{t_{(1)}}f(t)\,\dd t + \int_{t_{(1)}}^{t_{(2)}}f(t)\,\dd t + \dots + \int_{t_{(n)}}^{b}f(t)\,\dd t \mDot
\end{equation}
Das Riemann-Integral kann also in beliebig viele Teilabschnitte unterteilt werden, sofern die Teilabschnitte das gesamte Intervall $[a,b]$ abdecken.

\section{Unbestimmtes Integral und Stammfunktion}\label{sec: unbestimmtesIntegral_Stammfunktion}
\begin{rememberbox}[]{Stammfunktion}
    Die \textbf{Stammfunktion $F$} einer reellen Funktion $f$ ist eine differenzierbare Funktion, deren Ableitung $F'$ mit $f$ übereinstimmt: 
    \begin{equation}
        F'(x) = f(x), \quad \forall x \in D.
    \end{equation}
\end{rememberbox}
Ist $F$ eine Stammfunktion von $f$, so ist auch $F + C$ für eine beliebige Konstante $C\in\Real,\Complex$ eine Stammfunktion von $f$. Die Konstante $C$ wird Integrationskonstante genannt und muss bei jedem unbestimmten Integral berücksichtigt werden.

\begin{importantbox}{Hauptsätze der Differential- und Integralrechnung}
\begin{itemize}
    \item \textbf{Erster Hauptsatz:} Besitzt die Funktion $F:[a,b]\rightarrow\mathbb{R}$ eine Riemann-integrierbare Ableitung $f=F'$, dann gilt:
    \begin{equation}
        \int_{a}^{b}f(x)\,\dd x = \int_{a}^{b}F'(x)\,\dd x = F(b)-F(a) \mDot
    \end{equation}
    \item \textbf{Zweiter Hauptsatz:} Jede auf $[a, b]$ stetige Funktion $f$ besitzt eine Stammfunktion auf $[a, b]$. Eine solche Stammfunktion ist gegeben durch:
    \begin{equation}
        F(x) \defeq \int_{a}^{x}f(t)\,\dd t \quad \text{für } x \in [a,b] \mDot
    \end{equation}
\end{itemize}
\end{importantbox}


\section{Integration spezieller Funktionen}\label{sec: Integration_spezFunktionen}
Im Folgenden kann die Konstante $c \in \Real$ oder $c \in \Complex$.
\subsubsection{Integration einer Konstanten}
Für eine Funktion $f(x) = c$ ergibt das Integral
\begin{equation}
    \int c\,\dd x = c\cdot x + C.
\end{equation}

\subsubsection{Integration einer Polynomfunktion}
Für eine Funktion $f(x) = x^n$ ergibt das unbestimmte Integral
\begin{equation}
    \int x^{n} \,\dd x = \frac{x^{n+1}}{n+1} + C \mDot
\end{equation}
Ein Sonderfall ist die Funktion $f(x) = \frac{1}{x} = x^{-1}$
\begin{equation}
    \int \frac{1}{x} \,\dd x = c \ln{|x|} + C \mDot
\end{equation}

\subsubsection{Integration einer Exponentialfunktion}
Für eine Funktion $f(x) = a^x$ ergibt das unbestimmte Integral
\begin{equation}
    \int a^{x}\,\dd x = \frac{a^{x}}{\ln{a}} + C \mDot
\end{equation}
Im Falle der natürlichen Exponentialfunktion oder $e$-Funktion findet man
\begin{equation}
    \int e^{x}\,\dd x = e^{x} + C\mComma
\end{equation}
wobei wieder $\ln{e} = 1$ verwendet wurde.

\subsubsection{Integration einer Logarithmusfunktion}
Für eine Funktion $f(x) = \log_a{x}$ ergibt das unbestimmte Integral
\begin{equation}
    \int \log{x}\,\dd x = \frac{1}{\ln{a}}\left(x\ln{x} - x\right) + C\mComma
\end{equation}
und somit ist das Integral des natürlichen Logarithmus
\begin{equation}
    \int \ln{x}\,\dd x = x\ln{x} - x + C\mDot
\end{equation}

\subsubsection{Integration von Winkelfunktionen}
Die Winkelfunktionen haben folgende unbestimmten Integrale
\begin{align*}
    \int \sin(x) \,\dd x &= -\cos(x) + C\mComma \\
    \int \cos(x) \,\dd x &= \sin(x) + C\mComma \\
    \int \tan(x) \,\dd x &= -\ln|\cos(x)| + C \mDot
\end{align*}
Die Umkehrfunktionen haben folgende unbestimmte Integrale:
\begin{align*}
    \int \arcsin(x) \,\dd x &= x \cdot \arcsin(x) + \sqrt{1-x^2} + C\mComma \\
    \int \arccos(x) \,\dd x &= x \cdot \arccos(x) - \sqrt{1-x^2} + C\mComma \\
    \int \arctan(x) \,\dd x &= x \cdot \arctan(x) - \frac{1}{2} \ln(1+x^2) + C\mDot
\end{align*}        



\section{Integrationsregeln}\label{sec: Integrationsregeln}
Im Folgenden seien die Funktionen $f(x), g(x), h(x)$ differenzierbare Funktionen

\subsubsection{Konstante Faktoren}
\begin{rememberbox}[]{Konstante Faktoren}
Sei $\lambda \in \Real, \Complex$, dann
    \begin{equation}
        \int \lambda\cdot f(x)\,\dd x = \lambda\cdot\int f(x)\,\dd x \mDot
    \end{equation}
Konstante Faktoren (unabhängig von der Integrationsvariable) dürfen aus dem Integral gezogen werden.
\end{rememberbox}
\textit{Beispiele:}
\begin{itemize}
    \item $ \int 5x^2 \,\dd x = 5 \int x^2 \,\dd x = 5 \cdot \frac{x^3}{3} + C = \frac{5}{3}x^3 + C $
    \item $ \int -2\cos(x) \,\dd x = -2 \int \cos(x) \,\dd x = -2\sin(x) + C $
\end{itemize}


\subsubsection{Summenfunktion}
\begin{rememberbox}[]{Summenfunktion}
Für das Integral der Summenfunktion $f(x) = g(x) + h(x)$ gilt:
    \begin{equation}
        \int \left[g(x) + h(x)\right]\,\dd x = \int g(x)\,\dd x+\int h(x) \,\dd x \mDot
    \end{equation}
\end{rememberbox}
\textit{Beispiele:}
\begin{itemize}
    \item $ \int (x^3 + \sin(x)) \,\dd x = \int x^3 \,\dd x + \int \sin(x) \,\dd x = \frac{x^4}{4} - \cos(x) + C $
    \item $ \int (e^x - \frac{1}{x}) \,\dd x = \int e^x \,\dd x - \int \frac{1}{x} \,\dd x = e^x - \ln|x| + C $
\end{itemize}

\subsubsection{Partielle Integration}
\begin{rememberbox}[]{Partielle Integration}
Die partielle Integration ist die Umkehrung der Produktregel der Differenzialrechnung und wird verwendet, um das Integral von Produkten zweier Funktionen zu berechnen.
    \begin{equation}
    \int f'(x)\cdot g(x)\,\dd x = f(x)\cdot g(x)-\int f(x)\cdot g'(x)\,\dd x
        \end{equation}
\end{rememberbox}
\textit{Beispiele:}
\begin{itemize}
    \item $\int x \cdot e^x \,\dd x$ \\
    Setze $g(x) = x$ und $f'(x) = e^x$. Dann ist $g'(x) = 1$ und $f(x) = e^x$.
    $$ \int x \cdot e^x \,\dd x = x \cdot e^x - \int 1 \cdot e^x \,\dd x = x \cdot e^x - e^x + C \mDot$$
    \item $\int \ln(x) \,\dd x$ \\
    Setze $g(x) = \ln(x)$ und $f'(x) = 1$. Dann ist $g'(x) = \frac{1}{x}$ und $f(x) = x$.
    $$ \int 1 \cdot \ln(x) \,\dd x = x \cdot \ln(x) - \int x \cdot \frac{1}{x} \,\dd x = x \cdot \ln(x) - \int 1 \,\dd x = x \cdot \ln(x) - x + C \mDot$$
\end{itemize}


\subsubsection{Substitution}
\begin{rememberbox}[]{Substitution}
Die Substitutionsregel ist die Umkehrung der Kettenregel und vereinfacht ein Integral durch den Austausch der Integrationsvariable:
    \begin{equation}
        \int_{\phi(a)}^{\phi(b)}f(x)\,\dd x = \int_{a}^{b}f(\phi(t))\cdot\phi'(t)\,\dd t
    \end{equation}
\end{rememberbox}
\textit{Beispiele:}
\begin{itemize}
    \item $\int 2x \cdot \cos(x^2) \,\dd x$ \\
    Substituiere $t = x^2$. Dann ist $\frac{dt}{dx} = 2x$, also $dt = 2x \,\dd x$.
    $$ \int \cos(x^2) \underbrace{2x\,\dd x}_{\dd t} = \int \cos(t) \,\dd t = \sin(t) + C = \sin(x^2) + C \mDot$$
    \item $\int (3x+5)^4 \,\dd x$ \\
    Substituiere $t = 3x+5$. Dann ist $\frac{dt}{dx} = 3$, also $dx = \frac{dt}{3}$.
    $$ \int (3x+5)^4 \,\dd x = \int t^4 \cdot \frac{1}{3} \,dt = \frac{1}{3} \int t^4 \,dt = \frac{1}{3} \cdot \frac{t^5}{5} + C = \frac{(3x+5)^5}{15} + C \mDot$$
\end{itemize}


\section{Differentialgleichungen}\label{sec: differentialgleichungen}
Eine \textbf{Differentialgleichung} (DGL) ist eine mathematische Gleichung, die eine oder mehrere Funktionen und ihre Ableitungen enthält. Im Wesentlichen beschreibt eine DGL die Beziehung zwischen einer sich ändernden Größe und ihrer Änderungsrate.

Stellen wir uns eine Funktion $f(x)$ vor. Eine DGL verknüpft $x$, $f(x)$, die erste Ableitung $f'(x) = \frac{\dd f}{\dd x}$, die zweite Ableitung $f''(x) = \frac{\dd^2 f}{\dd x^2}$ und so weiter.

\subsubsection{Bedeutung von Differentialgleichungen}
Differentialgleichungen sind das Herzstück vieler wissenschaftlicher und technischer Disziplinen. Sie ermöglichen es uns, dynamische Prozesse zu modellieren, bei denen sich Größen im Laufe der Zeit oder im Raum ändern. Anwendungsgebiete sind vielfältig:
\begin{itemize}
    \item \textbf{Physik:} Bewegung von Objekten, Schwingungen, Wärmeleitung, Quantenmechanik.
    \item \textbf{Ingenieurwesen:} Regelungstechnik, elektrische Schaltkreise, Strömungsdynamik.
    \item \textbf{Biologie:} Populationswachstum, Ausbreitung von Krankheiten.
    \item \textbf{Wirtschaft:} Finanzmodelle, Marktdynamiken.
\end{itemize}

\subsubsection{Klassifizierung von Differentialgleichungen}
Man klassifiziert DGLs nach drei Hauptkriterien:

\begin{enumerate}
    \item \textbf{Gewöhnlich vs. Partiell:}
    \begin{itemize}
        \item \textbf{Gewöhnliche Differentialgleichungen (GDGL):} Enthalten nur Funktionen einer einzigen unabhängigen Variable und deren Ableitungen. Beispiel: $f'(x) + 2f(x) = \sin(x)$.
        \item \textbf{Partielle Differentialgleichungen (PDGL):} Enthalten Funktionen von mehreren unabhängigen Variablen und deren partielle Ableitungen.
    \end{itemize}
    \item \textbf{Ordnung:} Die Ordnung einer DGL ist die Ordnung der höchsten vorkommenden Ableitung.
    \begin{itemize}
        \item DGL 1. Ordnung: $f'(x) + 3 f(x) = 0$.
        \item DGL 2. Ordnung: $f''(x) + 4f'(x) + 3f(x) = x^2$.
    \end{itemize}
    \item \textbf{Linearität:} Eine DGL ist linear, wenn die gesuchte Funktion und ihre Ableitungen nur in der ersten Potenz und nicht miteinander multipliziert vorkommen.
\end{enumerate}

\subsection{Beispiel aus der Physik: Bewegung}
Ein klassisches Beispiel zur Veranschaulichung ist die geradlinige Bewegung eines Objekts entlang einer Dimension -- somit kann man das Problem mit Skalaren beschreiben, ohne Vektoren. Die physikalischen Größen Ort, Geschwindigkeit und Beschleunigung sind durch Ableitungen miteinander verknüpft.

Sei $s(t)$ der Ort eines Objekts zur Zeit $t$.
\begin{itemize}
    \item Die \textit{Geschwindigkeit} $v(t)$ ist die erste Ableitung des Ortes nach der Zeit:
    $$ v(t) = s'(t) = \frac{\dd s}{\dd t} $$
    \item Die \textit{Beschleunigung} $a(t)$ ist die erste Ableitung der Geschwindigkeit bzw. die zweite Ableitung des Ortes nach der Zeit:
    $$ a(t) = v'(t) = s''(t) = \frac{\dd v}{\dd t} = \frac{\dd^2 s}{\dd t^2} $$
\end{itemize}

\subsubsection{Fall 1: Konstante Beschleunigung}
Betrachten wir den freien Fall (ohne Luftwiderstand). Die Beschleunigung $a(t) = \const$ ist konstant und entspricht der Erdbeschleunigung $a = g \approx \SI{9.81}{\meter\per\second\squared}$. Nehmen wir an, die positive Richtung sei nach unten.
Die DGL für die Geschwindigkeit lautet:
$$ v'(t) = g $$
Dies ist eine sehr einfache DGL. Wir lösen sie durch direkte Integration:
$$ \int v'(t) \, \dd t = \int g \, \dd t \implies v(t) = g \cdot t + C_1 $$
Die Integrationskonstante $C_1$ ist die Anfangsgeschwindigkeit $v(0)$. Also:
$$ v(t) = g \cdot t + v_0 $$
Um nun den Ort $s(t)$ zu finden, lösen wir die DGL $s'(t) = v(t)$:
$$ s'(t) = g \cdot t + v_0 $$
Nochmalige Integration liefert:
$$ \int s'(t) \, \dd t = \int (g \cdot t + v_0) \, \dd t \implies s(t) = \frac{1}{2}g \cdot t^2 + v_0 \cdot t + C_2 $$
Die Integrationskonstante $C_2$ ist der Anfangsort $s(0)$, also $s_0$. Das Ergebnis ist die bekannte Formel für die gleichmäßig beschleunigte Bewegung:
$$ s(t) = \frac{1}{2}gt^2 + v_0 t + s_0 $$

\section{Lösungsmethoden für DGLs 1. Ordnung}
Die Lösung einer DGL ist eine Funktion, die die Gleichung erfüllt. Oft gibt es eine ganze Schar von Lösungen (abhängig von Integrationskonstanten). Eine eindeutige Lösung erhält man durch Angabe von \textbf{Anfangs- oder Randbedingungen} (\zB $f(x_0) = f_0 = 4.2$).

\subsection{Trennung der Variablen}
Eine der grundlegendsten Methoden ist die Trennung der Variablen. Sie ist anwendbar, wenn sich die DGL in die Form
$$ y' = f(x) \cdot g(y) $$
bringen lässt. Man schreibt $y' = \frac{\dd y}{\dd x}$ und "trennt" dann die Variablen $x$ und $y$ auf unterschiedliche Seiten der Gleichung:
$$ \frac{\dd y}{g(y)} = f(x) \, \dd x $$
Anschließend integriert man beide Seiten:
$$ \int \frac{\dd y}{g(y)} = \int f(x) \, \dd x + C $$

\subsubsection{Beispiel: Radioaktiver Zerfall}
Die Zerfallsrate eines radioaktiven Materials ist proportional zur vorhandenen Menge $N(t)$.
$$ N'(t) = -\lambda \cdot N(t) \quad (\lambda > 0 \text{ ist die Zerfallskonstante}) $$
Wir trennen die Variablen:
$$ \frac{\dd N}{N} = -\lambda \, \dd t $$
Integration beider Seiten:
$$ \int \frac{1}{N} \, \dd N = \int -\lambda \, \dd t $$
$$ \ln|N| = -\lambda t + C $$
Um $N$ zu isolieren, wenden wir die Exponentialfunktion an:
$$ |N| = e^{-\lambda t + C} = e^C \cdot e^{-\lambda t} $$
Da $N$ eine Menge ist ($N>0$) und $e^C$ eine positive Konstante ist, können wir dies als $N_0 = e^C$ (die Anfangsmenge bei $t = 0$) schreiben.
$$ N(t) = N_0 \cdot e^{-\lambda t} $$

\section{Ausblick: DGLs höherer Ordnung}
Lineare Differentialgleichungen zweiter Ordnung mit konstanten Koeffizienten haben die Form:
$$ a\cdot y'' + b\cdot y' + c\cdot y = 0 $$
Der Lösungsansatz hierfür ist $y(x) = e^{rx}$. Setzt man diesen Ansatz in die DGL ein, erhält man die \textbf{charakteristische Gleichung}:
$$ ar^2 + br + c = 0 $$
Die Lösungen dieser quadratischen Gleichung für $r$ bestimmen die Form der allgemeinen Lösung für $y(x)$. Dies ist ein grundlegendes Konzept in der Analyse von Schwingungssystemen (mechanisch oder elektrisch).

% Chapter end - always start new page after chapter
\newpage

\newpage
\chapter{Literaturempfehlungen}
\cite{Feilhauer2020a}
In diesem Kapitel finden Sie eine Auswahl an grundlegenden Lehrbüchern, die Ihr Verständnis der Experimentalphysik vertiefen und hervorragende Begleiter auf Ihrem akademischen Weg sind. Diese Werke haben sich als Standardwerke der theoretischen und experimentellen Physik etabliert und bieten eine ausgezeichnete Ergänzung zu den Vorlesungen. 
Die hier empfohlenen Bücher sind wertvolle Ressourcen, um die in den Vorlesungen behandelten Themen zu vertiefen und das Wissen zu festigen. Es kann sich durchaus lohnen, in eines dieser Standardwerke zu investieren und als ständigen Begleiter während des Studiums und auch außerhalb des Studiums zu nutzen.
\vspace{1cm}
\begin{figure}[h!]
    \centering
    \begin{minipage}[b]{0.28\textwidth}
        \centering
        \includegraphics[width=\textwidth]{Bilder/Literatur/Demtröder.jpg}
        % \caption{(Links) Demtröder, Wolfgang: Experimentalphysik
    \end{minipage}
    \hfill 
    \begin{minipage}[b]{0.253\textwidth}
        \centering
        \includegraphics[width=\textwidth]{Bilder/Literatur/Tipler.jpg}
        % \caption{Tipler, Paul A.; Mosca, Gene: Physik für Wissenschaftler und Ingenieure}
    \end{minipage}
    \hfill
    \begin{minipage}[b]{0.262\textwidth}
        \centering
        \includegraphics[width=\textwidth]{Bilder/Literatur/Giancoli.jpg}
        % \caption{Giancoli, Douglas C.: Physik: Prinzipien mit Anwendungen}
    \end{minipage}
     \caption{(Links) Demtröder, Wolfgang: Experimentalphysik. (Mitte) Tipler, Paul A.; Mosca, Gene: Physik für Wissenschaftler und Ingenieure. (Rechts) Giancoli, Douglas C.: Physik: Prinzipien mit Anwendungen.} 
    \label{fig: recommended_books_minipage}
\end{figure}

% \backmatter
%\phantomsection %needed for the correct anchor point of the bibliography 
%\addcontentsline{toc}{chapter}{Bibliography}

% \bibliographystyle{abbrv}
\bibliographystyle{alex_phd.bst}
\bibliography{skriptum_bib.bib}


\end{document}
