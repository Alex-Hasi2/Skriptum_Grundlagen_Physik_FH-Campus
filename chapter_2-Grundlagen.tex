\chapter{Grundlagen}\label{chap: Grundlagen}
Die Physik ist die Naturwissenschaft, die sich mit den Bausteinen der uns umgebenden Welt und deren gegenseitigen Wechselwirkungen beschäftigt. Es soll ein grundlegendes Verständnis auch komplizierter Körper aus ihrem Aufbau aus elementaren Teilchen und deren Wechselwirkungen geschaffen werden.\newline
Komplexe Naturvorgänge sollen auf einfache Gesetzmäßigkeiten zurückgeführt, quantifiziert und voraussagbar werden.
\begin{examplebox}[sidebyside, sidebyside align=top seam, lower separated=false, righthand width=7.5cm]{Beispiel}
    Radioaktivität
    \begin{itemize}
        \item Spontaner Zerfall der Atomkerne führt zu Emission von Teilchen oder Strahlung
        \item Quantifizierung dieses Prozesses über Zerfallsreihen 
        \item Voraussagbar über Halbwertszeit
    \end{itemize}
    \tcblower
    \centering
    \includegraphics[width=0.90\linewidth]{Bilder/Kapitel_Grundlagen/Zerfallsreihe_Uran238.png}
    \captionof{figure}[Zerfallsreihe von $^{238}$U (Quelle: \cite{Giancoli2010}, S.~1427)]{Zerfallsreihe von $^{238}\text{U}$. $Z$ ist die Anzahl der Protonen, $N$ die Anzahl der Neutronen und $A$ die Gesamtzahl der Kernteilchen. (Quelle:~\cite[S.~1427]{Giancoli2010}.)}\label{fig: zerfallsreihe_uran238}
\end{examplebox}

\section{Das Experiment}\label{sec: das_Experiment}
\begin{rememberbox}[]{}
    \textbf{Das Experiment} ist eine gezielte Frage an die Natur, auf die bei geeigneter experimenteller Anordnung eine eindeutige Antwort erhalten werden kann.
\end{rememberbox}
Das Ziel des Experimentes ist es einen Naturvorgang in einem kontrollierbaren und beliebig oft wiederholbaren Vorgang zu untersuchen. Die Bedingungen sind dabei strikt vorgegeben. 
Dabei wird besonderen Wert auf die Eliminierung aller störenden Einflüsse gelegt, die den untersuchten Effekt überlagern könnten. Im Sinne der Physik sollen damit Gesetzmäßigkeiten gefunden werden und eine Fülle von Beobachtungen in physikalische Gesetze überführt werden. 


\begin{examplebox}[sidebyside, sidebyside align=top seam, lower separated=false, righthand width=6.5cm]{Beispiel}
    Pendel-Experiment zur Bestimmung der Gravitationsbeschleunigung $\ivec{g}$.
    
\textbf{Hypothese:} Die Schwingungsdauer eines Pendels hängt von der Länge des Pendels und der Erdbeschleunigung ab, aber nicht von der Masse des Pendels.

\textbf{Störende Einflüsse:} Das Experiment wird in einem Raum ohne Luftzug durchgeführt. Die Startauslenkung kann mit bekannter Genauigkeit eingestellt werden.

\textbf{Reproduzierbarkeit:} Was muss dokumentiert werden, damit andere Forschende zu denselben Ergebnissen kommen? 
    \tcblower
    \centering
    \begin{tikzpicture}[
        scale=1.0,
        >={Stealth[length=3mm, width=3mm]}, % Pfeilspitzen größer machen
        font=\sffamily
        ]
        % Definiere Winkel und Längen
        \def\pendulumlength{5}
        \def\angle{20}
        \def\angleoffset{10}
        \def\radius{0.3}
        \def\lenG{2.0} % Länge des Gravitationsvektors
        
        % Koordinaten
        \coordinate (pivot) at (0,0);
        \coordinate (rest) at (0,-\pendulumlength);
        \coordinate (mass) at ({-\pendulumlength*sin(\angle)},{-\pendulumlength*cos(\angle)});
        \coordinate (masstop) at ({-(\pendulumlength-\radius)*sin(\angle)},{-(\pendulumlength-\radius)*cos(\angle)});
        
        % Aufhängung (horizontale Linien oben)
        \draw[line width=1.5pt] (-0.4,0) -- (0.4,0);
        \draw[line width=1pt] (-0.35,0.05) -- (-0.25,0.15);
        \draw[line width=1pt] (-0.15,0.05) -- (-0.05,0.15);
        \draw[line width=1pt] (0.05,0.05) -- (0.15,0.15);
        \draw[line width=1pt] (0.25,0.05) -- (0.35,0.15);
        
        % Ruheposition (gestrichelte vertikale Linie)
        \draw[gray, densely dotted, line width=0.8pt] (pivot) -- (rest);
        \node[right, gray] at ($(rest)+(0.1,-0.2)$) {Ruheposition};
        
        % Bogenbahn (gestrichelt)
        \draw[magenta!60, densely dashed, line width=0.7pt] 
            ({-\pendulumlength*sin(\angle+\angleoffset)},{-\pendulumlength*cos(\angle+\angleoffset)}) 
            arc[
                start angle={-90-\angle-\angleoffset}, 
                end angle={-90+\angle+\angleoffset}, 
                radius=\pendulumlength
            ];
        
        % Faden (Lila/Magenta)
        \draw[black, line width=1.5pt] (pivot) -- (masstop) node[midway, left=0.1cm, font=\Large] {$L$};
        
        % Winkel phi (oben am Aufhängepunkt)
        \draw[olive, line width=1.2pt, ->] (0,-1.6) arc[start angle=-90, end angle={-90-\angle}, radius=1.6] ;
        \node[olive, font=\Large] at ({-1.04*sin(\angle/2)},{-1.2*cos(\angle/2)}) {$\varphi$};

        % Graues Parallelogramm für die drei Pfeile (Gravitationskraft, Spannkraft, Rücktreibende Kraft)
        % Korrigiert: Linie parallel zur Spannkraft und Gravitationskraft
        \draw[gray!30, line width=1.2pt, dashed] (mass)++({-(\lenG*cos(\angle))*sin(\angle)}, {-(\lenG*cos(\angle))*cos(\angle)}) -- ++({+(\lenG*sin(\angle))*cos(\angle)}, {-(\lenG*sin(\angle))*sin(\angle)});
        \draw[gray!30, line width=1.2pt, dashed] (mass)++(0,-\lenG) -- ++({+(\lenG*cos(\angle))*sin(\angle)}, {+(\lenG*cos(\angle))*cos(\angle)});

        % Gravitationskraft (blauer Pfeil nach unten)
        \draw[->, blue!70!black, line width=2pt] (mass) -- ++(0,-\lenG) 
            node[below, right, font=\large] {$\ivec{g}$};
        
        % Spannkraft entlang des Fadens (grüner Pfeil)
        % Diese zeigt vom Massenpunkt in Richtung Aufhängepunkt
        \draw[->, green!60!black, line width=2pt] (mass) -- ++({-(\lenG*cos(\angle))*sin(\angle)}, {-(\lenG*cos(\angle))*cos(\angle)});
        
        % Rücktreibende Kraft (roter Pfeil tangential zur Kreisbahn)
        % Diese ist senkrecht zum Faden
        \draw[->, red, line width=2pt] (mass) -- ++({+(\lenG*sin(\angle))*cos(\angle)}, {-(\lenG*sin(\angle))*sin(\angle)});

        % Masse (gelber Kreis mit schwarzem Rand)
        \shade[ball color=yellow!80!orange, opacity=0.95] (mass) circle (\radius);
        \draw[black, line width=1.5pt] (mass) circle (\radius);

        % Winkel phi (unter der Masse)
        \draw[olive, line width=1.2pt, ->] (mass)++(0,-1.25) arc[start angle=-90, end angle={-90-\angle}, radius=1.25];
        \node[olive, font=\normalsize] at ($(mass)+({-0.8*sin(\angle/2)},{-0.9*cos(\angle/2)})$) {$\varphi$};
    
    \end{tikzpicture}
    % \includegraphics[width=0.95\linewidth]{Bilder/Kapitel_Grundlagen/Pendel_Experiment.png}
    \captionof{figure}[Pendel-Experiment zur Bestimmung der Gravitationsbeschleunigung]{Versuchsaufbau: Pendel-Experiment zur Bestimmung der Gravitationsbeschleunigung $\ivec{g}$.}\label{fig: pendel_experiment}
\end{examplebox}


\section{Bausteine der wissenschaftlichen Erkenntnis}\label{sec: Wissenschaftliche_Erkenntnis}
In den Wissenschaften, insbesondere in der Mathematik und Physik, wird Wissen nicht willkürlich gesammelt, sondern systematisch aufgebaut. Die Basis dieses Gebäudes bilden verschiedene Arten von Aussagen und Annahmen, die sich in ihrem Status, ihrer Funktion und ihrem Grad an Sicherheit fundamental unterscheiden. Von der ersten vagen Vermutung bis hin zum umfassenden Erklärungsmodell durchläuft eine wissenschaftliche Idee verschiedene Stufen der Prüfung und Etablierung. Die Grundbegriffe der Wissenschaftstheorie – Hypothese, Axiom, Satz, Gesetz und Theorie – sind die wesentlichen Werkzeuge, um die Welt zu beschreiben, zu verstehen und sich der Wahrheit schrittweise anzunähern. Sie bilden eine Hierarchie des Wissens, die von vorläufigen Annahmen bis zu in sich geschlossenen, widerspruchsfreien Gedankengebäuden reicht.
\subsection{Hypothese}\label{subsec: Hypothese}
Eine \textit{Hypothese} (Unterstellung, Annahme) ist der wissenschaftliche Ausgangspunkt für einen Erkenntnisgewinn. Sie ist eine vorläufige, testbare Annahme oder eine Vermutung, die ein Phänomen zu erklären versucht. Sie muss so formuliert sein, dass sie durch Beobachtungen oder Experimente prinzipiell widerlegt (falsifiziert) werden kann. Eine Hypothese wartet auf ihre Überprüfung.
\begin{rememberbox}[]{Hypothese}
    Eine \textbf{Hypothese} ist eine vorläufige Annahme oder ein Erklärungsversuch, die getestet werden soll. Sie ist eine Aussage, die durch Experimente oder Beobachtungen überprüft wird. 
\end{rememberbox}

\textit{Beispiele:}\newline
Pflanzen wachsen bei Bestrahlung mit rotem Licht schneller als bei Bestrahlung mit blauem Licht.\newline
Wenn man Salz in Wasser auflöst, dann erhöht sich der Siedepunkt des Wassers. 


\subsection{Axiom}\label{subsec: Axiom}
\textit{Axiome} sind die unbewiesenen Fundamente einer Theorie. Es sind grundlegende Annahmen oder Prinzipien, die innerhalb eines Systems als wahr, allgemein akzeptiert oder offensichtlich angesehen werden. Man kann sie nicht aus anderen Sätzen ableiten, sondern sie dienen als Basis, um weitere Sätze (Theoreme) logisch zu beweisen.
\begin{rememberbox}[]{Axiom}
    \textbf{Axiome} sind grundlegende Annahmen oder Prinzipien, die als bekannt, allgemein akzeptiert oder eindeutig wahr aufgefasst werden. Sie dienen als Basis für weitere Theorien und Beweise. 
\end{rememberbox}

\textit{Beispiele:} \newline
Durch zwei Punkte kann genau eine Gerade gelegt werden.\newline
Zwei parallele Linien schneiden sich nie.


\subsection{Satz (Theorem)}\label{subsec: Theorem}
Ein Satz, in der Wissenschaft oft als Theorem bezeichnet, ist eine logische Schlussfolgerung, die aus Axiomen und bereits bewiesenen Sätzen abgeleitet wird. Ein Satz muss erst streng bewiesen werden, bevor er als wahr akzeptiert wird. Er ist das Ergebnis eines logisch-deduktiven Prozesses.
\begin{rememberbox}[]{Satz}
    \textbf{Sätze} sind in der Mathematik und Logik Aussagen, die aus Axiomen und anderen bewiesenen Sätzen abgeleitet werden. Ein Satz muss erst bewiesen werden, um als wahr akzeptiert zu werden.
\end{rememberbox}

\textit{Beispiele:} \newline 
[Satz von Thales] Alle von einem Halbkreis umschriebenen Dreiecke sind rechtwinklig. \newline
[Satz von Fermat] Für $n > 2$ gibt es keine positiven ganzen Zahlen $a, b, c$ die die Gleichung $a^n + b^n = c^n$ erfüllen.


\subsection{Gesetz}\label{subsec: Gesetz}
Ein physikalisches Gesetz beschreibt eine grundlegende und regelmäßig wiederkehrende Beziehung in der Natur unter bestimmten Bedingungen. Es verknüpft messbare Größen miteinander, oft in Form einer präzisen mathematischen Gleichung. Ein Gesetz beschreibt, was passiert, liefert aber nicht zwingend eine Erklärung dafür, warum es passiert.
\begin{rememberbox}[]{Gesetz}
    Als \textbf{Gesetze} bezeichnet man Aussagen, die bestimmte Phänomene unter bestimmten Bedingungen beschreiben (meist durch Formeln). Anders als Theorien beschreiben Gesetze meist nur und liefern keine Erklärungen.
\end{rememberbox}

\textit{Beispiele:} \newline
[Ohm'sche Gesetz] ($U = R\cdot I$) beschreibt den Zusammenhang zwischen Spannung, Widerstand und Stromstärke, ohne die mikroskopischen Ursachen des elektrischen Widerstands zu erklären.\newline
[Gravitationsgesetz] Jede Masse zieht jede andere Masse mit einer Kraft an, die direkt proportional zum Produkt ihrer Massen und umgekehrt proportional zum Quadrat des Abstands zwischen ihren Schwerpunkten ist, $F = G \frac{M_1 \cdot M_2}{r^2}$.


\subsection{Theorie}\label{subsec: Theorie}
Eine physikalische Theorie ist die höchste Stufe wissenschaftlicher Erklärung. Sie ist ein umfassendes, in sich widerspruchsfreies System, das mehrere Gesetze und Prinzipien zusammenfasst. Eine Theorie liefert eine tiefgehende Erklärung für eine ganze Reihe von Naturvorgängen und erklärt beispielsweise die Vorgänge hinter den Gesetzen. Der Gültigkeitsbereich einer Theorie wird durch Experimente ständig überprüft, bestätigt und bei Bedarf eingeschränkt oder erweitert. Eine wissenschaftliche Theorie ist das bestmögliche Erklärungsmodell, das es gibt und unterscheidet sich daher radikal von der missbräuchlichen Verwendung im allgemeinen Sprachgebrauch -- hier wird ein unbewiesener Gedanke oder eine Hypothese oftmals als Theorie bezeichnet.
\begin{rememberbox}[]{Theorie}
    Eine physikalische \textbf{Theorie} ist die Zusammenfassung mehrerer physikalischer Gesetze und Prinzipien zu einem geschlossenen und in sich widerspruchsfreien Aufbau. Sie liefert eine Erklärung für Naturvorgänge. 
Der Gültigkeitsbereich einer physikalischen Theorie wird durch Experimente geprüft und eingeschränkt.

\end{rememberbox}

\textit{Beispiel:} \newline
Die Relativitätstheorie von Albert Einstein ist eine umfassende Theorie, die das Gesetz der Schwerkraft und die Bewegung von Objekten bei hohen Geschwindigkeiten erklärt und dabei auf grundlegenden Prinzipien aufbaut.

% Chapter end - always start new page after chapter
\newpage