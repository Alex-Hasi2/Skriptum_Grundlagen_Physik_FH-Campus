\documentclass[11pt, a4paper]{article}

\usepackage[utf8]{inputenc}
\usepackage[T1]{fontenc}
\usepackage{amsmath}
\usepackage{amssymb}
\usepackage{graphicx}
\usepackage{geometry}
\usepackage{fancyhdr}
\usepackage{enumitem}
\usepackage{array}
\usepackage{eso-pic} % Required for placing the logo

\usepackage{tikz}

\usepackage{siunitx}
\DeclareSIUnit{\litre}{l}
\sisetup{angle-symbol-degree = ^\circ}
\sisetup{locale = DE, separate-uncertainty, inter-unit-product = {},
range-units = brackets, list-units = single, per-mode=symbol-or-fraction}

\geometry{a4paper, top=2.5cm, bottom=2.5cm, left=2.5cm, right=2.5cm}


% Needed for the bibliography style (.bst files)
\providecommand{\Verfuegbar}{Verf{\"u}gbar}


%defs for the paper
\newcommand{\mDot}{\,.}
\newcommand{\mComma}{\,{,}\,}

% text subscripts
\newcommand{\kin}{\mathrm{kin}}
\newcommand{\pot}{\mathrm{pot}}
\newcommand{\rot}{\mathrm{rot}}
\newcommand{\trans}{\mathrm{trans}}
\newcommand{\atm}{\mathrm{atm}}
\newcommand{\minText}{\mathrm{min}}
\newcommand{\maxText}{\mathrm{max}}
% quantities with text subscripts
\newcommand{\Ekin}{E_{\kin}}
\newcommand{\Epot}{E_{\pot}}
\newcommand{\Erot}{E_{\rot}}
\newcommand{\Etrans}{E_{\trans}}
\newcommand{\kB}{k_{\mathrm{B}}}

% Text 
\newcommand{\Schro}{Schr\"o\-din\-ger }

% vectors 
\newcommand{\ivec}[1]{\vv{#1}} % using esvect package
\newcommand{\ivecS}[2]{\vv*{#1}{\!#2}} % using esvect package
% column vector
\newcommand{\icolTwo}[2]{\begin{pmatrix} #1 \\ #2 \end{pmatrix}}
\newcommand{\icolThree}[3]{\begin{pmatrix} #1 \\ #2 \\ #3 \end{pmatrix}}
% row vector (INLINE)
\newcommand{\inlrowTwo}[2]{(#1, #2)}
\newcommand{\inlrowThree}[3]{(#1, #2, #3)}

% point 
\newcommand{\ipTwo}[2]{(#1\!\mid\! #2)}
\newcommand{\ipThree}[3]{(#1\!\mid\! #2\!\mid\! #3)}

% rangle, langle 
\newcommand{\lrangle}[1]{{\langle{#1}\rangle}}
% measurement units
\newcommand{\Unit}[1]{\,\mathrm{#1}}

\newcommand{\msSp}{\;}
\newcommand{\mdSp}{\;\;}
\newcommand{\mtSp}{\;\;\;}
\newcommand{\mqSp}{\;\;\;\;}

  
%mathematical symbols
\newcommand{\defeq}{\vcentcolon=}
\newcommand{\eqdef}{=\vcentcolon}
\newcommand*\conj[1]{\bar{#1}}
\newcommand{\eqexcl}{\stackrel{!}{=}}
\newcommand{\eqquestion}{\stackrel{?}{=}}
\newcommand\equalhatInl{\mathrel{\stackon[1.0pt]{=}{\stretchto{%
    \scalerel*[\widthof{=}]{\wedge}{\rule{1ex}{3ex}}}{0.45ex}}}}
\newcommand\equalhat{\mathrel{\stackon[4.8pt]{=}{\stretchto{%
    \scalerel*[\widthof{=}]{\wedge}{\rule{1ex}{3ex}}}{0.45ex}}}}
\newcommand{\mAND}{\land}
\newcommand{\mOR}{\lor}
\newcommand{\mNOT}{\lnot}

% text editing
\newcommand{\textunderscript}[1]{$_{\text{#1}}$}
\newcommand{\textupperscript}[1]{$^{\text{#1}}$}
\newcommand{\eqqref}[1]{eq.\!~(\ref{#1})}
\newcommand{\Eqqref}[1]{Eq.\!~(\ref{#1})}
\newcommand{\figref}[1]{fig.\!~(\ref{#1})}
\newcommand{\Figref}[1]{Fig.\!~(\ref{#1})}
\newcommand{\secref}[1]{sec.\!~(\ref{#1})}
\newcommand{\Secref}[1]{Sec.\!~(\ref{#1})}

%general abbreviations (in German)
\newcommand{\wA}{\mbox{w.\,A.\ }}
\newcommand{\fA}{\mbox{f.\,A.\ }}
\newcommand{\zB}{\mbox{z.\,B.\ }}
\newcommand{\bzw}{\mbox{bzw.\ }}
\newcommand{\gDh}{\mbox{d.\,h.\ }}
\newcommand{\gDQ}[1]{\glqq #1\grqq}
\newcommand{\oBdA}{\mbox{o.\,B.\,d.\,A.\ }}
\newcommand{\sEUR}{\text{\euro}}

%latin abbreviations
\newcommand{\etal}{\mbox{\emph{et al.\ }}}
\newcommand{\exgrat}{\mbox{e.g.\ }}
\newcommand{\idest}{\mbox{i.e.\ }}

%general math terms
\newcommand{\const}{\mathrm{const}}
\newcommand{\bigO}{\mathcal{O}}

% Lorem ipsum
\newcommand*{\QEDA}{\hfill\ensuremath{\blacksquare}}%
\newcommand*{\QEDB}{\hfill\ensuremath{\square}}%

%  ------------------ abbreviations

%matrix operations
\newcommand{\T}{T}
\DeclareMathOperator{\arcsinh}{arcsinh}
\DeclareMathOperator{\Tr}{Tr}
\DeclareMathOperator{\argg}{arg}
\DeclareMathOperator{\Arg}{arg}
\DeclareMathOperator{\codim}{codim}
\DeclareMathOperator{\atanTwo}{atan2}
\DeclareMathOperator{\diag}{diag}

%real and complex numbers latin Letters
\newcommand{\Real}{\mathbb{R}}
\newcommand{\Complex}{\mathbb{C}}
\newcommand{\Integer}{\mathbb{N}}

% differentials 
\newcommand{\dd}{\mathrm{d}}



\pagestyle{fancy}
\fancyhf{}
% \rhead{Physikalische Grundlagen}
% \lhead{FH Campus Wien}
\cfoot{\thepage}

\begin{document}
% Command to add the logo to the top left of the first page
\AddToShipoutPictureBG*{%
  \AtPageUpperLeft{%
    \hspace{2.0cm}% Move logo to the right from the page edge
    \raisebox{-5.cm}{% Move logo down from the page edge
      \includegraphics[width=5cm]{Bilder/Allgemein/logo_hcw.png}% The logo file
    }%
  }%
}

\begin{center}
    {\Large \textbf{Physikalische Grundlagen}} \\[1em]
    {\large 3. Schriftliche Prüfung: 08.09.2025} \\[1em]
    {\large Studiengang: Clinical Engineering}
\end{center}

\vspace{1cm}

\noindent
\begin{tabular}{ll}
    \textbf{Vorname:} & \underline{\hspace{8cm}} \\[0.3cm]
    \textbf{Nachname:} & \underline{\hspace{8cm}} \\
\end{tabular}

\vspace{1cm}

\begin{itemize}[label={$\Diamond$}]
    \item Sie haben 90 min Zeit.
    \item Es sind keine Unterlagen erlaubt.
    \item Sie dürfen einen Taschenrechner verwenden.
    \item Geben Sie klare und verständliche Antworten!
    \item Schreiben Sie leserlich!
    \item Streichen Sie alle bis auf eine Lösung durch!
    \item Jeglicher Versuch unerlaubte Hilfsmittel zu verwenden oder von KommilitonInnen \\ abzuschreiben, wird mit einer negativen Bewertung des Tests geahndet.
\end{itemize}

\vspace{1cm}

\begin{center}
    \textbf{Viel Erfolg!}
\end{center}

\vspace{4cm}

\begin{flushright}
    \renewcommand{\arraystretch}{1.23} % This increases the row height by 50%
    \begin{tabular}{l p{2.3cm}}
        \hline
        \noalign{\vskip 0.15cm}
        \textbf{\Large{Punkte:}} & \textbf{\Large{/50}} \\
        \noalign{\vskip 0.1cm}
        \hline
        Beispiel 1: & /12 \\ 
        Beispiel 2: & /15 \\ 
        Beispiel 3: & /13 \\ 
        Beispiel 4: & /10
    \end{tabular}
\end{flushright}

\newpage

% ------------------------- BEISPIEL 1 -------------------------
\section*{Beispiel 1 - Heißes Metall in Wasser}

\begin{minipage}[t]{0.6\textwidth}
    \vspace{0pt}
    Eine massive Eisenkugel mit einer Masse von $m_{\text{Kugel}} = \SI{500}{\gram}$ hat nach der Schmelze eine Temperatur von $T_{\text{Kugel}} = \SI{250}{\celsius}$. Anschließend wird sie schnell in ein Gefäß mit \SI{5}{\litre} Wasser gegeben, das eine Anfangstemperatur von $T_{\text{Wasser}} = \SI{20}{\degreeCelsius}$ hat.\\
    \newline
    Gegeben sind die spezifische Wärmekapazität von Wasser $c_{\text{Wasser}} = \SI{4,18}{\kilo\joule\per(\kilogram\cdot\kelvin)}$, die spezifische Wärmekapazität von Eisen $c_{\text{Eisen}} = \SI{0,45}{\kilo\joule\per(\kilogram\cdot\kelvin)}$ und die Dichte von Wasser $\rho_{\text{Wasser}} = \SI{1}{\kilogram\per\deci\meter\cubed}$. 
\end{minipage}\hfill
\begin{minipage}[t]{0.35\textwidth}
    \vspace{0pt}
    \centering
    \includegraphics[width=0.7\textwidth]{Bilder/Test/3terAntritt/kugel_in_wasser_3.jpeg}
\end{minipage}

\subsection*{Aufgabe}
Berechnen Sie die Endtemperatur (Mischtemperatur) $T_{\text{Misch}}$, die sich im thermischen Gleichgewicht einstellt. Nehmen Sie an, dass keine Wärme an die Umgebung verloren geht.

\subsection*{Vorgehen}
\begin{enumerate}
    \item Stellen Sie die Formel für die vom Wasser aufgenommene Wärme $Q_{\text{Wasser,auf}}$ auf. Drücken Sie diese als Funktion der gesuchten Mischtemperatur $T_{\text{Misch}}$ aus.
    \item Stellen Sie die Formel für die von der Eisenkugel abgegebene Wärme $Q_{\text{Eisen,ab}}$ auf. Drücken Sie auch diese als Funktion der gesuchten Mischtemperatur $T_{\text{Misch}}$ aus.
    \item Nutzen Sie den Grundsatz des thermischen Gleichgewichts, indem Sie die beiden Wärmen gleichsetzen ($Q_{\text{Wasser,auf}} = Q_{\text{Eisen,ab}}$). Lösen Sie die resultierende Gleichung nach der Mischtemperatur $T_{\text{Misch}}$ auf.
\end{enumerate}

\vfill
\begin{flushright}
\renewcommand{\arraystretch}{1.23}
\begin{tabular}{|c | c | c|}
\hline
\multicolumn{3}{|l|}{\textbf{\large{Punkte: \quad\quad\quad /12}}}\\
\hline
% \textbf{1)} & \textbf{2)} & \textbf{3)} \\
% \quad/3 & \quad/3 & \quad/6 \\
% \hline
\end{tabular}
\end{flushright}

\newpage

% ------------------------- BEISPIEL 2 -------------------------
\section*{Beispiel 2 - Rutsche mit Reibung}
\begin{minipage}[t]{0.5\textwidth}
    \vspace{0pt}
    Ein Kind mit einer Masse von $m = \SI{35}{\kilogram}$ startet aus dem Stillstand ($v_0 = \SI{0}{\meter\per\second}$) am oberen Ende einer geraden Rutsche. Die Rutsche hat eine Höhe von $h = \SI{3}{\metre}$ und eine Länge von $L = \SI{5}{\metre}$. Während des Rutschens wirkt eine konstante Gleitreibungskraft zwischen dem Kind und der Rutsche. Der Gleitreibungskoeffizient beträgt $\mu_{\text{Gleit}} = 0,2$. Die Erdbeschleunigung lautet $g = \SI{9,81}{\metre\per\second\squared}$.
    \end{minipage}\hfill
\begin{minipage}[t]{0.49\textwidth}
    \vspace{0pt}
    \centering
    \includegraphics[width=0.99\textwidth]{Bilder/Test/3terAntritt/kind_rutsche.png}
\end{minipage}

\subsection*{Aufgabe}
Berechnen Sie die Geschwindigkeit des Kindes am unteren Ende der Rutsche. 

\subsection*{Vorgehen}
\begin{enumerate}
    \item Tragen Sie alle wirkenden Kräfte ein (Gewichtskraft, Normalkraft, Gleitreibungskraft) in der Grafik ein!
    \item Berechnen Sie den Neigungswinkel $\theta$ der Rutsche.
    \item Zerlegen Sie die Gewichtskraft in eine Komponente senkrecht, $F_{G, \perp}$, und eine parallel, $F_{G, \parallel}$, zur Rutsche. Geben Sie die Formeln für $F_{G, \perp}$ und $F_{G, \parallel}$ und deren Zahlenwerte an.
    \item Berechnen Sie die Normalkraft $F_N$, die auf das Kind wirkt. Berechnen Sie die Gleitreibungskraft $F_{R, \text{Gleit}}$ mithilfe der Normalkraft.
    \item Geben Sie die Hangabtriebskraft $F_A$ an, also die Komponente der Gewichtskraft, die das Kind die Rutsche hinab beschleunigt.
    \item Berechnen Sie die resultierende Nettokraft in Bewegungsrichtung ($x$-Richtung) und bestimmen Sie daraus die konstante Beschleunigung des Kindes entlang der Rutsche.
    \item Für die gleichförmig beschleunigte Bewegung ergibt sich der Weg $s$ und die Geschwindigkeit $v$ zu
    \begin{gather*}
        s(t) = s_0 + v_0\cdot t + \frac{1}{2}a\cdot t^2 \mComma\\
        v(t) = v_0 + a \cdot t \mDot
    \end{gather*}
    Leiten Sie aus diesen zeitabhängigen Formeln eine Formel für $v(L)$ her. Also die Geschwindigkeit des Kindes nachdem es unter dem Einfluss der Beschleunigung $a$ eine Strecke $s = L$ zurückgelegt hat. \\
    Tipp: Überlegen Sie, was für die Anfangsbedingungen $s_0$ und $v_0$ gilt. \\
    \textit{Lösung: } $v = \sqrt{2 a L}$
    \item Verwenden Sie die soeben hergeleitete Formel, um die Endgeschwindigkeit nach der zurückgelegten Strecke $L$ zu berechnen.
\end{enumerate}

\vfill
\begin{flushright}
\renewcommand{\arraystretch}{1.23}
\begin{tabular}{| c | c | c | c | c | c | c | c |}
\hline
\multicolumn{8}{|l|}{\textbf{\large{Punkte: \quad\quad\quad /15}}}\\
\hline
% \textbf{1)} & \textbf{2)} & \textbf{3)} & \textbf{4)} & \textbf{5)} & \textbf{6)} & \textbf{7)} & \textbf{8)} \\
% \quad/2 & \quad/1,5 & \quad/2 & \quad/2,5 & \quad/1,5 & \quad/2 & \quad/2,5 & \quad/1 \\
% \hline
\end{tabular}
\end{flushright}

\newpage
% ------------------------- BEISPIEL 3 -------------------------
\section*{Beispiel 3 - Holzklotz im Wasser}
\begin{center}
    \includegraphics[width=0.4\textwidth]{Bilder/Test/3terAntritt/holzklotz_eintauchen.png}
\end{center}
Ein quaderförmiger Holzklotz mit einer Kantenlänge von $L = \SI{20}{\centi\metre}$ und einer Dichte von $\rho_{\text{Holz}} = \SI{750}{\kilogram\per\metre\cubed}$ wird in ein Wasserbecken gelegt. Die Dichte von Wasser beträgt $\rho_{\text{Wasser}} = \SI{1000}{\kilogram\per\metre\cubed}$.

\subsection*{Aufgabe}
Berechnen Sie eine Formel für die Eintauchtiefe $h$ des Holzklotzes! Wie tief taucht der Holzklotz in der Angabe in das Wasser ein?

\subsection*{Vorgehen}
\begin{enumerate}
    \item Formulieren Sie die Bedingung für das Schwimmen eines Körpers. Welche beiden Kräfte müssen sich im Gleichgewicht befinden?
    \item Berechnen Sie das Gesamtvolumen $V_{\text{ges}}$ des Holzklotzes und daraus seine Gewichtskraft $F_G$ (als Funktion der Kantenlänge $L$).
    \item Stellen Sie eine Formel für die Auftriebskraft $F_A$ auf. Diese soll vom Volumen des eingetauchten Teils des Klotzes ($V_{\text{ein}}$) abhängen.
    \item Das eingetauchte Volumen $V_{\text{ein}}$ kann durch die Grundfläche $A = L^2$ des Klotzes und die gesuchte Eintauchtiefe $h$ ausgedrückt werden. Setzen Sie dies in die Formel für die Auftriebskraft aus (3) ein.
    \item Setzen Sie die Formel für die Gewichtskraft aus (2) und die der Auftriebskraft aus (4) in (1) ein und lösen Sie die Gleichung nach der Eintauchtiefe $h$ auf.\\
    \textit{Lösung:} $h = L \cdot (\rho_{\text{Holz}}/\rho_{\text{Wasser}})$
    \item Wie tief taucht der Holzblock aus der Angabe ein? 
\end{enumerate}

\vfill
\begin{flushright}
\renewcommand{\arraystretch}{1.23}
\begin{tabular}{|c | c | c | c | c | c|}
\hline
\multicolumn{6}{|l|}{\textbf{\large{Punkte: \quad\quad\quad /13}}}\\
\hline
% \textbf{1)} & \textbf{2)} & \textbf{3)} & \textbf{4)} & \textbf{5)} & \textbf{6)} \\
% \quad/1 & \quad/2 & \quad/2,5 & \quad/2,5 & \quad/3,5 & \quad/1,5 \\
% \hline
\end{tabular}
\end{flushright}


\newpage
% ------------------------- BEISPIEL 4 -------------------------
\section*{Beispiel 4 - Multiple-Choice}
\textit{Tragen Sie die gesuchte Antwort ein oder kreuzen Sie die korrekte Antwort an. Pro Frage ist nur eine Antwort richtig.}

\begin{enumerate}[label=\arabic*.,itemsep=18pt]
    \item Was ist die SI-Einheit der Leistung?
    \begin{itemize}[label={$\square$},itemsep=1pt,topsep=0pt]
        \item Joule [\si{\joule}]
        \item Watt [\si{\watt}]
        \item Newton [\si{\newton}]
        \item Pascal [\si{\pascal}]
    \end{itemize}

    \item Aus welchen SI-Basiseinheiten ist ein Joule zusammengesetzt?
    $$ 1\,\si{J} = \underline{\hspace{3cm}} .$$

    \item Ein Objekt mit Masse $m$ wird auf die doppelte Geschwindigkeit beschleunigt. Um welchen Faktor ändert sich seine kinetische Energie?
    \begin{itemize}[label={$\square$},itemsep=1pt,topsep=0pt]
        \item Faktor 2
        \item Faktor 8
        \item Faktor 4
        \item Faktor $\sqrt{2}$
    \end{itemize}

    \item Welche Aussage über Reibungskräfte ist im Allgemeinen korrekt?
    \begin{itemize}[label={$\square$},itemsep=1pt,topsep=0pt]
        \item Die Gleitreibungskraft ist größer als die maximale Haftreibungskraft.
        \item Die maximale Haftreibungskraft ist größer als die Gleitreibungskraft.
        \item Haft- und Gleitreibungskraft sind immer gleich groß.
        \item Reibungskräfte sind immer proportional zur Kontaktfläche.
    \end{itemize}

    \item Bei einem isochoren Prozess eines idealen Gases bleibt welche Zustandsgröße konstant?
    \begin{itemize}[label={$\square$},itemsep=1pt,topsep=0pt]
        \item Der Druck
        \item Die innere Energie
        \item Die Temperatur
        \item Das Volumen
    \end{itemize}
    
    \item Nach dem Archimedischen Prinzip ist die Auftriebskraft auf einen vollständig in einer Flüssigkeit eingetauchten Körper gleich:
    \begin{itemize}[label={$\square$},itemsep=1pt,topsep=0pt]
        \item dem Gewicht des Körpers.
        \item dem Volumen des Körpers mal der Erdbeschleunigung.
        \item dem Gewicht der vom Körper verdrängten Flüssigkeit.
        \item der Dichte des Körpers mal seinem Volumen.
    \end{itemize}

    \newpage
    \item Der erste Hauptsatz der Thermodynamik ($\Delta U = Q + W$) ist eine spezielle Formulierung des...
    \begin{itemize}[label={$\square$},itemsep=1pt,topsep=0pt]
        \item Impulserhaltungssatzes.
        \item Energieerhaltungssatzes.
        \item Drehimpulserhaltungssatzes.
        \item Massenerhaltungssatzes.
    \end{itemize}
    
    \item Ein Auto ($m=\SI{1000}{\kg}$) fährt mit $v = \SI{10}{\metre\per\second}$ durch eine Kurve mit $r = \SI{50}{\metre}$. Wie groß ist die benötigte Zentripetalkraft?
    \begin{itemize}[label={$\square$},itemsep=1pt,topsep=0pt]
        \item \SI{1000}{\newton}
        \item \SI{2000}{\newton}
        \item \SI{5000}{\newton}
        \item \SI{200}{\newton}
    \end{itemize}

    \item Der Wirkungsgrad $\eta$ einer Wärmekraftmaschine ist definiert als das Verhältnis von\\[9pt] \underline{\hspace{3.5cm}} \;zu\; \underline{\hspace{3.5cm}}.
    
    \item Wie lautet der Name der Zustandsgleichung für reale Gase, die das Eigenvolumen der Gasteilchen und die Anziehungskräfte zwischen ihnen berücksichtigt? \\[9pt]
    \underline{\hspace{6cm}}

    \item Die absolute Temperatur $T$ eines idealen Gases ist ein direktes Maß für die ...
    \begin{itemize}[label={$\square$},itemsep=1pt,topsep=0pt]
        \item mittlere potentielle Energie der Gasteilchen.
        \item mittlere Geschwindigkeit der Gasteilchen.
        \item mittlere kinetische Energie der Gasteilchen.
        \item Gesamtanzahl der Gasteilchen.
    \end{itemize}
\end{enumerate}

% Punkte Beispiel 5
\vfill
\begin{flushright}
\renewcommand{\arraystretch}{1.23}
\begin{tabular}{|c | c | c | c | c|}
\hline
\multicolumn{5}{|l|}{\textbf{\large{Punkte: \quad\quad\quad /10}}}\\
\hline
\end{tabular}
\end{flushright}




\end{document}