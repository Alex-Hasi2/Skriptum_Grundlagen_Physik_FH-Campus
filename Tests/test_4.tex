\documentclass[11pt, a4paper]{article}

\usepackage[utf8]{inputenc}
\usepackage[T1]{fontenc}
\usepackage{amsmath}
\usepackage{amssymb}
\usepackage{graphicx}
\usepackage{geometry}
\usepackage{fancyhdr}
\usepackage{enumitem}
\usepackage{array}
\usepackage{eso-pic} % Required for placing the logo

\usepackage{tikz}
\usepackage[b]{esvect} % For \vv (option [d] is a nice default)

\usepackage{siunitx}
\DeclareSIUnit{\litre}{l}
\sisetup{angle-symbol-degree = ^\circ}
\sisetup{locale = DE, separate-uncertainty, inter-unit-product = {},
range-units = brackets, list-units = single, per-mode=symbol-or-fraction}

\geometry{a4paper, top=2.5cm, bottom=2.5cm, left=2.5cm, right=2.5cm}


% Needed for the bibliography style (.bst files)
\providecommand{\Verfuegbar}{Verf{\"u}gbar}


%defs for the paper
\newcommand{\mDot}{\,.}
\newcommand{\mComma}{\,{,}\,}

% text subscripts
\newcommand{\kin}{\mathrm{kin}}
\newcommand{\pot}{\mathrm{pot}}
\newcommand{\rot}{\mathrm{rot}}
\newcommand{\trans}{\mathrm{trans}}
\newcommand{\atm}{\mathrm{atm}}
\newcommand{\minText}{\mathrm{min}}
\newcommand{\maxText}{\mathrm{max}}
% quantities with text subscripts
\newcommand{\Ekin}{E_{\kin}}
\newcommand{\Epot}{E_{\pot}}
\newcommand{\Erot}{E_{\rot}}
\newcommand{\Etrans}{E_{\trans}}
\newcommand{\kB}{k_{\mathrm{B}}}

% Text 
\newcommand{\Schro}{Schr\"o\-din\-ger }

% vectors 
\newcommand{\ivec}[1]{\vv{#1}} % using esvect package
\newcommand{\ivecS}[2]{\vv*{#1}{\!#2}} % using esvect package
% column vector
\newcommand{\icolTwo}[2]{\begin{pmatrix} #1 \\ #2 \end{pmatrix}}
\newcommand{\icolThree}[3]{\begin{pmatrix} #1 \\ #2 \\ #3 \end{pmatrix}}
% row vector (INLINE)
\newcommand{\inlrowTwo}[2]{(#1, #2)}
\newcommand{\inlrowThree}[3]{(#1, #2, #3)}

% point 
\newcommand{\ipTwo}[2]{(#1\!\mid\! #2)}
\newcommand{\ipThree}[3]{(#1\!\mid\! #2\!\mid\! #3)}

% rangle, langle 
\newcommand{\lrangle}[1]{{\langle{#1}\rangle}}
% measurement units
\newcommand{\Unit}[1]{\,\mathrm{#1}}

\newcommand{\msSp}{\;}
\newcommand{\mdSp}{\;\;}
\newcommand{\mtSp}{\;\;\;}
\newcommand{\mqSp}{\;\;\;\;}

  
%mathematical symbols
\newcommand{\defeq}{\vcentcolon=}
\newcommand{\eqdef}{=\vcentcolon}
\newcommand*\conj[1]{\bar{#1}}
\newcommand{\eqexcl}{\stackrel{!}{=}}
\newcommand{\eqquestion}{\stackrel{?}{=}}
\newcommand\equalhatInl{\mathrel{\stackon[1.0pt]{=}{\stretchto{%
    \scalerel*[\widthof{=}]{\wedge}{\rule{1ex}{3ex}}}{0.45ex}}}}
\newcommand\equalhat{\mathrel{\stackon[4.8pt]{=}{\stretchto{%
    \scalerel*[\widthof{=}]{\wedge}{\rule{1ex}{3ex}}}{0.45ex}}}}
\newcommand{\mAND}{\land}
\newcommand{\mOR}{\lor}
\newcommand{\mNOT}{\lnot}

% text editing
\newcommand{\textunderscript}[1]{$_{\text{#1}}$}
\newcommand{\textupperscript}[1]{$^{\text{#1}}$}
\newcommand{\eqqref}[1]{eq.\!~(\ref{#1})}
\newcommand{\Eqqref}[1]{Eq.\!~(\ref{#1})}
\newcommand{\figref}[1]{fig.\!~(\ref{#1})}
\newcommand{\Figref}[1]{Fig.\!~(\ref{#1})}
\newcommand{\secref}[1]{sec.\!~(\ref{#1})}
\newcommand{\Secref}[1]{Sec.\!~(\ref{#1})}

%general abbreviations (in German)
\newcommand{\wA}{\mbox{w.\,A.\ }}
\newcommand{\fA}{\mbox{f.\,A.\ }}
\newcommand{\zB}{\mbox{z.\,B.\ }}
\newcommand{\bzw}{\mbox{bzw.\ }}
\newcommand{\gDh}{\mbox{d.\,h.\ }}
\newcommand{\gDQ}[1]{\glqq #1\grqq}
\newcommand{\oBdA}{\mbox{o.\,B.\,d.\,A.\ }}
\newcommand{\sEUR}{\text{\euro}}

%latin abbreviations
\newcommand{\etal}{\mbox{\emph{et al.\ }}}
\newcommand{\exgrat}{\mbox{e.g.\ }}
\newcommand{\idest}{\mbox{i.e.\ }}

%general math terms
\newcommand{\const}{\mathrm{const}}
\newcommand{\bigO}{\mathcal{O}}

% Lorem ipsum
\newcommand*{\QEDA}{\hfill\ensuremath{\blacksquare}}%
\newcommand*{\QEDB}{\hfill\ensuremath{\square}}%

%  ------------------ abbreviations

%matrix operations
\newcommand{\T}{T}
\DeclareMathOperator{\arcsinh}{arcsinh}
\DeclareMathOperator{\Tr}{Tr}
\DeclareMathOperator{\argg}{arg}
\DeclareMathOperator{\Arg}{arg}
\DeclareMathOperator{\codim}{codim}
\DeclareMathOperator{\atanTwo}{atan2}
\DeclareMathOperator{\diag}{diag}

%real and complex numbers latin Letters
\newcommand{\Real}{\mathbb{R}}
\newcommand{\Complex}{\mathbb{C}}
\newcommand{\Integer}{\mathbb{N}}

% differentials 
\newcommand{\dd}{\mathrm{d}}



\pagestyle{fancy}
\fancyhf{}
% \rhead{Physikalische Grundlagen}
% \lhead{FH Campus Wien}
\cfoot{\thepage}

\begin{document}
% Command to add the logo to the top left of the first page
\AddToShipoutPictureBG*{%
  \AtPageUpperLeft{%
    \hspace{2.0cm}% Move logo to the right from the page edge
    \raisebox{-5.cm}{% Move logo down from the page edge
      \includegraphics[width=4cm]{Bilder/Allgemein/LogoFHCampus.png}% The logo file
    }%
  }%
}

\begin{center}
    {\Large \textbf{Physikalische Grundlagen}} \\[1em]
    {\large 4. Schriftliche Prüfung: 14.10.2025} \\[1em]
    {\large Studiengang: Clinical Engineering}
\end{center}

\vspace{1cm}

\noindent
\begin{tabular}{ll}
    \textbf{Vorname:} & \underline{\hspace{8cm}} \\[0.3cm]
    \textbf{Nachname:} & \underline{\hspace{8cm}} \\
\end{tabular}

\vspace{1cm}

\begin{itemize}[label={$\Diamond$}]
    \item Sie haben 90 min Zeit.
    \item Es sind keine Unterlagen erlaubt.
    \item Sie dürfen einen Taschenrechner verwenden.
    \item Geben Sie klare und verständliche Antworten!
    \item Schreiben Sie leserlich!
    \item Streichen Sie alle bis auf eine Lösung durch!
    \item Jeglicher Versuch unerlaubte Hilfsmittel zu verwenden oder von KommilitonInnen abzuschreiben, wird mit einer negativen Bewertung des Tests geahndet.
\end{itemize}

\vspace{1cm}

\begin{center}
    \textbf{Viel Erfolg!}
\end{center}

\vspace{1cm}

\begin{flushright}
    \renewcommand{\arraystretch}{1.23} % This increases the row height by 50%
    \begin{tabular}{l p{2.3cm}}
        \hline
        \noalign{\vskip 0.15cm}
        \textbf{\Large{Punkte:}} & \textbf{\Large{/50}} \\
        \noalign{\vskip 0.1cm}
        \hline
        Beispiel 1: & /13 \\ 
        Beispiel 2: & /14 \\ 
        Beispiel 3: & /13 \\ 
        Beispiel 4: & /10 \\
    \end{tabular}
\end{flushright}

\newpage

% ------------------------- BEISPIEL 1 -------------------------
\section*{Beispiel 1 - Energieerhaltung bei einer Feder}

Ein Block mit der Masse $m = \SI{2}{\kilogram}$ wird gegen eine horizontale Feder mit der Federkonstante $k = \SI{800}{\newton\per\meter}$ gedrückt und komprimiert diese um $\Delta x_0 = -\SI{20}{\centi\meter}$. Nachdem der Block losgelassen wird (siehe Skizze), gleitet er \textbf{reibungsfrei} über eine horizontale Fläche und anschließend eine Rampe hinauf. Die Erdbeschleunigung beträgt $g = \SI{9,81}{\metre\per\second\squared}$. Für die Feder gilt das Hooksche Gesetz 
\begin{equation*}
    F_{\text{Feder}} = -k \cdot \Delta x
\end{equation*}
\begin{center}
    \centering
    \includegraphics[width=0.85\textwidth]{Bilder/Test/4terAntritt/Block-Feder-Rampe.png}
\end{center}
    
\subsection*{Aufgabe}
Berechnen Sie die maximale Höhe $h$, die der Block auf der Rampe erreicht.
\subsection*{Vorgehen}
\begin{enumerate}
    \item Formulieren Sie den (mechanischen) Energieerhaltungssatz für dieses System. \\
    Tipp: Welche Energieformen sind zu den unterschiedlichen Zeitpunkten der Bewegung vorhanden?
    \item Berechnen Sie eine Formel für die potenzielle Energie $E_{\text{pot,Feder}}$, die anfangs in der komprimierten Feder gespeichert ist und geben Sie anschließend den numerischen Wert an. \\
    Tipp: $W_\mathrm{Feder} = \int F_\mathrm{Feder} \cdot \dd (\Delta x) = -E_\mathrm{pot,Feder}$ 
    \item Beschreiben Sie, was mit der potenziellen Energie des Blocks passiert, nachdem er von der Feder beschleunigt wurde. Welche Geschwindigkeit $v$ hat der Block am Fuße der Rampe?
    \item Am höchsten Punkt der Rampe, den der Block erreicht, ist die gesamte kinetische Energie in potenzielle Energie $E_{\text{pot,Höhe}}$ umgewandelt worden. Geben Sie eine Formel für die maximale Höhe $h$ als Funktion der Anfangsauslenkung $\Delta x_0$ an. \\ 
    Berechnen Sie auch den numerischen Wert.
\end{enumerate}

\vfill
\begin{flushright}
\renewcommand{\arraystretch}{1.23}
\begin{tabular}{|c | c | c | c |}
\hline
\multicolumn{4}{|l|}{\textbf{\large{Punkte: \quad\quad\quad /13}}}\\
\hline
\textbf{1)} & \textbf{2)} & \textbf{3)} & \textbf{4)} \\
\quad/\num{2} & \quad/\num{4} & \quad/\num{3} & \quad/\num{4} \\
\hline
\end{tabular}
\end{flushright}

\newpage

% ------------------------- BEISPIEL 2 -------------------------
\section*{Beispiel 2 - Konisches Pendel}

Ein Ball der Masse $m=\SI{0.5}{\kilogram}$ ist an einem Seil der Länge $l=\SI{1.5}{\meter}$ aufgehängt und rotiert auf einer horizontalen Kreisbahn mit einem Radius von $r=\SI{0.8}{\meter}$. Auf den Ball wirken die Gewichtskraft $\ivecS{F}{G}$ und die Seilkraft $\ivecS{F}{S}$.
\begin{center}
    \includegraphics[width=0.4\textwidth]{Bilder/Uebungsaufgaben/thetherbal_angabe.png}
\end{center}

\subsection*{Aufgabe}
Berechnen Sie die Winkelgeschwindigkeit $\omega$ des Balls.
\subsection*{Vorgehen}
\begin{enumerate}
    \item Zeichnen Sie ein Freikörperdiagramm mit allen auf den Ball wirkenden Kräften in der $(x,z)$-Ebene und teilen Sie dabei die Seilkraft dementsprechend in $F_{S,x}$ und $F_{S,z}$ auf.
    \item Geben Sie eine Formel für den Winkel $\theta$ des Seils zur Vertikalen mithilfe der gegebenen Längen $l$ und $r$. Berechnen Sie auch den numerischen Wert!
    \item Stellen Sie das Kräftegleichgewicht in vertikaler Richtung ($z$-Richtung) auf. Beachten Sie, dass sich der Ball in dieser Richtung nicht bewegen soll. \\
    Lösen Sie die entstandene Gleichung nach dem Betrag der Seilkraft $F_S$ auf.
    \item Stellen Sie das Kräftegleichgewicht in horizontaler ($x$-Richtung) auf. Die resultierende Kraft in horizontaler Richtung ist die notwendige Zentripetalkraft $F_{\text{Zp}}$.
    \item Setzen Sie den Ausdruck für die Seilkraft $F_S$ aus (3) in die Gleichung aus (4) ein und lösen Sie die resultierende Gleichung nach der Winkelgeschwindigkeit $\omega$ auf. Wie hängt die Winkelgeschwindigkeit $\omega$ mit $g,r,\theta$ zusammen? \\ Berechnen Sie auch den numerischen Wert von $\omega$.
\end{enumerate}

\vfill
\begin{flushright}
\renewcommand{\arraystretch}{1.23}
\begin{tabular}{|c | c | c | c | c|}
\hline
\multicolumn{5}{|l|}{\textbf{\large{Punkte: \quad\quad\quad /14}}}\\
\hline
\textbf{1)} & \textbf{2)} & \textbf{3)} & \textbf{4)} & \textbf{5)} \\
\quad/\num{2} & \quad/\num{2} & \quad/\num{4} & \quad/\num{2} & \quad/\num{4} \\
\hline
\end{tabular}
\end{flushright}

\newpage

% ------------------------- BEISPIEL 3 -------------------------
\section*{Beispiel 3 - Mischtemperatur}

Ein \SI{400}{\gram} schweres Stück Kupfer mit einer Temperatur von $T_{\text{Cu}} = \SI{95}{\celsius}$ wird in ein wärmeisoliertes Gefäß mit \SI{1.2}{\litre} Glykol gegeben, das eine Anfangstemperatur von $T_{\text{Glykol}} = \SI{18}{\degreeCelsius}$ hat.
\begin{center}
    \includegraphics[width=0.3\textwidth]{Bilder/Test/3terAntritt/kugel_in_wasser_3.jpeg}
\end{center}

Gegeben sind:
\begin{itemize}
    \item Spezifische Wärmekapazität von Kupfer: $c_{\text{Cu}} = \SI{0.385}{\kilo\joule\per(\kilogram\cdot\kelvin)}$
    \item Spezifische Wärmekapazität von Glykol: $c_{\text{Glykol}} = \SI{2.40}{\kilo\joule\per(\kilogram\cdot\kelvin)}$
    \item Dichte von Glykol: $\rho_{\text{Glykol}} = \SI{1.11}{\kilogram\per\deci\meter\cubed}$
\end{itemize}

\subsection*{Aufgabe}
Berechnen Sie die Endtemperatur (Mischtemperatur) $T_{\text{Misch}}$, die sich im thermischen Gleichgewicht einstellt.
\subsection*{Vorgehen}
\begin{enumerate}
    \item Formulieren Sie den Grundsatz des thermischen Gleichgewichts in einem isolierten System. Welche Beziehung besteht zwischen der vom Kupfer abgegebenen und der vom Glykol aufgenommenen Wärme?
    \item Berechnen Sie zunächst die Masse des Glykols.
    \item Stellen Sie eine Formel für die vom Kupfer abgegebene Wärme $Q_{\text{ab}}$ als Funktion von $T_{\text{Misch}}$ auf.
    \item Stellen Sie eine Formel für die vom Glykol aufgenommene Wärme $Q_{\text{auf}}$ als Funktion von $T_{\text{Misch}}$ auf.
    \item Setzen Sie die beiden Wärmeformeln gleich und lösen Sie die resultierende Gleichung nach der Mischtemperatur $T_{\text{Misch}}$ auf. Berechnen Sie die Mischtemperatur $T_{\text{Misch}}$, die sich einstellt.
\end{enumerate}

\vfill
\begin{flushright}
\renewcommand{\arraystretch}{1.23}
\begin{tabular}{|c | c | c | c | c|}
\hline
\multicolumn{5}{|l|}{\textbf{\large{Punkte: \quad\quad\quad /13}}}\\
\hline
\textbf{1)} & \textbf{2)} & \textbf{3)} & \textbf{4)} & \textbf{5)} \\
\quad/\num{1} & \quad/\num{2} & \quad/\num{2} & \quad/\num{2} & \quad/\num{6} \\
\hline
\end{tabular}
\end{flushright}

\newpage

% ------------------------- BEISPIEL 4 -------------------------
\section*{Beispiel 4 - Single-Choice}
\textit{Kreuzen Sie die korrekte Antwort an. Pro Frage ist nur eine Antwort richtig.}

\begin{enumerate}[label=\arabic*.,itemsep=18pt]
    \item Was ist die SI-Einheit der Kraft?
    \begin{itemize}[label={$\square$},itemsep=1pt,topsep=0pt]
        \item Joule [\si{\joule}]
        \item Watt [\si{\watt}]
        \item Newton [\si{\newton}]
        \item Pascal [\si{\pascal}]
    \end{itemize}

    \item Aus welchen SI-Basiseinheiten ist ein Pascal (\si{\pascal}) zusammengesetzt?
    \begin{itemize}[label={$\square$},itemsep=1pt,topsep=0pt]
        \item \si{\kilogram \cdot \metre^{-1} \cdot\, \second^{-2}}
        \item \si{\kilogram \cdot\metre^{2} \cdot\, \second^{-2}}
        \item \si{\kilogram \cdot\metre^{-1} \cdot\, \second^{-1}}
        \item \si{\kilogram \cdot\metre \cdot\, \second^{-2}}
    \end{itemize}

    \item Ein Auto ($m=\SI{1200}{\kg}$) beschleunigt von \SI{50}{\kilo\meter\per\hour} auf \SI{100}{\kilo\meter\per\hour}. Um welchen Faktor erhöht sich seine kinetische Energie?
    \begin{itemize}[label={$\square$},itemsep=1pt,topsep=0pt]
        \item Faktor 2
        \item Faktor 4
        \item Faktor $\sqrt{2}$
        \item Faktor 8
    \end{itemize}

    \item Bei einem adiabatischen Prozess eines idealen Gases bleibt was konstant?
    \begin{itemize}[label={$\square$},itemsep=1pt,topsep=0pt]
        \item Der Druck
        \item Die innere Energie
        \item Die Temperatur
        \item Die Wärme
        \item Die Arbeit
    \end{itemize}

    \item Ein Planet hat den doppelten Radius und die doppelte Masse der Erde. Wie groß ist die Fallbeschleunigung $g'$ an seiner Oberfläche im Vergleich zur Erdbeschleunigung $g$? \\
    ($g = GM/R^2$)
    \begin{itemize}[label={$\square$},itemsep=1pt,topsep=0pt]
        \item $g' = 2g$
        \item $g' = g$
        \item $g' = g/2$
        \item $g' = g/4$
    \end{itemize}

    \item Der erste Hauptsatz der Thermodynamik ($\Delta U = Q + W$) ist eine Formulierung des...
    \begin{itemize}[label={$\square$},itemsep=1pt,topsep=0pt]
        \item Impulserhaltungssatzes.
        \item Energieerhaltungssatzes.
        \item Drehimpulserhaltungssatzes.
        \item Massenerhaltungssatzes.
    \end{itemize}

    \newpage
    \item Welche der folgenden Aussagen beschreibt das 3. Newtonsche Axiom?
    \begin{itemize}[label={$\square$},itemsep=1pt,topsep=0pt]
        \item Ein Körper bleibt in Ruhe, wenn keine Kraft auf ihn wirkt.
        \item Die Beschleunigung ist proportional zur Kraft ($F=ma$).
        \item Kräfte treten immer paarweise auf; übt Körper A eine Kraft auf B aus, übt B eine gleich große, entgegengesetzte Kraft auf A aus.
        \item Die Energie in einem abgeschlossenen System ist konstant.
    \end{itemize}

    \item Ein Rad mit Radius $r=\SI{0.5}{\meter}$ dreht sich mit einer Winkelgeschwindigkeit von $\omega = \SI{4}{\radian\per\second}$. Wie groß ist die Bahngeschwindigkeit $v$ eines Punktes am Rand des Rades?
    \begin{itemize}[label={$\square$},itemsep=1pt,topsep=0pt]
        \item \SI{1}{\meter\per\second}
        \item \SI{2}{\meter\per\second}
        \item \SI{4}{\meter\per\second}
        \item \SI{8}{\meter\per\second}
    \end{itemize}

    \item Welche Größe wird in der Einheit Watt (\si{\watt}) gemessen?
    \begin{itemize}[label={$\square$},itemsep=1pt,topsep=0pt]
        \item Energie
        \item Arbeit
        \item Kraft
        \item Leistung
    \end{itemize}

    \item Ein Objekt wird über eine Strecke von \SI{5}{\meter} mit einer konstanten Kraft von \SI{20}{\newton} verschoben, die parallel zum Weg wirkt. Welche Arbeit wird verrichtet?
    \begin{itemize}[label={$\square$},itemsep=1pt,topsep=0pt]
        \item \SI{4}{\joule}
        \item \SI{25}{\joule}
        \item \SI{100}{\joule}
        \item \SI{0.25}{\joule}
    \end{itemize}
    
    \item Sie tragen einen Koffer mit $m = \SI{3}{\kilo\gram}$ auf einer ebenen Strecke von $s = \SI{500}{\meter}$ unter dem Einfluss der Erdbeschleunigung $g$. Welche Arbeit wird dabei verrichtet? 
    \begin{itemize}[label={$\square$},itemsep=1pt,topsep=0pt]
        \item \SI{0}{\joule}
        \item \SI{1500}{\joule}
        \item \SI{14715}{\joule}
        \item Es kommt auf die Geschwindigkeit an.
    \end{itemize}

    \item Scheinkräfte treten in welcher Art von Bezugssystemen auf?
    \begin{itemize}[label={$\square$},itemsep=1pt,topsep=0pt]
        \item In Inertialsystemen
        \item Nur in rotierenden Bezugssystemen
        \item In beschleunigten Bezugssystemen
        \item In allen Bezugssystemen
    \end{itemize}
\end{enumerate}

\vfill
\begin{flushright}
\renewcommand{\arraystretch}{1.23}
\begin{tabular}{|l c|}
\hline
\multicolumn{2}{|l|}{\textbf{\large{Punkte: \quad\quad\quad /10}}}\\
\hline
\end{tabular}
\end{flushright}

\end{document}