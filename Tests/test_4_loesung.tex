\documentclass[11pt, a4paper]{article}

\usepackage[utf8]{inputenc}
\usepackage[T1]{fontenc}
\usepackage{amsmath}
\usepackage{amssymb}
\usepackage{graphicx}
\usepackage{geometry}
\usepackage{fancyhdr}
\usepackage{enumitem}
\usepackage{array}
\usepackage{eso-pic} % Required for placing the logo

\usepackage{tikz}
\usepackage[b]{esvect} % For \vv (option [d] is a nice default)

\usepackage{siunitx}
\DeclareSIUnit{\litre}{l}
\sisetup{
    locale = DE,
    inter-unit-product = \cdot,
    per-mode = symbol-or-fraction
}

\geometry{a4paper, top=2.5cm, bottom=2.5cm, left=2.5cm, right=2.5cm}

% Needed for the bibliography style (.bst files)
\providecommand{\Verfuegbar}{Verf{\"u}gbar}


%defs for the paper
\newcommand{\mDot}{\,.}
\newcommand{\mComma}{\,{,}\,}

% text subscripts
\newcommand{\kin}{\mathrm{kin}}
\newcommand{\pot}{\mathrm{pot}}
\newcommand{\rot}{\mathrm{rot}}
\newcommand{\trans}{\mathrm{trans}}
\newcommand{\atm}{\mathrm{atm}}
\newcommand{\minText}{\mathrm{min}}
\newcommand{\maxText}{\mathrm{max}}
% quantities with text subscripts
\newcommand{\Ekin}{E_{\kin}}
\newcommand{\Epot}{E_{\pot}}
\newcommand{\Erot}{E_{\rot}}
\newcommand{\Etrans}{E_{\trans}}
\newcommand{\kB}{k_{\mathrm{B}}}

% Text 
\newcommand{\Schro}{Schr\"o\-din\-ger }

% vectors 
\newcommand{\ivec}[1]{\vv{#1}} % using esvect package
\newcommand{\ivecS}[2]{\vv*{#1}{\!#2}} % using esvect package
% column vector
\newcommand{\icolTwo}[2]{\begin{pmatrix} #1 \\ #2 \end{pmatrix}}
\newcommand{\icolThree}[3]{\begin{pmatrix} #1 \\ #2 \\ #3 \end{pmatrix}}
% row vector (INLINE)
\newcommand{\inlrowTwo}[2]{(#1, #2)}
\newcommand{\inlrowThree}[3]{(#1, #2, #3)}

% point 
\newcommand{\ipTwo}[2]{(#1\!\mid\! #2)}
\newcommand{\ipThree}[3]{(#1\!\mid\! #2\!\mid\! #3)}

% rangle, langle 
\newcommand{\lrangle}[1]{{\langle{#1}\rangle}}
% measurement units
\newcommand{\Unit}[1]{\,\mathrm{#1}}

\newcommand{\msSp}{\;}
\newcommand{\mdSp}{\;\;}
\newcommand{\mtSp}{\;\;\;}
\newcommand{\mqSp}{\;\;\;\;}

  
%mathematical symbols
\newcommand{\defeq}{\vcentcolon=}
\newcommand{\eqdef}{=\vcentcolon}
\newcommand*\conj[1]{\bar{#1}}
\newcommand{\eqexcl}{\stackrel{!}{=}}
\newcommand{\eqquestion}{\stackrel{?}{=}}
\newcommand\equalhatInl{\mathrel{\stackon[1.0pt]{=}{\stretchto{%
    \scalerel*[\widthof{=}]{\wedge}{\rule{1ex}{3ex}}}{0.45ex}}}}
\newcommand\equalhat{\mathrel{\stackon[4.8pt]{=}{\stretchto{%
    \scalerel*[\widthof{=}]{\wedge}{\rule{1ex}{3ex}}}{0.45ex}}}}
\newcommand{\mAND}{\land}
\newcommand{\mOR}{\lor}
\newcommand{\mNOT}{\lnot}

% text editing
\newcommand{\textunderscript}[1]{$_{\text{#1}}$}
\newcommand{\textupperscript}[1]{$^{\text{#1}}$}
\newcommand{\eqqref}[1]{eq.\!~(\ref{#1})}
\newcommand{\Eqqref}[1]{Eq.\!~(\ref{#1})}
\newcommand{\figref}[1]{fig.\!~(\ref{#1})}
\newcommand{\Figref}[1]{Fig.\!~(\ref{#1})}
\newcommand{\secref}[1]{sec.\!~(\ref{#1})}
\newcommand{\Secref}[1]{Sec.\!~(\ref{#1})}

%general abbreviations (in German)
\newcommand{\wA}{\mbox{w.\,A.\ }}
\newcommand{\fA}{\mbox{f.\,A.\ }}
\newcommand{\zB}{\mbox{z.\,B.\ }}
\newcommand{\bzw}{\mbox{bzw.\ }}
\newcommand{\gDh}{\mbox{d.\,h.\ }}
\newcommand{\gDQ}[1]{\glqq #1\grqq}
\newcommand{\oBdA}{\mbox{o.\,B.\,d.\,A.\ }}
\newcommand{\sEUR}{\text{\euro}}

%latin abbreviations
\newcommand{\etal}{\mbox{\emph{et al.\ }}}
\newcommand{\exgrat}{\mbox{e.g.\ }}
\newcommand{\idest}{\mbox{i.e.\ }}

%general math terms
\newcommand{\const}{\mathrm{const}}
\newcommand{\bigO}{\mathcal{O}}

% Lorem ipsum
\newcommand*{\QEDA}{\hfill\ensuremath{\blacksquare}}%
\newcommand*{\QEDB}{\hfill\ensuremath{\square}}%

%  ------------------ abbreviations

%matrix operations
\newcommand{\T}{T}
\DeclareMathOperator{\arcsinh}{arcsinh}
\DeclareMathOperator{\Tr}{Tr}
\DeclareMathOperator{\argg}{arg}
\DeclareMathOperator{\Arg}{arg}
\DeclareMathOperator{\codim}{codim}
\DeclareMathOperator{\atanTwo}{atan2}
\DeclareMathOperator{\diag}{diag}

%real and complex numbers latin Letters
\newcommand{\Real}{\mathbb{R}}
\newcommand{\Complex}{\mathbb{C}}
\newcommand{\Integer}{\mathbb{N}}

% differentials 
\newcommand{\dd}{\mathrm{d}}



\pagestyle{fancy}
\fancyhf{}
\cfoot{\thepage}

\begin{document}
\noindent
\vspace{9cm}
\begin{center}
    {\Huge \textbf{Lösung}} \\[1.5em]
    {\Large \textbf{Physikalische Grundlagen}} \\[1em]
    % --- Angepasstes Datum ---
    {\large 4. Schriftliche Prüfung: 14.10.2025} \\[1em]
    {\large Studiengang: Clinical Engineering} 
\end{center}

\newpage

% ------------------------- LÖSUNG BEISPIEL 1 -------------------------
\section*{Lösung Beispiel 1 - Energieerhaltung bei einer Feder}

\begin{enumerate}
    \item \textbf{Formulieren Sie den (mechanischen) Energieerhaltungssatz für dieses System.} \\
    Da die Bewegung reibungsfrei ist, bleibt die mechanische Gesamtenergie erhalten: 
    \begin{enumerate}
        \item Am Anfang (A) ist die gesamte Energie als potenzielle Energie in der Feder gespeichert ($E_A = E_{\text{pot,Feder}}$)
        \item Am Fuße der Rampe (Mitte [M]) ist die gesamte Energie in Form von kinetischer Energie des Blocks gespeichert ($E_M = E_{\text{kin,Block}}$) 
        \item Am höchsten Punkt der Rampe (Ende [E]) ist die Geschwindigkeit null, und die gesamte Energie ist in potenzielle Energie der Höhe umgewandelt ($E_E = E_{\text{pot,Block}}$)
        \end{enumerate}
        Daher gilt:
    $$ E_\text{pot,Feder} + E_\text{kin,Block} + E_\text{pot,Block} = \const \mDot $$
    

    \item \textbf{Berechnen Sie eine Formel für die potenzielle Energie $E_{\text{pot,Feder}}$.} \\
    Die in der Feder gespeicherte potenzielle Energie entspricht der Arbeit, die verrichtet werden muss, um sie zu komprimieren. Mit dem Hookschen Gesetz $F_{\text{Feder}} = k \cdot \Delta x$ ergibt sich die Arbeit aus dem Integral der Kraft über den Weg:
    \begin{multline}
        W_\text{Feder} = \int_{0}^{\Delta x_0} F_{\text{Feder}} \cdot \dd(\Delta x) = \int_{0}^{\Delta x_0} -k \cdot \Delta x \cdot \dd(\Delta x) = \\
        -\frac{1}{2} k \Delta x^2 \bigg|_{0}^{\Delta x_0} =  -\frac{1}{2} k \Delta x_0^2 = -E_{\text{pot,Feder}} \mDot
    \end{multline}
    Lösen des Integrals führt damit zur Formel für die potenzielle Energie:
    $$ E_{\text{pot,Feder}} = \frac{1}{2}k (\Delta x_0)^2 \mDot$$
    Einsetzen der Werte ($\Delta x_0 = -\SI{20}{\centi\meter} = -\SI{0.2}{\meter}$):
    $$ E_{\text{pot,Feder}} = \frac{1}{2} \cdot \left(\SI{800}{\newton\per\meter}\right) \cdot (-\SI{0.2}{\meter})^2 = \SI{16}{\joule} $$
    
    \item \textbf{Beschreiben Sie, was mit der Energie des Blocks passiert.} \\
    Nach dem Loslassen wird die potenzielle Energie der Feder vollständig in kinetische Energie des Blocks umgewandelt ($E_{\text{kin}} = \frac{1}{2}mv^2$). Am Fuße der Rampe ist die kinetische Energie also $E_{\text{kin}} = \SI{16}{\joule}$. Die Geschwindigkeit beträgt:
    $$ v = \sqrt{\frac{2 E_{\text{kin}}}{m}} = \sqrt{\frac{2 \cdot \SI{16}{\joule}}{\SI{2}{\kilogram}}} = \SI{4}{\meter\per\second} = \SI{14.4}{\kilo\meter\per\hour}$$

    \item \textbf{Geben Sie eine Formel für die maximale Höhe $h$ an als Funktion von $\Delta x_0$.} \\
    Die potentielle Energie auf der maximalen Höhe $h$ ist $ E_{\text{pot,Höhe}} = mgh$.
    Gleichsetzen der potenziellen Energien liefert
    $$ E_{\text{pot,Feder}} = E_{\text{pot,Höhe}} \Rightarrow \frac{1}{2}k(\Delta x_0)^2 = mgh $$
    Wir lösen nach $h$ auf:
    $$ h = \frac{E_{\text{pot,Feder}}}{mg} = \frac{\frac{1}{2}k(\Delta x_0)^2}{mg} = \frac{k(\Delta x_0)^2}{2mg}$$
    Einsetzen der Werte:
    $$ h = \frac{\SI{800}{\newton\per\meter} \cdot (\SI{0.2}{\meter})^2}{2 \cdot \SI{2}{\kilogram} \cdot \SI{9.81}{\metre\per\second\squared}} \approx \SI{0.815}{\meter} $$
    Der Block erreicht eine maximale Höhe von ca. \SI{81.5}{\centi\meter}.
\end{enumerate}

\newpage

% ------------------------- LÖSUNG BEISPIEL 2 -------------------------
\section*{Lösung Beispiel 2 - Konisches Pendel}

\begin{enumerate}
    \item \textbf{Zeichnen Sie ein Freikörperdiagramm.} \\
    \begin{center}
        \includegraphics[width=0.4\textwidth]{Bilder/Uebungsaufgaben/thetherbal_Kraftschnitt.png}
    \end{center}
    Die wirkenden Kräfte sind die Gewichtskraft $\vec{F}_G = m\vec{g}$ (nach unten) und die Seilkraft $\vec{F}_S$ (entlang des Seils). Die Seilkraft wird in eine vertikale Komponente $F_{S,z}$ und eine horizontale (zentripetale) Komponente $F_{S,x}$ zerlegt.
    
    \item \textbf{Geben Sie eine Formel für den Winkel $\theta$ an und berechnen Sie ihn.} \\
    Aus der Geometrie ergibt sich ein rechtwinkliges Dreieck mit Hypotenuse $l$ und Gegenkathete $r$.
    $$ \sin(\theta) = \frac{r}{l} \implies \theta = \arcsin\left(\frac{r}{l}\right) $$
    $$ \theta = \arcsin\left(\frac{\SI{0.8}{\meter}}{\SI{1.5}{\meter}}\right) \approx \SI{32.23}{\degree} $$

    \item \textbf{Stellen Sie das Kräftegleichgewicht in z-Richtung auf und lösen Sie nach $F_S$.} \\
    Die vertikale Komponente der Seilkraft muss die Gewichtskraft ausgleichen, da keine vertikale Beschleunigung stattfindet.
    $$ \sum F_z = F_S \cos(\theta) - mg = 0 $$
    $$ F_S = \frac{mg}{\cos(\theta)} = \frac{\SI{0.5}{\kg} \cdot \SI{9.81}{\m\per\s\squared}}{\cos(\SI{32.23}{\degree})} \approx \SI{5.80}{\newton} $$
    
    \item \textbf{Stellen Sie das Kräftegleichgewicht in x-Richtung auf.} \\
    Die horizontale Komponente der Seilkraft wirkt als Zentripetalkraft.
    $$ \sum F_x = F_S \sin(\theta) = F_{\text{Zp}} = m \omega^2 r $$

    \item \textbf{Setzen Sie die Ausdrücke zusammen und lösen Sie nach $\omega$.} \\
    Wir setzen den Ausdruck für $F_S$ aus (3) in die Gleichung aus (4) ein:
    $$ \left(\frac{mg}{\cos(\theta)}\right) \sin(\theta) = m \omega^2 r $$
    $$ mg \tan(\theta) = m \omega^2 r $$
    Wir kürzen die Masse $m$ und lösen nach $\omega$ auf:
    $$ \omega^2 = \frac{g \tan(\theta)}{r} \implies \omega = \sqrt{\frac{g \tan(\theta)}{r}} = \sqrt{\frac{g}{r} \tan\left(\arcsin\left(\frac{r}{l}\right)\right)}$$
    Einsetzen der Werte ergibt die Kreisfrequenz der rotierenden Masse:
    $$ \omega = \sqrt{\frac{\SI{9.81}{\m\per\s\squared} \cdot \tan(\SI{32.23}{\degree})}{\SI{0.8}{\meter}}} \approx \sqrt{\frac{\num{6.18}}{\num{0.8}}} \approx \SI{2.78}{\radian\per\second} $$
\end{enumerate}

\newpage

% ------------------------- LÖSUNG BEISPIEL 3 -------------------------
\section*{Lösung Beispiel 3 - Mischtemperatur}

\begin{enumerate}
    \item \textbf{Grundsatz des thermischen Gleichgewichts:}
    In einem isolierten System ist die Summe der ausgetauschten Wärmemengen null. Das bedeutet, die vom wärmeren Körper (Kupfer) abgegebene Wärme $Q_{\text{ab}}$ ist gleich der vom kälteren Körper (Glykol) aufgenommenen Wärme $Q_{\text{auf}}$.
    $$ Q_{\text{ab}} = Q_{\text{auf}} $$

    \item \textbf{Masse des Glykols:}
    $$ m_{\text{Glykol}} = \rho_{\text{Glykol}} \cdot V = \SI{1.11}{\kilogram\per\deci\meter\cubed} \cdot \SI{1.2}{\litre} = \SI{1.332}{\kilogram} $$

    \item \textbf{Abgegebene Wärme des Kupfers:}
    $$ Q_{\text{ab}} = m_{\text{Cu}} \cdot c_{\text{Cu}} \cdot (T_{\text{Cu}} - T_{\text{Misch}}) $$
    
    \item \textbf{Aufgenommene Wärme des Glykols:}
    $$ Q_{\text{auf}} = m_{\text{Glykol}} \cdot c_{\text{Glykol}} \cdot (T_{\text{Misch}} - T_{\text{Glykol}}) $$

    \item \textbf{Gleichsetzen und $T_{\text{Misch}}$ berechnen:}
    $$ m_{\text{Cu}} c_{\text{Cu}} (T_{\text{Cu}} - T_{\text{Misch}}) = m_{\text{Glykol}} c_{\text{Glykol}} (T_{\text{Misch}} - T_{\text{Glykol}}) $$
    $$ T_{\text{Misch}} = \frac{m_{\text{Cu}} c_{\text{Cu}} T_{\text{Cu}} + m_{\text{Glykol}} c_{\text{Glykol}} T_{\text{Glykol}}}{m_{\text{Cu}} c_{\text{Cu}} + m_{\text{Glykol}} c_{\text{Glykol}}} $$
    Einsetzen der Werte (Masse in kg):
    $$ T_{\text{Misch}} = \frac{(\SI{0.4}{\kg} \cdot \SI{0.385}{\frac{\kilo\joule}{\kg\K}} \cdot \SI{95}{\celsius}) + (\SI{1.332}{\kg} \cdot \SI{2.40}{\frac{\kilo\joule}{\kg\K}} \cdot \SI{18}{\celsius})}{(\SI{0.4}{\kg} \cdot \SI{0.385}{\frac{\kilo\joule}{\kg\K}}) + (\SI{1.332}{\kg} \cdot \SI{2.40}{\frac{\kilo\joule}{\kg\K}})} $$
    $$ T_{\text{Misch}} = \frac{\num{14.63} + \num{57.5424}}{\num{0.154} + \num{3.1968}} = \frac{\num{72.1724}}{\num{3.3508}} \approx \SI{21.54}{\celsius} $$
\end{enumerate}

\newpage

% ------------------------- LÖSUNG BEISPIEL 4 -------------------------
\section*{Lösung Beispiel 4 - Single-Choice}

\begin{enumerate}[label=\arabic*.,itemsep=18pt]
    \item Was ist die SI-Einheit der Kraft?
    \begin{itemize}[label={$\square$},itemsep=1pt,topsep=0pt]
        \item Joule [\si{\joule}]
        \item Watt [\si{\watt}]
        \item[$\boxtimes$] Newton [\si{\newton}]
        \item Pascal [\si{\pascal}]
    \end{itemize}

    \item Aus welchen SI-Basiseinheiten ist ein Pascal (\si{\pascal}) zusammengesetzt?
    \begin{itemize}[label={$\square$},itemsep=1pt,topsep=0pt]
        \item[$\boxtimes$] \si{\kilogram \cdot \metre^{-1} \cdot\, \second^{-2}} ($\text{Pa} = \text{N}/\text{m}^2 = (\text{kg} \cdot \text{m}/\text{s}^2)/\text{m}^2$)
        \item \si{\kilogram \cdot\metre^{2} \cdot\, \second^{-2}}
        \item \si{\kilogram \cdot\metre^{-1} \cdot\, \second^{-1}}
        \item \si{\kilogram \cdot\metre \cdot\, \second^{-2}}
    \end{itemize}

    \item Ein Auto ($m=\SI{1200}{\kg}$) beschleunigt von \SI{50}{\kilo\meter\per\hour} auf \SI{100}{\kilo\meter\per\hour}. Um welchen Faktor erhöht sich seine kinetische Energie?
    \begin{itemize}[label={$\square$},itemsep=1pt,topsep=0pt]
        \item Faktor 2
        \item[$\boxtimes$] Faktor 4 ($E_{\text{kin}} \propto v^2$. Bei doppelter Geschwindigkeit $v \to 2v$ wird die Energie $E_{\text{kin}} \propto (2v)^2 \propto 4v^2$)
        \item Faktor $\sqrt{2}$
        \item Faktor 8
    \end{itemize}

    \item Bei einem adiabatischen Prozess eines idealen Gases bleibt was konstant?
    \begin{itemize}[label={$\square$},itemsep=1pt,topsep=0pt]
        \item Der Druck
        \item Die innere Energie
        \item Die Temperatur
        \item[$\boxtimes$] Die Wärme ($\Delta Q = 0$)
        \item Die Arbeit
    \end{itemize}

    \item Ein Planet hat den doppelten Radius und die doppelte Masse der Erde. Wie groß ist die Fallbeschleunigung $g'$ an seiner Oberfläche im Vergleich zur Erdbeschleunigung $g$? ($g = GM/R^2$)
    \begin{itemize}[label={$\square$},itemsep=1pt,topsep=0pt]
        \item $g' = 2g$
        \item $g' = g$
        \item[$\boxtimes$] $g' = g/2$ ($g' = G(2M)/(2R)^2 = G(2M)/(4R^2) = (1/2) \cdot GM/R^2 = g/2$)
        \item $g' = g/4$
    \end{itemize}

    \item Der erste Hauptsatz der Thermodynamik ($\Delta U = Q + W$) ist eine Formulierung des...
    \begin{itemize}[label={$\square$},itemsep=1pt,topsep=0pt]
        \item Impulserhaltungssatzes.
        \item[$\boxtimes$] Energieerhaltungssatzes.
        \item Drehimpulserhaltungssatzes.
        \item Massenerhaltungssatzes.
    \end{itemize}
    
    \newpage
    \item Welche der folgenden Aussagen beschreibt das 3. Newtonsche Axiom?
    \begin{itemize}[label={$\square$},itemsep=1pt,topsep=0pt]
        \item Ein Körper bleibt in Ruhe, wenn keine Kraft auf ihn wirkt. (1. Axiom)
        \item Die Beschleunigung ist proportional zur Kraft ($F=ma$). (2. Axiom)
        \item[$\boxtimes$] Kräfte treten immer paarweise auf; übt Körper A eine Kraft auf B aus, übt B eine gleich große, entgegengesetzte Kraft auf A aus.
        \item Die Energie in einem abgeschlossenen System ist konstant. (Energieerhaltungssatz)
    \end{itemize}

    \item Ein Rad mit Radius $r=\SI{0.5}{\meter}$ dreht sich mit einer Winkelgeschwindigkeit von $\omega = \SI{4}{\radian\per\second}$. Wie groß ist die Bahngeschwindigkeit $v$ eines Punktes am Rand des Rades?
    \begin{itemize}[label={$\square$},itemsep=1pt,topsep=0pt]
        \item \SI{1}{\meter\per\second}
        \item[$\boxtimes$] \SI{2}{\meter\per\second} ($v = \omega \cdot r = \SI{4}{\radian\per\second} \cdot \SI{0.5}{\meter} = \SI{2}{\meter\per\second}$)
        \item \SI{4}{\meter\per\second}
        \item \SI{8}{\meter\per\second}
    \end{itemize}

    \item Welche Größe wird in der Einheit Watt (\si{\watt}) gemessen?
    \begin{itemize}[label={$\square$},itemsep=1pt,topsep=0pt]
        \item Energie
        \item Arbeit
        \item Kraft
        \item[$\boxtimes$] Leistung
    \end{itemize}

    \item Ein Objekt wird über eine Strecke von \SI{5}{\meter} mit einer konstanten Kraft von \SI{20}{\newton} verschoben, die parallel zum Weg wirkt. Welche Arbeit wird verrichtet?
    \begin{itemize}[label={$\square$},itemsep=1pt,topsep=0pt]
        \item \SI{4}{\joule}
        \item \SI{25}{\joule}
        \item[$\boxtimes$] \SI{100}{\joule} ($W = F \cdot s = \SI{20}{\newton} \cdot \SI{5}{\meter} = \SI{100}{\joule}$)
        \item \SI{0.25}{\joule}
    \end{itemize}
    
    \item Sie tragen einen Koffer mit $m = \SI{3}{\kilo\gram}$ auf einer ebenen Strecke von $s = \SI{500}{\meter}$ unter dem Einfluss der Erdbeschleunigung $g$. Welche Arbeit wird dabei verrichtet?
    \begin{itemize}[label={$\square$},itemsep=1pt,topsep=0pt]
        \item[$\boxtimes$] \SI{0}{\joule} (Die Haltekraft wirkt vertikal, der Weg ist horizontal. Da die Kraft senkrecht zum Weg steht, ist die Arbeit $W = \vec{F} \cdot \vec{s} = 0$.)
        \item \SI{1500}{\joule}
        \item \SI{14715}{\joule}
        \item Es kommt auf die Geschwindigkeit an.
    \end{itemize}

    \item Scheinkräfte treten in welcher Art von Bezugssystemen auf?
    \begin{itemize}[label={$\square$},itemsep=1pt,topsep=0pt]
        \item In Inertialsystemen
        \item Nur in rotierenden Bezugssystemen
        \item[$\boxtimes$] In beschleunigten Bezugssystemen
        \item In allen Bezugssystemen
    \end{itemize}
\end{enumerate}

\end{document}