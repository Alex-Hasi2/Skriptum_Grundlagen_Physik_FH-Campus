\documentclass[11pt, a4paper]{article}

\usepackage[utf8]{inputenc}
\usepackage[T1]{fontenc}
\usepackage{amsmath}
\usepackage{amssymb}
\usepackage{graphicx}
\usepackage{geometry}
\usepackage{fancyhdr}
\usepackage{enumitem}
\usepackage{array}
\usepackage{eso-pic} % Erforderlich für die Platzierung des Logos

% --- siunitx Konfiguration ---
\usepackage{siunitx}
\DeclareSIUnit{\litre}{l}
\sisetup{
    angle-symbol-degree = ^\circ,
    locale = DE, 
    separate-uncertainty, 
    inter-unit-product = \cdot, % Multiplikationspunkt zwischen Einheiten
    range-units = brackets, 
    list-units = single, 
    per-mode=symbol-or-fraction
}

% --- Seitenlayout ---
\geometry{a4paper, top=2.5cm, bottom=2.5cm, left=2.5cm, right=2.5cm}

% --- Kopf- und Fußzeile ---
\pagestyle{fancy}
\fancyhf{}
\cfoot{\thepage}

% --- Benutzerdefinierte Befehle (aus defs.tex) ---
\newcommand{\mDot}{\,.}
\newcommand{\mComma}{\, ,\,}
\newcommand{\dd}{\mathrm{d}}
\newcommand{\eqexcl}{\stackrel{!}{=}}

\begin{document}
% --- Logo auf der ersten Seite oben links ---
\AddToShipoutPictureBG*{%
  \AtPageUpperLeft{%
    \hspace{2.0cm}%
    \raisebox{-5.cm}{%
      \includegraphics[width=4cm]{Bilder/Allgemein/LogoFHCampus.png}% Logo-Datei hier anpassen
    }%
  }%
}

% --- Titelblock ---
\begin{center}
    {\Large \textbf{Physikalische Grundlagen}} \\[1em]
    {\large 1. Schriftliche Prüfung: 17.06.2025} \\[1em]
    {\large Studiengang: CLINICAL ENGINEERING}
\end{center}

\vspace{1cm}

% --- Studenteninformation ---
\noindent
\begin{tabular}{ll}
    \textbf{Vorname:} & \underline{\hspace{8cm}} \\[0.5cm]
    \textbf{Nachname:} & \underline{\hspace{8cm}} \\
\end{tabular}

\vspace{1cm}

% --- Anweisungen ---
\begin{itemize}[label={$\Diamond$}]
    \item Sie haben 90 min Zeit.
    \item Es sind keine Unterlagen erlaubt.
    \item Sie dürfen einen Taschenrechner verwenden.
    \item Geben Sie klare und verständliche Antworten!
    \item Schreiben Sie leserlich!
    \item Streichen Sie alle bis auf eine Lösung durch!
    \item Jeglicher Versuch unerlaubte Hilfsmittel zu verwenden oder von KommilitonInnen abzuschreiben, wird mit einer negativen Bewertung des Tests geahndet.
\end{itemize}

\vspace{1cm}

\begin{center}
    \textbf{Viel Erfolg!}
\end{center}

\vspace{1cm}

% --- Punktetabelle ---
\begin{flushright}
    \renewcommand{\arraystretch}{1.2} 
    \begin{tabular}{l p{2.3cm}}
        \hline
        \noalign{\vskip 0.15cm}
        \textbf{\Large{Punkte:}} & \textbf{\Large{/50}} \\
        \noalign{\vskip 0.1cm}
        \hline
        Beispiel 1: & /10 \\ 
        Beispiel 2: & /8 \\ 
        Beispiel 3: & /9 \\ 
        Beispiel 4: & /9 \\ 
        Beispiel 5: & /8 \\
        Beispiel 6: & /6 \\
    \end{tabular}
\end{flushright}

\newpage

% ------------------------- BEISPIEL 1 -------------------------
\section*{Beispiel 1 - Wasserkocher vergessen}

\begin{minipage}[t]{0.6\textwidth}
    \vspace{0pt}
    Ein Wasserkocher habe eine Heizleistung von \SI{2000}{\watt}. Sie haben \SI{1}{\litre} Wasser ($\rho_{W}=\SI{1}{\kilogram\per\deci\metre\cubed}$) für Tee bei $T=\SI{20}{\degreeCelsius}$ eingefüllt und den Wasserkocher eingeschaltet. Ihr Wasserkocher hat \textbf{keine Selbstabschaltung} und Sie gehen \SI{25}{\minute} einkaufen.
    \vspace{0.5cm}
    
    Wenn das Wasser eine spezifische Wärmekapazität $c_{W}=\SI{4,18}{\kilo\joule\per(\kilogram\cdot\kelvin)}$ hat und eine spezifische Verdampfungswärme von $\lambda_{\text{verd}}=\SI{2256}{\kilo\joule\per\kilogram}$, kommen Sie dann rechtzeitig vom Einkaufen zurück, bevor das gesamte Wasser verdampft ist und der Kocher zu verschmoren droht?
\end{minipage}\hfill
\begin{minipage}[t]{0.35\textwidth}
    \vspace{0pt}
    \centering
    \includegraphics[width=\textwidth]{Bilder/Test/1terAntritt/wasserkocher.png}
\end{minipage}

\subsection*{Aufgaben bzw. Vorgehen}
\begin{enumerate}
    \item Skizzieren Sie den zeitlichen Ablauf in einem Diagramm ($T_{W}$ über t)! Beschriften Sie die einzelnen Abschnitte des Prozesses!
    \item Berechnen Sie die Energie zum Erhitzen des Wassers auf \SI{100}{\degreeCelsius}.
    \item Berechnen Sie die Verdampfungswärme, die zum Verdampfen des Wassers benötigt wird.
    \item Berechnen Sie, wie viel Energie der Wasserkocher in \SI{25}{\minute} liefert.
    \item Fazit?
\end{enumerate}

\vfill
\begin{flushright}
\renewcommand{\arraystretch}{1.23}
\begin{tabular}{|c | c | c | c | c|}
\hline
\multicolumn{5}{|l|}{\textbf{\large{Punkte: \quad\quad\quad /10}}}\\
\hline
\textbf{1)} & \textbf{2)} & \textbf{3)} & \textbf{4)} & \textbf{5)} \\
\quad/2 & \quad/2 & \quad/2 & \quad/2 & \quad/2 \\
\hline
\end{tabular}
\end{flushright}

\newpage

% ------------------------- BEISPIEL 2 -------------------------
\section*{Beispiel 2 - Jump-Bag}

\begin{center}
    \includegraphics[width=0.7\textwidth]{Bilder/Test/1terAntritt/jump_bag.png}
\end{center}

Zwei Männer mit jeweils \SI{80}{\kilogram} Körpermasse springen gemeinsam von einem Sprungturm mit \SI{6}{\metre} Höhe auf einen „Jump-Bag". Dort übertragen Sie Ihre gesamte kinetische Energie verlustfrei auf eine Frau am anderen Ende, die eine Masse von $m_{\text{Frau}}=\SI{65}{\kilogram}$ hat. Es gibt keine Reibungsverluste oder anders gearteten Verluste.

\subsection*{Aufgaben}
\begin{enumerate}[itemsep=10pt]
    \item Wie lautet die Bewegungsform der Frau nach der Energieübertragung? Begründen Sie Ihre Entscheidung! 
    \begin{itemize}[label={$\square$}, itemsep=2pt, topsep=1pt]
        \item Gleichförmige Bewegung
        \item Gleichförmig beschleunigte Bewegung
        \item Gleichförmige Kreisbewegung
    \end{itemize}
    \item Berechnen Sie die maximale Flughöhe $h_{\text{max}}$ der Frau!
    \item Mit welcher maximalen Geschwindigkeit $v_{\text{max}}$ kommt sie auf der Wasseroberfläche bei $h=0$ m auf?
\end{enumerate}

\vfill
\begin{flushright}
\renewcommand{\arraystretch}{1.23}
\begin{tabular}{|c | c | c|}
\hline
\multicolumn{3}{|l|}{\textbf{\large{Punkte: \quad\quad\quad /8}}}\\
\hline
\textbf{1)} & \textbf{2)} & \textbf{3)} \\
\quad/2 & \quad/3 & \quad/3 \\
\hline
\end{tabular}
\end{flushright}

\newpage

% ------------------------- BEISPIEL 3 -------------------------
\section*{Beispiel 3 - Ideales Gas}
Die Zustandsgleichung eines idealen Gases mit Volumen $V$ [\si{\metre\cubed}], Druck p [\si{\pascal}] und der Temperatur T [\si{\kelvin}] lautet
$$p \cdot V = n \cdot R \cdot T \mComma$$
wobei $n$ [\si{\mole}] die Anzahl der Mole des Gases ist und $R = \SI{8,314}{\joule\per\mole\per\kelvin}$. Betrachten Sie ein ideales Gas mit $n=1$ mol.
\subsection*{Aufgaben}
\begin{enumerate}[itemsep=8pt]
    \item In der Grafik ist eine Isotherme eingezeichnet und mit (I) markiert. Welche Temperatur hat diese Isotherme?
    \item Zeichnen Sie eine weitere Isotherme mit $T=\SI{250}{\kelvin}$ ein und markieren Sie sie mit (II).
    \item Zeichnen Sie eine beliebige Isobare ein und beschriften Sie sie mit (III).
    \item Zeichnen Sie eine beliebige Isochore ein und beschriften Sie sie mit (IV).
    \item Berechnen Sie das Volumen eines idealen Gases unter Normalbedingungen, d.h. $n = 1$ mol, $p=\SI{1,013}{\bar}$, $T=\SI{0}{\degreeCelsius}$!
\end{enumerate}
\begin{center}
    \includegraphics[width=0.8\textwidth]{Bilder/Test/1terAntritt/ideales_gas.png}
\end{center}

\vfill
\begin{flushright}
\renewcommand{\arraystretch}{1.23}
\begin{tabular}{|c | c | c | c | c|}
\hline
\multicolumn{5}{|l|}{\textbf{\large{Punkte: \quad\quad\quad /9}}}\\
\hline
\textbf{1)} & \textbf{2)} & \textbf{3)} & \textbf{4)} & \textbf{5)} \\
\quad/2 & \quad/2 & \quad/1 & \quad/1 & \quad/3 \\
\hline
\end{tabular}
\end{flushright}

\newpage

% ------------------------- BEISPIEL 4 -------------------------
\section*{Beispiel 4 - Carnot'scher Kreisprozess}
Der Carnot'sche Kreisprozess beschreibt eine theoretische Wärmekraftmaschine, die einen Zyklus aus 4 dynamischen Prozessen durchläuft, bevor es zum Ausgangszustand zurückkehrt. Diese 4 Schritte sind (ungeordnete Reihenfolge):
\begin{itemize}[label={$\Diamond$}, itemsep=2pt, topsep=4pt, leftmargin=2cm]
    \item Isotherme Expansion
    \item Adiabatische Expansion
    \item Isotherme Kompression
    \item Adiabatische Kompression
\end{itemize}
Als Wirkungsgrad $\eta$ der Maschine definiert man die von ihr verrichtete Arbeit $\Delta W$ dividiert durch die aufgenommene Wärmeenergie $\Delta Q_{1}$:
$$\eta= \left|\frac{\Delta W}{\Delta Q_{1}}\right| = \frac{T_{1}-T_{2}}{T_{1}}$$
wobei $T_{1}>T_{2}$ gilt.
\begin{center}
    \includegraphics[width=0.4\textwidth]{Bilder/Test/1terAntritt/kreisprozess_waermekraftmaschine.png}
\end{center}

\subsection*{Aufgaben}
\begin{enumerate}
    \item Ordnen Sie die 4 Schritte den richtigen Seiten in der Grafik zu.
    \item In welchen Schritten verrichtet die Maschine eine Arbeit ($\Delta W_{ij}<0$) und in welchen Schritten wird Arbeit zugeführt ($\Delta W_{ij}>0$)? Tragen Sie die Arbeit mit Pfeilen auf den jeweiligen Seiten ein!
    \item Woran erkennt man in der Grafik, dass eine Nettoarbeit $\Delta W<0$ von der Maschine verrichtet wird? Tragen Sie die Nettoarbeit $\Delta W$ in der Grafik ein!
    \item Wie maximieren Sie den physikalischen Wirkungsgrad dieser Wärmekraftmaschine? Was ergeben sich für technische Schwierigkeiten, um diesen Wirkungsgrad praktisch zu erreichen?
\end{enumerate}
\vfill
\begin{flushright}
\renewcommand{\arraystretch}{1.23}
\begin{tabular}{|c | c | c | c|}
\hline
\multicolumn{4}{|l|}{\textbf{\large{Punkte: \quad\quad\quad /9}}}\\
\hline
\textbf{1)} & \textbf{2)} & \textbf{3)} & \textbf{4)} \\
\quad/2 & \quad/2 & \quad/2 & \quad/3 \\
\hline
\end{tabular}
\end{flushright}

\newpage

% ------------------------- BEISPIEL 5 -------------------------
\section*{Beispiel 5 - Phasendiagramm}

In der Grafik sehen Sie das Phasendiagramm von Wasser ($H_{2}O$).
\begin{center}
    \includegraphics[width=0.44\textwidth]{Bilder/Test/1terAntritt/phasendiagramm_wasser.png}
\end{center}

\subsection*{Aufgaben}
\begin{enumerate}[itemsep=8pt]
    \item Beschriften Sie die Schmelzkurve, die Dampfdruckkurve und die Sublimationskurve in der Grafik!
    \item Welches Vorzeichen hat die Ableitung $\frac{\dd p}{\dd T}$ entlang dieser drei Kurven?
    \item Die Schmelzwärme wird beschrieben durch folgende Gleichung:
    $$\Lambda_{\text{Schmelz}} = T \frac{\dd p}{\dd T}(V_{\text{Fl}} - V_{\text{Fest}})$$
    Erklären Sie anhand dieser Formel und der Grafik die Anomalie des Wassers.
    \item Erklären Sie, was passieren würde, wenn Sie eine Glasflasche bis zum Rand mit Wasser bei $T=\SI{20}{\degreeCelsius}$ füllen und verschlossen in den Gefrierschrank legen, um es gefrieren zu lassen!
\end{enumerate}

\vfill
\begin{flushright}
\renewcommand{\arraystretch}{1.23}
\begin{tabular}{|c | c | c | c|}
\hline
\multicolumn{4}{|l|}{\textbf{\large{Punkte: \quad\quad\quad /8}}}\\
\hline
\textbf{1)} & \textbf{2)} & \textbf{3)} & \textbf{4)} \\
\quad/2 & \quad/1 & \quad/3 & \quad/2 \\
\hline
\end{tabular}
\end{flushright}

\newpage

% ------------------------- BEISPIEL 6 -------------------------
\section*{Beispiel 6 - Raumstation}
\begin{minipage}[t]{0.55\textwidth}
    \vspace{0pt}
    Sie entwerfen eine Raumstation, die aus einem großen, ringförmigen Habitat besteht. Der Ring dreht sich im Weltraum, um durch Zentrifugalkraft eine künstliche Schwerkraft zu erzeugen. \\
    Der Ring hat einen Durchmesser von \SI{300}{\metre} und soll so konstruiert werden, dass die Bewohner auf der Innenseite des Rings dieselbe Beschleunigung erfahren wie auf der Erde, also $g=\SI{9.81}{\metre\per\second\squared}$.
\end{minipage}\hfill
\begin{minipage}[t]{0.42\textwidth}
    \vspace{0pt}
    \includegraphics[width=\textwidth]{Bilder/Test/1terAntritt/raumstation.png}
    \centering
\end{minipage}


\subsection*{Aufgaben}
\begin{enumerate}
    \item Berechnen Sie, mit welcher Winkelgeschwindigkeit ($\omega$ in \si{\radian\per\second}) sich der Ring drehen muss, damit die Bewohner genau die Erdbeschleunigung g spüren.
    \item Geben Sie zusätzlich an, wie viele Umdrehungen pro Minute das sind. \\
    Tipp: $\omega=2\pi f$, wobei $[f]=\si{\per\second}$.
\end{enumerate}

\vfill
\begin{flushright}
\renewcommand{\arraystretch}{1.23}
\begin{tabular}{|c | c|}
\hline
\multicolumn{2}{|l|}{\textbf{\large{Punkte: \quad /6}}}\\
\hline
\textbf{1)} & \textbf{2)} \\
\quad/3 & \quad/3 \\
\hline
\end{tabular}
\end{flushright}

\end{document}