\documentclass[11pt, a4paper]{article}

\usepackage[utf8]{inputenc}
\usepackage[T1]{fontenc}
\usepackage{amsmath}
\usepackage{amssymb}
\usepackage{graphicx}
\usepackage{geometry}
\usepackage{fancyhdr}
\usepackage{enumitem}
\usepackage{array}
\usepackage{eso-pic} % Required for placing the logo

% --- siunitx Konfiguration ---
\usepackage{siunitx}
\DeclareSIUnit{\litre}{l}
\sisetup{
    angle-symbol-degree = ^\circ,
    locale = DE, 
    separate-uncertainty, 
    inter-unit-product = \cdot, % Multiplikationspunkt zwischen Einheiten
    range-units = brackets, 
    list-units = single, 
    per-mode=symbol-or-fraction
}

% --- Seitenlayout ---
\geometry{a4paper, top=2.5cm, bottom=2.5cm, left=2.5cm, right=2.5cm}

% --- Benutzerdefinierte Befehle (aus defs.tex) ---
\newcommand{\mDot}{\,.}
\newcommand{\mComma}{\, ,\,}
\newcommand{\dd}{\mathrm{d}}
\newcommand{\eqexcl}{\stackrel{!}{=}}

% --- Kopf- und Fußzeile ---
\pagestyle{fancy}
\fancyhf{}
\cfoot{\thepage}

\begin{document}
% --- Titelseite ---
\noindent
\vspace{9cm}
\begin{center}
    {\Huge \textbf{Lösung}} \\[1.5em]
    {\Large \textbf{Physikalische Grundlagen}} \\[1em]
    {\large 3. Schriftliche Prüfung: 08.09.2025} \\[1em]
    {\large Studiengang: Clinical Engineering} 
\end{center}

\newpage

% ------------------------- LÖSUNG BEISPIEL 1 -------------------------
\section*{Lösung Beispiel 1 - Heißes Metall in Wasser}

Zuerst wird die Masse des Wassers berechnet:
$$ m_{\text{Wasser}} = \rho_{\text{Wasser}} \cdot V_{\text{Wasser}} = \SI{1}{\kilogram\per\deci\metre\cubed} \cdot \SI{5}{\litre} = \SI{5}{\kilogram} \mDot $$

\begin{enumerate}
    \item \textbf{Stellen Sie die Formel für die vom Wasser aufgenommene Wärme $Q_{\text{Wasser,auf}}$ auf.} \\
    Die vom Wasser aufgenommene Wärme berechnet sich aus der Masse, der spezifischen Wärmekapazität und der Temperaturdifferenz zwischen der Endtemperatur $T_{\text{Misch}}$ und der Anfangstemperatur $T_{\text{Wasser}}$.
    $$ Q_{\text{Wasser,auf}} = m_{\text{Wasser}} \cdot c_{\text{Wasser}} \cdot (T_{\text{Misch}} - T_{\text{Wasser}}) $$
    Das Wasser wird sich erwärmen, daher ist $(T_{\text{Misch}} - T_{\text{Wasser}}) > 0$. 

    \item \textbf{Stellen Sie die Formel für die von der Eisenkugel abgegebene Wärme $Q_{\text{Eisen,ab}}$ auf.} \\
    Analog dazu gibt die heiße Eisenkugel Wärme ab, bis sie die Mischtemperatur erreicht hat. Die Temperaturdifferenz ist hier die anfängliche Temperatur der Kugel $T_{\text{Kugel}}$ abzüglich der Mischtemperatur $T_{\text{Misch}}$.
    $$ Q_{\text{Eisen,ab}} = m_{\text{Kugel}} \cdot c_{\text{Eisen}} \cdot (T_{\text{Kugel}} - T_{\text{Misch}}) $$
    Die Eisenkugel kühlt sich ab, daher ist $(T_{\text{Kugel}} - T_{\text{Misch}}) > 0$. 

    
    \item \textbf{Nutzen Sie den Grundsatz des thermischen Gleichgewichts, indem Sie die beiden Wärmen gleichsetzen ($Q_{\text{Wasser,auf}} = Q_{\text{Eisen,ab}}$). Lösen Sie die resultierende Gleichung nach der Mischtemperatur $T_{\text{Misch}}$ auf.} \\
    Im thermischen Gleichgewicht ist die aufgenommene Wärme gleich der abgegebenen Wärme:
    \begin{align*}
        Q_{\text{Wasser,auf}} &\eqexcl Q_{\text{Eisen,ab}} \\
        m_{\text{Wasser}} \cdot c_{\text{Wasser}} \cdot (T_{\text{Misch}} - T_{\text{Wasser}}) &= m_{\text{Kugel}} \cdot c_{\text{Eisen}} \cdot (T_{\text{Kugel}} - T_{\text{Misch}})
    \end{align*}
    Diese Gleichung wird nun nach $T_{\text{Misch}}$ umgeformt:
    \begin{align*}
        m_W c_W T_{\text{Misch}} - m_W c_W T_W &= m_E c_E T_E - m_E c_E T_{\text{Misch}} && | \; + m_E c_E T_{\text{Misch}} \\
        (m_W c_W + m_E c_E) T_{\text{Misch}} - m_W c_W T_W &= m_E c_E T_E && | \; + m_W c_W T_W \\
        (m_W c_W + m_E c_E) T_{\text{Misch}} &= m_E c_E T_E + m_W c_W T_W  && | : (\ldots) \\
        T_{\text{Misch}} &= \frac{m_E c_E T_E + m_W c_W T_W}{m_W c_W + m_E c_E}
    \end{align*}
    Nun werden die gegebenen Werte eingesetzt:
    \begin{align*}
        T_{\text{Misch}} &= \frac{ \SI{0,5}{\kilo\gram} \cdot \SI{0,45}{\kilo\joule\per\kilo\gram\per\kelvin} \cdot \SI{250}{\celsius} + \SI{5}{\kilo\gram} \cdot \SI{4,18}{\kilo\joule\per\kilo\gram\per\kelvin} \cdot \SI{20}{\celsius}}{ \SI{5}{\kilo\gram} \cdot \SI{4,18}{\kilo\joule\per\kilo\gram\per\kelvin} + \SI{0,5}{\kilo\gram} \cdot \SI{0,45}{\kilo\joule\per\kilo\gram\per\kelvin}} \\
        T_{\text{Misch}} &= \frac{ \num{474,25} }{ \num{21,125}} \,\si{\celsius} \approx \SI{22,45}{\celsius}
    \end{align*}
    Die Mischtemperatur beträgt also rund \textbf{\SI{22,45}{\degreeCelsius}}.

\end{enumerate}



\section*{Lösung Beispiel 2 - Rutsche mit Reibung}

\begin{enumerate}
    \item \textbf{Tragen Sie alle wirkenden Kräfte in der Grafik ein!} \\
    Die wirkenden Kräfte sind die Gewichtskraft $F_G$ (vertikal nach unten), die Normalkraft $F_N$ (senkrecht zur Rutschfläche) und die Gleitreibungskraft $F_{R, \text{Gleit}}$ (entgegen der Bewegungsrichtung, also die Rutsche hinauf).
    \begin{center}
        \includegraphics[width=0.5\textwidth]{Bilder/Test/3terAntritt/kind_rutsche_lösung.png}
    \end{center}

    \item \textbf{Berechnen Sie den Neigungswinkel $\theta$ der Rutsche.} \\
    Der Winkel $\theta$ kann über die Höhe $h$ und die Länge $L$ der Rutsche mit dem Sinus bestimmt werden:
    $$ \sin(\theta) = \frac{\text{Gegenkathete}}{\text{Hypotenuse}} = \frac{h}{L} = \frac{\SI{3}{\metre}}{\SI{5}{\metre}} = 0,6 $$
    $$ \theta = \arcsin(0,6) \approx \SI{36,87}{\degree} = \SI{0.6435}{\radian} \mDot $$

    \item \textbf{Zerlegen Sie die Gewichtskraft in eine Komponente senkrecht, $F_{G, \perp}$, und eine parallel, $F_{G, \parallel}$, zur Rutsche.} \\
    Die Gewichtskraft ist 
    $$F_G = m \cdot g = \SI{35}{\kg} \cdot \SI{9,81}{\m\per\s\squared} = \SI{343,35}{\newton} \mDot$$
    Die Zerlegung der Gewichtskraft ist $F_{G,x} = F_{G,\parallel}$ und $F_{G,y} = F_{G, \perp}$: 
    $$ F_{G,x} = F_{G, \parallel} = F_G \cdot \sin(\theta) = \SI{343,35}{\newton} \cdot 0,6 = \SI{206,01}{\newton} $$
    $$ F_{G,y} = F_{G, \perp} = F_G \cdot \cos(\theta) = \SI{343,35}{\newton} \cdot 0,8 = \SI{274,68}{\newton} $$
    
    % \item \textbf{Berechnen Sie die Normalkraft und daraus die Gleitreibungskraft $F_{R, \text{Gleit}}$.} \\
    % Die Normalkraft $F_N$ entspricht genau der Komponente $F_{G, \perp}$, somit ist $F_N = F_G\cdot \cos(\theta) = \SI{274,68}{\newton}$.
    % Die Gleitreibungskraft ist das Produkt aus dem Gleitreibungskoeffizienten und der Normalkraft:
    % $$ F_{R, \text{Gleit}} = \mu_{\text{Gleit}} \cdot F_N = 0,2 \cdot \SI{274,68}{\newton} = \SI{54,936}{\newton} $$

   \item \textbf{Berechnen Sie die Normalkraft $F_N$ und die Gleitreibungskraft $F_{R, \text{Gleit}}$.} \\
    Die Normalkraft wirkt der senkrechten Komponente der Gewichtskraft entgegen und ist daher betragsmäßig gleich groß:
    $$ F_N = F_{G, \perp} = F_G \cdot \cos(\theta) = \SI{274,68}{\newton} $$
    Daraus ergibt sich die Gleitreibungskraft:
    $$ F_{R, \text{Gleit}} = \mu_{\text{Gleit}} \cdot F_N = 0,2 \cdot \SI{274,68}{\newton} = \SI{54,936}{\newton} $$

    \item \textbf{Berechnen Sie die Hangabtriebskraft $F_A$.} \\
    Die Hangabtriebskraft ist die Kraft, die das Kind die Rutsche hinabzieht. Sie entspricht der parallelen Komponente der Gewichtskraft:
    $$ F_A = F_{G, \parallel} = \SI{206,01}{\newton} $$

    \item \textbf{Berechnen Sie die resultierende Nettokraft und die Beschleunigung.} \\
    Die Nettokraft $F_{\text{netto}}$ in Bewegungsrichtung ($x$-Richtung) ist die Differenz aus Hangabtriebskraft und der entgegenwirkenden Gleitreibungskraft.
    $$ F_{\text{netto}} = F_A - F_{R, \text{Gleit}} = \SI{206,01}{\newton} - \SI{54,936}{\newton} = \SI{151,074}{\newton} \mDot$$
    Nach dem zweiten Newtonschen Gesetz ($F=ma$) ergibt sich die Beschleunigung $a$:
    $$ a_\text{netto} = \frac{F_{\text{netto}}}{m} = \frac{\SI{151,074}{\newton}}{\SI{35}{\kg}} \approx \SI{4,316}{\metre\per\second\squared} $$

    



    \item \textbf{Leiten Sie aus diesen zeitabhängigen Formeln eine Formel für $v(L)$ her.} \\
    Die gegebenen Formeln sind:
    \begin{gather*}
        s(t) = s_0 + v_0\cdot t + \frac{1}{2}a\cdot t^2 \\
        v(t) = v_0 + a \cdot t
    \end{gather*}
    Die Anfangsbedingungen sind \textbf{Start aus dem Stillstand} ($v_0 = 0$) am \textbf{Startpunkt} ($s_0 = 0$). Die Gleichungen vereinfachen sich zu:
    \begin{gather*}
        s(t) = \frac{1}{2}a\cdot t^2\\
        v(t) = a \cdot t
    \end{gather*}
    Aus der ersten Gleichung erhalten wir für $s = L$
    $$ L = \frac{1}{2}a \cdot t^2 \implies t = \sqrt{\frac{2 L}{a}}$$
    Dies setzen wir in die zweite Gleichung ein und erhalten die gesuchte Beziehung
    $$ v = a \sqrt{\frac{2 L}{a}} = \sqrt{\frac{2L a^2}{a}}\implies \boldsymbol{v = \sqrt{2aL}} $$

    \item \textbf{Verwenden Sie die soeben hergeleitete Formel, um die Endgeschwindigkeit nach der zurückgelegten Strecke $L$ zu berechnen.} \\
    Mit der in Schritt 6 berechneten Beschleunigung $a$ und der Streckenlänge $L$ erhalten wir:
    $$ v = \sqrt{2 \cdot a \cdot L} = \sqrt{2 \cdot \SI{4,316}{\m\per\s\squared} \cdot \SI{5}{\m}} = \sqrt{\SI{43,16}{\m\squared\per\s\squared}} \approx \SI{6,57}{\metre\per\second} $$
    Die Endgeschwindigkeit des Kindes beträgt somit ca. \textbf{\SI{6,57}{\metre\per\second}}.
\end{enumerate}


\newpage
\section*{Lösung Beispiel 3 - Holzklotz im Wasser}

\begin{enumerate}
    \item \textbf{Formulieren Sie die Bedingung für das Schwimmen eines Körpers.} \\
    Ein Körper schwimmt, wenn sich seine abwärts gerichtete Gewichtskraft ($F_G$) und die aufwärts gerichtete Auftriebskraft ($F_A$) im Gleichgewicht befinden. Es gilt also:
    $$ F_G = F_A $$

    \item \textbf{Berechnen Sie das Gesamtvolumen $V_{\text{ges}}$ des Holzklotzes und daraus seine Gewichtskraft $F_G$.} \\
    Die Kantenlänge beträgt $L = \SI{20}{\centi\metre} = \SI{0,2}{\metre}$. Das Gesamtvolumen des Würfels ist:
    $$ V_{\text{ges}} = L^3 = (\SI{0,2}{\metre})^3 = \SI{0,008}{\metre\cubed} \mDot$$
    Die Gewichtskraft ist das Produkt aus Dichte, Volumen und Erdbeschleunigung:
    $$ F_G = \rho_{\text{Holz}} \cdot V_{\text{ges}} \cdot g = \rho_{\text{Holz}} \cdot L^3 \cdot g =  \SI{750}{\kg\per\m\cubed} \cdot \SI{0,008}{\m\cubed} \cdot \SI{9,81}{\m\per\s\squared} = \SI{58,86}{\newton} \mDot$$

    \item \textbf{Stellen Sie eine Formel für die Auftriebskraft $F_A$ auf.} \\
    Nach dem Archimedischen Prinzip entspricht die Auftriebskraft dem Gewicht der verdrängten Flüssigkeit (hier Wasser). Sie hängt vom eingetauchten Volumen $V_{\text{ein}}$ ab
    $$ F_A = \rho_{\text{Wasser}} \cdot V_{\text{ein}} \cdot g \mDot$$

    \item \textbf{Drücken Sie das eingetauchte Volumen $V_{\text{ein}}$ durch die Eintauchtiefe $h$ aus.} \\
    Das eingetauchte Volumen ist das Produkt aus der Grundfläche $A = L^2$ des Klotzes und der gesuchten Eintauchtiefe $h$
    $$ V_{\text{ein}} = A \cdot h = L^2 \cdot h \mDot$$
    Eingesetzt in die Formel für die Auftriebskraft ergibt sich:
    $$ F_A = \rho_{\text{Wasser}} \cdot V_{\text{ein}} \cdot g = \rho_{\text{Wasser}} \cdot L^2 \cdot h \cdot g $$

    \item \textbf{Setzen Sie die Kräfte gleich und lösen Sie nach der Eintauchtiefe $h$ auf.} \\
    Wir setzen die in (1) formulierte Gleichgewichtsbedingung mit den hergeleiteten Termen an:
    \begin{align*}
        F_G &= F_A \\
        \rho_{\text{Holz}} \cdot V_{\text{ges}} \cdot g &= \rho_{\text{Wasser}} \cdot V_{\text{ein}} \cdot g \\
        \rho_{\text{Holz}} \cdot L^3 \cdot g &= \rho_{\text{Wasser}} \cdot L^2 \cdot h \cdot g
    \end{align*}
    Wir können auf beiden Seiten durch $g$ und $L^2$ kürzen:
    $$ \rho_{\text{Holz}} \cdot L = \rho_{\text{Wasser}} \cdot h $$
    Nun lösen wir nach $h$ auf:
    $$ h = L \cdot \frac{\rho_{\text{Holz}}}{\rho_{\text{Wasser}}} $$
    \item \textbf{Wie tief taucht der Holzblock ein?}
    Einsetzen der Zahlenwerte:
    $$ h = \SI{0,2}{\metre} \cdot \frac{\SI{750}{\kg\per\m\cubed}}{\SI{1000}{\kg\per\m\cubed}} = \SI{0,2}{\metre} \cdot 0,75 = \SI{0,15}{\metre} $$
    Die Eintauchtiefe des Holzklotzes beträgt \textbf{\SI{15}{\centi\metre}}.
\end{enumerate}


\newpage
\section*{Lösung Beispiel 4 - Multiple-Choice}

\begin{enumerate}[label=\arabic*.,itemsep=18pt]
    \item Was ist die SI-Einheit der Leistung?
    \begin{itemize}[label={$\square$},itemsep=1pt,topsep=0pt]
        \item Joule [\si{\joule}]
        \item[$\boxtimes$] Watt [\si{\watt}]
        \item Newton [\si{\newton}]
        \item Pascal [\si{\pascal}]
    \end{itemize}

    \item Aus welchen SI-Basiseinheiten ist ein Joule zusammengesetzt?
    $$ 1\,\si{J} = \textbf{\si{\kilogram\cdot\metre\squared\per\second\squared}} $$

    \item Ein Objekt mit Masse $m$ wird auf die doppelte Geschwindigkeit beschleunigt. Um welchen Faktor ändert sich seine kinetische Energie? ($E_{\text{kin}} \propto v^2 \implies (2v)^2 = 4v^2$)
    \begin{itemize}[label={$\square$},itemsep=1pt,topsep=0pt]
        \item Faktor 2
        \item Faktor 8
        \item[$\boxtimes$] Faktor 4
        \item Faktor $\sqrt{2}$
    \end{itemize}

    \item Welche Aussage über Reibungskräfte ist im Allgemeinen korrekt?
    \begin{itemize}[label={$\square$},itemsep=1pt,topsep=0pt]
        \item Die Gleitreibungskraft ist größer als die maximale Haftreibungskraft.
        \item[$\boxtimes$] Die maximale Haftreibungskraft ist größer als die Gleitreibungskraft.
        \item Haft- und Gleitreibungskraft sind immer gleich groß.
        \item Reibungskräfte sind immer proportional zur Kontaktfläche.
    \end{itemize}

    \item Bei einem isochoren Prozess eines idealen Gases bleibt welche Zustandsgröße konstant?
    \begin{itemize}[label={$\square$},itemsep=1pt,topsep=0pt]
        \item Der Druck
        \item Die innere Energie
        \item Die Temperatur
        \item[$\boxtimes$] Das Volumen
    \end{itemize}
    
    \item Nach dem Archimedischen Prinzip ist die Auftriebskraft auf einen vollständig in einer Flüssigkeit eingetauchten Körper gleich:
    \begin{itemize}[label={$\square$},itemsep=1pt,topsep=0pt]
        \item dem Gewicht des Körpers.
        \item dem Volumen des Körpers mal der Erdbeschleunigung.
        \item[$\boxtimes$] dem Gewicht der vom Körper verdrängten Flüssigkeit.
        \item der Dichte des Körpers mal seinem Volumen.
    \end{itemize}

    \newpage
    \item Der erste Hauptsatz der Thermodynamik ($\Delta U = Q + W$) ist eine spezielle Formulierung des...
    \begin{itemize}[label={$\square$},itemsep=1pt,topsep=0pt]
        \item Impulserhaltungssatzes.
        \item[$\boxtimes$] Energieerhaltungssatzes.
        \item Drehimpulserhaltungssatzes.
        \item Massenerhaltungssatzes.
    \end{itemize}
    
    \item Ein Auto ($m=\SI{1000}{\kg}$) fährt mit $v = \SI{10}{\metre\per\second}$ durch eine Kurve mit $r = \SI{50}{\metre}$. Wie groß ist die benötigte Zentripetalkraft?
    ($F_Z = \frac{mv^2}{r} = \frac{\SI{1000}{\kg} \cdot (\SI{10}{\m\per\s})^2}{\SI{50}{\m}} = \SI{2000}{\newton}$)
    \begin{itemize}[label={$\square$},itemsep=1pt,topsep=0pt]
        \item \SI{1000}{\newton}
        \item[$\boxtimes$] \SI{2000}{\newton}
        \item \SI{5000}{\newton}
        \item \SI{200}{\newton}
    \end{itemize}

    \item Der Wirkungsgrad $\eta$ einer Wärmekraftmaschine ist definiert als das Verhältnis von \\[9pt] \textbf{gewonnener (Netto-)Arbeit} zu \textbf{zugeführter Wärme}.
    
    \item Wie lautet der Name der Zustandsgleichung für reale Gase, die das Eigenvolumen der Gasteilchen und die Anziehungskräfte zwischen ihnen berücksichtigt? \\[9pt]
    \textbf{Van-der-Waals-Gleichung}

    \item Die absolute Temperatur $T$ eines idealen Gases ist ein direktes Maß für die ...
    \begin{itemize}[label={$\square$},itemsep=1pt,topsep=0pt]
        \item mittlere potentielle Energie der Gasteilchen.
        \item mittlere Geschwindigkeit der Gasteilchen.
        \item[$\boxtimes$] mittlere kinetische Energie der Gasteilchen.
        \item Gesamtanzahl der Gasteilchen.
    \end{itemize}
\end{enumerate}



\end{document}