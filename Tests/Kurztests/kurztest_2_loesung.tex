\documentclass[11pt, a4paper]{article}

\usepackage[utf8]{inputenc}
\usepackage[T1]{fontenc}
\usepackage{amsmath}
\usepackage{amssymb}
\usepackage{graphicx}
\usepackage{geometry}
\usepackage{fancyhdr}
\usepackage{enumitem}
\usepackage{array}
\usepackage{eso-pic} % Required for placing the logo
\usepackage{esvect}
\usepackage[most]{tcolorbox} % Needed for solution box

\usepackage{siunitx}
\DeclareSIUnit{\litre}{l}
\sisetup{
    locale = DE,
    inter-unit-product = \cdot,
    per-mode = symbol-or-fraction
}

\geometry{a4paper, top=2.5cm, bottom=2.5cm, left=2.5cm, right=2.5cm}


% Needed for the bibliography style (.bst files)
\providecommand{\Verfuegbar}{Verf{\"u}gbar}


%defs for the paper
\newcommand{\mDot}{\,.}
\newcommand{\mComma}{\,{,}\,}

% text subscripts
\newcommand{\kin}{\mathrm{kin}}
\newcommand{\pot}{\mathrm{pot}}
\newcommand{\rot}{\mathrm{rot}}
\newcommand{\trans}{\mathrm{trans}}
\newcommand{\atm}{\mathrm{atm}}
\newcommand{\minText}{\mathrm{min}}
\newcommand{\maxText}{\mathrm{max}}
% quantities with text subscripts
\newcommand{\Ekin}{E_{\kin}}
\newcommand{\Epot}{E_{\pot}}
\newcommand{\Erot}{E_{\rot}}
\newcommand{\Etrans}{E_{\trans}}
\newcommand{\kB}{k_{\mathrm{B}}}

% Text 
\newcommand{\Schro}{Schr\"o\-din\-ger }

% vectors 
\newcommand{\ivec}[1]{\vv{#1}} % using esvect package
\newcommand{\ivecS}[2]{\vv*{#1}{\!#2}} % using esvect package
% column vector
\newcommand{\icolTwo}[2]{\begin{pmatrix} #1 \\ #2 \end{pmatrix}}
\newcommand{\icolThree}[3]{\begin{pmatrix} #1 \\ #2 \\ #3 \end{pmatrix}}
% row vector (INLINE)
\newcommand{\inlrowTwo}[2]{(#1, #2)}
\newcommand{\inlrowThree}[3]{(#1, #2, #3)}

% point 
\newcommand{\ipTwo}[2]{(#1\!\mid\! #2)}
\newcommand{\ipThree}[3]{(#1\!\mid\! #2\!\mid\! #3)}

% rangle, langle 
\newcommand{\lrangle}[1]{{\langle{#1}\rangle}}
% measurement units
\newcommand{\Unit}[1]{\,\mathrm{#1}}

\newcommand{\msSp}{\;}
\newcommand{\mdSp}{\;\;}
\newcommand{\mtSp}{\;\;\;}
\newcommand{\mqSp}{\;\;\;\;}

  
%mathematical symbols
\newcommand{\defeq}{\vcentcolon=}
\newcommand{\eqdef}{=\vcentcolon}
\newcommand*\conj[1]{\bar{#1}}
\newcommand{\eqexcl}{\stackrel{!}{=}}
\newcommand{\eqquestion}{\stackrel{?}{=}}
\newcommand\equalhatInl{\mathrel{\stackon[1.0pt]{=}{\stretchto{%
    \scalerel*[\widthof{=}]{\wedge}{\rule{1ex}{3ex}}}{0.45ex}}}}
\newcommand\equalhat{\mathrel{\stackon[4.8pt]{=}{\stretchto{%
    \scalerel*[\widthof{=}]{\wedge}{\rule{1ex}{3ex}}}{0.45ex}}}}
\newcommand{\mAND}{\land}
\newcommand{\mOR}{\lor}
\newcommand{\mNOT}{\lnot}

% text editing
\newcommand{\textunderscript}[1]{$_{\text{#1}}$}
\newcommand{\textupperscript}[1]{$^{\text{#1}}$}
\newcommand{\eqqref}[1]{eq.\!~(\ref{#1})}
\newcommand{\Eqqref}[1]{Eq.\!~(\ref{#1})}
\newcommand{\figref}[1]{fig.\!~(\ref{#1})}
\newcommand{\Figref}[1]{Fig.\!~(\ref{#1})}
\newcommand{\secref}[1]{sec.\!~(\ref{#1})}
\newcommand{\Secref}[1]{Sec.\!~(\ref{#1})}

%general abbreviations (in German)
\newcommand{\wA}{\mbox{w.\,A.\ }}
\newcommand{\fA}{\mbox{f.\,A.\ }}
\newcommand{\zB}{\mbox{z.\,B.\ }}
\newcommand{\bzw}{\mbox{bzw.\ }}
\newcommand{\gDh}{\mbox{d.\,h.\ }}
\newcommand{\gDQ}[1]{\glqq #1\grqq}
\newcommand{\oBdA}{\mbox{o.\,B.\,d.\,A.\ }}
\newcommand{\sEUR}{\text{\euro}}

%latin abbreviations
\newcommand{\etal}{\mbox{\emph{et al.\ }}}
\newcommand{\exgrat}{\mbox{e.g.\ }}
\newcommand{\idest}{\mbox{i.e.\ }}

%general math terms
\newcommand{\const}{\mathrm{const}}
\newcommand{\bigO}{\mathcal{O}}

% Lorem ipsum
\newcommand*{\QEDA}{\hfill\ensuremath{\blacksquare}}%
\newcommand*{\QEDB}{\hfill\ensuremath{\square}}%

%  ------------------ abbreviations

%matrix operations
\newcommand{\T}{T}
\DeclareMathOperator{\arcsinh}{arcsinh}
\DeclareMathOperator{\Tr}{Tr}
\DeclareMathOperator{\argg}{arg}
\DeclareMathOperator{\Arg}{arg}
\DeclareMathOperator{\codim}{codim}
\DeclareMathOperator{\atanTwo}{atan2}
\DeclareMathOperator{\diag}{diag}

%real and complex numbers latin Letters
\newcommand{\Real}{\mathbb{R}}
\newcommand{\Complex}{\mathbb{C}}
\newcommand{\Integer}{\mathbb{N}}

% differentials 
\newcommand{\dd}{\mathrm{d}}



% --- Lösungsbox-Umgebung ---
\definecolor{boxcol_back_lgreen}{RGB}{220, 255, 220}
\definecolor{boxcol_back_white}{RGB}{255, 255, 255}
\definecolor{boxcol_frame_green}{RGB}{30, 180, 30}
\newtcolorbox{solutionbox}[1][]{
  colback=boxcol_back_white,
  colframe=boxcol_frame_green,
  colbacktitle=boxcol_back_lgreen,
  fonttitle=\bfseries,
  coltitle=black,
  boxsep=5pt,
  arc=1mm,
  enhanced,
  attach boxed title to top left={yshift=-2mm, xshift=3mm},
  title=Lösung,
  #1
}
% ---

\pagestyle{fancy}
\fancyhf{}
\cfoot{\thepage}

\begin{document}
% Command to add the logo to the top left of the first page
\AddToShipoutPictureBG*{%
  \AtPageUpperLeft{%
    \hspace{2.0cm}% Move logo to the right from the page edge
    \raisebox{-5.cm}{% Move logo down from the page edge
      \includegraphics[width=4cm]{../../Bilder/Allgemein/LogoFHCampus.png}% The logo file
    }%
  }%
}

\begin{center}
    {\Large \textbf{Physikalische Grundlagen}} \\[1em]
    {\large Kurztest 2, 03.03.2026 - \textbf{LÖSUNG}} \\[1em]
    {\large Studiengang: Clinical Engineering}
\end{center}

\vspace{1cm}

\noindent
\vspace{1cm}
\noindent
\begin{tabular}{ll}
    \textbf{Kürzel:} & \underline{\hspace{8cm}} \\[0.5cm]
    \textbf{Gruppe: } & \LARGE\textbf{A}  \\[0.3cm]
\end{tabular}

\begin{itemize}[label={$\circ$}]
    \item Sie haben 20 min Zeit.
    \item Sie dürfen einen Taschenrechner verwenden.
    \item Geben Sie klare und verständliche Antworten!
    \item Schreiben Sie leserlich!
    \item Streichen Sie alle bis auf eine Lösung durch!
\end{itemize}

Viel Erfolg!

\vspace{2cm}

\begin{flushright}
    \renewcommand{\arraystretch}{1.23}
    \begin{tabular}{l p{2.3cm}}
        \hline
        \noalign{\vskip 0.15cm}
        \textbf{\Large{Punkte:}} & \textbf{\Large{/48}} \\
        \noalign{\vskip 0.1cm}
        \hline
    \end{tabular}
\end{flushright}

\newpage

\section*{Kurztest 2: Grundlagen, Metrologie und Vektorrechnung}
    
\begin{enumerate}[label=\textbf{\arabic*.},itemsep=0.50cm]
    
    % --- Kapitel 1/3: SI-Basiseinheiten ---
    \item \textbf{SI-Basiseinheiten:} \\
    Nenne mindestens 3 SI-Basiseinheiten mit ihren Einheitensymbolen.
    \begin{solutionbox}
        \begin{itemize}[itemsep=0pt, topsep=0pt]
            \item Meter (m) für die Länge
            \item Kilogramm (kg) für die Masse
            \item Sekunde (s) für die Zeit
            \item (Weitere: Ampere (A), Kelvin (K), Mol (mol), Candela (cd))
        \end{itemize}
    \end{solutionbox}
    
    % --- Kapitel 3: Einheitenumrechnung ---
    \item \textbf{Einheitenumrechnung:} \\
    Ein Auto fährt mit einer Geschwindigkeit von \SI{72}{\kilo\meter\per\hour}. Rechne diese Geschwindigkeit in \unit{\meter\per\second} um und zeige deine Rechnung.    
    \begin{solutionbox}
        \[ \SI{72}{\kilo\meter\per\hour} = 72 \cdot \frac{\SI{1000}{\meter}}{\SI{3600}{\second}} = \frac{72000}{3600} \si{\meter\per\second} = \frac{720}{36} \si{\meter\per\second} = \SI{20}{\meter\per\second} \]
        Oder einfacher, indem man durch 3,6 dividiert:
        \[ \frac{72}{3.6} = \SI{20}{\meter\per\second} \]
    \end{solutionbox}
    
    % --- Dimensionsanalyse ---
    \item \textbf{Dimensionsanalyse:} \\
    Sie wissen, dass die Fallzeit $t$ eines Objekts von der Anfangshöhe $h$ und von der Erdbeschleunigung $g$ abhängt.
    Verwenden Sie die Dimensionsanalyse, um die korrekte Proportionalitätsrelation zwischen der Fallzeit $t$ und den Größen $h$ und $g$ herzuleiten. \\
    \textit{Hinweis:} $[t] = T$  (Zeit), $[h] = L$ (Länge) und $[g] = L/T^2$ (Länge pro Zeit zum Quadrat).
    \begin{solutionbox}
        Ansatz: $t \propto h^a \cdot g^b$.
        Dimensionsgleichung:
        \[ [t] = [h]^a \cdot [g]^b \implies T = L^a \cdot \left(\frac{L}{T^2}\right)^b = L^{a+b} \cdot T^{-2b} \]
        Koeffizientenvergleich für die Dimensionen:
        \begin{itemize}
            \item $L: \quad 0 = a+b \implies a = -b$
            \item $T: \quad 1 = -2b \implies b = -1/2$
        \end{itemize}
        Daraus folgt $a = -(-1/2) = 1/2$. Die Proportionalität ist also:
        \[ t \propto h^{1/2} \cdot g^{-1/2} \implies t \propto \sqrt{\frac{h}{g}} \]
    \end{solutionbox}
    
    % --- Kapitel 3: Radiant und Grad ---
    \item \textbf{Radiant und Grad:} \\
    Gib die Beziehung zwischen Radiant und Grad für einen Vollkreis an. \\
    Wie viele \si{\radian} entsprechen demnach \SI{1}{\degree}?
    \begin{solutionbox}
        Ein Vollkreis hat \SI{360}{\degree}, was $2\pi$ Radiant entspricht.
        \[ \SI{360}{\degree} = 2\pi \, \si{\radian} \]
        Daraus folgt für \SI{1}{\degree}:
        \[ \SI{1}{\degree} = \frac{2\pi}{360} \, \si{\radian} = \frac{\pi}{180} \, \si{\radian} \approx \SI{0.01745}{\radian} \]
    \end{solutionbox}
    
    % --- Kapitel 4: Systematische vs. Statistische Fehler ---
    \item \textbf{Fehlerarten:} \\
    Sie möchten die Periodenlänge $T$ eines Pendels mit einer Stoppuhr messen und übersehen, dass der Knoten des Fadens im Aufhängepunkt zu einer nicht vernachlässigbaren Reibung führt. 
    Was passiert mit Ihren Messwerten $x_i$ im Vergleich zum Wert $x_W$? Um welche Art von Fehler handelt es sich dabei? 
    \begin{solutionbox}
        Die Reibung führt zu einer Dämpfung, wodurch die gemessene Periodendauer tendenziell länger ist als die des idealen, reibungsfreien Pendels. Die Messwerte $x_i$ werden also systematisch größer sein als der wahre Wert $x_W$.
        Es handelt sich um einen \textbf{systematischen Fehler}.
    \end{solutionbox}
    
    % --- Anhang: Vektorrechnung - Eigenschaften ---
    \item \textbf{Eigenschaften von Vektoren:} \\
    Was sind die beiden definierenden Eigenschaften von Vektoren?
    \begin{solutionbox}
        Ein Vektor ist durch seine \textbf{Länge} (Betrag) und seine \textbf{Richtung} definiert.
    \end{solutionbox}
    
    % --- Anhang: Betrag eines 3D-Vektors ---
    \item \textbf{Betrag eines Vektors:} \\
    Berechne den Betrag (die Länge) des Vektors $\ivec{a} = (2, -3, 6)$ in $\mathbb{R}^3$ und vereinfache das Ergebnis.
    \begin{solutionbox}
        \[ |\ivec{a}| = \sqrt{2^2 + (-3)^2 + 6^2} = \sqrt{4 + 9 + 36} = \sqrt{49} = 7 \]
    \end{solutionbox}
    
    % --- Anhang: Einheitsvektor berechnen ---
    \item \textbf{Normierung eines Vektors:} \\
    Gib den Einheitsvektor $\ivecS{e}{a}$ des Vektors $\ivec{a} = (1, 0, 2)$ an.
    \begin{solutionbox}
        \[ |\ivec{a}| = \sqrt{1^2 + 0^2 + 2^2} = \sqrt{5} \]
        \[ \ivecS{e}{a} = \frac{\ivec{a}}{|\ivec{a}|} = \frac{1}{\sqrt{5}} \begin{pmatrix} 1 \\ 0 \\ 2 \end{pmatrix} = \begin{pmatrix} 1/\sqrt{5} \\ 0 \\ 2/\sqrt{5} \end{pmatrix} \]
    \end{solutionbox}
    
    % --- Anhang: Inneres Produkt ---
    \item \textbf{Skalarprodukt:} \\
    Berechne das innere Produkt (Skalarprodukt) der Vektoren $\ivec{u} = (7, 1)$ und $\ivec{v} = (-3, -4)$.
    Mit welcher Formel könnten Sie den Winkel $\alpha$ zwischen den beiden Vektoren berechnen?
    \begin{solutionbox}
        Skalarprodukt: \[\ivec{u} \cdot \ivec{v} = (7)(-3) + (1)(-4) = -25 .\] \\[0.1cm]
        Formel für den Winkel zwischen zwei Vektoren:
        \[ \alpha = \arccos\left(\frac{\ivec{u} \cdot \ivec{v}}{|\ivec{u}| \cdot |\ivec{v}|}\right) \]
    \end{solutionbox}
    
    % --- Anhang: Orthogonalität ---
    \item \textbf{Orthogonale Vektoren:} \\
    Überprüfe, ob die Vektoren $\ivec{u} = (3, 2, -1)$ und $\ivec{v} = (1, -2, -1)$ orthogonal (senkrecht) zueinander sind. 
    Begründe deine Antwort.
    \begin{solutionbox}
        Zwei Vektoren sind orthogonal, wenn ihr Skalarprodukt Null ist.
        \[ \ivec{u} \cdot \ivec{v} = (3)(1) + (2)(-2) + (-1)(-1) = 3 - 4 + 1 = 0 \]
        Da das Skalarprodukt 0 ist, sind die Vektoren orthogonal.
    \end{solutionbox}
    
\end{enumerate}

\vspace{1cm}

\end{document}