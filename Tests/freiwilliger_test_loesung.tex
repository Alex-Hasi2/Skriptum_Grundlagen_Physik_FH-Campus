\documentclass[11pt, a4paper, ngerman]{article}

\usepackage[utf8]{inputenc}
\usepackage[T1]{fontenc}
\usepackage{amsmath}
\usepackage{amssymb}
\usepackage{graphicx}
\usepackage{geometry}
\usepackage{fancyhdr}
\usepackage{enumitem}
\usepackage{array}
\usepackage{eso-pic} % Required for placing the logo

\usepackage{siunitx}
\DeclareSIUnit{\litre}{l}
\sisetup{
    locale = DE,
    inter-unit-product = \cdot,
    per-mode = symbol-or-fraction,
    range-phrase = { bis }
}

\geometry{a4paper, top=2.5cm, bottom=2.5cm, left=2.5cm, right=2.5cm}

% Needed for the bibliography style (.bst files)
\providecommand{\Verfuegbar}{Verf{\"u}gbar}


%defs for the paper
\newcommand{\mDot}{\,.}
\newcommand{\mComma}{\,{,}\,}

% text subscripts
\newcommand{\kin}{\mathrm{kin}}
\newcommand{\pot}{\mathrm{pot}}
\newcommand{\rot}{\mathrm{rot}}
\newcommand{\trans}{\mathrm{trans}}
\newcommand{\atm}{\mathrm{atm}}
\newcommand{\minText}{\mathrm{min}}
\newcommand{\maxText}{\mathrm{max}}
% quantities with text subscripts
\newcommand{\Ekin}{E_{\kin}}
\newcommand{\Epot}{E_{\pot}}
\newcommand{\Erot}{E_{\rot}}
\newcommand{\Etrans}{E_{\trans}}
\newcommand{\kB}{k_{\mathrm{B}}}

% Text 
\newcommand{\Schro}{Schr\"o\-din\-ger }

% vectors 
\newcommand{\ivec}[1]{\vv{#1}} % using esvect package
\newcommand{\ivecS}[2]{\vv*{#1}{\!#2}} % using esvect package
% column vector
\newcommand{\icolTwo}[2]{\begin{pmatrix} #1 \\ #2 \end{pmatrix}}
\newcommand{\icolThree}[3]{\begin{pmatrix} #1 \\ #2 \\ #3 \end{pmatrix}}
% row vector (INLINE)
\newcommand{\inlrowTwo}[2]{(#1, #2)}
\newcommand{\inlrowThree}[3]{(#1, #2, #3)}

% point 
\newcommand{\ipTwo}[2]{(#1\!\mid\! #2)}
\newcommand{\ipThree}[3]{(#1\!\mid\! #2\!\mid\! #3)}

% rangle, langle 
\newcommand{\lrangle}[1]{{\langle{#1}\rangle}}
% measurement units
\newcommand{\Unit}[1]{\,\mathrm{#1}}

\newcommand{\msSp}{\;}
\newcommand{\mdSp}{\;\;}
\newcommand{\mtSp}{\;\;\;}
\newcommand{\mqSp}{\;\;\;\;}

  
%mathematical symbols
\newcommand{\defeq}{\vcentcolon=}
\newcommand{\eqdef}{=\vcentcolon}
\newcommand*\conj[1]{\bar{#1}}
\newcommand{\eqexcl}{\stackrel{!}{=}}
\newcommand{\eqquestion}{\stackrel{?}{=}}
\newcommand\equalhatInl{\mathrel{\stackon[1.0pt]{=}{\stretchto{%
    \scalerel*[\widthof{=}]{\wedge}{\rule{1ex}{3ex}}}{0.45ex}}}}
\newcommand\equalhat{\mathrel{\stackon[4.8pt]{=}{\stretchto{%
    \scalerel*[\widthof{=}]{\wedge}{\rule{1ex}{3ex}}}{0.45ex}}}}
\newcommand{\mAND}{\land}
\newcommand{\mOR}{\lor}
\newcommand{\mNOT}{\lnot}

% text editing
\newcommand{\textunderscript}[1]{$_{\text{#1}}$}
\newcommand{\textupperscript}[1]{$^{\text{#1}}$}
\newcommand{\eqqref}[1]{eq.\!~(\ref{#1})}
\newcommand{\Eqqref}[1]{Eq.\!~(\ref{#1})}
\newcommand{\figref}[1]{fig.\!~(\ref{#1})}
\newcommand{\Figref}[1]{Fig.\!~(\ref{#1})}
\newcommand{\secref}[1]{sec.\!~(\ref{#1})}
\newcommand{\Secref}[1]{Sec.\!~(\ref{#1})}

%general abbreviations (in German)
\newcommand{\wA}{\mbox{w.\,A.\ }}
\newcommand{\fA}{\mbox{f.\,A.\ }}
\newcommand{\zB}{\mbox{z.\,B.\ }}
\newcommand{\bzw}{\mbox{bzw.\ }}
\newcommand{\gDh}{\mbox{d.\,h.\ }}
\newcommand{\gDQ}[1]{\glqq #1\grqq}
\newcommand{\oBdA}{\mbox{o.\,B.\,d.\,A.\ }}
\newcommand{\sEUR}{\text{\euro}}

%latin abbreviations
\newcommand{\etal}{\mbox{\emph{et al.\ }}}
\newcommand{\exgrat}{\mbox{e.g.\ }}
\newcommand{\idest}{\mbox{i.e.\ }}

%general math terms
\newcommand{\const}{\mathrm{const}}
\newcommand{\bigO}{\mathcal{O}}

% Lorem ipsum
\newcommand*{\QEDA}{\hfill\ensuremath{\blacksquare}}%
\newcommand*{\QEDB}{\hfill\ensuremath{\square}}%

%  ------------------ abbreviations

%matrix operations
\newcommand{\T}{T}
\DeclareMathOperator{\arcsinh}{arcsinh}
\DeclareMathOperator{\Tr}{Tr}
\DeclareMathOperator{\argg}{arg}
\DeclareMathOperator{\Arg}{arg}
\DeclareMathOperator{\codim}{codim}
\DeclareMathOperator{\atanTwo}{atan2}
\DeclareMathOperator{\diag}{diag}

%real and complex numbers latin Letters
\newcommand{\Real}{\mathbb{R}}
\newcommand{\Complex}{\mathbb{C}}
\newcommand{\Integer}{\mathbb{N}}

% differentials 
\newcommand{\dd}{\mathrm{d}}



\pagestyle{fancy}
\fancyhf{}
\cfoot{\thepage}

\begin{document}
\noindent
\vspace{9cm}
\begin{center}
    {\Huge \textbf{Lösung}} \\[1.5em]
    {\Large \textbf{Physikalische Grundlagen}} \\[1em]
    {\large Freiwillige Selbsteinschätzung} \\[1em]
    {\large Studiengang: Clinical Engineering} 
\end{center}

\newpage

\section*{Lösung Beispiel 1 – Eiswürfel in Limonade}

\begin{enumerate}
    \item \textbf{Reicht die Wärme der Limonade zum Schmelzen?} \\
    Zuerst berechnen wir die Masse der Eiswürfel:
    $$ m_{\text{Eis}} = 2 \cdot \SI{25}{\gram} = \SI{50}{\gram} = \SI{0,05}{\kilogram} \mDot $$
    Die zum Schmelzen des Eises benötigte Energie (Schmelzwärme) ist:
    $$ Q_{\text{Schmelz}} = m_{\text{Eis}} \cdot \lambda_{\text{Schmelz}} = \SI{0,05}{\kg} \cdot \SI{332,8}{\kilo\joule\per\kilogram} = \SI{16,64}{\kilo\joule} \mDot $$
    Die maximal verfügbare Wärme der Limonade beim Abkühlen von \SI{33}{\degreeCelsius} auf \SI{0}{\degreeCelsius} ist:
    $$ Q_{\text{Limo, max}} = m_{\text{Limo}} \cdot c_{\text{Limo}} \cdot \Delta T = \SI{0,24}{\kg} \cdot \SI{4,18}{\kilo\joule\per\kilogram\per\kelvin} \cdot (\SI{33}{\kelvin}) = \SI{33,1}{\kilo\joule} \mDot $$
    Da $Q_{\text{Limo, max}} > Q_{\text{Schmelz}}$ (\SI{33,1}{\kilo\joule} > \SI{16,64}{\kilo\joule}), schmelzen die Eiswürfel vollständig.

    \item \textbf{Berechnung der Endtemperatur $T_{\text{Ende}}$} \\
    Wir setzen die von der Limonade abgegebene Wärme gleich der vom Eis aufgenommenen Wärme (Schmelzen + Erwärmen).
    $$ Q_{\text{Limo, abgegeben}} = Q_{\text{Eis, aufgenommen}} $$
    $$ m_{\text{Limo}} c_{\text{Limo}} \underbrace{(T_{\text{Limo,Anfang}} - T_{\text{Ende}})}_{>0} = m_{\text{Eis}} \lambda_{\text{Schmelz}} + m_{\text{Eis}} c_{\text{Wasser}} \underbrace{(T_{\text{Ende}} - T_{\text{Eis,Anfang}})}_{>0} $$
    Wir setzen die Werte ein (Temperaturen in Celsius, da es sich um Differenzen bzw. eine Endtemperatur handelt):
    $$ \SI{0,24}{\kg} \cdot \SI{4180}{\joule\per\kg\per\kelvin} \cdot (\SI{33}{\celsius} - T_{\text{Ende}}) = \SI{16640}{\joule} + \SI{0,05}{\kg} \cdot \SI{4180}{\joule\per\kg\per\kelvin} \cdot (T_{\text{Ende}} - \SI{0}{\celsius}) $$
    $$ \num{1003,2} \cdot (\num{33} - T_{\text{Ende}}) = \num{16640} + \num{209} \cdot T_{\text{Ende}} $$
    $$ \num{33105,6} - \num{1003,2} \cdot T_{\text{Ende}} = \num{16640} + \num{209} \cdot T_{\text{Ende}} $$
    $$ \num{16465,6} = \num{1212,2} \cdot T_{\text{Ende}} $$
    $$ T_{\text{Ende}} = \frac{\num{16465,6}}{\num{1212,2}} \approx \SI{13,58}{\celsius} $$
    Die Endtemperatur der Mischung beträgt ca. \SI{13,6}{\celsius}.
\end{enumerate}

\newpage
\section*{Lösung Beispiel 2 - Reibungsfreies Schieben}

\begin{enumerate}
    \item \textbf{Kräfteskizze für Masse m} \\
    \begin{center}
        \includegraphics[width=0.4\textwidth]{Bilder/Test/FreiwilligerSelbsttest/schiefe_ebene_loesung.png}
    \end{center}
    Auf die Masse $m$ wirken die Gewichtskraft $F_G = mg$ (senkrecht nach unten) und die Normalkraft $F_N$ (senkrecht zur Auflagefläche). Das System wird mit $a_x$ nach rechts beschleunigt.
    
    \item \textbf{Kräftegleichgewicht in $x$- und $y$-Richtung} \\
    Wir zerlegen die Normalkraft $F_N$ in ihre $x$- und $y$-Komponenten.
    $$ \sum F_y = F_N \cos(\theta) - mg \eqexcl 0 \quad \Rightarrow \quad (1) \ F_N \cos(\theta) = mg $$
    In $x$-Richtung bewirkt die resultierende Kraft die Beschleunigung $a_x$:
    $$ \sum F_x = F_N \sin(\theta) + 0  \eqexcl m a_x \quad \Rightarrow \quad (2) \ F_N \sin(\theta) = m a_x $$

    \item \textbf{Herleitung der Beschleunigung $a_x$} \\
    Aus Gleichung (1) erhalten wir für die Normalkraft:
    $$ F_N = \frac{mg}{\cos(\theta)} $$
    Dies setzen wir in Gleichung (2) ein:
    \begin{gather*}
        \left(\frac{mg}{\cos(\theta)}\right) \sin(\theta) = m a_x \\
        \implies mg \tan(\theta) = m a_x \\
         a_x = g \tan(\theta)
    \end{gather*}
    
    \item \textbf{Berechnung der Gesamtkraft $F$} \\
    Die Kraft $F$ muss die Gesamtmasse $M+m$ mit der Beschleunigung $a_x$ bewegen:
    $$ F = (M+m) a_x = (M+m) g \tan(\theta) \mDot$$
\end{enumerate}

\newpage
\section*{Lösung Beispiel 3 – Warme Autoreifen}

\begin{enumerate}
    \item \textbf{Druck bei \SI{10}{\degreeCelsius}} \\
    Für ein ideales Gas bei konstantem Volumen können wir die beiden Volumina gleichsetzen:
    \begin{align*}
        \frac{n \cdot R \cdot T_1}{p_1} &= \frac{n \cdot R \cdot T_2}{p_2} \\
        \implies \frac{p_1}{T_1} &= \frac{p_2}{T_2}
    \end{align*}
    Die Temperaturen müssen in Kelvin umgerechnet werden:
    $T_1 = \SI{40}{\degreeCelsius} = \SI{313,15}{\kelvin}$ \\
    $T_2 = \SI{10}{\degreeCelsius} = \SI{283,15}{\kelvin}$ \\
    Nun lösen wir nach $p_2$ auf:
    $$ p_2 = p_1 \cdot \frac{T_2}{T_1} = \SI{2,2}{bar} \cdot \frac{\SI{283,15}{\kelvin}}{\SI{313,15}{\kelvin}} \approx \SI{1,99}{bar} $$
    Der Druck im Reifen beträgt bei \SI{10}{\degreeCelsius} ca. \SI{1,99}{bar}.

    \item \textbf{Anzahl der Mol im Reifen} \\
    Wir verwenden die ideale Gasgleichung $pV = nRT$ und lösen nach $n$ auf. Die Einheiten müssen SI-konform sein: \\
    $p = \SI{2,2}{bar} = \SI{220000}{\pascal}$ \\
    $V = \SI{40}{\litre} = \SI{0,04}{\metre\cubed}$ \\
    $T = \SI{40}{\celsius} = \SI{313,15}{\kelvin}$
    $$ n = \frac{pV}{RT} = \frac{\SI{220000}{\pascal} \cdot \SI{0,04}{\metre\cubed}}{\SI{8,314}{\joule\per(\mole\cdot\kelvin)} \cdot \SI{313,15}{\kelvin}} \approx \SI{3,38}{mol} $$
    
    \item \textbf{Phasendiagramm} \\
    Der typische Betriebsbereich eines Autoreifens (Drücke um \SIrange{2}{3}{bar} und Temperaturen von \zB \SIrange{-40}{50}{\celsius}) liegt im Phasendiagramm von Luft weit im gasförmigen Bereich. Eine Verflüssigung oder Erstarrung ist unter normalen Bedingungen ausgeschlossen.
    \begin{center}
        \includegraphics[width=0.6\textwidth]{Bilder/Test/FreiwilligerSelbsttest/phasendiagramm_luft_loesung.png}
    \end{center}
\end{enumerate}

\newpage
\section*{Lösung Beispiel 4 - Thermische Ausdehnung}

\begin{enumerate}
    \item \textbf{Längenänderungen $\Delta l$ und $\Delta b$} \\
    Gemäß der Formel für die lineare Ausdehnung gilt:
    $$ \Delta l = \alpha l \Delta T \quad \text{und} \quad \Delta b = \alpha b \Delta T $$

    \item \textbf{Formel für die Flächenänderung $\Delta A$} \\
    Die neue Fläche $A'$ ist das Produkt der neuen Seitenlängen $l' = l + \Delta l$ und $b' = b + \Delta b$. Die Flächenänderung $\Delta A = A' - A$:
    $$ \Delta A = \underbrace{(l + \Delta l)(b + \Delta b)}_{A'} - \underbrace{l\cdot b}_A = l \Delta b + b \Delta l + \Delta l \Delta b $$
    
    \item \textbf{Vereinfachung} \\
    Wir ersetzen $\Delta l$ und $\Delta b$ mit den Ausdrücken aus (1):
    $$ \Delta A = l (\alpha b \Delta T) + b (\alpha l \Delta T) + (\alpha l \Delta T)(\alpha b \Delta T) $$
    $$ \implies \Delta A = 2\alpha (l b) \Delta T + (l b) (\alpha \Delta T)^2 $$
    Da $A = l b$ und der Term $(\alpha \Delta T)^2$ laut Angabe vernachlässigbar ist, ergibt sich:
    $$ \Delta A \approx 2\alpha A \Delta T \mDot$$
    
    \item \textbf{Numerisches Beispiel} \\
    Eisenplatte mit $\alpha = \SI{12e-6}{\kelvin^{-1}}$ mit $l = \SI{2,3}{\meter}$ und $b = \SI{1,2}{\meter}$ bei einer Erwärmung um $\Delta T = \SI{40}{\celsius}$. \\
    Zuerst berechnen wir die Anfangsfläche $A$:
    $$ A = l \cdot b = \SI{2,3}{\m} \cdot \SI{1,2}{\m} = \SI{2,76}{\metre\squared} $$
    Nun setzen wir die Werte in die hergeleitete Formel ein:
    $$ \Delta A = 2 \cdot (\SI{12e-6}{\kelvin^{-1}}) \cdot \SI{2,76}{\metre\squared} \cdot \SI{40}{\kelvin} \approx \SI{0,00265}{\metre\squared} = \SI{26,5}{\centi\metre\squared} $$
\end{enumerate}

\newpage
\section*{Lösung Beispiel 5 - Looping}

\begin{enumerate}
    \item \textbf{Geschwindigkeit am höchsten Punkt} \\
    Am höchsten Punkt des Loopings muss die Zentripetalkraft mindestens so groß sein wie die Gewichtskraft. Im Grenzfall ($F_N=0$) liefert die Gewichtskraft allein die gesamte Zentripetalkraft:
    \begin{align*}
        F_G &= F_{\text{Zp}} \\
        \implies mg &= \frac{m v_{\text{oben}}^2}{r}
    \end{align*}
    Daraus folgt für die Geschwindigkeit am höchsten Punkt ($r = d/2 = \SI{15}{\m}$):
    $$ v_{\text{oben}} = \sqrt{gr} = \sqrt{\SI{9,81}{\m\per\s\squared} \cdot \SI{15}{\m}} \approx \SI{12,13}{\m\per\s} $$

    \item \textbf{Anfangsgeschwindigkeit mittels Energieerhaltung} \\
    Die Gesamtenergie am Start ($h=0$) muss gleich der Gesamtenergie am höchsten Punkt ($h=d=2r$) sein.
    $$ E_{\text{unten}} = E_{\text{oben}} \implies \frac{1}{2}m v_{0,\text{min}}^2 = \frac{1}{2}m v_{\text{oben}}^2 + mg(2r) $$
    Wir setzen $v_{\text{oben}}^2 = gr$ aus (1) ein:
    $$ \frac{1}{2}m v_{0,\text{min}}^2 = \frac{1}{2}m(gr) + 2mgr = \frac{5}{2}mgr $$
    Auflösen nach $v_{0,\text{min}}$:
    $$ v_{0,\text{min}}^2 = 5gr \implies v_{0,\text{min}} = \sqrt{5gr} $$

    \item \textbf{Numerische Berechnung} \\
    Wir setzen die gegebenen Werte ein:
    $$ v_{0,\text{min}} = \sqrt{5 \cdot \SI{9,81}{\m\per\s\squared} \cdot \SI{15}{\m}} = \sqrt{\SI{735,75}{\m\squared\per\s\squared}} \approx \SI{27,12}{\m\per\s} $$
    Umrechnung in \si{\kilo\metre\per\hour}:
    $$ v_{0,\text{min}} \approx \SI{27,12}{\m\per\s} \cdot 3,6 \approx \SI{97,65}{\kilo\metre\per\hour} $$
\end{enumerate}

\end{document}