\chapter{Rotation starrer Körper}\label{chap: rotation_starrer_koerper}
Nachdem wir uns bisher mit der Bewegung von Massenpunkten beschäftigt haben (Translation), erweitern wir unser Modell nun auf ausgedehnte Körper, die sich nicht verformen: die sogenannten \textbf{starren Körper}. Neben der Translation können diese Körper auch eine Rotation um eine oder mehrere Achsen ausführen.

\section{Modell eines starren Körpers}\label{sec: model_starrer_koerper}
Ein starrer Körper ist ein physikalisches Modell, bei dem die Abstände zwischen allen Punkten des Körpers konstant bleiben, unabhängig von den auf ihn wirkenden Kräften. Das bedeutet, dass der Körper sich nicht verformt und seine Form unter Belastung beibehält. Dieses Modell ist eine Vereinfachung der realen Welt, da alle materiellen Objekte in gewissem Maße verformbar sind. Dennoch ist das Modell des starren Körpers in vielen Bereichen der Physik und Technik äußerst nützlich, da es die Analyse von Bewegungen und Kräften erheblich vereinfacht -- insbesondere Rotationen werden dadurch deutlich einfacher zu beschreiben.

\subsection{Volumenintegral}
Um die Bewegung eines starren Körpers zu analysieren, betrachten wir ihn als eine Ansammlung von vielen kleinen Massenpunkten $i = 1,2,\ldots,N$, die alle starr miteinander verbunden sind. Diese Massenpunkte haben die Volumina $\Delta V_i$ und die Massen $\Delta m_i$. Das Volumen des Körpers $V$ und seine Gesamtmasse $M$ ergibt sich aus der Summe über alle Elemente 
\begin{equation}\label{eq: volumen_masse_endlicheTeile_starrer_koerper}
    V = \sum_{i=1}^{N} \Delta V_i \mComma \quad M = \sum_{i=1}^{N} \Delta m_i \mDot
\end{equation}
Das Verhältnis der Masse eines Volumenelements zu seinem Volumen ist die Dichte $\rho$,
\begin{equation}\label{eq: dichte_starrer_koerper}
    \rho_i = \frac{\Delta m_i}{\Delta V_i} \mComma
\end{equation}
die die Einheit $[\rho] = \si{\kilogram\per\meter^{3}}$ hat. Die Dichte kann eine Funktion des Ortes im Körper sein, also $\rho = \rho(\ivec{r})$. Oftmals nimmt man aber an, dass die Dichte homogen ist, also überall im Körper den gleichen Wert hat ($\rho = \const$). 

Mit Hilfe von \cref{eq: dichte_starrer_koerper} können wir die Gesamtmasse des starren Körpers auch als 
\begin{equation}
    M = \sum_{i=1}^{N} \rho_i \cdot \Delta V_i 
\end{equation}
schreiben. Im Grenzfall $N \to \infty$ -- die Volumenelemente werden also infinitesimal klein -- geht die Summe in ein Integral über, und wir erhalten die Gesamtmasse des starren Körpers als Integral über alle Volumenelemente
\begin{align}
    V &= \lim_{N \to \infty} \sum_{i=1}^{N} \Delta V_i = \int_V \dd V \mComma \label{eq: gesamtvolumen_starre_koerper_allg} \\
    M &= \lim_{N \to \infty} \sum_{i=1}^{N} \Delta m_i = \int_V \rho(\ivec{r}) \,\dd V \mDot \label{eq: gesamtmasse_starrer_koerp_allg}
\end{align}
Das Integral in \cref{eq: gesamtvolumen_starre_koerper_allg} ist ein sogenanntes \textit{Volumenintegral} und ist eine Abkürzung für ein dreifaches Integral über die räumlichen Koordinaten, beispielsweise $x$, $y$ und $z$ in kartesischen Koordinaten: 
\begin{equation}\label{eq: volumenintegral_3fach}
    V = \int_V \dd V = \int_{z_{\min}}^{z_{\max}} \left[ \int_{y_{\min}}^{y_{\max}} \left( \int_{x_{\min}}^{x_{\max}} \dd x \, \right) \dd y \, \right] \dd z \mDot
\end{equation}
Die Berechnung solcher Integrale wird in \cref{sec: Mehrdimensionale_Integration} ausführlich erklärt.

Bei homogenen Körpern mit konstanter Dichte $\rho$ vereinfacht sich \cref{eq: gesamtmasse_starrer_koerp_allg} zu
\begin{equation}\label{eq: gesamtmasse_starrer_koerp_homogen}
    M = \int_V \rho \, \dd V = \rho \int_V \dd V = \rho \cdot V \mComma
\end{equation}
was der gewohnten Formel für die Masse eines homogenen Körpers entspricht.

\subsection{Massenschwerpunkt}\label{subsec: massenschwerpunkt}
Der \textbf{Massenschwerpunkt} (auch Schwerpunkt oder Massenmittelpunkt genannt) eines starren Körpers ist der gewichtete Mittelpunkt aller Massenpunkte des Körpers. Die Gewichtung erfolgt über den Abstand der Massenpunkte. Der Schwerpunkt eines Körpers muss nicht unbedingt innerhalb des physischen Körpers liegen, \zB bei einem Hohlkörper oder einem zweiatomigen Molekül.

Der Ortsvektor des Schwerpunkts $\ivec{r}_S$ berechnet sich über 
\begin{equation}
    \ivecS{r}{S} = \frac{\sum_{i=1}^{N} \ivecS{r}{i} \, \Delta m_i}{\sum_{i=1}^{N} \Delta m_i} = \frac{1}{M} \sum_{i=1}^{N} \ivecS{r}{i} \, \Delta m_i = \frac{1}{M} \sum_{i=1}^{N} \ivecS{r}{i} \, \rho_i \, \Delta V_i \mDot
\end{equation}
\begin{importantbox}{Schwerpunkt eines starren Körpers}
    Beim Übergang zu infinitesimalen Volumenelementen ($N \to \infty$) geht die Summe in ein Integral über und wir erhalten die allgemeine Formel für den Ortsvektor des Schwerpunkts
    \begin{equation}\label{eq: schwerpunkt_starrer_koerp_allg}
        \ivecS{r}{S} = \frac{1}{M} \int_V \ivec{r} \, \rho(\ivec{r}) \, \dd V \mComma
    \end{equation}
    wobei sich die Formel für homogene Körper mit konstanter Dichte zu $\ivecS{r}{S} = \frac{1}{V} \int_V \ivec{r} \, \dd V$ vereinfacht. 
\end{importantbox}

\subsubsection{Überlagerung von Translation und Rotation}\label{subsubsec: ueberlagerung_schwerpunktsbew_rotation}
Die Bedeutung des Schwerpunkts wird ersichtlich, wenn man die Bewegung von einzelnen Volumenelementen $\dd V_i$ innerhalb eines starren Körpers betrachet. Der Ortsvektor $\ivecS{r}{i}$ eines Volumenelements lässt sich schreiben als 
\begin{equation}\label{eq: ortsvektor_ri_schwerpunkt}
    \ivecS{r}{i} = \ivecS{r}{S} + \ivecS{r}{i,S} \mDot
\end{equation}
Differenzieren wir \cref{eq: ortsvektor_ri_schwerpunkt} nach der Zeit, erhalten wir für die Geschwindigkeit des Volumenelements
\begin{equation}\label{eq: geschwindigkeit_ri_schwerpunkt}
    \ivecS{v}{i} = \ivecS{v}{S} + \frac{\dd}{\dd t}\ivecS{r}{i,S} \mDot
\end{equation}
Gleichzeitig wissen wir aber, dass der Körper starr ist, \gDh die Abstände zwischen den Volumenelementen bleiben konstant: $| \ivecS{r}{i,S} | = \const$. Diese Beziehung ist äquivalent zu $\ivecS{r}{i,S} \cdot \ivecS{r}{i,S} = \const$. Differenzieren wir diese Gleichung zweimal nach der Zeit, erhalten wir
\begin{equation}
    \frac{\dd}{\dd t} \left( \ivecS{r}{i,S} \cdot \ivecS{r}{i,S} \right) = 2 \, \ivecS{r}{i,S} \cdot \ivecS{v}{i,S} = 0 \mComma
\end{equation}
woraus folgt, dass $\ivecS{r}{i,S} \perp \ivecS{v}{i,S}$ ist. Somit beschreibt das Volumenelement eine Kreisbewegung um eine Achse durch den Schwerpunkt $S$ des Körpers. Deshalb kann die Geschwindigkeit des Volumenelements auch als 
\begin{equation}\label{eq: geschwindigkeit_ri_schwerpunkt_rotation}
    \ivecS{v}{i,S} = \ivec{\omega} \times \ivecS{r}{i,S} \mDot
\end{equation}
geschrieben werden. Eingesetzt in \cref{eq: geschwindigkeit_ri_schwerpunkt} ergibt sich das folgende Resultat

\begin{rememberbox}{Überlagerung von Translation und Rotation}
    Die Geschwindigkeit $\ivecS{v}{i}$ eines Volumenelements $\dd V_i$ innerhalb eines starren Körpers relativ zum Koordinatenursprung setzt sich zusammen aus der Geschwindigkeit des Schwerpunkts $\ivecS{v}{S}$ und der Rotationsgeschwindigkeit um den Schwerpunkt:
    \begin{equation}\label{eq: ueberlagerung_vi_vs_plus_omega_riS}
        \ivecS{v}{i} = \ivecS{v}{S} + \ivec{\omega} \times \ivecS{r}{i,S} \mDot
    \end{equation}
    Da dies für alle Volumenelemente $\dd V_i$ gilt, kann die gesamte Bewegung eines starren Körpers als Überlagerung einer Translation des Schwerpunkts und einer Rotation um den Schwerpunkt beschrieben werden.
\end{rememberbox}


\begin{figure}[ht]
    \centering
    \begin{tikzpicture}
        
        % 1. Include the image as a node named 'myImage'
        \node[anchor=south west, inner sep=0] (myImage) at (0,0) {
            \includegraphics[width=0.44\linewidth]{Bilder/Kapitel_Rotation/schwerpunkt_rs_ris_ri_Kugel.png}
        };
        
        % 2. Create a scope relative to the image size
        % (0,0) is bottom-left, (1,1) is top-right
        \begin{scope}[x={(myImage.south east)}, y={(myImage.north west)}]
            % --- LABELS ---
            % Adjust the numbers (x, y) to move the text
            \node at (0.09, 0.03) {\Large $x$};       % Bottom right area
            \node at (0.95, 0.24) {\Large $y$};       % Far right area
            \node at (0.24, 0.92) {\Large $z$};       % Top center
            \node at (0.22, 0.25) {\Large $0$};       % Origin
            
            \node at (0.7, 0.94) {\large $\ivec{\omega}$};  % Near top right
            \node at (0.60, 1.0) {\large $\ivec{A}$};  % Near top right
            \node at (0.4, 0.5) {\large $\ivecS{r}{S}$}; % Schwerpunkt
            \node at (0.53, 0.41) {\large $\ivecS{r}{i}$}; % Massenpunkt i
            \node at (0.67, 0.64) {\large $\ivecS{r}{i,S}$}; % Vektor riS 

            \node at (0.85, 0.66) {\large $\dd V_i$}; % Volumenelemnt dV_i label
            \node at (0.56, 0.61) {\large $S$}; % Schwerpunkt S
        \end{scope}
       
    \end{tikzpicture}
    \caption{Der Ortsvektor $\protect\ivecS{r}{i}$ eines Massenpunkts $i$ innerhalb eines starren Körpers lässt sich in den Ortsvektor des Schwerpunkts $\protect\ivecS{r}{S}$ und den Ortsvektor relativ zum Schwerpunkt $\protect\ivecS{r}{i,S}$ zerlegen.}\label{fig: aufteilung_bewegung_schwerpunkt}
\end{figure}


\section{Rotationskinematik}\label{sec: rotationskinematik}
Analog zur Translation, die durch Ort, Geschwindigkeit und Beschleunigung beschrieben wird, nutzen wir für die Rotation den Winkel, die Winkelgeschwindigkeit und die Winkelbeschleunigung.
\begin{figure}[ht]
    \centering
    \begin{tikzpicture}
        
        % 1. Include the image as a node named 'myImage'
        \node[anchor=south west, inner sep=0] (myImage) at (0,0) {
            \includegraphics[width=0.6\linewidth]{Bilder/Kapitel_Rotation/RotatingCylinder.pdf}
        };
        
        % 2. Create a scope relative to the image size
        % (0,0) is bottom-left, (1,1) is top-right
        \begin{scope}[x={(myImage.south east)}, y={(myImage.north west)}]
            % --- LABELS ---
            % Adjust the numbers (x, y) to move the text
            \node at (0.83, 0.24) {\Large $x$};       % Bottom right area
            \node at (0.77, 0.69) {\Large $y$};       % Far right area
            \node at (0.50, 1.05) {\Large $z$};       % Top center
            
            \node[text=red] at (0.58, 0.7) {\large $\protect\ivec{\omega}$};  % Near top right
            \node at (0.76, 0.51) {\large $\varphi$}; % Near top left
            \draw (0.587, 0.565) -- (0.72, 0.51); 
        \end{scope}
    \end{tikzpicture}
    \caption{Darstellung eines rotierenden Körpers. Der Vektor der Winkelgeschwindigkeit $\protect\ivec{\omega}$ gibt die Drehachse (Richtung) und den Betrag der Winkelgeschwindigkeit $|\protect\ivec{\omega}|$ an.}\label{fig: rotation_definition}
\end{figure}



Die \textbf{Winkelgeschwindigkeit} $\ivec{\omega}$ ist ein Vektor, dessen Richtung die Drehachse angibt (Rechte-Hand-Regel) und dessen Betrag die Schnelligkeit der Drehung beschreibt.

\begin{rememberbox}{Winkelgeschwindigkeit}
    Die momentane Winkelgeschwindigkeit $\omega$ ist die zeitliche Ableitung des überstrichenen Winkels $\varphi$:
    \begin{equation}\label{eq: winkelgeschwindigkeit}
        \omega = \frac{\mathrm{d}\varphi}{\mathrm{d}t} \mDot
    \end{equation}
    Die Einheit ist \si{\radian\per\second} (\bzw \si{\per\second}).
\end{rememberbox}

Ändert sich die Winkelgeschwindigkeit zeitlich, sprechen wir von einer \textbf{Winkelbeschleunigung} $\alpha$:
\begin{equation}\label{eq: winkelbeschleunigung}
    \alpha = \frac{\mathrm{d}\omega}{\mathrm{d}t} = \frac{\mathrm{d}^2\varphi}{\mathrm{d}t^2} \mDot
\end{equation}

\section{Drehmoment und Trägheitsmoment}\label{sec: drehmoment_traegheit}

So wie eine Kraft notwendig ist, um den Bewegungszustand (Translation) eines Körpers zu ändern, ist ein \textbf{Drehmoment} notwendig, um den Rotationszustand eines Körpers zu verändern -- die Rotation wird dann beschleunigt oder gebremst.

\begin{importantbox}{Drehmoment}
    Das Drehmoment $\ivec{M}$ resultiert aus einer Kraft $\ivec{F}$, die an einem Punkt mit dem Ortsvektor $\ivec{r}$ (vom Drehzentrum aus) angreift. Es ist definiert als das Kreuzprodukt:
    \begin{equation}\label{eq: drehmoment_def_M_r_cross_F}
        \ivec{M} = \ivec{r} \times \ivec{F} \mDot
    \end{equation}
    Der Betrag ist maximal, wenn die Kraft senkrecht zum Hebelarm steht.
\end{importantbox}
In \cref{fig: drehmoment_definition} ist schematisch eine Situation dargestellt, in der eine Kraft $\ivec{F}$ im Punkt $P$ mit dem Ortsvektor $\ivec{r}$ an einem Körper angreift, der um eine \textbf{starre Achse} $\ivec{A}$ gelagert ist. Die Kraft $\ivec{F}$ kann in drei Anteile zerlegt werden: den normalen Anteil $\ivecS{F}{n} \parallel \ivec{r}$, den tangentialen Anteil $\ivecS{F}{t} \perp \ivec{A} \, \wedge \, \ivecS{F}{t} \perp \ivec{r}$ und den axialen Anteil $\ivec{F}_z \parallel \ivec{A}$. Nur die Anteile $\ivecS{F}{z}$ und $\ivecS{F}{t}$ tragen zum Drehmoment bei, da $\ivecS{F}{n} \parallel \ivec{r}$ ist und somit $\ivec{r} \times \ivecS{F}{n} = 0$. 
Der axiale Anteil $\ivecS{F}{z}$ bewirkt eine Richtungsänderung der Drehachse, die bei einer fest gelagerten Drehachse von den Achsenlagern aufgefangen wird. Übrig bleibt lediglich der tangentiale Anteil $\ivecS{F}{t}$, der das Drehmoment erzeugt:
\begin{equation}\label{eq: drehmoment_betrag}
    M = r \cdot F_t = r \cdot F \cdot \sin(\alpha) \mComma
\end{equation}
wobei $\alpha$ der Winkel zwischen $\ivec{r}$ und $\ivec{F}$ ist. Bei raumfester Rotationsachse ist das Drehmoment also immer parallel zur Drehachse, \gDh $\ivec{M} \parallel \ivec{A}$.

\begin{figure}[htb]
    \centering
    % Platzhalter für ein Bild: Hebelarm mit Kraft und Drehmoment
    \includegraphics[width=0.5\linewidth]{Bilder/Kapitel_Rotation/Drehmoment_Fn_Ft_Fz.png} 
    \caption{Die Kraft $\protect\ivec{F}$, die das Drehmoment $\protect\ivec{M}$ erzeugt, kann in drei Anteile $\protect\ivecS{F}{n}$, $\protect\ivecS{F}{t}$ und $\protect\ivecS{F}{z}$ zerlegt werden. Nur die Anteile $\protect\ivecS{F}{t}$ und $\protect\ivecS{F}{z}$ tragen zum Drehmoment bei, wobei $\protect\ivecS{F}{z}$ bei starrer Drehachse von den Lagern aufgenommen wird. (Quelle:~\cite[S.~133]{Demtroeder2018})}\label{fig: drehmoment_definition}
\end{figure}

\subsection{Das Trägheitsmoment}
Ein Körper setzt einer Änderung seines Bewegungszustandes einen Widerstand entgegen. Bei der Translation ist der Proportionalitätsfaktor die Masse $m$, \gDh für eine Beschleunigung $\ivec{a}$ benötigt man eine Kraft $\ivec{F} = m \ivec{a}$. \\
Bei der Rotation hängt dieser Widerstand nicht nur von der Masse ab, sondern auch davon, wie die Masse bezüglich der Drehachse verteilt ist. Diese Größe nennen wir \textbf{Trägheitsmoment} $J$. Für eine Winkelbeschleunigung $\ivec{\alpha}$ ist das erforderliche Drehmoment $\ivec{M} = J \cdot \ivec{\alpha}$ (siehe \cref{eq: grundgleichung_rotation}), was analog zur Translation ist, wenn man die Analogien aus \cref{tab: analogie_translation_rotation} betrachtet.

Für ein System aus diskreten Massenpunkten gilt:
\begin{equation}\label{eq: traegheitsmoment_diskret}
    J = \sum_{i} m_i r_{\perp, i}^2 \mComma
\end{equation}
wobei $r_{\perp, i}$ der senkrechte Abstand der Masse $m_i$ zur Drehachse ist. Für kontinuierliche Körper geht die Summe in ein Integral über und wir erhalten die allgemeine Form des Trägheitsmoments. 

\begin{importantbox}{Trägheitsmoment}
    Das Trägheitsmoment $J$ eines starren Körpers bezüglich einer Drehachse ist definiert als
    \begin{equation}\label{eq: traegheitsmoment_integral}
    J = \int r_{\perp}^2 \,\dd m = \int r_{\perp}^2 \,\rho(\ivec{r}) \,\dd V \mComma
    \end{equation}
    wobei $r_{\perp, i}$ der senkrechte Abstand des Volumenelements $\dd V$ zur Drehachse ist. Das Trägheitsmoment hat die Einheit $[J] = \si{\kilogram\, \meter^{2}}$.
    
    Der Integrand $r_{\perp}^2 \rho(\ivec{r})$ geht für homogene Körper in $r_{\perp}^2 \rho$ über und somit wird das Integral zu 
    \begin{equation}\label{eq: traegheitsmoment_integral_homogeneDichte}
    J = \rho \int r_{\perp}^2 \,\dd V \mComma
    \end{equation}
    wobei $\rho$ die Dichte des Körpers (Materials) ist. 
\end{importantbox}

Da in \cref{eq: traegheitsmoment_integral,eq: traegheitsmoment_integral_homogeneDichte} der Normalabstand $r_{\perp}$ quadratisch eingeht, hängt das Trägheitsmoment stark davon ab, wie die Massenelemente relativ zur Drehachse verteilt sind.

Das Trägheitsmoment ist immer bezüglich einer bestimmten Drehachse definiert. Für verschiedene Achsen durch denselben Körper kann das Trägheitsmoment sehr unterschiedlich sein und muss im Allgemeinen für jede Achse neu berechnet werden.

\subsubsection{Trägheitsmomente einiger Standardkörper}
In \cref{fig: traegheitsmomente_standardkoerper} sind die Trägheitsmomente einiger Standardkörper für verschiedene Drehachsen zusammengefasst. Die Herleitungen dieser Formeln erfolgen durch die Berechnung des Integrals in \cref{eq: traegheitsmoment_integral} für die jeweiligen Geometrien. Die Herleitungen sind für einige Körper in \cref{sec: App_Herleitungen_Traegheitsmomente} zu finden.
\begin{figure}
    \centering
    \includegraphics[width=0.75\linewidth]{Bilder/Kapitel_Rotation/Traegheitsmomente_verschKoerper.png}
    \caption{Standardkörper mit ihren Trägheitsmomenten bezüglich verschiedener Drehachsen. (Quelle:~\cite[S.~137]{Demtroeder2018})}\label{fig: traegheitsmomente_standardkoerper}
\end{figure}

\subsubsection{Satz von Steiner}\label{subsubsec: steiner_satz}
Der Satz von Steiner (auch Parallelachsen-Theorem genannt) erlaubt es, das Trägheitsmoment eines Körpers bezüglich einer beliebigen Achse zu berechnen, wenn das Trägheitsmoment bezüglich einer parallelen Achse durch den Schwerpunkt bekannt ist.
\begin{importantbox}{Satz von Steiner}
    Das Trägheitsmoment $J_A$ eines Körpers bezüglich einer Achse $A$, die einen Abstand $d$ vom Schwerpunkt hat, ist gegeben durch
    \begin{equation}\label{eq: steiner_satz}
        J_A = J_{S} + M \cdot d^2 \mComma
    \end{equation}
    wobei $J_{S}$ das Trägheitsmoment bezüglich der parallelen Achse durch den Schwerpunkt und $M$ die Gesamtmasse des Körpers ist.
\end{importantbox}



\subsection{Die Rotationsenergie}\label{subsec: rotationsenergie_herleitung}
Wir betrachten noch einmal den starren Körper in \cref{fig: drehmoment_definition}, der sich um eine feste Achse $\ivec{A}$ mit der Winkelgeschwindigkeit $\ivec{\omega}$ dreht und möchten dafür die Rotationsenergie berechnen. 

Ein Volumenelement $\Delta V_i$ im senkrechten Abstand $r_{\perp,i}$ zur Drehachse hat die Masse $\Delta m_i = \rho_i \cdot \Delta V_i$. Dieses Volumenelement bewegt sich mit der Winkelgeschwindigkeit $\omega$ um die Achse, wodurch es eine Bahngeschwindigkeit $v_i = r_{\perp,i} \cdot \omega$ hat (siehe \cref{subsubsec: geschwindigkeit_ebeneKreisbewegung}). Die kinetische Energie dieses Volumenelements ist
\begin{equation}\label{eq: kin_energie_volumenelement}
    \Delta E_{\kin,i} = \frac{1}{2} \Delta m_i \cdot v_i^2 = \frac{1}{2} \left(\rho_i \cdot \Delta V_i \right) \cdot {\left(r_{\perp,i} \cdot \omega\right)}^2 = \frac{1}{2} \omega^2\, \rho_i \, r_{\perp,i}^2 \, \Delta V_i \mDot
\end{equation}
Die gesamte kinetische Energie des rotierenden Körpers ergibt sich durch Summation über alle Volumenelemente:
\begin{equation}\label{eq: kin_energie_starrer_koerp_sum}
    E_{\rot} = \sum_{i} \Delta E_{\kin,i} = \sum_{i} \frac{1}{2} \omega^2 \, \rho_i \, r_{\perp,i}^2 \, \Delta V_i = \frac{1}{2} \omega^2 \sum_{i} \rho_i \, r_{\perp,i}^2 \, \Delta V_i \mDot
\end{equation}
Im Grenzfall $N \to \infty$ (infinitesimale Volumenelemente) geht die Summe in ein Integral über:
\begin{equation}\label{eq: kin_energie_starrer_koerp_integral}
    E_{\rot} = \frac{1}{2} \omega^2 \int_V \rho \, r_{\perp}^2 \,\mathrm{d}V \mDot
\end{equation}
Das Integral in \cref{eq: kin_energie_starrer_koerp_integral} identifizieren wir als das Trägheitsmoment $J$ bezüglich der Drehachse $\ivec{A}$:
\begin{equation}\label{eq: traegheitsmoment_herleitung}
    J = \int_V \rho\, r_{\perp}^2 \,\mathrm{d}V \mDot
\end{equation}
Damit lässt sich die Rotationsenergie als
\begin{equation}\label{eq: rot_energie_traegheitsmoment}
    E_{\rot} = \frac{1}{2} \, J \, \omega^2
\end{equation}
schreiben. Diese Formel ist analog zur kinetischen Energie eines Massenpunkts ($E_{\kin} = \frac{1}{2} m v^2$), wenn man die Analogien aus \cref{tab: analogie_translation_rotation} betrachtet. 

\begin{rememberbox}{Rotationsenergie}
    Die Rotationsenergie eines starren Körpers ist die Summe der kinetischen Energien aller Volumenelemente, die sich mit der Winkelgeschwindigkeit $\omega$ um eine Achse drehen.
\end{rememberbox}



\section{Grundgleichung der Rotationsdynamik}\label{sec: rotationsdynamik}
Das zweite Newton'sche Gesetz ($\ivec{F} = m \cdot \ivec{a}$) hat ein direktes Äquivalent in der Rotation.

\begin{importantbox}{Grundgleichung der Rotation}
    Das wirkende Gesamtdrehoment ist proportional zur Winkelbeschleunigung. Der Proportionalitätsfaktor ist das Trägheitsmoment:
    \begin{equation}\label{eq: grundgleichung_rotation}
        \ivec{M} = J \cdot \ivec{\alpha} \mDot
    \end{equation}
    Für den Betrag des Drehmoments gilt dann 
    \begin{equation}\label{eq: grundgleichung_rotation_betrag}
        M = J \cdot \alpha = J \cdot \frac{\dd \omega}{\dd t} = J \cdot \frac{\dd^2 \varphi}{\dd t^2}\mDot
    \end{equation}
\end{importantbox}
In der Dynamik haben wir bisher Kräfte $\ivec{F}$ betrachtet, die auf Massenpunkte wirken und so deren Translation ($\ivec{a}, \ivec{v}, \ivec{r}$) beeinflussen. Wenn nun bei starren Körpern Drehmomente $\ivec{M}$ wirken, beeinflussen diese die Rotation ($\ivec{\alpha}, \ivec{\omega}, \varphi$) des Körpers entsprechend den \cref{eq: grundgleichung_rotation,eq: grundgleichung_rotation_betrag}.

\subsection{Rollen auf schiefer Ebene}
Ein klassisches Beispiel zur Veranschaulichung der Rotationsdynamik ist das Rollen eines starren Körpers (z.\,B. einer Kugel oder eines Zylinders) auf einer schiefen Ebene mit Steigungswinkel $\alpha$ wie dargestellt in \cref{fig: rollen_schiefe_ebene}.
\begin{figure}[ht]
    \centering
    \begin{tikzpicture}[
        scale=1.0,
        >=Latex,
        font=\small\sffamily
    ]

        % --- Definitionen ---
        \def\R{1.2} % Radius des Rades
        \def\alphaV{25} % Winkel der schiefen Ebene
        \def\lenEbene{8.2} % Länge der schiefen Ebene
        \def\fgLen{3.7} % Länge der Gewichtskraft
        \definecolor{darkGreen}{rgb}{0.0, 0.5, 0.0}

        % --- Schiefe Ebene zeichnen ---
        \fill[red!10] (0,0) -- (\lenEbene,0) -- (\lenEbene, {\lenEbene*tan(\alphaV)}) -- (0,0) -- cycle;
        \draw[thick] (0,0) -- (\lenEbene,0) -- (\lenEbene, {\lenEbene*tan(\alphaV)}) -- cycle;
        % --- Rad zeichnen ---
        \coordinate (Center) at (6, {6*tan(\alphaV) + \R*(1/cos(\alphaV))}); % Mittelpunkt des Rades
        \draw[line width=1.5pt] (Center) circle (\R); % Rad
        \fill[gray!30] (Center) circle (\R); % graue Füllung
        \fill[black] (Center) circle (2pt); % Schwerpunkt

        % --- Kräfte zeichnen ---
        % Gewichtskraft F_G
        \draw[-{latex}, black, line width=1.5pt] (Center) -- ++(0,-\fgLen) node[midway, pos=0.7, right=0pt] {\large $\ivecS{F}{G} = M \ivec{g}$};

        % Normalkraft F_N setzt am Ende von F_N an und zeigt senkrecht zur schiefen Ebene nach unten
        \draw[->, darkGreen, thick] (Center)++({\alphaV+180}:{sin(\alphaV)*\fgLen}) -- ++({\alphaV+90}:-{cos(\alphaV)*\fgLen}) node[midway, pos=0.6, left] {\large $\ivecS{F}{N}$};


        % Kraft im Schwerpunkt parallel zur schiefen Ebene nach unten
        \draw[->, orange!70!black, thick] (Center) -- ++({\alphaV+180}:{sin(\alphaV)*\fgLen}) node[midway, above] {\large $\ivecS{F}{\parallel}$};

        % Radius r zum Kontaktpunkt
        \draw[<-, blue!70!black, line width=1.2pt] (Center) -- ++({\alphaV-90}:\R) node[midway, right] {\large $\ivec{r}$};

        % Kontaktpunkt markieren
        \fill[black] (Center)++({\alphaV-90}:\R) circle (1.3pt) node[below right=-1pt] {\large $A$};

        % --- Winkel alpha in der schiefen Ebene beschriften ---
        \draw[-, line width=1.2pt] (2.0,0) arc(0:\alphaV:2.0) node[midway, left=3pt, pos=0.45] {\large $\alpha$};

        % --- Omega Pfeil für Drehung des Rades ---
        \draw[-{Stealth}, gray!70!black, line width=1.pt] (Center)++({\alphaV-30}:\R*0.5) arc({\alphaV-30}:{\alphaV+90}:\R*0.5);
    \end{tikzpicture}
    \caption{Rollen eines starren Körpers mit rundem Querschnitt und Radius $r$ auf einer schiefen Ebene. Die Gewichtskraft $\protect\ivecS{F}{G}$ greift im Schwerpunkt an und wird in eine Normalkraft $\protect\ivecS{F}{N}$ und eine parallele Komponente $\protect\ivecS{F}{\parallel}$ zerlegt. Die Kraftkomponente $\protect\ivecS{F}{\parallel}$ erzeugt um den Kontaktpunkt $A$ ein Drehmoment $\protect\ivec{M} = \protect\ivec{r} \times \protect\ivecS{F}{\parallel}$.}\label{fig: rollen_schiefe_ebene}
\end{figure}

Beim Rollen ohne Gleiten wirkt die Gewichtskraft $\ivecS{F}{G} = M \ivec{g}$ im Schwerpunkt des Körpers. Diese Kraft kann in eine Normalkraft $\ivecS{F}{N}$ senkrecht zur Ebene und eine parallele Komponente $\ivecS{F}{\parallel}$ zerlegt werden. Die parallele Komponente erzeugt um den Kontaktpunkt $A$ ein Drehmoment $\ivec{M} = \ivec{r} \times \ivecS{F}{\parallel}$, das zu einer Winkelbeschleunigung $\ivec{\alpha}$ des Körpers führt und damit wiederum zu einer Erhöhung des Betrags $\omega$ der Winkelgeschwindigkeit $\ivec{\omega}$.

\subsubsection{Aufteilung der kinetischen Energie beim Rollen ohne Gleiten}
Beim Rollen ohne Gleiten besitzt der Körper sowohl eine translatorische als auch eine rotatorische kinetische Energie. Die gesamte kinetische Energie ist die Summe der beiden Anteile, $E_{\kin} = E_{\trans} + E_{\rot}$. Legt der Körper eine Strecke $s$ zurück, so wird potentielle Energie in kinetische Energie umgewandelt und es stellt sich die Frage, wie sich diese kinetische Energie aufteilt.
Hat der Körper die Strecke $s$ entlang der schiefen Ebene zurückgelegt, so hat sich die Höhe um $\Delta h = s \sin(\alpha)$ verringert. Die dadurch freigesetzte potentielle Energie ist
\begin{equation}\label{eq: potentielle_Energie_schiefeEbene_rollen}
    \Delta E_{\pot} = M g \Delta h = M g s \sin(\alpha) \mDot
\end{equation}
Diese Energie wird aufgeteilt in die beiden kinetischen Energieanteile
\begin{equation}\label{eq: kinetische_Energie_aufteilung_rollen}
    E_{\kin} = E_{\trans} + E_{\rot} = \frac{1}{2} M v^2 + \frac{1}{2} J_S \,\omega^2 \mComma
\end{equation}
wobei $J_S$ das Trägheitsmoment bezüglich der Drehachse durch den Schwerpunkt ist -- hier hilft uns der \textit{Steiner'sche Satz}. Da der Körper ohne Gleiten rollt, besteht die Beziehung $v = r \cdot \omega$ zwischen der Bahngeschwindigkeit $v$ des Schwerpunkts und der Winkelgeschwindigkeit $\omega$. Damit lässt sich \cref{eq: kinetische_Energie_aufteilung_rollen} umschreiben zu
\begin{equation}\label{eq: kinetische_Energie_aufteilung_rollbew}
    E_{\kin} = \frac{1}{2} M v^2 + \frac{1}{2} J_S \left(\frac{v}{r}\right)^2 = \frac{1}{2} M v^2 \left( 1 + \frac{J_S}{M r^2} \right) \mDot
\end{equation}
Der dimensionslose Term $b = \frac{J_S}{M r^2}$ ($b \in [0,1]$) gibt an, welcher Anteil der kinetischen Energie in die Rotationsenergie fließt.\footnote{Für eine Punktmasse wäre $J_S = 0$ und somit $b=0$. Eine Punktmasse hat nur translatorische kinetische Energie.} Setzen wir nun die freigesetzte potentielle Energie aus \cref{eq: potentielle_Energie_schiefeEbene_rollen} gleich der kinetischen Energie aus \cref{eq: kinetische_Energie_aufteilung_rollbew}, so erhalten wir
\begin{equation}\begin{aligned}\label{eq: energieerhaltung_rollen_schiefe_ebene}
    M g s \sin(\alpha) &= \frac{1}{2} M v^2 \left( 1 + b \right) \\
    v^2 &= \frac{2 g s \sin(\alpha)}{1 + b} \\
    \implies v &= \sqrt{\frac{2 g s \sin(\alpha)}{1 + b}} \mDot
\end{aligned}\end{equation}
Die zugehörige Beschleunigung $a$ bekommt man am einfachsten über das Ableiten von $v^2$ nach der Zeit $t$, also 
\begin{equation}\begin{aligned}\label{eq: beschleunigung_rollen_schiefe_ebene_1}
    \frac{\dd}{\dd t} v^2 &= 2 v \cdot a = \frac{2 g \sin(\alpha)}{1 + b} \cdot \frac{\dd s}{\dd t} = \frac{2 g v \sin(\alpha)}{1 + b} \\
    \implies a &= \frac{\dd v}{\dd t} = \frac{g \sin(\alpha)}{1 + b} \mDot
\end{aligned}\end{equation}
Die Beschleunigung $a$ ist also um den Faktor $\beta = \frac{1}{1+b} < 1$ kleiner als die Beschleunigung eines gleitenden Körpers auf der schiefen Ebene ($a = g \sin(\alpha)$, siehe \cref{eq: beschleunigung_reibungsfreies_gleiten}). Je größer also der Anteil $b$ der Rotationsenergie ist, desto geringer ist die Beschleunigung des rollenden Körpers. Die Geschwindigkeit ist um einen Faktor $\sqrt{\beta} = \sqrt{\frac{1}{1+b}}$ kleiner als die Geschwindigkeit eines gleitenden Körpers nach der Strecke $s$.

\subsubsection{Trägheitszahl $b$ einiger Standardkörper}\label{subsubsec: traegheit_b_standardkoerper}
Die vorhergehenden Ergebnisse für $E_{\kin}$ [\cref{eq: kinetische_Energie_aufteilung_rollbew}], die Beschleunigung $a$ [\cref{eq: beschleunigung_rollen_schiefe_ebene_1}] und die Geschwindigkeit $v$ [\cref{eq: energieerhaltung_rollen_schiefe_ebene}] gelten ganz allgemein für starre Körper mit Radius $r$ und Masse $M$, die ohne Gleiten auf einer schiefen Ebene rollen -- beispielsweise für Vollzylinder, Hohlzylinder, Vollkugeln und Hohlkugeln. Der einzige Unterschied zwischen den verschiedenen Körpern liegt in der Trägheitszahl $b = \frac{J_S}{M r^2}$, die vom Trägheitsmoment $J_S$ bezüglich der Drehachse durch den Schwerpunkt abhängt. Je größer das Trägheitsmoment eines Körpers ist, desto langsamer rollt er die schiefe Ebene hinunter, weil anteilig mehr Energie in die Rotation fließt. In \cref{tab: traegheitsfaktor_standardkoerper} sind die Trägheitsmomente und die zugehörigen Trägheitszahlen und Trägheitsfaktoren einiger Standardkörper zusammengefasst.

Der Hohlzylinder hat das größte Trägheitsmoment, weil die gesamte Masse in einem dünnen Ring am weitesten von der Drehachse entfernt ist. Damit hat er auch die maximale Trägheitszahl $b=1$ und rollt am langsamsten die schiefe Ebene hinunter. Die Hohlkugel hat das nächstkleinere Trägheitsmoment und die Trägheitszahl $b=\frac{2}{3} \approx 0.66 $. Danach kommt der Vollzylinder mit  $b=\frac{1}{2}= 0.5$ und schließlich die Vollkugel mit dem kleinsten Trägheitsmoment und einer Trägheitszahl von $b=\frac{2}{5}= 0.4$, die somit am schnellsten die schiefe Ebene hinunterrollt.
\begin{table}[htb]
    \centering
    \renewcommand{\arraystretch}{1.3}
    \begin{tabularx}{0.9\textwidth}{Xc|c|c}
        \toprule
        \textbf{Körper} & \textbf{Trägheitsmoment $J_S$} & \textbf{Trägheitszahl $b$} & \textbf{Trägheitsfaktor $\beta$} \\
        \midrule
        Vollzylinder & $\frac{1}{2} M r^2$ & $\frac{1}{2}$ & $\frac{2}{3}$ \\
        Hohlzylinder & $M r^2$ & $1$ & $\frac{1}{2}$ \\
        Vollkugel & $\frac{2}{5} M r^2$ & $\frac{2}{5}$ & $\frac{5}{7}$ \\
        Hohlkugel & $\frac{2}{3} M r^2$ & $\frac{2}{3}$ & $\frac{3}{5}$ \\
        \bottomrule
    \end{tabularx}
    \caption{Trägheitsmomente $J_S$, Trägheitszahlen $b$ und Trägheitsfaktoren $\beta$ einiger Standardkörper.}\label{tab: traegheitsfaktor_standardkoerper}
\end{table} 


\section{Analogie: Translation und Rotation}\label{sec: analogie_trans_rot}
Die mathematische Struktur der Formeln für Translation und Rotation ist identisch. Dies erleichtert das Lernen erheblich, da man oft nur die Variablen austauschen muss. Die wichtigsten korrespondieren physikalischen Größen sind in \cref{tab: analogie_translation_rotation} zusammengefasst.

\begin{table}[ht]
    \centering
    \renewcommand{\arraystretch}{1.5}
    \begin{tabularx}{0.9\textwidth}{Xc|cX}
        \toprule
        \textbf{Translation} & \textbf{Symbol} & \textbf{Symbol} & \textbf{Rotation} \\
        \midrule
        Ort & $\ivec{r}$ & $\varphi$ & Winkel \\
        Geschwindigkeit & $\ivec{v}$ & $\ivec{\omega}$ & Winkelgeschwindigkeit \\
        Beschleunigung & $\ivec{a}$ & $\ivec{\alpha}$ & Winkelbeschleunigung \\
        Masse & $m$ & $J$ & Trägheitsmoment \\
        Kraft & $\ivec{F}$ & $\ivec{M}$ & Drehmoment \\
        Impuls & $\ivec{p} = m\ivec{v}$ & $\ivec{L} = J\ivec{\omega}$ & Drehimpuls \\
        Kinetische Energie & $E_{\text{kin}} = \frac{1}{2}mv^2$ & $E_{\text{rot}} = \frac{1}{2}J\omega^2$ & Rotationsenergie \\
        Grundgleichung & $\ivec{F} = m\ivec{a}$ & $\ivec{M} = J\ivec{\alpha}$ & Grundgleichung \\
        \bottomrule
    \end{tabularx}
    \caption{Gegenüberstellung der physikalischen Größen von Translation und Rotation.}\label{tab: analogie_translation_rotation}
\end{table}

\section{Drehimpuls und Erhaltungssatz}\label{sec: drehimpuls}
Ähnlich wie der Impuls $\ivec{p}$ ein Maß für den translatorischen Schwung ist, beschreibt der \textbf{Drehimpuls} $\ivec{L}$ den \gDQ{Drall} eines Körpers. Der Drehimpuls eines Massenelements $\Delta m_i$ bezüglich einer Rotationsachse $\ivec{A}$ ist definiert als das Kreuzprodukt aus dem Normalabstandsvektor $\ivecS{r}{\perp, i}$ und dem Impuls $\ivecS{p}{i}$:
\begin{equation}\label{eq: drehimpuls}
    \ivecS{L}{i} = \ivecS{r}{\perp, i} \times \ivecS{p}{i} = \ivecS{r}{\perp, i} \times (\Delta m_i \ivecS{v}{i}) = r_{\perp,i}^2 \Delta m_i \ivec{\omega}\mDot
\end{equation}
Auf gleiche Weise wie in der Herleitung der Rotationsenergie (siehe \cref{subsec: rotationsenergie_herleitung}) können wir den gesamten Drehimpuls eines starren Körpers durch Integration über alle Volumenelemente berechnen. Für einen starren Körper, der sich mit der Winkelgeschwindigkeit $\ivec{\omega}$ dreht, ergibt sich der Gesamtdrehimpuls zu
\begin{equation}\label{eq: gesamtdrehimpuls_starrer_koerp}
    \ivec{L} = \left(\int_M r_{\perp}^2 \dd m \right) \cdot \ivec{\omega}  = \left(\int_V r_{\perp}^2 \, \rho(\ivec{r}) \, \dd V \right) \cdot \ivec{\omega} = J \cdot \ivec{\omega} \mDot
\end{equation}
Hier sei der Vollständigkeit halber erwähnt, dass \cref{eq: gesamtdrehimpuls_starrer_koerp} nur für Rotationen um sogenannte \textbf{Hauptträgheitsachsen} gilt -- für Körper mit gewisser Symmetrie fallen die Symmetrieachsen und Hauptträgheitsachsen meist zusammen. Für Rotationen um beliebige Achsen muss das Trägheitsmoment über den Trägheitstensor beschrieben werden, sodass $\ivec{L} \nparallel \ivec{\omega}$ gilt -- das übersteigt jedoch den Rahmen dieses Skriptums.\\

Eines der wichtigsten Prinzipien der Physik ist die \textbf{Drehimpulserhaltung}.
\begin{importantbox}{Drehimpulserhaltung}
    Wirkt auf ein System kein äußeres Drehmoment ($\ivec{M}_{\text{ext}} = 0$), so bleibt der Gesamtdrehimpuls konstant:
    \begin{equation}\label{eq: drehimpulserhaltung}
        \frac{\mathrm{d}\ivec{L}}{\mathrm{d}t} = 0 \quad \implies \quad \ivec{L} = \const \mDot
    \end{equation}
    
    Da $\ivec{L} = J \cdot \ivec{\omega}$ ist, führt eine Änderung des Trägheitsmoments $J$ (z.\,B. wenn ein Eiskunstläufer die Arme anzieht) zu einer Änderung der Winkelgeschwindigkeit $\omega$, damit das Produkt konstant bleibt.
\end{importantbox}
Es ist anzumerken, dass \cref{eq: drehimpulserhaltung} eine Vektor-Gleichung ist. Sowohl Betrag als auch Richtung des Drehimpulses bleiben konstant, wenn kein äußeres Drehmoment wirkt.