\chapter*{Allgemeines}\label{chap: Allgemeines}
\addcontentsline{toc}{chapter}{Allgemeines}

\section*{Das griechische Alphabet}\label{sec: griechisches_Alphabet}
\addcontentsline{toc}{section}{Das griechische Alphabet} 
Das griechische Alphabet wird in den Naturwissenschaften und der Technik häufig für Variablennamen verwendet. Beachten Sie, dass manche griechische Buchstaben zwei Arten der Notation haben. \\
\begin{center}
\definecolor{lightgray}{gray}{0.9}
\rowcolors{1}{white}{lightgray}
\begin{tabular}{l c c @{\quad\quad\vrule\quad\quad} l c c} 
%\hline
\textbf{Alpha} & A & $\alpha$ & \textbf{Ny} & N & $\nu$ \\
\textbf{Beta} & B & $\beta$ & \textbf{Xi} & $\Xi$ & $\xi$ \\
\textbf{Gamma} & $\Gamma$ & $\gamma$ & \textbf{Omikron} & O & $o$ \\
\textbf{Delta} & $\Delta$ & $\delta$ & \textbf{Pi} & $\Pi$ & $\pi$ \\
\textbf{Epsilon} & E & $\epsilon$, $\varepsilon$ & \textbf{Rho} & P & $\rho$ \\
\textbf{Zeta} & Z & $\zeta$ & \textbf{Sigma} & $\Sigma$ & $\sigma$ \\
\textbf{Eta} & H & $\eta$ & \textbf{Tau} & T & $\tau$ \\
\textbf{Theta} & $\Theta$ & $\theta$, $\vartheta$ & \textbf{Ypsilon} & Y & $\upsilon$ \\
\textbf{Iota} & I & $\iota$ & \textbf{Phi} & $\Phi$ & $\phi$, $\varphi$ \\
\textbf{Kappa} & K & $\kappa$ & \textbf{Chi} & X & $\chi$ \\
\textbf{Lambda} & $\Lambda$ & $\lambda$ & \textbf{Psi} & $\Psi$ & $\psi$ \\
\textbf{My} & M & $\mu$ & \textbf{Omega} & $\Omega$ & $\omega$ \\
%\hline 
\end{tabular}
\end{center}

\section*{Mathematische Symbole und Formeln}\label{sec: mathematische_Symbole-Formeln}
\addcontentsline{toc}{section}{Mathematische Symbole und Formeln} 
Um die Sprache der Mathematik zu verstehen, ist es wichtig, die Bedeutung der einzelnen Symbole zu kennen. In der Mathematik haben sich unzählige Symbole und Abkürzungen eingebürgert, die allerdings wie \gDQ{Wörter} zu lesen sind. Die meisten Formeln lesen sich daher wie ein Satz. Die gängigsten Symbole und Formeln, die zum Verständnis dieses Skriptums notwendig sind, sind hier aufgelistet.

\begin{center}
\definecolor{lightgray}{gray}{0.9}
\rowcolors{1}{lightgray}{white} 
\begin{tabular}{c l @{\qquad\vrule\qquad} c l}
$=$ & ist gleich & $\Delta x$ & Differenz \\
$\neq$ & ist ungleich & $dx$ & Differenzial \\
$\equiv$ & äquivalent & $x'$ & 1. Ableitung \\
$\approx$ & ungefähr gleich & $\dot{x}$ & 1. Ableitung nach der Zeit \\
$\propto$ & proportional zu & $\int$ & Integral \\
$\gg$ & viel größer als & $\Sigma$ & Summe \\
$\ll$ & viel kleiner als & $:=$ & Definition \\
$\hat{=}$ & entspricht & $\stackrel{?}{=}$ & Hypothese/Frage \\
$\wedge$ & mathematisches „Und“ & $\stackrel{!}{=}$ & bekannte Gleichheit \\
$\vee$ & mathematisches „Oder“ & $\infty$ & unendlich \\
\oBdA & ohne Beschränkung der Allgemeinheit & & \\
\end{tabular}
\end{center}
Der mathematische Begriff \textbf{ohne Beschränkung der Allgemeinheit}, oft mit \oBdA abgekürzt, ist eine gängige Formulierung in Beweisen. Sie signalisiert, dass der Beweis für einen speziellen Fall geführt wird, dieser aber so gewählt ist, dass er alle anderen möglichen Fälle repräsentiert. Die Gültigkeit des Beweises für diesen einen Fall überträgt sich somit auf alle anderen Fälle, ohne dass die Allgemeingültigkeit der Aussage eingeschränkt wird.

\section*{Häufig vorkommende Begriffe in der Physik}\label{sec: häufige_Begriffe_Physik}
\addcontentsline{toc}{section}{Physikalische Begriffe} 
Manchmal kommt es vor, dass man spezielle (physikalische) Begriffe zwar schon häufig gehört hat, deren Bedeutung aber nicht gänzlich verstanden wird. Diese Liste dient als Nachschlagewerk, um Begriffe, die nicht im allgemeinen Sprachgebrauch vorkommen, zu definieren. 
\vspace{0.3cm}
\begin{center}
\definecolor{lightgray}{gray}{0.9}
\rowcolors{2}{white}{lightgray} % Ab der zweiten Zeile abwechselnde Farben
\begin{tabularx}{0.9\textwidth}{l X} 
\toprule
\textbf{Begriff} & \textbf{Erklärung} \\
\midrule
quantitativ & zahlenmäßig \\
qualitativ & beschreibend, interpretierend \\
monochromatisch & aus einer einzelnen Wellenlänge oder Frequenz bestehend \\
homogen & einheitlich; gleichartig aufgebaut/zusammengesetzt \\
heterogen & uneinheitlich; nicht gleichartig aufgebaut/zusammengesetzt \\
Abszisse & „x-Achse“ \\
Ordinate & „y-Achse“ \\
normal, orthogonal & rechtwinklig \\
orthonormal & orthogonal und normiert \\
normiert & Länge ist 1 (bei Vektoren) \\
isotrop & unabhängig von der Richtung \\
isotherm & gleiche Temperatur \\
isobar & gleicher Druck \\
isochor & gleiches Volumen \\
isentrop & gleiche Entropie \\
Isotop & gleiche Protonenzahl \\
\bottomrule
\end{tabularx}
\end{center}

\newpage
\section*{Physikalische Größen}\label{sec: Physikalische_Größen}
\addcontentsline{toc}{section}{Physikalische Größen} 
Die folgende Tabelle stellt die gängigsten physikalischen Größen, die in diesem Skriptum vorkommen, übersichtlich dar. Die Basiseinheiten sind allesamt SI-Einheiten. Für diejenigen Größen, für die kein Einheitensymbol angegeben ist, verwendet man üblicherweise die Basiseinheiten. 
\begin{center}
\definecolor{lightgray}{gray}{0.9}
\rowcolors{2}{white}{lightgray}
\begin{tabularx}{\textwidth}{l c c c c}
\toprule
% \textbf{Größe} & \textbf{Einheit} & \textbf{\shortstack{Einheiten-\\symbol}} & \textbf{\shortstack{Basis-\\einheiten}} & \textbf{\shortstack{Formel-\\symbol}} \\
\thead{Größe} & \thead{Einheit} & \thead{Einheiten-\\symbol} & \thead{Basis-\\einheiten} & \thead{Formel-\\symbol} \\
\midrule
Zeit & Sekunde & \si{\second} & \si{\second} & $t$ \\
Ort, Weg, Länge & Meter & \si{\meter} & \si{\meter} & $\vec{s}$ \\
Masse & Kilogramm & \si{\kilogram} & \si{\kilogram} & $m$ \\
Trägheitsmoment & & & \si{\kilogram\,\meter^2} & $I, J$ \\
Temperatur & Kelvin & \si{\kelvin} & \si{\kelvin} & $T$ \\
Winkel & Radiant & \si{\radian} & 1 & $\alpha, \beta, \gamma, \varphi, \theta, \dots$ \\
Raumwinkel & Steradiant & \si{\steradian} & 1 & $\Omega$ \\
Geschwindigkeit & & & \si{\meter\per\second} & $\vec{v}$ \\
Beschleunigung & & & \si{\meter\per\second^2} & $\vec{a}$ \\
Winkelgeschwindigkeit & & & \si{\radian\per\second} & $\vec{\omega}$ \\
Winkelbeschleunigung & & & \si{\radian\per\second^2} & $\vec{\alpha}$ \\
Impuls & & & \si{\kilogram\meter\per\second} & $\vec{p}$ \\
Kraft & Newton & \si{\newton} & \si{\kilogram\meter\per\second^2} & $\vec{F}$ \\
Drehimpuls & & & \si{\kilogram\meter^2\per\second} & $\vec{L}$ \\
Drehmoment & & \si{\newton\,\meter} & \si{\kilogram\meter^2\per\second^2} & $\vec{M}$ \\
Wirkung & & \si{\joule\second} & \si{\kilogram\meter^2\per\second} & $S$ \\
Energie, Arbeit & Joule & \si{\joule} & \si{\kilogram\meter^2\per\second^2} & $E, W$ \\
Leistung & Watt & \si{\watt} & \si{\kilogram\meter^2\per\second^3} & $P$ \\
Frequenz & Hertz & \si{\hertz} & \si{\per\second} & $f, \nu$ \\
Wellenlänge & & & \si{\meter} & $\lambda$ \\
Lichtgeschwindigkeit & & & \si{\meter\per\second} & $c$ \\
Druck & Pascal & \si{\pascal} & \si{\kilogram\per(\meter\,\second^2)} & $p$ \\
Dichte & & & \si{\kilogram\per\meter^3} & $\rho$ \\
Volumen & & & \si{\meter^3} & $V$\\
Spannung & Volt & \si{\volt} & \si{\kilogram\meter^2\per(\second^3\,\ampere)} & $U$ \\
Stromstärke & Ampere & \si{\ampere} & \si{\ampere} & $I$ \\
Elektrischer Widerstand & Ohm & \si{\ohm} & \si{\kilogram\meter^2\per(\second^3\,\ampere^2)} & $R$ \\
Ladung & Coulomb & \si{\coulomb} & \si{\ampere\,\second} & $Q, q$ \\
\bottomrule
\end{tabularx}
\end{center}
\newpage


\section*{Einheitenpräfixe}\label{sec: Einheitenpräfixe}
\addcontentsline{toc}{section}{Einheitenpräfixe} 
Einheitenpräfixe, auch Vorsilben genannt, vereinfachen den Umgang mit (sehr) großen oder (sehr) kleinen Zahlenwerten. Ihre Aufgabe ist es, eine Basiseinheit (wie Meter, Gramm oder Sekunde) mit einer festen Zehnerpotenz zu skalieren (multiplizieren).

Man verwendet sie, um lange Ziffernfolgen zu vermeiden und Zahlen lesbarer und verständlicher zu machen. Anstatt zum Beispiel $\SI{1000}{\meter}$ zu schreiben, verwendet man das Präfix \textit{Kilo} und schreibt \SI{1}{\kilo\meter}. Genauso ist es bei sehr kleinen Werten praktischer, \SI{1}{\nano\meter} (Nanometer) statt \SI{0,000000001}{\meter} zu schreiben. Diese kompakte Schreibweise reduziert die Fehleranfälligkeit und erleichtert das alltägliche Arbeiten mit physikalischen Größen erheblich.

% \begin{center}
% \definecolor{lightgray}{gray}{0.9}
% \rowcolors{2}{white}{lightgray} % Abwechselnde Farben ab der 2. Datenzeile
% \begin{tabular}{l l l @{\qquad\vrule\qquad} l l l}
% \toprule
% \textbf{Präfix} & \textbf{Symbol} & \textbf{Faktor} & \textbf{Präfix} & \textbf{Symbol} & \textbf{Faktor} \\
% \midrule
% atto   & a      & $10^{-18}$ & exa    & E  & $10^{18}$ \\
% femto  & f      & $10^{-15}$ & peta   & P  & $10^{15}$ \\
% pico   & p      & $10^{-12}$ & tera   & T  & $10^{12}$ \\
% nano   & n      & $10^{-9}$  & giga   & G  & $10^{9}$  \\
% mikro  & \textmu & $10^{-6}$  & mega   & M  & $10^{6}$  \\
% milli  & m      & $10^{-3}$  & kilo   & k  & $10^{3}$  \\
% zenti  & c      & $10^{-2}$  & hekto  & h  & $10^{2}$  \\
% dezi   & d      & $10^{-1}$  & deka   & da & $10^{1}$  \\
% \bottomrule
% \end{tabular}
% \end{center}

\begin{center}
% Die Farbe wird wie gewohnt definiert
\definecolor{lightgray}{gray}{0.9}

% Verwende die NiceTabular-Umgebung
\begin{NiceTabular}{l c c @{\qquad\vrule\qquad} l c c}
\CodeBefore
  \rowcolors{2}{white}{lightgray}
\Body
\toprule
\textbf{Präfix} & \textbf{Symbol} & \textbf{Faktor} & \textbf{Präfix} & \textbf{Symbol} & \textbf{Faktor} \\
\midrule
atto    & a       & $10^{-18}$ & exa    & E  & $10^{18}$ \\
femto   & f       & $10^{-15}$ & peta   & P  & $10^{15}$ \\
pico    & p       & $10^{-12}$ & tera   & T  & $10^{12}$ \\
nano    & n       & $10^{-9}$  & giga   & G  & $10^{9}$  \\
mikro   & \textmu & $10^{-6}$  & mega   & M  & $10^{6}$  \\
milli   & m       & $10^{-3}$  & kilo   & k  & $10^{3}$  \\
zenti   & c       & $10^{-2}$  & hekto  & h  & $10^{2}$  \\
dezi    & d       & $10^{-1}$  & deka   & da & $10^{1}$  \\
\bottomrule
\end{NiceTabular}
\end{center}

\vspace{1em} % Ein kleiner vertikaler Abstand

Jede dieser Vorsilben kann mit beliebigen Einheiten kombiniert werden. Das Präfix steht dabei immer direkt vor der Einheit, die skaliert werden soll: Die Einheit $\si{\kilo\gram\meter\second^{-1}}$ ist demnach nicht dasselbe wie $\si{\gram\kilo\meter\second^{-1}}$.

\subsubsection*{Beispiele:}
\begin{multicols}{3}
\begin{itemize}
    \item \si{\kilo\gram}
    \item \si{\hecto\liter}
    \item \si{\mega\hertz}
    \item \si{\kilo\watt\hour}
    \item \si{\giga\byte}
    \item \si{\nano\second}
    \item \si{\micro\meter}
    \item \si{\milli\farad}
    \item \si{\kilo\ohm}
    \item \si{\peta\joule}
\end{itemize}
\end{multicols}